\documentclass[12pt]{ctexart}
\usepackage{amsfonts,amssymb,amsmath,amsthm,geometry,enumerate,fixdif}
\usepackage[colorlinks,linkcolor=blue,anchorcolor=blue,citecolor=green]{hyperref}
\usepackage[all]{xy}

%introduce theorem environment
\theoremstyle{definition}
\newtheorem{definition}{定义}
\newtheorem{theorem}{定理}
\newtheorem{lemma}{引理}
\newtheorem{corollary}{推论}
\newtheorem{property}{性质}
\newtheorem{proposition}{命题}
\newtheorem{example}{例}

\theoremstyle{plain}
\newtheorem*{solution}{解}
\newtheorem*{remark}{注}
\geometry{a4paper,scale=0.8}

\newcommand{\iu}{\mathrm{i}}
\newcommand{\eu}{\mathrm{e}}
\newcommand{\Arg}{\operatorname{Arg}}
\newcommand{\Ln}{\operatorname{Ln}}
\renewcommand{\Re}{\operatorname{Re}}
\renewcommand{\Im}{\operatorname{Im}}

%article info
\title{\vspace{-2em}\textbf{初等复变函数}\vspace{-2em}}
\date{ }
\begin{document}
	\maketitle
	\begin{definition}
		设$z\in\mathbb{C}$,定义$\eu^{z}=\displaystyle\sum_{n=0}^{\infty}\frac{z^n}{n!}$.
	\end{definition}
	\begin{property}
		$|\eu^z|=\eu^x>0$.
	\end{property}
	\begin{property}
	$\eu^{z_1}\eu^{z_2}=\eu^{z_1+z_2}$.
	\end{property}
	\begin{property}
		$\eu^z$以$2\pi\iu$为周期.
	\end{property}
	
	由Euler公式,有
	$$\eu^{\iu y}=\cos y+\iu\sin y.$$
	得
	$$\cos y=\frac{\eu^{\iu y}+\eu^{-\iu y}}{2},\qquad\sin y=\frac{\eu^{\iu y}-\eu^{-\iu y}}{2\iu}.$$
	于是对$z\in\mathbb{C}$,定义
	$$\cos z=\frac{\eu^{\iu z}+\eu^{-\iu z}}{2},\qquad\sin y=\frac{\eu^{\iu z}-\eu^{-\iu z}}{2\iu}.$$
	
	依然可以定义$\tan z=\dfrac{\sin z}{\cos z}$.
	\begin{property}
		$\sin z$和$\cos z$是无界的.
	\end{property}
	\begin{definition}
		满足$\eu^w=z$的复数$w$称为$z$的\textbf{对数},记作$\Ln z$.
	\end{definition}
	类似地可以定义$\arcsin z$,$\arccos z$以及$\arctan z$.
	\begin{property}
		$w=\Ln z$是多值函数.
	\end{property}
	\begin{definition}
		对于复数$\alpha$,定义幂函数
		$$w=z^{\alpha}=\eu^{\alpha\Ln z}.$$
	\end{definition}
	\begin{property}
		当$\alpha=\dfrac{m}{n}$,$m,n\in\mathbb{Z}$且$(m,n)=1$时,$w=z^{\alpha}$是$n$值的. 当$\alpha$是无理数时,$w=z^{\alpha}$有无穷多值.
	\end{property}
\end{document}