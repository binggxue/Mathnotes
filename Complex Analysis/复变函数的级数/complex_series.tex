\documentclass[12pt]{ctexart}
\usepackage{amsfonts,amssymb,amsmath,amsthm,geometry,enumerate,fixdif}
\usepackage[colorlinks,linkcolor=blue,anchorcolor=blue,citecolor=green]{hyperref}
\usepackage[all]{xy}

%introduce theorem environment
\theoremstyle{definition}
\newtheorem{definition}{定义}
\newtheorem{theorem}{定理}
\newtheorem{lemma}{引理}
\newtheorem{corollary}{推论}
\newtheorem{property}{性质}
\newtheorem{proposition}{命题}
\newtheorem{example}{例}

\theoremstyle{plain}
\newtheorem*{solution}{解}
\newtheorem*{remark}{注}
\geometry{a4paper,scale=0.8}

\newcommand{\iu}{\mathrm{i}}
\newcommand{\eu}{\mathrm{e}}
\newcommand{\Arg}{\operatorname{Arg}}
\renewcommand{\Re}{\operatorname{Re}}
\renewcommand{\Im}{\operatorname{Im}}

%article info
\title{\vspace{-2em}\textbf{复变函数项级数}\vspace{-2em}}
\date{ }
\begin{document}
	\maketitle
	\begin{definition}
		对任意$z_0\in\varOmega$,任意$\varepsilon>0$,存在$N\in\mathbb{N}^+$,使得对任意$n>N$,有
		$$\left|f_n(z_0)-f_n(z)\right|<\varepsilon,$$
		则称函数列$\left\{f_n\right\}$\textbf{逐点收敛}于$f$,记作
		$$f_n\rightarrow f.$$
	\end{definition}
	\begin{definition}
		对任意$\varepsilon>0$,存在$N\in\mathbb{N}^+$,使得对任意$n>N$,任意$z_0\in\varOmega$,有
		$$\left|f_n(z_0)-f_n(z)\right|<\varepsilon,$$
		则称函数列$\left\{f_n\right\}$\textbf{一致收敛}于$f$,记作
		$$f_n\rightrightarrows f.$$
	\end{definition}
	\begin{definition}
		对$\varOmega$的任意紧子集$K$,$f_n$在$K$上一致收敛于$f$,则称$f_n$在$\varOmega$上\textbf{紧一致收敛},记作$f_n\stackrel{c}{\rightrightarrows}f$.
	\end{definition}
	下面主要讨论一致收敛.
	\begin{theorem}[Cauchy]
		函数列$\left\{f_n\right\}$一致收敛当且仅当对任意$\varepsilon>0$,存在$N\in\mathbb{N}^+$,对任意$n,m>N$,任意$z_0\in\varOmega$,有
		$$\left\|f_n(z_0)-f_m(z_0)\right\|<\varepsilon.$$
	\end{theorem}
	\begin{theorem}[Weierstrass]
		设$a_n\geqslant 0$,若$|f_n|\leqslant a_n,\ \forall z\in\varOmega$且$\sum_{n\geqslant 0}^{\infty}a_n<\infty$,则$\sum_{n\geqslant 0}^{\infty}f_n(z)$一致收敛.
	\end{theorem}
	\begin{definition}
		设$a_n\in\mathbb{C}$,则形如
		$$\sum_{n=0}^{\infty}a_n(z-z_0)^{n}$$
		的级数称为\textbf{幂级数}.
	\end{definition}
	由于可以作平移变换$\tilde{z}=z-z_0$,于是下面研究
	\begin{equation}\label{power}
		\sum_{n=0}^{\infty}a_nz^n
	\end{equation}
	的情形. 首先考虑它的绝对收敛性. 注意到如果对于$z_0\in\mathbb{C}$,级数(\ref{power})绝对收敛,那么对任意$z$满足$|z|<|z_0|$,级数(\ref{power})也是绝对收敛的. 于是有以下定理.
	\begin{theorem}[Abel]
		对于幂级数$\sum_{n=0}^{\infty}a_nz^n$,存在一个数$0\leqslant R\leqslant\infty$使得
		\begin{enumerate}
			\item 若$|z|<R$,则幂级数绝对收敛;
			\item 若$|z|>R$,则幂级数发散.
		\end{enumerate}
		且有Cauchy-Hadamard公式:
		$$R=\left(\limsup_{n\to\infty}|a_n|^{1/n}\right)^{-1}.$$
	\end{theorem}
	\begin{theorem}
		函数$f(z)=\sum_{n=0}^{\infty}a_nz^n$是它的收敛域上的全纯函数,收敛半径$R>0$,则
	$$f'(z)=\sum_{n=0}^{\infty}na_nz^{n-1},$$
	且$f'$和$f$有相同的收敛半径.
	\end{theorem}
	\begin{proof}
		设$f_1=\sum_{n=0}^{\infty}na_nz^{n-1}$,
		
		设$f$的收敛半径为$R$,由于$\lim\limits_{n\to\infty}n^{1/n}=1$,于是
		$$R^{-1}=\limsup_{n\to\infty}|a_n|^{1/n}=\limsup_{n\to\infty}|na_n|^{1/n},$$
		$f_1$和$f$有相同的收敛半径.
		
		设$|z_0|<r<R$,设$f(z)=S_N(z)+E_N(z)$,$f_1(z)=S_N'(z)+T_N(z)$,其中
		$$S_N(z)=\sum_{n=0}^{N}a_nz^n,\quad E_N(z)=\sum_{N+1}^{\infty}a_nz^n,\quad S_N'(z)=\sum_{n=1}^{N}na_nz^{n-1},\quad T_N(z)=\sum_{n=N+1}^{\infty}na_nz^{n-1}.$$
		于是
		$$\left|\frac{f(z)-f(z_0)}{z-z_0}-f_1(z_0)\right|=\left|\frac{S_N(z)-S_N(z_0)}{z-z_0}-S_N'(z_0)+\frac{E_N(z)-E_N(z_0)}{z-z_0}-T_N(z_0)\right|,$$
		
		由于当$N\to\infty$时,$S_N'(z)\to f_1(z)$,即$T_N\to 0$,对任意$\varepsilon>0$,存在$N_1\in\mathbb{N}^+$,当$N>N_1$时,有$|T_N(z_0)|<\varepsilon$.
		
		已知$|z_0|<r$,不妨设$|z|<r$,则
		\begin{equation}
			\begin{aligned}
				\left|\frac{E_N(z)-E_N(z_0)}{z-z_0}\right|&=\left|\sum_{n=N+1}^{\infty}a_n\frac{z^n-z_0^n}{z-z_0}\right|\\
				&=\left|\sum_{n=N+1}^{\infty}a_n\left(z^{n-1}+z^{n-2}z_0+\cdots+zz_0^{n-2}+z_0^{n-1}\right)\right|\\
				&\leqslant\sum_{n=N+1}^{\infty}|a_n|(nr^{n-1})
			\end{aligned}
		\end{equation}
		因为$\lim\limits_{n\to\infty}\left(|a_n|n\right)^{n-1}=R^{-1}$,于是$\sum_{n=1}^{\infty}|a_n|(nr^{n-1})$收敛. 于是存在$N_2\in\mathbb{N}^+$,当$N>N_2$,$|z|<r$时,
		$$\left|\frac{E_N(z)-E_N(z_0)}{z-z_0}\right|<\varepsilon.$$
		
		对于$N>\max\{N_1,N_2\}$,
		$$\limsup_{z\to z_0}\left|\frac{f(z)-f(z_0)}{z-z_0}-f_1(z_0)\right|\leqslant\limsup_{z\to z_0}\left|\frac{S_N(z)-S_N(z_0)}{z-z_0}-S_N'(z_0)\right|+2\varepsilon=2\varepsilon.$$
		于是
		$$\limsup_{z\to z_0}\left|\frac{f(z)-f(z_0)}{z-z_0}-f_1(z_0)\right|=0,$$
		$$\lim\limits_{z\to z_0}\frac{f(z)-f(z_0)}{z-z_0}=f_1(z_0).$$
		即$$f'(z_0)=f_1(z_0)=\sum_{n=0}^{\infty}na_nz_0^{n-1}.$$
	\end{proof}
	重复进行,得到
	$$f^{(k)}=k!a_k+\frac{(k+1)!}{1!}a_{k+1}z+\frac{(k+2)!}{2!}a_{k+2}z^2+\cdots$$
	对任意正整数$k$都成立.由此可得$a_k=\dfrac{f^k(0)}{k!}$,于是
	$$f(z)=f(0)+f'(0)z+\frac{f''(0)}{2!}z^2+\cdots+\frac{f^{(k)}(0)}{k!}z^k+\cdots,$$
	称为\textbf{Taylor-Maclaurin级数}.
	\begin{theorem}
		设$f_n:\varOmega\to\mathbb{C}$,若$f_n\in C(\varOmega)$,$f_n\rightrightarrows f$,则$f\in C(\varOmega)$.
	\end{theorem}
	\begin{theorem}
		设$f_n:\varOmega\to\mathbb{C}$,若$\gamma$是可求长曲线,$f_n\in C(\gamma)$,在$\gamma$上$f_n\rightrightarrows f$,则
		$$\int_{\gamma}f(z)\d z=\lim\limits_{n\to\infty}\int_{\gamma}f_n(z)\d z.$$
	\end{theorem}
	\begin{theorem}
		设$f_n\in H(\varOmega)$,若$f_n\stackrel{c}{\rightrightarrows}f$,则$f\in H(\varOmega)$.
	\end{theorem}
	\begin{theorem}[Osgood]
		设$f_n\in H(\varOmega)$,若$f_n\rightarrow f$,则存在$\varOmega$的一个开稠子集$\tilde{\varOmega}$,使得$f\in H(\tilde{\varOmega})$.
	\end{theorem}
\end{document}