\documentclass[lang=cn,10pt]{elegantbook}

\usepackage{amsfonts,amssymb,fixdif}

\title{常微分方程讲义}
\subtitle{Ordinary Differential Equations}

\author{薛冰}
\date{\today}

%table of content depth,目录显示的深度
\setcounter{tocdepth}{3} 

% 修改封面的颜色带
\definecolor{customcolor}{RGB}{252,134,150}
\colorlet{coverlinecolor}{customcolor}

\cover{Cover.png}

\begin{document}
	\maketitle
	
	\frontmatter
	\tableofcontents
	\mainmatter
	
	\chapter{实数理论}

数学分析研究的基本对象是定义在实数集上的函数.
\section{实数的定义}%或许可以改成“有理数的扩充”
\subsection{建立实数的原则}
\begin{definition}[数域]\label{def:field}
	设$P$是由一些复数组成的集合,其中包括0和1,如果$P$中任意两个数的和、差、积、商(除数不为0)仍为$P$中的数,则称$P$为一个{\heiti 数域}.
\end{definition}
由定义\ref{def:field}可知,数域对加、减、乘、除(除数不为0)四则运算具有封闭性,即结果仍在数域本身中。例如,全体有理数所构成的集合$\mathbb{Q}$是一个数域,称为有理数域.此外,常见的数域还有复数域$\mathbb{C}$,读者可自行验证.
\begin{example}
	证明全体有理数所构成的集合$\mathbb{Q}$是一个数域.
	\begin{proof}
		对$\forall a=\frac{p}{q},\ b=\frac{s}{t}$,其中$p,q,s,t\in \mathbb{Z}$且$st\neq0$,则由有理数的定义知,$a,b\in \mathbb{Q}$
		
		显然$0,1\in\mathbb{Q}$,
		
		$a+b=\frac{pt+qs}{qt}\in \mathbb{Q}$
		
		$a-b=\frac{pt-qs}{qt}\in \mathbb{Q}$
		
		$a\cdot b=\frac{ps}{qt}\in \mathbb{Q}$
		
		$\frac{a}{b}=\frac{pt}{qs}\in \mathbb{Q}$
		
		故$\mathbb{Q}$是一个数域.$\hfill\blacksquare$
	\end{proof}
\end{example}
\begin{definition}[阿基米德有序域]
	集合$F$构成一个{\heiti 阿基米德有序域},是说它满足以下三个条件:
	\begin{enumerate}
		\item $F$是域\qquad 在$F$中定义了加法“$+$”和乘法“$\cdot$”两种运算,使得对于$F$中任意元素$a$,$b$,$c$成立:\par
		加法的结合律:\ $(a+b)+c=a+(b+c)$;\par
		加法的交换律:\ $a+b=b+a$;\par
		乘法的结合律:\
		$(a\cdot b)\cdot c=a\cdot (b\cdot c)$;\par
		乘法的交换律:\ $a\cdot b=b\cdot a$;\par
		乘法关于加法的分配律:
		$(a+b)\cdot c=a\cdot c+b\cdot c$;\par
		$F$中对加法存在{\heiti 零元素}和{\heiti 负元素}(即存在加法的逆运算减法);
		
		$F$中对乘法存在{\heiti 单位元素}和{\heiti 逆元素}(即存在乘法的逆运算除法).
		\item $F$是有序域\qquad
		在$F$中定义了{\heiti 序}关系“\textless”具有如下{\heiti 全序}的性质:\par
		传递性:$\forall a,b,c \in F$,若$a<b$,$b<c$,则$a<c$;\par
		三歧性:$\forall a,b \in F$,$a>b$,$a<b$,$a=b$三者必居其一,也只居其一;\par
		加法保序性:$\forall a,b,c \in F$,若$a<b$,则$a+c<b+c$;\par
		乘法保序性:$\forall a,b,c \in F$,若$a<b$,则$ac<bc$\ (c>0).
		\item $F$中元素满足阿基米德性\qquad 对$F$中两个正元素$a$,$b$,必存在自然数$n$,使得$na>b$.
	\end{enumerate}
\end{definition}
\subsection{Dedekind分割}
设$A/B$是有理数域$\mathbb{Q}$上的一个分割,即把$\mathbb{Q}$中的元素分为$A$、$B$两个集合,使得$\forall a \in A,\ b \in B$有$a<b$成立.则从逻辑上分为下列四种情况:
\begin{enumerate}
	\item $A$有最大值$a_0$,$B$有最小值$b_0$;
	\item $A$有最大值$a_0$,$B$无最小值;
	\item $A$无最大值,$B$有最小值$b_0$;
	\item $A$无最大值,$B$无最小值.
\end{enumerate}	

而对于第1种情况,取$\frac{a_0+b_0}{2}\in \mathbb{Q}$,它将不属于集合$A$、$B$中的任何一个.\par 
对于第4种情况,说明分割到的数不在$\mathbb{Q}$内(我们将这种划分称为{\heiti 无端划分}),因此这是我们通过“切割”构造出来的“新数”.将所有这样切割出来的新数与原来的$\mathbb{Q}$取并集,并设新的集合为$\mathbb{R}$.将$\mathbb{R}$中的元素再次进行分割,设$A'/B'$为$\mathbb{R}$上的一个分割,则$\forall a \in A',\ b \in B'$有$a<b$成立.同理,在排除上述的第1种情况后,对$\mathbb{R}$的分割分为以下三种情况:
\begin{enumerate}
	\item $A'$有最大值$a_0$,$B'$无最小值;
	\item $A'$无最大值,$B'$有最小值$b_0$;
	\item $A'$无最大值,$B'$无最小值.
\end{enumerate}	

由此,我们给出Dedekind定理.
\begin{theorem}[Dedekind定理]
	设$A'$、$B'$是$\mathbb{R}$的两个子集,且满足:
	\begin{enumerate}
		\item $A'$和$B'$均不为空集;
		\item $A'\cup B'=\mathbb{R}$;
		\item $\forall a\in A',\ b\in B'$有$a<b$.
	\end{enumerate}
	则或者$A'$有最大元素,或者$B'$有最小元素.
\end{theorem}
Dedikind定理指出,我们在上述对$\mathbb{R}$进行分割时,只会出现第1种或第2种情况,而不可能$A'$无最大值且$B'$也无最小值。
\begin{proof}
	设$A$是$A'$中所有有理数的集合,设$B$是$B'$中所有有理数的集合,则$A/B$有三种情况:
	\begin{enumerate}
		\item $A$有最大值$a_0$,$B$无最小值;
		\item $A$无最大值,$B$有最小值$b_0$;
		\item $A$无最大值,$B$无最小值;
	\end{enumerate}	
	
	对于第1种情况,设$a'\in A'$,使得$a'>a_0$,则在$(a_0,a')$之间必定存在有理数$a$,这与$A$有最大值$a_0$矛盾.因此$a_0$也是$A'$的最大值;
	
	同理,对于第2种情况,我们可以得到$b_0$也是$B'$的最小值;
	
	对于第3种情况,设$c$是由$A/B$得到的无理数,则$a_0<c<b_0$,可知$c$或者是$A'$中的元素,或者是$B'$中的元素.不妨设$c\in A'$,可以证明c为$A'$的最大元素,因为如果存在$a$是$A$的最大元素,那么区间$(c,a)$之间必定存在有理数大于$a_0$,这与$A$的最大值是$a_0$矛盾.
	
	以上我们就证明了或者$A'$有最大元素,或者$B'$有最小元素.$\hfill\blacksquare$
\end{proof}
\subsection{实数的公理化定义}
由前两节的理论基础,我们可以定义{\heiti 实数}构成一个阿基米德有序域,且满足Dedekind定理.实数分为有理数和无理数,其中无理数由有理数的无端划分产生.
\section{确界原理}
确界原理是极限理论的基础.
\subsection{确界的定义}
\begin{definition}
	设$S$为$\mathbb{R}$中的一个数集,若存在数$M(L)$,使得对一切$x\in S$,都有$x\leqslant M(x\geqslant L)$,则称$S$为{\heiti 有上界(下界)的数集},数$M(L)$称为$S$的一个{\heiti 上界(下界)}.
\end{definition}
\begin{definition}[上确界]
	设$S$是$\mathbb{R}$中的一个数集,若数$\eta$满足:
	
	(i)$\eta$是$S$的上界;
	
	(ii)$\forall \alpha < \eta,\ \exists x_0\in S,\ s.t.x_0>\alpha$,即$\eta$又是$S$的最小上界,\\
	则称$\eta$为数集$S$的{\heiti 上确界},记作$\eta=\sup S$
\end{definition}
\begin{definition}[下确界]
	设$S$是$\mathbb{R}$中的一个数集,若数$\xi$满足:
	
	(i)$\xi$是$S$的下界;
	
	(ii)$\forall \beta > \xi,\ \exists x_0\in S,\ s.t.x_0<\beta$,即$\xi$又是$S$的最小下界,\\
	则称$\xi$为数集$S$的{\heiti 下确界},记作$\xi=\inf S$
\end{definition}
上(下)确界也可以由$\varepsilon$语言定义.
\begin{definition}[上确界的$\varepsilon$语言定义]
	设$S$为$\mathbb{R}$中的一个数集,若S的一个上界$M$,$\forall \varepsilon>0$,$\exists a\in S$,s.t.\ $a>M-\varepsilon$,则数$M$称为$S$的一个{\heiti 上确界}.
\end{definition}
\begin{definition}[下确界的$\varepsilon$语言定义]
	设$S$为$\mathbb{R}$中的一个数集,若S的一个下界$L$,$\forall \varepsilon>0$,$\exists b\in S$,s.t.\ $b<L+\varepsilon$,则数$L$称为$S$的一个{\heiti 下确界}.
\end{definition}
上确界和下确界统称为确界.
\subsection{确界原理及其证明}
\begin{theorem}[确界原理]
	设$S$为$\mathbb{R}$中的一个数集,若$S$有上界,则必有上确界;若$S$有下界,则必有下确界.
\end{theorem}
\begin{proof}
	设数集$S$有上界,下面证明$S$有上确界.
	
	设$B$是数集$S$所有上界组成的集合,记$A=\mathbb{R}\textbackslash B$.若$B$有最小元素,则$B$的最小元素$b_0$即为$S$的上确界.
	
	设$x\in A$,$x$不是$S$的上界,则$\exists t\in S,\ s.t.\ x<t$,取$x'=\frac{x+t}{2}$,则$x'>x$,因此对$A$中任何一个元素$x$,都有$x'>x$存在,即$A$没有最大元素.由Dedekind定理,$B$一定有最小元素,即$S$必有上确界.$\hfill\blacksquare$
\end{proof}
\begin{example}
	利用确界原理证明Dedekind定理,即证明二者的等价关系.
\end{example}
\begin{proof}
	设$\mathbb{R}$上任意一个Dedekind分割为$A/B$,易知$B$中的每个元素都是$A$的一个上界.由确界原理,$A$一定有上确界.设$m=\sup A$,
	
	若$m\in B$,则$A$中无最大元素,假设$B$中无最小元素,则$\exists m'\in B\ s.t.\ m'<m$,由于$m=\sup A$,推出$m'\in A$,这与$m'\in B$矛盾.故$B$中有最小元素.
	
	若$m\in A$,则$A$中有最大元素$m$,假设$B$中有最小元素$n$,则$\frac{m+n}{2}$不属于$A$和$B$,这与$A\cup B=\mathbb{R}$是矛盾的,故$B$中无最小元素.
	
	于是我们就证明了Dedekind定理.$\hfill\blacksquare$
\end{proof}
\section{实数的完备性}
本节内容是基于第二章中数列极限的前提下展开的,建议在学完第二章后进行学习.数列极限的定义与基本性质在此不再赘述.

\subsection{关于实数集完备性的基本定理}
前面我们已经学习了Dedekind分割与确界原理,从不同的角度反映了实数集的特性,通常称为{\heiti 实数的完备性}或{\heiti 实数的连续性}公理.下面我们将介绍另外的几个实数的完备性公理.

\begin{theorem}[单调有界收敛定理]
	在实数系中,有界的单调数列必有极限.
\end{theorem}
\begin{proof}
	不妨设$\left\{a_n\right\}$为有上界的递增数列,由确界原理,$\left\{a_n\right\}$必有上确界,设$a=\sup \left\{a_n\right\}$,根据上确界的定义,
	
	$\forall \varepsilon>0,\ \exists a_N \ s.t.\ a_N>a-\varepsilon$
	
	由$\left\{a_n\right\}$的递增性,当$n\geqslant N$时,有
	$$a-\varepsilon<a_N\leqslant a_n$$
	
	又因为$$a_n\leqslant a<a+\varepsilon$$
	
	故$$a-\varepsilon<a_n<a+\varepsilon$$
	
	即$${\lim_{n \to +\infty}a_n}=a$$
	$\hfill\blacksquare$
\end{proof}

\begin{theorem}[致密性定理]
	任何有界数列必定有收敛的子列.
\end{theorem}
要证明此定理,可以先证明以下引理.
\begin{lemma}\label{zilie}
	任何数列都存在单调子列.
\end{lemma}
\begin{proof}
	设数列为$\left\{a_n\right\}$,下面分两种情况讨论:
	\begin{enumerate}
		\item 若$\forall k\in \mathbb{Z}_+$,$\left\{a_{k+n}\right\}$都有最大项,记$\left\{a_{1+n}\right\}$的最大项为$a_{n_1}$,则$a_{{n_1}+n}$也有最大项,记作$a_{n_2}$,显然有$a_{n_1}\geqslant a_{n_2}$,同理,有$$a_{n_2}\geqslant a_{n_3}$$
		$$.........$$
		由此得到一个单调递减的子列$\left\{a_{n_k}\right\}$
		\item 若至少存在一个正整数$k$,使得$\left\{a_{k+n}\right\}$没有最大项,先取$n_1=k+1$,总存在$a_{n_1}$后面的项$a_{n_2}$($n_2>n_1$)使得$$a_{n_2}>a_{n_1}$$,同理,总存在$a_{n_2}$后面的项$a_{n_3}$($n_3>n_2$)使得$$a_{n_3}>a_{n_2}$$
		$$.........$$
		由此得到一个严格递增的子列$\left\{a_{n_k}\right\}$
	\end{enumerate}
	
	综上,命题得证.$\hfill\blacksquare$
\end{proof}
下面是对致密性定理的证明:
\begin{proof}
	设数列$\left\{a_n\right\}$有界,由引理\ref{zilie},数列$\left\{a_n\right\}$存在单调且有界的子列,由单调有界收敛定理得出该子列是收敛的.$\hfill\blacksquare$
\end{proof}
\begin{theorem}[柯西(Cauchy)收敛准则]
	数列$\left\{a_n\right\}$收敛的充要条件是:\par 
	$\forall \varepsilon>0,\ \exists N\in \mathbb{Z}_+,\ s.t.\ n,\ m>N$时,有
	$$\lvert a_n - a_m \rvert<\varepsilon$$
\end{theorem}
单调有界只是数列收敛的充分条件,而柯西收敛准则给出了数列收敛的充要条件.
\begin{proof}
	{\heiti 必要性}\qquad 设$\lim\limits_{n \to +\infty}a_n=A$,则$\forall \varepsilon>0,\ \exists N\in \mathbb{Z}_+\ s.t.\ n,\ m>N$时,有$$\lvert a_n-A\rvert <\frac{\varepsilon}{2},\ \lvert a_n-A\rvert <\frac{\varepsilon}{2}$$
	
	因而$$\lvert a_n-a_m\rvert \leqslant \lvert a_n-A\rvert + \lvert a_m-A\rvert=\varepsilon$$
	
	{\heiti 充分性}\qquad 先证明该数列必定有界.取$\varepsilon=1$,因为$\left\{a_n\right\}$满足柯西收敛准则的条件,所以$\exists N_0,\ \forall n>N_0$,有
	$$\lvert a_n-a_{N_0+1}\rvert <1$$
	
	取$M=\max\left\{\lvert a_1 \rvert,\ \lvert a_2 \rvert,\ \cdot\cdot\cdot\,\ \lvert a_{N_0} \rvert,\ \lvert a_{N_0+1} \rvert+1\right\}$,则对一切$n$,成立$$\lvert a_n \rvert\leqslant M$$
	
	由致密性原理,在$\left\{a_n\right\}$中必有收敛子列$$\lim_{k \to +\infty}a_{n_k}=\xi$$
	
	由条件,$\forall \varepsilon>0,\ \exists N$,当$n,\ m>N$时,有$$\lvert a_n-a_m\rvert <\frac{\varepsilon}{2}$$
	
	在上式中取$a_m=a_{n_k}$,其中$k$充分大,满足$n_k>N$,并且令$k \to \infty$,于是得到
	$$\lvert a_n-\xi \rvert\leqslant \frac{\varepsilon}{2}< \varepsilon $$
	
	即数列$\left\{a_n\right\}$收敛.$\hfill\blacksquare$
\end{proof}
\begin{definition}[闭区间套]
	设闭区间列$\left\{\left[a_n,b_n\right]\right\}$具有如下性质:
	\begin{enumerate}
		\item $\left[a_n,b_n\right]\supset \left[a_n+1,b_n+1\right],\ n=1,2,\cdots$;
		\item $\lim\limits_{n \to +\infty}(b_n-a_n)=0$.
	\end{enumerate}
	则称$\left\{\left[a_n,b_n\right]\right\}$为{\heiti 闭区间套},或简称{\heiti 区间套}.
\end{definition}

由性质1,构成闭区间套的闭区间列是前一个套着后一个的,即各闭区间端点满足如下不等式:
\begin{equation}\label{chuan}
	a_1\leqslant a_2\leqslant \cdots\leqslant a_n\leqslant\cdots\leqslant b_n\leqslant\cdots\leqslant b_2\leqslant b_1.
\end{equation}
\begin{theorem}[闭区间套定理]
	若$\left\{\left[a_n,b_n\right]\right\}$是一个闭区间套,则在实数系中存在唯一的一点$\xi$,使得$\xi\in\left[a_n,b_n\right],\ n=1,2,\cdots$,即$$a_n\leqslant\xi\leqslant b_n,\ n=1,2,\cdots.$$
\end{theorem}
\begin{proof}
	由式\ref{chuan}可以看出,数列$\left\{a_n\right\}$是递增数列且有界,$\left\{b_n\right\}$是递减数列且有界,由单调有界收敛定理,可知$\left\{a_n\right\}$和$\left\{b_n\right\}$都收敛.设$\lim\limits_{n\to \infty}a_n=\xi$,由闭区间套的第2条性质,得$\lim\limits_{n\to \infty}b_n=\xi.$
	
	$\left\{a_n\right\}$是递增数列,有$a_n\leqslant\xi,\ n=1,2,\cdots;$
	
	$\left\{b_n\right\}$是递减数列,有$b_n\geqslant\xi,\ n=1,2,\cdots.$\\
	所以有$a_n\leqslant\xi\leqslant b_n,\ n=1,2,\cdots.$\\
	下面证明$\xi$的唯一性:\\
	假设存在$\xi'$满足$a_n\leqslant\xi'\leqslant b_n,\ n=1,2,\cdots.$,则\\
	$$\lvert\xi'-\xi\rvert\leqslant b_n-a_n,\ n=1,2,\cdots,$$\\
	由闭区间套的第2条性质,有
	$$\lvert\xi'-\xi\rvert\leqslant \lim\limits_{n\to\infty}(b_n-a_n)=0,\ n=1,2,\cdots,$$\\
	故$\xi'=\xi$.$\hfill\blacksquare$
\end{proof}
\begin{definition}
	设$S$为数轴上的点集,$H$为开区间的集合(即$H$的每一个元素都是形如$(\alpha,\beta)$的开区间).若$S$中的任何一点都含在$H$中至少一个开区间内,则称$H$为$S$的一个{\heiti 开覆盖},或称$H$覆盖$S$.若$H$中开区间的个数是无限(有限)的,则称$H$为$S$的一个{\heiti 无限开覆盖(有限开覆盖)}.若存在$S$的开覆盖$H'\subseteq H$,则称$H'$是$H$的{\heiti 子覆盖},特别地,当$H'$中含有的开区间的个数为有限个时,称$H'$为$H$的{\heiti 有限子覆盖}.
\end{definition}
\begin{theorem}[Heine-Borel有限覆盖定理]
	设$H$是闭区间$\left[a,b\right]$的一个(无限)开覆盖,则从$H$中能选出有限个开区间来覆盖$\left[a,b\right]$.
	
	即:有限闭区间的任一开覆盖都存在一个有限子覆盖.
\end{theorem}
\begin{proof}
	设$H$是闭区间$\left[a,b\right]$的一个开覆盖,定义集合
	$$S=\left\{x|x\in \left(a,b \right],\ \mbox{且}\left[a,x\right]\mbox{存在开覆盖}H\mbox{的一个有限子覆盖} \right\}.$$
	
	因为$H$是$\left[a,b\right]$的一个开覆盖,所以存在一个区间$I_0\in H$使得$a\in I_0$,则存在$x_0\in I_0$满足$x_0>a$,所以$S\neq\varnothing$.显然$b$是$S$的一个上界,由确界原理,$S$一定有上确界.设$M=\sup S\leqslant b$,下面证明$M=b$:
	
	反证法\qquad 假设$M<b$,则$M\in \left(a,b \right]$,$\left[a,M\right]$存在$H$的一个有限子覆盖.假设开区间$I_1$包含$M$,则存在$\delta >0$使得$(M-\delta,M+\delta)\subseteq I_1$,因为$M$是$S$的上确界,所以$M-\delta\in S$,记$\left[a,M-\delta\right]$的有限开覆盖为$H'$,则$\left[a,M+\delta\right]$也有有限开覆盖$H'\cup I_1$,得$M+\delta\in S$这与$M$是$S$的上确界矛盾.所以$M=b$,即$\left[a,b\right]$的开覆盖$H$存在一个有限子覆盖.$\hfill\blacksquare$
\end{proof}
\begin{remark}
	法国数学家Borel于1895年第一次陈述并证明了现代形式的Heine-Borel定理.此定理只对有限闭区间成立,而对开区间则不一定成立.例如开区间集合$$\left\{(\frac{1}{n+1},1)\right\},\ (n=1,2,\cdots)$$构成了开区间$(0,1)$的开覆盖,但不能从中选出有限个开区间覆盖住$(0,1)$.
\end{remark}
从上面的讨论我们发现,如果从数轴上取下一段“紧致无缝”的集合(含端点),那么就可以从它的任意开覆盖中取出一个有限子覆盖,否则就不行.这表明我们找到了一个刻画实数集完备性的新方法,我们形象地将这个性质称为“紧致性”.
\begin{definition}[紧致集]
	设集合$E\in\mathbb{R}$,若集合$E$的任一开覆盖都存在一个有限子覆盖,则称$E$为$\mathbb{R}$上的一个{\heiti 紧致集}.
\end{definition}
\begin{remark}
	紧致集也称紧集,是一个重要的拓扑概念.
\end{remark}
\begin{remark}
	以后我们会将以上条件称为"Heine-Borel条件".
\end{remark}
我们可以重新表述Heine-Borel有限覆盖定理.
\begin{theorem}[Heine-Borel有限覆盖定理]
	$\mathbb{R}$中的任一有限闭区间都是紧致集.
\end{theorem}
\begin{definition}[邻域]
	设$a\in\mathbb{R},\ \delta>0$,将满足$\lvert x-a\rvert<\delta$的全体$x$的集合称为{\heiti $a$的$\delta$邻域},记作$U(a,\delta)$.将满足$0<\lvert x-a\rvert<\delta$的全体$x$的集合称为{\heiti $a$的$\delta$去心邻域},记作$\mathring{U}(a,\delta)$.
\end{definition}
显然,邻域与去心邻域的区别在于去心邻域不包含中心点$a$.
\begin{definition}[聚点]
	设$S$是数轴上的点集,$\xi$是一个定点(可以在$S$中也可以不在$S$中),若$\xi$的任一邻域中都含有$S$中无穷多个点,则称$\xi$为$S$的一个聚点.
\end{definition}
聚点的另一定义如下:
\begin{definition}
	若存在各项互异的收敛数列$\left\{x_n\right\}\subset S$,则其极限$\lim\limits_{n\to \infty}x_n=\xi$是$S$的一个聚点.
\end{definition}
\begin{theorem}[Weierstrass聚点定理]
	实轴上任一有界无限点集$S$至少有一个聚点.
\end{theorem}
由聚点的等价定义,该定理也可叙述为:{\heiti 有界数列必有收敛子列},即致密性定理.
\subsection{实数集完备性定理的等价关系}
通过前面的学习,我们共有以下8个基本定理来叙述实数的完备性:
\begin{enumerate}
	\item Dedekind定理;
	\item 确界原理;
	\item 单调有界收敛定理;
	\item 致密性定理;
	\item 柯西收敛准则;
	\item 闭区间套定理;
	\item Heine-Borel有限覆盖定理;
	\item Weierstrass聚点定理.
\end{enumerate}
可以证明,这8个基本定理都是等价的.(证明会在后续修正时给出)
	\newpage
	
\chapter{初等积分法}
所谓微分方程的初等积分法,就是通过初等函数及其有限次积分的表达式求解微分方程的方法.
\section{恰当方程}
\begin{definition}[恰当方程]
	考虑对称形式的一阶微分方程
	\begin{equation}\label{equ:exactequation}
		P(x,y)\d x+Q(x,y)\d y=0.
	\end{equation}
	如果存在一个可微函数$\varPhi(x,y)$,使得它的全微分为
	$$\d \varPhi(x,y)=P(x,y)\d x+Q(x,y)\d y,$$
	则称方程\ref{equ:exactequation}为{\heiti 恰当方程}(exact\ equation)或{\heiti 全微分方程}.
\end{definition}
因此,当方程\ref{equ:exactequation}为恰当方程时,可将它改写为全微分的形式
$$\d\varPhi(x,y)=P(x,y)\d x+Q(x,y)\d y=0,$$
从而
\begin{equation}\label{equ:generalint}
	\varPhi(x,y)=C,
\end{equation}
其中$C$为任意常数,我们称\ref{equ:generalint}式为方程\ref{equ:exactequation}的一个{\heiti 通积分}(general integration)或{\heiti 通解}(general solution).

事实上,将任意常数$C$取定后,利用逆推法容易验证:由\ref{equ:generalint}式确定的隐函数$y=u(x)$(或$x=v(y)$)就是方程\ref{equ:exactequation}的一个解. 反之,若$y=u(x)$(或$x=v(y)$)是微分方程\ref{equ:exactequation}的一个解,则有
$$\d\varPhi(x,y)=P(x,y)\d x+Q(x,y)\d y=0,$$
其中$y=u(x)$(或$x=v(y)$). 从而$y=u(x)$(或$x=v(y)$)满足\ref{equ:generalint}式,其中积分常数$C$取决于解$y=u(x)$(或$x=v(y)$)的初值$(x_0,y_0)$,亦即$C=\varPhi(x_0,y_0)$.

在一般情况下,我们需要解决的问题是:
\begin{enumerate}[(1)]
	\item 如何判断一个给定的微分方程是否为恰当方程?
	\item 当它是恰当方程时,如何求出相应全微分的原函数?
	\item 当它不是恰当方程时,能否将它的求解问题转化为一个与之相关的恰当方程的求解问题?
\end{enumerate}

下面的定理对问题(1)和(2)给出了完满的解答. 至于问题(3)则是贯穿本章随后各节的一个中心问题.
\begin{theorem}
	设函数$P(x,y)$和$Q(x,y)$在区域$R=(\alpha,\beta)\times(\gamma.\delta)$上连续,且有连续的一阶偏导数$\dfrac{\partial P}{\partial y}$与$\dfrac{\partial Q}{\partial x}$,则微分方程
	$$P(x,y)\d x+Q(x,y)\d y=0$$
	是恰当方程的充要条件为恒等式
	$$\frac{\partial}{\partial y}P(x,y)\equiv\frac{\partial}{\partial x}Q(x,y)$$
	在$R$内成立. 且方程的通积分为
	$$\int_{x_0}^{x}P(x,y)\d x+\int_{y_0}^{y}Q(x_0,y)\d y=C,$$
	或者
	$$\int_{x_0}^{x}P(x,y_0)\d x+\int_{y_0}^{y}Q(x,y)\d y=C,$$
	其中$(x_0,y_0)$是$R$中任意取定的一点.
\end{theorem}
\begin{proof}
	{\heiti 必要性}\qquad 方程为恰当方程,则存在$\varPhi(x,y)$使得
	$$\frac{\partial\varPhi}{\partial x}=P(x,y),\qquad\frac{\partial\varPhi}{\partial y}=Q(x,y).$$
	则
	$$\frac{\partial P}{\partial y}=\frac{\partial^2\varPhi}{\partial y\partial x},\qquad\frac{\partial Q}{\partial x}=\frac{\partial^2\varPhi}{\partial x\partial y}.$$
	由偏导数的连续性假设,有
	$$\frac{\partial^2\varPhi}{\partial y\partial x}=\frac{\partial^2\varPhi}{\partial x\partial y}.$$
	即
	$$\frac{\partial P}{\partial y}=\frac{\partial Q}{\partial x}.$$
	
	{\heiti 充分性}\qquad 已知$\dfrac{\partial P}{\partial y}=\dfrac{\partial Q}{\partial x}$,我们要构造$\varPhi(x,y)$使
	$$\frac{\partial\varPhi}{\partial x}=P(x,y),\qquad\frac{\partial\varPhi}{\partial y}=Q(x,y).$$
	
	令
	$$\varPhi(x,y)=\int_{x_0}^{x}P(x,y)\d x+\psi(y),$$
	则显然$\dfrac{\partial\varPhi}{\partial x}=P(x,y)$.
	而
	\begin{align*}
		\frac{\partial \varPhi}{\partial y}
		&=\frac{\partial}{\partial y}\int_{x_0}^{x}P(x,y)\d x+\psi'(y)\\
		&=\int_{x_0}^{x}\frac{\partial P}{\partial y}\d x+\psi'(y)\\
		&=\int_{x_0}^{x}\frac{\partial Q}{\partial x}\d x+\psi'(y)\\
		&=Q(x,y)-Q(x_0,y)+\psi'(y).
	\end{align*}
	令
	$$\psi'(y)=Q(x_0,y),$$
	则有
	$$\frac{\partial\varPhi}{\partial y}=Q(x,y).$$
	此时
	$$\psi(y)=\int_{y_0}^{y}Q(x_0,y)\d y,$$
	所以有
	$$\varPhi(x,y)=\int_{x_0}^{x}P(x,y)\d x+\int_{y_0}^{y}Q(x_0,y)\d y.$$
	同理,我们令
	$$\varPhi(x,y)=\psi(x)+\int_{y_0}^{y}Q(x,y)\d y$$
	可类似得到另一个函数
	$$\varPhi(x,y)=\int_{x_0}^{x}P(x,y_0)\d x+\int_{y_0}^{y}Q(x,y)\d y.$$
	$\hfill\blacksquare$
\end{proof}
\begin{remark}
	求解恰当方程的关键是构造相应全微分的原函数$\varPhi(x,y)$,这实际上就是场论中的位势问题. 在单连通区域$R$上,条件
	$$\frac{\partial P}{\partial y}=\frac{\partial Q}{\partial x}$$
	保证了曲线积分
	$$\varPhi(x,y)=\int_{(x_0,y_0)}^{(x,y)}P(x,y)\d x+Q(x,y)\d y$$
	与积分的路径无关. 因此,上式确定了一个单值函数$\varPhi(x,y)$. 如果区域不是单连通的,那么一般而言$\varPhi(x,y)$也许是多值的.
\end{remark}
\begin{remark}
	事实上,这也可由Green公式简单推出.
\end{remark}
\begin{proposition}
	若函数$p(x),q(x)$在区间$I$上连续可微,则方程
	$$p(x)\d x+q(x)\d y$$
	是恰当的,其通解为
	$$\int_{x_0}^{x}p(x)\d x+\int_{y_0}^{y}q(x)\d y=C.$$
\end{proposition}
\section{变量分离方程}
\begin{definition}[变量分离方程]
	如果微分方程
	$$P(x,y)\d x+Q(x,y)\d y=0$$
	中的函数$P(x,y)$和$Q(x,y)$均能表示为关于$x$的函数与关于$y$的函数的乘积,则称该微分方程为{\heiti 变量分离方程}.
\end{definition}
由上述定义,我们可以将$P(x),Q(x)$分别写成
$$P(x)=X(x)Y_1(y),\quad Q(x)=X_1(x)Y(y).$$
则变量分离方程可以写成
$$X(x)Y_1(y)\d x+X_1(x)Y(y)\d y=0.$$

考虑特殊情形:$P(x)=X(x)$和$Q(y)=Y(y)$,则微分方程为
$$X(x)\d x+Y(y)\d y=0.$$
这显然是一个恰当方程,且其一个通解为
$$\int X(x)\d x+\int Y(y)=C.$$

一般而言,变量分离方程不一定是恰当方程. 但它的名字揭示了:我们可以把变量进行分离. 如果我们用因子$X_1(x)Y_1(y)$去除变量分离方程,就得到
$$\frac{X(x)}{X_1(x)}\d x+\frac{Y(y)}{Y_1(y)}\d y=0.$$
这是一个恰当方程,它的通解为
$$\int \frac{X(x)}{X_1(x)}\d x+\int \frac{Y(y)}{Y_1(y)}\d y=C.$$

当$X_1(x)Y_1(y)\neq 0$时,上述方程和变量分离方程是同解的. 假设存在实数$a$使得$X_1(a)=0$,则函数$x=a$也是变量分离方程的解,但不是分离变量后方程的解. 因此,在分离变量后,{\heiti 要注意补上这些可能丢失的解}. 即补上形如
$$x=a_i\quad (i=1,2,\cdots)$$
和
$$y=b_i\quad (i=1,2,\cdots)$$
的特解. 其中$a_i$是$X_i(x)=0$的根,$b_i$是$Y_1(y)=0$的根.
\section{一阶线性方程}
我们先给出一阶线性方程的定义.
\begin{definition}[一阶线性方程]
	形如
	$$\frac{\d y}{\d x}+p(x)y=q(x)$$
	的微分方程称为{\heiti 一阶线性方程}. 其中$p(x)$和$q(x)$在区间$I=(a,b)$上连续. 
	
	当$q(x)\equiv 0$时,即得
	$$\frac{\d y}{\d x}+p(x)y=0.$$
	此时我们称一阶线性方程是{\heiti 齐次的}. 当$q(x)$不恒为零时,则称一阶线性方程为{\heiti 非齐次的}.
\end{definition}
\begin{remark}
	这里的“齐次”与“非齐次”指的是有关未知函数$y$的式$(y,y',y'',\cdots)$次数相同(将不含有关$y$的式看作$0$次的).
\end{remark}
下面我们首先讨论齐次线性方程
$$\frac{\d y}{\d x}+p(x)y=0$$
的解法.
\begin{solution}
	改写成对称形式,即
	$$\d y+p(x)y\d x=0,$$
	这是一个分离变量方程. 当$y\neq 0$时,方程两侧同除以$y$,得
	$$\frac{\d y}{y}+p(x)=0.$$
	积分后,即得齐次线性方程的解
	$$y=C\mathrm{e}^{-\int p(x)\d x}\quad (C\neq 0).$$
	当$C=0$时,对应于方程的特解$y=0$,因此,$C$可以是任意常数,我们得到了齐次线性方程的通解.$\hfill\blacksquare$
\end{solution}

下面我们继续来看非齐次线性方程的解法. 
\begin{solution}
	同样地,我们改写成
	$$\d y+p(x)y\d x=q(x)\d x.$$
	一般地,上述方程并非恰当方程. 但如果我们将方程两侧同乘以一个非零因子$\mu(x)=\mathrm{e}^{\int p(x)\d x}$,我们得到
	$$\mathrm{e}^{\int p(x)\d x}\d y+\mathrm{e}^{\int p(x)\d x}p(x)y\d x=\mathrm{e}^{\int p(x)\d x}q(x)\d x,$$
	它是全微分的形式
	$$\d (\mathrm{e}^{\int p(x)\d x}y)=\d \int q(x)\mathrm{e}^{\int p(x)\d x}\d x.$$
	直接积分,得到通积分
	$$\mathrm{e}^{\int p(x)\d x}y=\int q(x)\mathrm{e}^{\int p(x)\d x}\d x+C.$$
	得到通解
	$$y=\mathrm{e}^{-\int p(x)\d x}\left(C+\int q(x)\mathrm{e}^{\int p(x)\d x}\d x\right),$$
	其中$C$是任意常数.$\hfill\blacksquare$
\end{solution}
\begin{remark}
	上述方法称为{\heiti 积分因子法}. 因为我们用因子$\mu(x)$乘方程的两侧后,他就转化为了一个全微分方程,从而获得它的积分. 此外,我们还有{\heiti 常数变易法},将在后续学习中提及.
\end{remark}

为确定起见,通常把一阶线性方程通解中的不定积分写成变上限的定积分,即
$$y=\mathrm{e}^{-\int_{x_0}^{x} p(t)\d t}\left[C+\int_{x_0}^{x}q(s)\mathrm{e}^{-\int_{x_0}^{s} p(t)\d t}\d s\right]\quad (x_0\in I),$$
或
$$y=C\mathrm{e}^{-\int_{x_0}^{x} p(t)\d t}+\int_{x_0}^{x}q(s)\mathrm{e}^{-\int_{s}^{x} p(t)\d t}\d s.$$
利用这种形式,容易得到Cauchy初值问题
$$\frac{\d y}{\d x}+p(x)y=q(x),\quad y(x_0)=y_0$$
的解为
$$y=y_0\mathrm{e}^{-\int_{x_0}^{x} p(t)\d t}+\int_{x_0}^{x}q(s)\mathrm{e}^{-\int_{s}^{x} p(t)\d t}\d s,$$
其中$p(x)$和$q(x)$在区间$I$上连续.

下面给出线性微分方程的一些性质.
\begin{theorem}
	齐次线性方程的解或者恒等于零,或者恒不等于零.
\end{theorem}
\begin{theorem}
	线性方程的解是整体存在的,即任一解都在$p(x)$和$q(x)$有定义且连续的整个区间$I$上存在.
\end{theorem}
\begin{theorem}
	齐次线性方程的任何解的线性组合仍是它的解;齐次线性方程的任一解与非齐次线性方程的任一解之和是非齐次线性方程的解;非齐次线性方程的任意两解之差必是相应的齐次线性方程的解.
\end{theorem}
\begin{theorem}
	非齐次线性方程的任一解与相应的齐次线性方程的通解之和构成非齐次线性方程的通解.
\end{theorem}
\begin{theorem}
	线性方程的Cauchy初值问题的解存在且唯一.
\end{theorem}

\section{初等变换法}
下面介绍几个标准类型的微分方程,它们可以通过适当的初等变换转化为变量分离方程或一阶线性方程.
\subsection{齐次方程}
\begin{definition}[齐次方程]
	如果微分方程
	$$P(x,y)\d x+Q(x,y)\d y=0$$
	中的函数$P(x,y)$和$Q(x,y)$都是$x$和$y$的同次(如$m$次)齐次函数,即
	$$P(tx,ty)=t^mP(x,y),\quad Q(tx,ty)=t^mQ(x,y),$$
	则称方程为{\heiti 齐次方程}.
\end{definition}
\begin{remark}
	这与上节定义的齐次线性方程不是一回事.
\end{remark}

对于齐次方程的解法,我们可以作变量替换. 令
$$y=ux.$$
其中$u$是新的未知函数,$x$仍为自变量. 则
\begin{equation*}
	\left\{
	\begin{aligned}
		&P(x,y)=P(x,xu)=x^mP(1,u),\\
		&Q(x,y)=Q(x,xu)=x^mQ(1,u).
	\end{aligned}
	\right.
\end{equation*}
代入齐次方程
$$P(x,y)\d x+Q(x,y)\d y=0,$$
得
$$x^m\left[P(1,u)+uQ(1,u)\right]\d x+x^{m+1}Q(1,u)\d u=0,$$
这就将齐次方程转化为了变量分离方程.

\begin{remark}
	易知方程
	$$P(x,y)\d x+Q(x,y)\d y=0$$
	为齐次方程的一个等价定义是,它可以化为如下形式:
	$$\frac{\d y}{\d x}=\varPhi\left(\frac{y}{x}\right).$$
\end{remark}
\begin{remark}
	容易看出,$x=0$是我们化为的变量分离方程的一个特解. 但它未必是原方程的解. 这是因为变换$y=ux$在$x=0$时是不可逆的.
\end{remark}
\subsection{Bernoulli方程}
\begin{definition}[Bernoulli方程]
	形如
	$$\frac{\d y}{\d x}+p(x)y=q(x)y^n$$
	的方程称为\textbf{Bernoulli}{\heiti 方程},其中$n$为常数且$n\neq 0,1$.
\end{definition}
对于Bernoulli方程的解法,我们先以$(1-n)y^{-n}$乘方程两边,得到
$$(1-n)y^{-n}\frac{\d y}{\d x}+(1-n)y^{1-n}p(x)=(1-n)q(x).$$
令$z=y^{1-n}$,有
$$\frac{\d z}{\d x}+(1-n)p(x)z=(1-n)q(x),$$
这就将其转化为了关于未知函数$z$的一阶线性方程.
\subsection{Riccati方程}
\begin{definition}[Riccati方程]
	假如一阶微分方程
	$$\frac{\d y}{\d x}=f(x,y)$$
	的右端函数$f(x,y)$是一个关于$y$的二次多项式,即该方程可以写成
	$$\frac{\d y}{\d x}=p(x)y^2+q(x)y+r(x),$$
	的形式,其中$p(x),q(x)$和$r(x)$在区间$I$上连续且$p(x)$不恒为零,则称该方程为{\heiti 二次方程}或\textbf{Riccati}{\heiti 方程}. 
\end{definition}
\begin{remark}
	Riccati方程是形式上最简单的非线性方程. 但一般而言,它不能用初等积分法求解.
\end{remark}
\begin{theorem}
	设已知Riccati方程的一个特解$y=\varphi_1(x)$,则可用积分法求得它的通解.
\end{theorem}
\begin{proof}
	对Riccati方程
	$$\frac{\d y}{\d x}=p(x)y^2+q(x)y+r(x)$$
	作变换$y=u+\varphi_1(x)$,其中$u$是新的未知函数. 代入Riccati方程,得到
	$$\frac{\d u}{\d x}+\frac{\d \varphi_1}{\d x}=p(x)\left[u^2+2\varphi_1(x)u+\varphi_1^2(x)\right]+q(x)\left[u+\varphi_1(x)\right]+r(x).$$
	由于$y=\varphi_1(x)$是Riccati方程的解,从上式消去相关的项后,就有
	$$\frac{\d u}{\d x}=\left[2p(x)\varphi_1(x)+q(x)\right]u+p(x)u^2,$$
	这是一个Bernoulli方程,可用积分法求出通解.$\hfill\blacksquare$
\end{proof}
\begin{theorem}
	设Riccati方程
	$$\frac{\d y}{\d x}+ay^2=bx^m,$$
	其中$a,b,m$都是常数且$a\neq 0$. 又设$x\neq 0$和$y\neq 0$,则当$m$为
	$$0,\ -2,\ \frac{-4k}{2k+1},\ \frac{-4k}{2k-1}\ (k=1,2,\cdots)$$
	时,方程可通过适当的变换化为变量分离方程.
\end{theorem}
\begin{proof}
	不妨设$a=1$(否则作自变量变换$\overline{x}=ax$即可),我们考虑
	\begin{equation}\label{riccati}
		\frac{\d y}{\d x}+y^2=bx^m.
	\end{equation}
	
	当$m=0$时,方程\ref{riccati}是一个变量分离方程
	$$\frac{\d y}{\d x}=b-y^2.$$
	
	当$m=-2$时,作变换$z=xy$,其中$z$是新未知函数. 然后代入方程\ref{riccati},得到
	$$\frac{\d z}{\d x}=\frac{b+z-z^2}{x}.$$
	这也是一个变量分离方程.
	
	\hspace*{\fill}
	
	当$m=\displaystyle\frac{-4k}{2k+1}$时,作变换
	$$x=\xi^{\frac{1}{m+1}},\quad y=\frac{b}{m+1}\eta^{-1},$$
	其中$\xi$和$\eta$分别为新的自变量和未知函数,则方程\ref{riccati}变为
	\begin{equation}\label{transriccati1}
		\frac{\d \eta}{\d \xi}+\eta^2=\frac{b}{(m+1)^2}\xi^n,
	\end{equation}
	其中$n=\dfrac{-4k}{2k-1}$. 再作变换
	$$\xi=\frac{1}{t},\quad \eta=t-zt^2,$$
	其中$t$和$z$分别是新的自变量和未知函数,则方程\ref{transriccati1}变为
	\begin{equation}\label{transriccati2}
		\frac{\d z}{\d t}+z^2=\frac{b}{(m+1)^2}t^l,
	\end{equation}
	其中$l=\dfrac{-4(k-1)}{2(k-1)+1}$.
	
	\hspace*{\fill}
	
	上述方程与方程\ref{riccati}在形式上一样,只是右端自变量的指数从$m$变为$l$. 比较$m$与$l$对$k$的依赖关系不难看出,只要将上述变换的过程重复$k$次,就能把方程\ref{riccati}化为$m=0$的情形.
	
	\hspace*{\fill}
	
	当$m=\displaystyle\frac{-4k}{2k-1}$时,原微分方程就是\ref{transriccati1}的类型,因此可以把它化为微分方程\ref{transriccati2}的形式,从而可以化归到$m=0$的情形. 至此证毕.$\hfill\blacksquare$
\end{proof}
\begin{remark}
	此定理由Daniel\ Bernoullli在1725年得到. 这个定理指出,对于Riccati方程能用初等积分法求解,$m$的取值是充分的. 实际上,Liouville在1841年进而证明了这个条件还是一个必要条件.
\end{remark}
\begin{remark}
	Riccati方程在历史上和近代都有重要应用. 例如,它曾用于证明Bessel方程的解不是初等函数,另外它也出现在现代控制论和向量场分支理论的一些问题中.
\end{remark}
\subsection{Gronwall不等式}
Gronwall不等式在一阶常微分方程解的存在唯一性定理的证明过程中起到核心作用,该不等式在PDE和FPDE中也有重要应用,它的作用是给出相关未知函数的上界估计.
\begin{theorem}[Gronwall-Bellman不等式]
	设$K$为非负常数,$f(t)$与$g(t)$为区间$\left[\alpha,\beta\right]$上的非负连续函数,且满足
	$$f(t)\leqslant K+\int_{\alpha}^{t}f(s)g(s)\d s,\qquad \alpha\leqslant t\leqslant\beta,$$
	则有
	$$f(t)\leqslant K\mathrm{e}^{\int_{\alpha}^{t}g(s)\d s}.$$
\end{theorem}
\begin{proof}
	设
	$$V(t)=K+\int_{\alpha}^{t}f(s)g(s)\d s,\qquad \alpha\leqslant\beta,$$
	则
	$$V'(t)=f(t)g(t)\leqslant g(t)V(t),$$
	即$V'(t)-g(t)V(t)\leqslant 0$,两边同乘$\mathrm{e}^{-\int_{\alpha}^{t}g(s)\d s}$,得
	$$\left[V(t)\mathrm{e}^{-\int_{\alpha}^{t}g(s)\d s}\right]'\leqslant 0.$$
	由单调性得
	$$V(t)\mathrm{e}^{-\int_{\alpha}^{t}g(s)\d s}\leqslant V(\alpha)=K.$$
	故
	$$f(t)\leqslant V(t)\leqslant K\mathrm{e}^{\int_{\alpha}^{t}g(s)\d s}.$$
	$\hfill\blacksquare$
\end{proof}
	
\chapter{存在和唯一性定理}
\section{Picard存在和唯一性定理}
\begin{definition}[Lipschitz条件]
	设函数$f(x,y)$在区域$D$内满足不等式
	$$|f(x,y_1)-f(x,y_2)|\leqslant L|y_1-y_2|,$$
	其中常数$L>0$,则称函数$f(x,y)$在区域$D$内对$y$满足\textbf{Lipschitz}{\heiti 条件}. 称$L$为Lipschitz常数.
\end{definition}
Lipschitz条件是一个比通常连续更强的光滑性条件。直觉上,Lipschitz连续函数限制了函数改变的速度,符合Lipschitz条件的函数的斜率,必小于一个称为Lipschitz常数的实数.

易知,若函数$f(x,y)$在凸区域$D$内对$y$有连续的偏微商,并且$D$是有界闭区域,则$f(x,y)$在$D$内对$y$满足Lipschitz条件;反之,结论不一定正确. 例如$f(x,y)=|y|$对$y$满足Lipschitz条件,但当$y=0$时它对$y$没有微商.

现在,我们要证明下述Picard定理.
\begin{theorem}[Picard定理]
	设初值问题
	$$(E):\ \frac{\d y}{\d x}=f(x,y),\quad y(x_0)=y_0,$$
	其中$f(x,y)$在矩形区域
	$$R:\ \left[x_0-a,x_0+a\right]\times\left[y_0-b,y_0+b\right]$$
	内连续,而且对$y$满足Lipschitz条件. 则$(E)$在区间$I=\left[x_0-h,x_0+h\right]$上有且仅有一个解,其中常数
	$$h=\min\left\{a,\frac{b}{M}\right\},\quad M>\max\limits_{(x,y)\in R}|f(x,y)|.$$
\end{theorem}
为了突出思路,我们把证明分成四步:
\begin{enumerate}[(1)]
	\item 将微分方程转化为对应的积分方程.
	\item 构造Picard序列$\{y_n(x)\}$.
	\item 证明Picard序列$y_n(x)\rightrightarrows y(x)$是方程的解.
	\item 证明解的唯一性.
\end{enumerate}
\begin{proof}
	(1) 先证明初值问题$(E)$有解$y=y(x)$,等价于积分方程
	\begin{equation}\label{inte}
		y=y_0+\int_{x_0}^{x}f(t,y)\d t
	\end{equation}
	有解$y=y(x)$. 
	
	设$y=y(x)\ (x\in I)$是$(E)$的解,则有
	$$y'(x)=f(x,y(x))\ (x\in I)$$
	和
	$$y(x_0)=y_0.$$
	由此,对上述微分方程进行积分并利用初值条件,得到
	$$y(x)=y_0+\int_{x_0}^{x}f(x,y(x))\d x\ (x\in I),$$
	即$y=y(x)$是积分方程\ref{inte}的解.
	
	反之,设$y=y(x)\ (x\in I)$是积分方程\ref{inte}的解,则只要逆转上面的推导就可知道$y=y(x)$是$(E)$的解.
	
	因此,Picard定理的证明等价于证明积分方程\ref{inte}在区间$I$上有且仅有一个解.
	
	(2) 采用不动点的思想,用逐次迭代法构造Picard序列
	$$y_{n+1}(x)=y_0+\int_{x_0}^{x}f(t,y_n(t))\d t\ (x\in I,\ n=0,1,2,\cdots),$$
	其中$y_0(x)=y_0$.
	
	当$n=0$时,注意到$f(x,y_0(x))$是$I$上的连续函数,所以由递推式,有
	$$y_1(x)=y_0(x)+\int_{x_0}^{x}f(t,y_0(t))\d t\ (x\in I)$$
	在$I$上是连续可微的,而且满足不等式
	$$|y_1(x)-y_0(x)|\leqslant\left|\int_{x_0}^{x}|f(t,y_0(t))|\d t\right|\leqslant M|x-x_0|.$$
	这就是说,在区间$I$上$|y_1(x)-y_0|\leqslant Mh\leqslant b$.
	
	因此,$f(x,y_1(x))$在$I$上是连续的. 所以由递推式,有
	$$y_2(x)=y_0(x)+\int_{x_0}^{x}f(t,y_1(t))\d t\ (x\in I)$$
	在$I$上是连续可微的,而且满足不等式
	$$|y_2(x)-y_1(x)|\leqslant\left|\int_{x_0}^{x}|f(t,y_1(t))\d t|\right|\leqslant M|x-x_0|,$$
	从而我们有:$|y_2(x)-y_0|\leqslant Mh\leqslant b\ (x\in I)$.
	
	如此类推,用归纳法不难证明:由递推式给出的Picard序列$\{y_n(x)\}$在$I$上是连续的,而且满足不等式
	$$|y_n(x)-y_0|\leqslant M|x-x_0|\ (n=0,1,2,\cdots).$$
	
	(3)现证:Picard序列$\{y_n(x)\}$在区间$I$上一致收敛到积分方程\ref{inte}的解.
	
	序列$\{y_n(x)\}$的收敛性等价于级数
	$$\sum_{n=1}^{\infty}\left[y_{n+1}(x)-y_n(x)\right]$$
	的收敛性. 
	\begin{align*}
		|y_2(x) - y_1(x)| &= \left|\int_{x_0}^{x}[f(t, y_1(t)) - f(t, y_0(t))]\d t\right|\\
		&\leqslant \left|\int_{x_0}^{x}|f(t, y_1(t)) - f(t, y_0(t))|\d t\right|\\
		&\leqslant L\left|\int_{x_0}^{x}|y_1(t) - y_0(t)|\d t\right|\\
		&\leqslant LM\left|\int_{x_0}^{x}|t - x_0|\d t\right|\\
		&= \frac{LM}{2}(x-x_0)^2 = \frac{M}{L}\frac{[L(x-x_0)]^2}{2}
	\end{align*}
	同理,可得
	$$|y_{n+1}(x)-y_n(x)|\leqslant L\left|\int_{x_0}^{x}|y_n(t)-y_{n-1}(t)|\d t\right|,$$
	根据归纳法,可以证明
	$$|y_{n+1}(x)-y_n(x)|\leqslant\frac{M}{L}\frac{\left[L|x-x_0|\right]^{n+1}}{(n+1)!}\leqslant\frac{M}{L}\frac{(Lh)^{n+1}}{(n+1)!}.$$
	故
	$$\sum_{j=0}^{\infty}|y_{j+1}(x)-y_j(x)|\leqslant\frac{M}{L}\sum_{j=0}^{\infty}\frac{(Lh)^{j+1}}{(j+1)!}.$$
	则根据函数项级数的Weierstrass判别法,可知$\{y_n(x)\}$一致收敛,因此对Picard序列的递推式取极限,得
	$$y(x)=\lim\limits_{n\to\infty}y_n(x)=y_0+\int_{x_0}^{x}f(t,y(t))\d t.$$
	故Picard序列的极限$y(x)$是方程的解.
	
	(4)最后证明解的唯一性. 若方程有两个互异的解$\varphi(x), \psi(x)$,记$u(x)=\varphi(x)-\psi(x)$,则有
	\begin{align*}
		|u(x)|&=\left|\int_{x_0}^{x}f(t,\varphi(t))-f(t,\psi(t))\d t\right|\\
		&\leqslant L\int_{x_0}^{x}|\varphi(t)-\psi(t)|\d t\\
		&=L\int_{x_0}^{x}|u(t)|\d t.
	\end{align*}
	根据Gronwall不等式可得
	$$|u(x)|\leqslant 0,$$
	故$\varphi(x)=\psi(x)$.$\hfill\blacksquare$
\end{proof}
\begin{remark}
	若$f(x,y)$不满足Lipschitz条件,则有Picard序列可能不收敛,解仍存在的情形.
\end{remark}
\begin{remark}
	$f$仅连续但不满足Lipschitz条件时,解可能不唯一. 例如$y'=\dfrac{3}{2}y^{\frac{1}{3}}$.
\end{remark}
\hspace*{\fill}

\begin{remark}
	我们也可以利用压缩映像原理来证明解的唯一性. 下面我们给出该原理但不再证明,感兴趣的读者可以参考泛函分析相关教材.
\end{remark}
\begin{theorem}[压缩映像原理]
	设$X$是完备的度量空间,$T$是$X$上的压缩映射,那么$T$有且仅有一个不动点,即$Tx=x$有且只有一个解.
\end{theorem}
下面我们介绍Osgood条件,它是一个比Lipschitz条件更弱的条件.
\begin{definition}[Osgood条件]
	设函数$f(x,y)$在区域$G$内连续,而且满足不等式
	$$|f(x,y_1)-f(x,y_2)|\leqslant F(|y_1-y_2|),$$
	其中$F(r)>0$是$r>0$的连续函数,且
	$$\int_{0}^{r_1}\frac{\d r}{F(r)}=+\infty,$$
	则称$f(x,y)$在$G$内对$y$满足\textbf{Osgood}{\heiti 条件}.
\end{definition}
\begin{remark}
	Lipschitz条件是Osgood条件的特例,因为$F(r)=Lr$满足上述要求.
\end{remark}





	
\end{document}