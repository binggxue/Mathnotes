\documentclass[lang=cn,10pt]{elegantbook}

\usepackage{amsfonts,amssymb,fixdif}

\title{概率论与数理统计讲义}
\subtitle{Probability and Statistics}

\author{薛冰}
\date{\today}

%table of content depth,目录显示的深度
\setcounter{tocdepth}{3} 

% 修改封面的颜色带
\definecolor{customcolor}{RGB}{255,185,0}
\colorlet{coverlinecolor}{customcolor}

\cover{Cover.png}

\begin{document}
	\maketitle
	
	\frontmatter
	\tableofcontents
	\mainmatter

\chapter{实数理论}

数学分析研究的基本对象是定义在实数集上的函数.
\section{实数的定义}%或许可以改成“有理数的扩充”
\subsection{建立实数的原则}
\begin{definition}[数域]\label{def:field}
	设$P$是由一些复数组成的集合,其中包括0和1,如果$P$中任意两个数的和、差、积、商(除数不为0)仍为$P$中的数,则称$P$为一个{\heiti 数域}.
\end{definition}
由定义\ref{def:field}可知,数域对加、减、乘、除(除数不为0)四则运算具有封闭性,即结果仍在数域本身中。例如,全体有理数所构成的集合$\mathbb{Q}$是一个数域,称为有理数域.此外,常见的数域还有复数域$\mathbb{C}$,读者可自行验证.
\begin{example}
	证明全体有理数所构成的集合$\mathbb{Q}$是一个数域.
	\begin{proof}
		对$\forall a=\frac{p}{q},\ b=\frac{s}{t}$,其中$p,q,s,t\in \mathbb{Z}$且$st\neq0$,则由有理数的定义知,$a,b\in \mathbb{Q}$
		
		显然$0,1\in\mathbb{Q}$,
		
		$a+b=\frac{pt+qs}{qt}\in \mathbb{Q}$
		
		$a-b=\frac{pt-qs}{qt}\in \mathbb{Q}$
		
		$a\cdot b=\frac{ps}{qt}\in \mathbb{Q}$
		
		$\frac{a}{b}=\frac{pt}{qs}\in \mathbb{Q}$
		
		故$\mathbb{Q}$是一个数域.$\hfill\blacksquare$
	\end{proof}
\end{example}
\begin{definition}[阿基米德有序域]
	集合$F$构成一个{\heiti 阿基米德有序域},是说它满足以下三个条件:
	\begin{enumerate}
		\item $F$是域\qquad 在$F$中定义了加法“$+$”和乘法“$\cdot$”两种运算,使得对于$F$中任意元素$a$,$b$,$c$成立:\par
		加法的结合律:\ $(a+b)+c=a+(b+c)$;\par
		加法的交换律:\ $a+b=b+a$;\par
		乘法的结合律:\
		$(a\cdot b)\cdot c=a\cdot (b\cdot c)$;\par
		乘法的交换律:\ $a\cdot b=b\cdot a$;\par
		乘法关于加法的分配律:
		$(a+b)\cdot c=a\cdot c+b\cdot c$;\par
		$F$中对加法存在{\heiti 零元素}和{\heiti 负元素}(即存在加法的逆运算减法);
		
		$F$中对乘法存在{\heiti 单位元素}和{\heiti 逆元素}(即存在乘法的逆运算除法).
		\item $F$是有序域\qquad
		在$F$中定义了{\heiti 序}关系“\textless”具有如下{\heiti 全序}的性质:\par
		传递性:$\forall a,b,c \in F$,若$a<b$,$b<c$,则$a<c$;\par
		三歧性:$\forall a,b \in F$,$a>b$,$a<b$,$a=b$三者必居其一,也只居其一;\par
		加法保序性:$\forall a,b,c \in F$,若$a<b$,则$a+c<b+c$;\par
		乘法保序性:$\forall a,b,c \in F$,若$a<b$,则$ac<bc$\ (c>0).
		\item $F$中元素满足阿基米德性\qquad 对$F$中两个正元素$a$,$b$,必存在自然数$n$,使得$na>b$.
	\end{enumerate}
\end{definition}
\subsection{Dedekind分割}
设$A/B$是有理数域$\mathbb{Q}$上的一个分割,即把$\mathbb{Q}$中的元素分为$A$、$B$两个集合,使得$\forall a \in A,\ b \in B$有$a<b$成立.则从逻辑上分为下列四种情况:
\begin{enumerate}
	\item $A$有最大值$a_0$,$B$有最小值$b_0$;
	\item $A$有最大值$a_0$,$B$无最小值;
	\item $A$无最大值,$B$有最小值$b_0$;
	\item $A$无最大值,$B$无最小值.
\end{enumerate}	

而对于第1种情况,取$\frac{a_0+b_0}{2}\in \mathbb{Q}$,它将不属于集合$A$、$B$中的任何一个.\par 
对于第4种情况,说明分割到的数不在$\mathbb{Q}$内(我们将这种划分称为{\heiti 无端划分}),因此这是我们通过“切割”构造出来的“新数”.将所有这样切割出来的新数与原来的$\mathbb{Q}$取并集,并设新的集合为$\mathbb{R}$.将$\mathbb{R}$中的元素再次进行分割,设$A'/B'$为$\mathbb{R}$上的一个分割,则$\forall a \in A',\ b \in B'$有$a<b$成立.同理,在排除上述的第1种情况后,对$\mathbb{R}$的分割分为以下三种情况:
\begin{enumerate}
	\item $A'$有最大值$a_0$,$B'$无最小值;
	\item $A'$无最大值,$B'$有最小值$b_0$;
	\item $A'$无最大值,$B'$无最小值.
\end{enumerate}	

由此,我们给出Dedekind定理.
\begin{theorem}[Dedekind定理]
	设$A'$、$B'$是$\mathbb{R}$的两个子集,且满足:
	\begin{enumerate}
		\item $A'$和$B'$均不为空集;
		\item $A'\cup B'=\mathbb{R}$;
		\item $\forall a\in A',\ b\in B'$有$a<b$.
	\end{enumerate}
	则或者$A'$有最大元素,或者$B'$有最小元素.
\end{theorem}
Dedikind定理指出,我们在上述对$\mathbb{R}$进行分割时,只会出现第1种或第2种情况,而不可能$A'$无最大值且$B'$也无最小值。
\begin{proof}
	设$A$是$A'$中所有有理数的集合,设$B$是$B'$中所有有理数的集合,则$A/B$有三种情况:
	\begin{enumerate}
		\item $A$有最大值$a_0$,$B$无最小值;
		\item $A$无最大值,$B$有最小值$b_0$;
		\item $A$无最大值,$B$无最小值;
	\end{enumerate}	
	
	对于第1种情况,设$a'\in A'$,使得$a'>a_0$,则在$(a_0,a')$之间必定存在有理数$a$,这与$A$有最大值$a_0$矛盾.因此$a_0$也是$A'$的最大值;
	
	同理,对于第2种情况,我们可以得到$b_0$也是$B'$的最小值;
	
	对于第3种情况,设$c$是由$A/B$得到的无理数,则$a_0<c<b_0$,可知$c$或者是$A'$中的元素,或者是$B'$中的元素.不妨设$c\in A'$,可以证明c为$A'$的最大元素,因为如果存在$a$是$A$的最大元素,那么区间$(c,a)$之间必定存在有理数大于$a_0$,这与$A$的最大值是$a_0$矛盾.
	
	以上我们就证明了或者$A'$有最大元素,或者$B'$有最小元素.$\hfill\blacksquare$
\end{proof}
\subsection{实数的公理化定义}
由前两节的理论基础,我们可以定义{\heiti 实数}构成一个阿基米德有序域,且满足Dedekind定理.实数分为有理数和无理数,其中无理数由有理数的无端划分产生.
\section{确界原理}
确界原理是极限理论的基础.
\subsection{确界的定义}
\begin{definition}
	设$S$为$\mathbb{R}$中的一个数集,若存在数$M(L)$,使得对一切$x\in S$,都有$x\leqslant M(x\geqslant L)$,则称$S$为{\heiti 有上界(下界)的数集},数$M(L)$称为$S$的一个{\heiti 上界(下界)}.
\end{definition}
\begin{definition}[上确界]
	设$S$是$\mathbb{R}$中的一个数集,若数$\eta$满足:
	
	(i)$\eta$是$S$的上界;
	
	(ii)$\forall \alpha < \eta,\ \exists x_0\in S,\ s.t.x_0>\alpha$,即$\eta$又是$S$的最小上界,\\
	则称$\eta$为数集$S$的{\heiti 上确界},记作$\eta=\sup S$
\end{definition}
\begin{definition}[下确界]
	设$S$是$\mathbb{R}$中的一个数集,若数$\xi$满足:
	
	(i)$\xi$是$S$的下界;
	
	(ii)$\forall \beta > \xi,\ \exists x_0\in S,\ s.t.x_0<\beta$,即$\xi$又是$S$的最小下界,\\
	则称$\xi$为数集$S$的{\heiti 下确界},记作$\xi=\inf S$
\end{definition}
上(下)确界也可以由$\varepsilon$语言定义.
\begin{definition}[上确界的$\varepsilon$语言定义]
	设$S$为$\mathbb{R}$中的一个数集,若S的一个上界$M$,$\forall \varepsilon>0$,$\exists a\in S$,s.t.\ $a>M-\varepsilon$,则数$M$称为$S$的一个{\heiti 上确界}.
\end{definition}
\begin{definition}[下确界的$\varepsilon$语言定义]
	设$S$为$\mathbb{R}$中的一个数集,若S的一个下界$L$,$\forall \varepsilon>0$,$\exists b\in S$,s.t.\ $b<L+\varepsilon$,则数$L$称为$S$的一个{\heiti 下确界}.
\end{definition}
上确界和下确界统称为确界.
\subsection{确界原理及其证明}
\begin{theorem}[确界原理]
	设$S$为$\mathbb{R}$中的一个数集,若$S$有上界,则必有上确界;若$S$有下界,则必有下确界.
\end{theorem}
\begin{proof}
	设数集$S$有上界,下面证明$S$有上确界.
	
	设$B$是数集$S$所有上界组成的集合,记$A=\mathbb{R}\textbackslash B$.若$B$有最小元素,则$B$的最小元素$b_0$即为$S$的上确界.
	
	设$x\in A$,$x$不是$S$的上界,则$\exists t\in S,\ s.t.\ x<t$,取$x'=\frac{x+t}{2}$,则$x'>x$,因此对$A$中任何一个元素$x$,都有$x'>x$存在,即$A$没有最大元素.由Dedekind定理,$B$一定有最小元素,即$S$必有上确界.$\hfill\blacksquare$
\end{proof}
\begin{example}
	利用确界原理证明Dedekind定理,即证明二者的等价关系.
\end{example}
\begin{proof}
	设$\mathbb{R}$上任意一个Dedekind分割为$A/B$,易知$B$中的每个元素都是$A$的一个上界.由确界原理,$A$一定有上确界.设$m=\sup A$,
	
	若$m\in B$,则$A$中无最大元素,假设$B$中无最小元素,则$\exists m'\in B\ s.t.\ m'<m$,由于$m=\sup A$,推出$m'\in A$,这与$m'\in B$矛盾.故$B$中有最小元素.
	
	若$m\in A$,则$A$中有最大元素$m$,假设$B$中有最小元素$n$,则$\frac{m+n}{2}$不属于$A$和$B$,这与$A\cup B=\mathbb{R}$是矛盾的,故$B$中无最小元素.
	
	于是我们就证明了Dedekind定理.$\hfill\blacksquare$
\end{proof}
\section{实数的完备性}
本节内容是基于第二章中数列极限的前提下展开的,建议在学完第二章后进行学习.数列极限的定义与基本性质在此不再赘述.

\subsection{关于实数集完备性的基本定理}
前面我们已经学习了Dedekind分割与确界原理,从不同的角度反映了实数集的特性,通常称为{\heiti 实数的完备性}或{\heiti 实数的连续性}公理.下面我们将介绍另外的几个实数的完备性公理.

\begin{theorem}[单调有界收敛定理]
	在实数系中,有界的单调数列必有极限.
\end{theorem}
\begin{proof}
	不妨设$\left\{a_n\right\}$为有上界的递增数列,由确界原理,$\left\{a_n\right\}$必有上确界,设$a=\sup \left\{a_n\right\}$,根据上确界的定义,
	
	$\forall \varepsilon>0,\ \exists a_N \ s.t.\ a_N>a-\varepsilon$
	
	由$\left\{a_n\right\}$的递增性,当$n\geqslant N$时,有
	$$a-\varepsilon<a_N\leqslant a_n$$
	
	又因为$$a_n\leqslant a<a+\varepsilon$$
	
	故$$a-\varepsilon<a_n<a+\varepsilon$$
	
	即$${\lim_{n \to +\infty}a_n}=a$$
	$\hfill\blacksquare$
\end{proof}

\begin{theorem}[致密性定理]
	任何有界数列必定有收敛的子列.
\end{theorem}
要证明此定理,可以先证明以下引理.
\begin{lemma}\label{zilie}
	任何数列都存在单调子列.
\end{lemma}
\begin{proof}
	设数列为$\left\{a_n\right\}$,下面分两种情况讨论:
	\begin{enumerate}
		\item 若$\forall k\in \mathbb{Z}_+$,$\left\{a_{k+n}\right\}$都有最大项,记$\left\{a_{1+n}\right\}$的最大项为$a_{n_1}$,则$a_{{n_1}+n}$也有最大项,记作$a_{n_2}$,显然有$a_{n_1}\geqslant a_{n_2}$,同理,有$$a_{n_2}\geqslant a_{n_3}$$
		$$.........$$
		由此得到一个单调递减的子列$\left\{a_{n_k}\right\}$
		\item 若至少存在一个正整数$k$,使得$\left\{a_{k+n}\right\}$没有最大项,先取$n_1=k+1$,总存在$a_{n_1}$后面的项$a_{n_2}$($n_2>n_1$)使得$$a_{n_2}>a_{n_1}$$,同理,总存在$a_{n_2}$后面的项$a_{n_3}$($n_3>n_2$)使得$$a_{n_3}>a_{n_2}$$
		$$.........$$
		由此得到一个严格递增的子列$\left\{a_{n_k}\right\}$
	\end{enumerate}
	
	综上,命题得证.$\hfill\blacksquare$
\end{proof}
下面是对致密性定理的证明:
\begin{proof}
	设数列$\left\{a_n\right\}$有界,由引理\ref{zilie},数列$\left\{a_n\right\}$存在单调且有界的子列,由单调有界收敛定理得出该子列是收敛的.$\hfill\blacksquare$
\end{proof}
\begin{theorem}[柯西(Cauchy)收敛准则]
	数列$\left\{a_n\right\}$收敛的充要条件是:\par 
	$\forall \varepsilon>0,\ \exists N\in \mathbb{Z}_+,\ s.t.\ n,\ m>N$时,有
	$$\lvert a_n - a_m \rvert<\varepsilon$$
\end{theorem}
单调有界只是数列收敛的充分条件,而柯西收敛准则给出了数列收敛的充要条件.
\begin{proof}
	{\heiti 必要性}\qquad 设$\lim\limits_{n \to +\infty}a_n=A$,则$\forall \varepsilon>0,\ \exists N\in \mathbb{Z}_+\ s.t.\ n,\ m>N$时,有$$\lvert a_n-A\rvert <\frac{\varepsilon}{2},\ \lvert a_n-A\rvert <\frac{\varepsilon}{2}$$
	
	因而$$\lvert a_n-a_m\rvert \leqslant \lvert a_n-A\rvert + \lvert a_m-A\rvert=\varepsilon$$
	
	{\heiti 充分性}\qquad 先证明该数列必定有界.取$\varepsilon=1$,因为$\left\{a_n\right\}$满足柯西收敛准则的条件,所以$\exists N_0,\ \forall n>N_0$,有
	$$\lvert a_n-a_{N_0+1}\rvert <1$$
	
	取$M=\max\left\{\lvert a_1 \rvert,\ \lvert a_2 \rvert,\ \cdot\cdot\cdot\,\ \lvert a_{N_0} \rvert,\ \lvert a_{N_0+1} \rvert+1\right\}$,则对一切$n$,成立$$\lvert a_n \rvert\leqslant M$$
	
	由致密性原理,在$\left\{a_n\right\}$中必有收敛子列$$\lim_{k \to +\infty}a_{n_k}=\xi$$
	
	由条件,$\forall \varepsilon>0,\ \exists N$,当$n,\ m>N$时,有$$\lvert a_n-a_m\rvert <\frac{\varepsilon}{2}$$
	
	在上式中取$a_m=a_{n_k}$,其中$k$充分大,满足$n_k>N$,并且令$k \to \infty$,于是得到
	$$\lvert a_n-\xi \rvert\leqslant \frac{\varepsilon}{2}< \varepsilon $$
	
	即数列$\left\{a_n\right\}$收敛.$\hfill\blacksquare$
\end{proof}
\begin{definition}[闭区间套]
	设闭区间列$\left\{\left[a_n,b_n\right]\right\}$具有如下性质:
	\begin{enumerate}
		\item $\left[a_n,b_n\right]\supset \left[a_n+1,b_n+1\right],\ n=1,2,\cdots$;
		\item $\lim\limits_{n \to +\infty}(b_n-a_n)=0$.
	\end{enumerate}
	则称$\left\{\left[a_n,b_n\right]\right\}$为{\heiti 闭区间套},或简称{\heiti 区间套}.
\end{definition}

由性质1,构成闭区间套的闭区间列是前一个套着后一个的,即各闭区间端点满足如下不等式:
\begin{equation}\label{chuan}
	a_1\leqslant a_2\leqslant \cdots\leqslant a_n\leqslant\cdots\leqslant b_n\leqslant\cdots\leqslant b_2\leqslant b_1.
\end{equation}
\begin{theorem}[闭区间套定理]
	若$\left\{\left[a_n,b_n\right]\right\}$是一个闭区间套,则在实数系中存在唯一的一点$\xi$,使得$\xi\in\left[a_n,b_n\right],\ n=1,2,\cdots$,即$$a_n\leqslant\xi\leqslant b_n,\ n=1,2,\cdots.$$
\end{theorem}
\begin{proof}
	由式\ref{chuan}可以看出,数列$\left\{a_n\right\}$是递增数列且有界,$\left\{b_n\right\}$是递减数列且有界,由单调有界收敛定理,可知$\left\{a_n\right\}$和$\left\{b_n\right\}$都收敛.设$\lim\limits_{n\to \infty}a_n=\xi$,由闭区间套的第2条性质,得$\lim\limits_{n\to \infty}b_n=\xi.$
	
	$\left\{a_n\right\}$是递增数列,有$a_n\leqslant\xi,\ n=1,2,\cdots;$
	
	$\left\{b_n\right\}$是递减数列,有$b_n\geqslant\xi,\ n=1,2,\cdots.$\\
	所以有$a_n\leqslant\xi\leqslant b_n,\ n=1,2,\cdots.$\\
	下面证明$\xi$的唯一性:\\
	假设存在$\xi'$满足$a_n\leqslant\xi'\leqslant b_n,\ n=1,2,\cdots.$,则\\
	$$\lvert\xi'-\xi\rvert\leqslant b_n-a_n,\ n=1,2,\cdots,$$\\
	由闭区间套的第2条性质,有
	$$\lvert\xi'-\xi\rvert\leqslant \lim\limits_{n\to\infty}(b_n-a_n)=0,\ n=1,2,\cdots,$$\\
	故$\xi'=\xi$.$\hfill\blacksquare$
\end{proof}
\begin{definition}
	设$S$为数轴上的点集,$H$为开区间的集合(即$H$的每一个元素都是形如$(\alpha,\beta)$的开区间).若$S$中的任何一点都含在$H$中至少一个开区间内,则称$H$为$S$的一个{\heiti 开覆盖},或称$H$覆盖$S$.若$H$中开区间的个数是无限(有限)的,则称$H$为$S$的一个{\heiti 无限开覆盖(有限开覆盖)}.若存在$S$的开覆盖$H'\subseteq H$,则称$H'$是$H$的{\heiti 子覆盖},特别地,当$H'$中含有的开区间的个数为有限个时,称$H'$为$H$的{\heiti 有限子覆盖}.
\end{definition}
\begin{theorem}[Heine-Borel有限覆盖定理]
	设$H$是闭区间$\left[a,b\right]$的一个(无限)开覆盖,则从$H$中能选出有限个开区间来覆盖$\left[a,b\right]$.
	
	即:有限闭区间的任一开覆盖都存在一个有限子覆盖.
\end{theorem}
\begin{proof}
	设$H$是闭区间$\left[a,b\right]$的一个开覆盖,定义集合
	$$S=\left\{x|x\in \left(a,b \right],\ \mbox{且}\left[a,x\right]\mbox{存在开覆盖}H\mbox{的一个有限子覆盖} \right\}.$$
	
	因为$H$是$\left[a,b\right]$的一个开覆盖,所以存在一个区间$I_0\in H$使得$a\in I_0$,则存在$x_0\in I_0$满足$x_0>a$,所以$S\neq\varnothing$.显然$b$是$S$的一个上界,由确界原理,$S$一定有上确界.设$M=\sup S\leqslant b$,下面证明$M=b$:
	
	反证法\qquad 假设$M<b$,则$M\in \left(a,b \right]$,$\left[a,M\right]$存在$H$的一个有限子覆盖.假设开区间$I_1$包含$M$,则存在$\delta >0$使得$(M-\delta,M+\delta)\subseteq I_1$,因为$M$是$S$的上确界,所以$M-\delta\in S$,记$\left[a,M-\delta\right]$的有限开覆盖为$H'$,则$\left[a,M+\delta\right]$也有有限开覆盖$H'\cup I_1$,得$M+\delta\in S$这与$M$是$S$的上确界矛盾.所以$M=b$,即$\left[a,b\right]$的开覆盖$H$存在一个有限子覆盖.$\hfill\blacksquare$
\end{proof}
\begin{remark}
	法国数学家Borel于1895年第一次陈述并证明了现代形式的Heine-Borel定理.此定理只对有限闭区间成立,而对开区间则不一定成立.例如开区间集合$$\left\{(\frac{1}{n+1},1)\right\},\ (n=1,2,\cdots)$$构成了开区间$(0,1)$的开覆盖,但不能从中选出有限个开区间覆盖住$(0,1)$.
\end{remark}
从上面的讨论我们发现,如果从数轴上取下一段“紧致无缝”的集合(含端点),那么就可以从它的任意开覆盖中取出一个有限子覆盖,否则就不行.这表明我们找到了一个刻画实数集完备性的新方法,我们形象地将这个性质称为“紧致性”.
\begin{definition}[紧致集]
	设集合$E\in\mathbb{R}$,若集合$E$的任一开覆盖都存在一个有限子覆盖,则称$E$为$\mathbb{R}$上的一个{\heiti 紧致集}.
\end{definition}
\begin{remark}
	紧致集也称紧集,是一个重要的拓扑概念.
\end{remark}
\begin{remark}
	以后我们会将以上条件称为"Heine-Borel条件".
\end{remark}
我们可以重新表述Heine-Borel有限覆盖定理.
\begin{theorem}[Heine-Borel有限覆盖定理]
	$\mathbb{R}$中的任一有限闭区间都是紧致集.
\end{theorem}
\begin{definition}[邻域]
	设$a\in\mathbb{R},\ \delta>0$,将满足$\lvert x-a\rvert<\delta$的全体$x$的集合称为{\heiti $a$的$\delta$邻域},记作$U(a,\delta)$.将满足$0<\lvert x-a\rvert<\delta$的全体$x$的集合称为{\heiti $a$的$\delta$去心邻域},记作$\mathring{U}(a,\delta)$.
\end{definition}
显然,邻域与去心邻域的区别在于去心邻域不包含中心点$a$.
\begin{definition}[聚点]
	设$S$是数轴上的点集,$\xi$是一个定点(可以在$S$中也可以不在$S$中),若$\xi$的任一邻域中都含有$S$中无穷多个点,则称$\xi$为$S$的一个聚点.
\end{definition}
聚点的另一定义如下:
\begin{definition}
	若存在各项互异的收敛数列$\left\{x_n\right\}\subset S$,则其极限$\lim\limits_{n\to \infty}x_n=\xi$是$S$的一个聚点.
\end{definition}
\begin{theorem}[Weierstrass聚点定理]
	实轴上任一有界无限点集$S$至少有一个聚点.
\end{theorem}
由聚点的等价定义,该定理也可叙述为:{\heiti 有界数列必有收敛子列},即致密性定理.
\subsection{实数集完备性定理的等价关系}
通过前面的学习,我们共有以下8个基本定理来叙述实数的完备性:
\begin{enumerate}
	\item Dedekind定理;
	\item 确界原理;
	\item 单调有界收敛定理;
	\item 致密性定理;
	\item 柯西收敛准则;
	\item 闭区间套定理;
	\item Heine-Borel有限覆盖定理;
	\item Weierstrass聚点定理.
\end{enumerate}
可以证明,这8个基本定理都是等价的.(证明会在后续修正时给出)
\newpage

\chapter{随机变量及其分布}
\section{基本概念}
\subsection{随机变量的定义}
\begin{definition}[随机变量]
	定义在样本空间$\varOmega$上的实值函数$X=X(\omega)$称为{\heiti 随机变量},常用大写字母$X,Y,Z$等表示随机变量,其取值用小写字母$x,y,z$等表示.
	
	假如一个随机变量仅可能取有限个值或可列个值,则称其为{\heiti 离散随机变量}. 假如一个随机变量的可能取值充满数轴上的一个区间$(a,b)$,则称其为{\heiti 连续随机变量},其中$a$可以是$-\infty$,$b$可以是$+\infty$.
\end{definition}
由定义,随机变量$X$是样本点$\omega$的一个映射. 这个映射的原像(定义域)可以是数,也可以不是数,但像一定是实数.

与微积分中的变量不同,概率论中的随机变量$X$是一种“随机取值的变量且伴随一个分布”. 以离散随机变量为例,我们不仅要知道$X$可能的取值,而且还要知道去这些值的概率各是多少,这就需要分布的概念. 有没有分布也是区分一般变量和随机变量的主要标志.
\subsection{分布函数}
\begin{definition}[分布函数]
	设$X$是一个随机变量,对任意实数$x$,称
	$$F(x)=P(X\leqslant x)$$
	为随机变量$X$的{\heiti 分布函数}. 且称$X$服从$F(x)$,记为$X\sim F(x)$. 有时也用$F_X(x)$以表明是$X$的分布函数.
\end{definition}
由定义可知,{\heiti 任一随机变量$X$(离散的或连续的)都有一个分布函数}. 有了分布函数,就可据此计算与随机变量$X$有关的事件的概率.

下面给出了分布函数的三个性质.
\begin{theorem}[分布函数的基本性质]
	设分布函数$F(x)$,则它有基本性质:
	\begin{enumerate}[(1)]
		\item {\heiti 单调性}\qquad $F(x)$是定义在整个实轴$(-\infty,+\infty)$上的单调增函数,即对任意$x_1<x_2$,有$F(x_1)\leqslant f(x_2)$.
		\item {\heiti 有界性}\qquad 对任意$x$,有$0\leqslant F(x)\leqslant 1$,且
		$$F(-\infty)=\lim\limits_{x\to -\infty}F(x)=0.$$
		$$F(+\infty)=\lim\limits_{x\to +\infty}F(x)=1.$$
		\item {\heiti 右连续性}\qquad $F(x)$是$x$的右连续函数,即对任意的$x_0$,有
		$$\lim\limits_{x\to x_0+0}F(x)=F(x_0).$$
		即
		$$F(x_0+0)=F(x_0)$$
	\end{enumerate}
\end{theorem}
\begin{proof}
	(1)是显然的. 下证(2).
	
	由于$F(x)$是事件$\{X\leqslant x\}$的概率,所以$0\leqslant F(x)\leqslant 1$. 由$F(x)$的单调性知,对任意整数$m$和$n$,有
	$$\lim\limits_{x\to -\infty}F(x)=\lim\limits_{m\to -\infty}F(m),\quad \lim\limits_{x\to +\infty}F(x)=\lim\limits_{n\to +\infty}F(n)$$
	都存在. 又由概率的可列可加性得
	\begin{align*}
		1&=P(-\infty<X<+\infty)=P\left(\bigcup_{i=-\infty}^{+\infty}\{i-1<X\leqslant i\}\right)\\
		&=\sum_{i=-\infty}^{+\infty}P(i-1<X\leqslant i)=\lim_{\substack{n\to +\infty\\ m\to -\infty}}\sum_{i=m}^{n}P(i-1<X\leqslant i)\\
		&=\lim\limits_{n\to +\infty}F(n)-\lim\limits_{m\to -\infty}F(m),
	\end{align*}
	由此可得
	$$\lim\limits_{x\to -\infty}F(x)=0,\quad \lim\limits_{x\to +\infty}F(x)=1.$$
	
	再证(3). 因为$F(x)$是单调有界增函数,所以其任一点$x_0$的右极限$F(x_0+0)$必存在. 为证右连续性,只要对任意单调减的数列$x_1>x_2>\cdots>x_n>\cdots>x_0$,当$x_n\to x_0\ (n\to\infty)$时,证明$\lim\limits_{n\to\infty}F(x_n)=F(x_0)$成立即可. 因为
	\begin{align*}
		F(x_1)-F(x_0)&=P(x_0<X\leqslant x_1)=P\left(\bigcup_{i=1}^{\infty}\{x_{i+1}<X\leqslant x_i\}\right)\\
		&=\sum_{i=1}^{\infty}P(x_{i+1}<X\leqslant x_i)=\sum_{i=1}^{\infty}\left[F(x_i)-F(x_{i+1})\right]\\
		&=\lim\limits_{n\to\infty}\left[F(x_1)-F(x_n)\right]=F(x_1)-\lim\limits_{n\to\infty}F(x_n),
	\end{align*}
	由此得
	$$F(x_0)=\lim\limits_{n\to\infty}F(x_n)=F(x_0+0).$$
	至此三条基本性质证毕.$\hfill\blacksquare$
\end{proof}
我们还可以证明满足上述三条性质的函数是某个随机变量的分布函数. 从而这三条基本性质是某个函数是分布函数的充要条件.
\begin{example}
	函数
	$$F(x)=\frac{1}{\pi}\left(\arctan x+\frac{\pi}{2}\right),\ -\infty<x<+\infty$$
	满足分布函数的三条基本性质,所以它是一个分布函数,特别地,它称为{\heiti Cauchy分布函数}.
\end{example}
\subsection{离散随机变量的概率分布列}
\begin{definition}[概率分布列]
	设$X$是一个离散随机变量,如果$X$的所有可能取值是$x_1,x_2,\cdots,x_n,\cdots$,则称$X$取$x_i$的概率
	$$p_i=p(x_i)=P(X=x_i),\ i=1,2,\cdots,n,\cdots$$
	为$X$的{\heiti 概率分布列}或{\heiti 分布列},记为$X\sim \{P_i\}$.
\end{definition}
分布列也可用如下列表的方式来表示.

\begin{tabular}{c|ccccc}
	$X$ & $x_1$ & $x_2$ & $\cdots$ & $x_n$ & $\cdots$\\
	\hline
	$P$ & $p(x_1)$ & $p(x_2)$ & $\cdots$ & $p(x_n)$ & $\cdots$\\
\end{tabular}

\begin{theorem}
	分布列有以下基本性质:
	\begin{enumerate}[(1)]
		\item {\heiti 非负性}\quad $p(x_i\geqslant 0),\ i=1,2,\cdots.$
		\item {\heiti 正则性}\quad $\displaystyle\sum_{i=1}^{\infty}p(x_i)=1$.
	\end{enumerate}
\end{theorem}
由离散随机变量$X$的分布列很容易写出$X$的分布函数
$$F(x)=\sum_{x_j\leqslant x}p(x_i).$$
它的图形是至多可数级的阶梯函数. 不过在离散场合,常用来描述分布的是分布列,很少用到分布函数.

特别地,常量$c$可看作仅取一个值的随机变量$X$,即
$$P(X=c)=1.$$
这个分布常称为{\heiti 单点分布}或{\heiti 退化分布},它的分布函数是
\begin{equation*}
	F(x)=
	\left\{
	\begin{aligned}
		&0,\quad x<c\\
		&1,\quad x\geqslant c.
	\end{aligned}
	\right.
\end{equation*}
在具体求随机变量$X$的分布列时,关键是求出$X$的所有可能取值及取这些值的概率.
\subsection{连续随机变量的概率密度函数}
\begin{definition}
	设随机变量$X$的分布函数为$F(x)$,如果存在实数轴上的一个非负可积函数$p(x)$,使得对任意实数$x$有
	$$F(x)=\int_{-\infty}^{x}p(t)\d t,$$
	则称$p(x)$为$X$的{\heiti 概率密度函数}(probability density function),简称{\heiti 密度函数}或{\heiti 密度}. 同时称$X$为{\heiti 连续随机变量},称$F(x)$为{\heiti 连续分布函数}.
\end{definition}
在$F(x)$的可导点上有
$$F'(x)=p(x).$$
其中$F(x)$是(累积)概率函数,其导函数$F'(x)$是概率密度函数,这也是$p(x)$被称为概率密度函数的理由.
\begin{theorem}[密度函数基本性质]
	\begin{enumerate}[(1)]
		\item {\heiti 非负性}\quad $p(x)\geqslant 0$.
		\item {\heiti 正则性}\quad $\displaystyle\int_{-\infty}^{+\infty}p(x)]\d x=1$.
	\end{enumerate}
\end{theorem}
\section{数学期望}
\subsection{数学期望的定义}
下面我们先定义离散随机变量的数学期望.
\begin{definition}[离散随机变量的数学期望]
	设离散随机变量$X$的分布列为
	$$p(x_i)=P(X=x_i),\ i=1,2,\cdots,n,\cdots.$$
	如果
	$$\sum_{i=1}^{\infty}|x_i|p(x_i)<\infty,$$
	则称
	$$E(X)=\sum_{i=1}^{\infty}x_ip(x_i)$$
	为随机变量$X$的{\heiti 数学期望}(expectation),或称为该分布的数学期望,简称{\heiti 期望}或{\heiti 均值}. 若级数$\displaystyle\sum_{i=1}^{\infty}|x_i|p(x_i)$不收敛,则称$X$的数学期望不存在.
\end{definition}
\begin{remark}
	以上定义中,要求级数绝对收敛的目的在于使数学期望唯一. 因为随机变量的取值可正可负,取值次序可先可后,由无穷级数理论可知,如果无穷级数绝对收敛,则可保证其和不受次序变动的影响. 
\end{remark}
\begin{remark}
	由于有限项的和不受次序变动的影响,故取有限个可能值的随机变量的数学期望总是存在的.
\end{remark}
类似地,我们可以定义连续随机变量的数学期望.
\begin{definition}[连续随机变量的数学期望]
	设连续随机变量$X$的密度函数为$p(x)$. 如果
	$$\int_{-\infty}^{+\infty}|x|p(x)\d x<\infty,$$
	则称
	$$E(x)=\int_{-\infty}^{+\infty}xp(x)\d x$$
	为$X$的{\heiti 数学期望}(expectation),或称为该分布$p(x)$的数学期望,简称{\heiti 期望}或{\heiti 均值}. 若$\displaystyle\int_{-\infty}^{+\infty}|x|p(x)\d x$不收敛,则称$X$的数学期望不存在.
\end{definition}

数学期望的理论意义是深刻的,它是消除随机性的主要手段.
\subsection{数学期望的性质}
下面均假设所涉及的数学期望存在,给出数学期望的一些性质.
\begin{theorem}
	若随机变量$X$的分布用分布列$p(x_i)$或用密度函数$p(x)$表示,则$X$的某一函数$g(X)$的数学期望为
	\begin{equation}\label{expectation}
		E\left[g(X)\right]=
		\left\{
		\begin{aligned}
			&\sum_{i}g(x_i)p(x_i),\quad&\text{在离散场合},\\
			&\int_{-\infty}^{+\infty}g(x)p(x)\d x,\quad&\text{在连续场合}.
		\end{aligned}
		\right.
	\end{equation}
\end{theorem}
\begin{proof}
	
\end{proof}
\begin{theorem}
	若$c$是常数,则$E(c)=c$.
\end{theorem}
\begin{proof}
	如果将$c$看作仅取一个值的随机变量$X$,则有$P(X=c)=1$,从而其数学期望$E(c)=E(X)=c\times 1=c$.$\hfill\blacksquare$
\end{proof}
\begin{theorem}
	对任意常数$a$,有
	$$E(aX)=aE(X).$$
\end{theorem}
\begin{proof}
	在\ref{expectation}式中令$g(x)=ax$,然后把$a$从求和号或积分号中提出来即得.$\hfill\blacksquare$
\end{proof}
\begin{theorem}
	对任意的两个函数$g_1(x)$和$g_2(x)$,有
	$$E\left[g_1(X)\pm g_2(X)\right]=E\left[g_1(X)\right]\pm E\left[g_2(X)\right].$$
\end{theorem}
\begin{proof}
	在\ref{expectation}式中令$g(x)=g_1(x)\pm g_2(x)$,然后把和式分解成两个和式,或把积分分解成两个积分即得.$\hfill\blacksquare$
\end{proof}
\subsection{Markov不等式}
\begin{theorem}[Markov不等式]
	设非负随机变量$X$的数学期望$E(X)$存在,则对任意常数$a>0$,有
	$$P(X\geqslant a)\leqslant\frac{E(X)}{a}.$$
\end{theorem}
\begin{proof}
	设连续的非负随机变量$X$的密度函数为$p(x)$,数学期望为$E(X)$,则
	\begin{align*}
		E(X)&=\int_{-\infty}^{\infty}xp(x)\d x=\int_{0}^{\infty}xp(x)\d x=\int_{0}^{a}xp(x)\d x+\int_{a}^{\infty}xp(x)\d x\\
		&\geqslant\int_{a}^{\infty}xp(x)\d x\geqslant\int_{a}^{\infty}ap(x)\d x=a\int_{a}^{\infty}p(x)\d x=aP(X\geqslant a).
	\end{align*}
	即
	$$P(X\geqslant a)\leqslant\frac{E(X)}{a}.$$
	$\hfill\blacksquare$
\end{proof}
Markov不等式的一个实际应用是,超过$n$倍平均工资的人数不会超过总人数的$\dfrac{1}{n}$.
\section{方差与标准差}
\subsection{方差与标准差的定义}
方差和标准差是度量随机变量取值波动大小最重要的两个特征数.
\begin{definition}[方差与标准差的定义]
	若随机变量$X^2$的数学期望$E(X^2)$存在,则称偏差平方$(X-E(X))^2$的数学期望$E(X-E(X))^2$为随机变量$X$或相应分布的{\heiti 方差}(variance),记为
	\begin{equation*}
		\text{Var}(X)=E(X-E(X))^2=
		\left\{
		\begin{aligned}
			&\sum_{i}(x_i-E(X))^2p(x_i),\quad&\text{在离散场合,}\\
			&\int_{-\infty}^{+\infty}(x-E(X))^2p(x)\d x,\quad&\text{在连续场合}.
		\end{aligned}
		\right.
	\end{equation*}
	称方差的正平方根$\sqrt{\text{Var}(X)}$为随机变量的{\heiti 标准差}(standrad deviation),记为$\sigma(X)$或$\sigma_X$.
\end{definition}
方差与标准差功能相似,它们都可以用来描述随机变量取值的集中和分散程度的两个特征数. 方差和标准差越小,随机变量的取值越集中;反之,方差和标准差越大,随机变量的取值就越分散.

方差和标准差之间的差别主要在量纲上. 由于标准差与数学期望有相同的量纲,因此$E(x)\pm k\sigma$是有意义的($k$为正实数),所以在实际中,人们比较乐意选用标准差,但标准差的计算必须通过方差才能算得.

\subsection{方差的性质}
以下均假定随机变量的方差是存在的.
\begin{theorem}
	$\mathrm{Var}(X)=E(X^2)-\left[E(X)\right]^2$.
\end{theorem}
\begin{proof}
	因为
	$$\mathrm{Var}(X)=E\left[X-E(X)\right]^2=E\left(X^2-2X\cdot E(X)+\left[E(X)\right]^2\right),$$
	由数学期望的性质,得
	$$\mathrm{Var}(X)=E(X^2)-2E(X)\cdot E(X)+\left[E(X)\right]^2=E(X^2)-\left[E(X)\right]^2.$$
	$\hfill\blacksquare$
\end{proof}
\begin{remark}
	在实际计算方差时,这个性质往往比定义$\mathrm{Var}(X)=E\left[X-E(X)\right]^2$更常用.
\end{remark}
\begin{theorem}
	常数的方差为$0$.
\end{theorem}
\begin{proof}
	设$c$为常数,则
	$$\mathrm{Var}(c)=E\left[c-E(c)\right]^2=E(c-c)^2=0.$$
	$\hfill\blacksquare$
\end{proof}
\begin{theorem}
	若$a,b$是常数,则$\mathrm{Var}(aX+b)=a^2\mathrm{Var}(X)$.
\end{theorem}
\begin{proof}
	因$a,b$是常数,则
	$$\mathrm{Var}(aX+b)=E\left[aX+b-E(aX+b)\right]^2=E\left[a(X-E(X))\right]^2=a^2\mathrm{Var}(X).$$
	$\hfill\blacksquare$
\end{proof}
\subsection{Chebyshev不等式}
前面我们介绍了Markov不等式,实际上它给出的界太过宽松,为进一步精细化,我们介绍下面的Chebyshev不等式. 事实上,它是Markov不等式的一种特殊情形.
\begin{theorem}[Chebyshev不等式]
	设随机变量$X$的数学期望和方差都存在,则对任意常数$\varepsilon>0$,有
	$$P(|X-E(X)|\geqslant\varepsilon)\leqslant\frac{\mathrm{Var}(X)}{\varepsilon^2},$$
	或
	$$P(|X-E(X)|<\varepsilon)\geqslant 1-\frac{\mathrm{Var}(X)}{\varepsilon^2}.$$
\end{theorem}
\begin{proof}
	只需证第一个不等式. 设$X$是一个连续型随机变量,其密度函数为$p(x)$. 记$E(X)=a$,$A=\{x:|x-a|\geqslant\varepsilon\}$. 我们有
	$$P(|X-a|\geqslant\varepsilon)=\int_{A}p(x)\d x\leqslant\int_{A}\frac{(x-a)^2}{\varepsilon^2}p(x)\d x\leqslant\frac{1}{\varepsilon^2}\int_{-\infty}^{\infty}(x-a)^2p(x)\d x=\frac{\mathrm{Var}(X)}{\varepsilon^2}.$$
	对于离散型随机变量亦可类似进行证明.$\hfill\blacksquare$
\end{proof}
在概率论中,事件$\{|X-E(X)|\geqslant\varepsilon\}$称为{\heiti 大偏差},其概率$P(|X-E(X)|\geqslant\varepsilon)$称为{\heiti 大偏差发生概率}. Chebyshev不等式给出了大偏差发生概率的上界,这个上界与方差成正比,方差越大上界也越大.

下面定理进一步说明了方差为$0$就意味着随机变量的取值几乎集中在一点上.
\begin{theorem}
	若随机变量$X$的方差存在,则$\mathrm{Var}(X)=0$的充要条件是$X$几乎处处为某个常数$a$,即$P(X=a)=1$.
\end{theorem}
\begin{proof}
	充分性显然成立. 下证必要性.
	
	设$\mathrm{Var}(X)=0$,这时$E(X)$存在. 因为
	$$\{|X-E(X)|>0\}=\bigcup_{n=1}^{\infty}\left\{|X-E(X)|\geqslant\frac{1}{n}\right\},$$
	所以有
	\begin{align*}
		P(|X-E(X)|>0)&=P\left(\bigcup_{n=1}^{\infty}\left\{|X-E(X)|\geqslant\frac{1}{n}\right\}\right)\\
		&\leqslant\sum_{n=1}^{\infty}P\left(|X-E(X)|\geqslant\frac{1}{n}\right)\\
		&\leqslant\sum_{n=1}^{\infty}\frac{\mathrm{Var}(X)}{(1/n)^2}=0,
	\end{align*}
	其中最后一个不等式用到了Chebyshev不等式. 由此可知
	$$P(|X-E(X)|>0)=0,$$
	因而有
	$$P(|X-E(X)|=0)=1,$$
	即
	$$P(X=E(X))=1,$$
	这就证明了结论,且其中的常数$a$就是$E(X)$.$\hfill\blacksquare$
\end{proof}
\begin{remark}
	设零测集$E_0\subseteq E$,$P$是关于$E$中元素的命题,若对$\forall x\in E\backslash E_0$,命题$P$成立,那么我们说命题$P$在$E$上{\heiti 几乎处处}(almost everywhere)成立.
\end{remark}
\subsection{随机变量的中心化与标准化}
下面我们介绍随机变量的中心化和标准化.
\begin{definition}[随机变量的中心化]
	已知$X$是任意的随机变量,当$E(x)$和$\mathrm{Var}(x)$存在时,对$X$作变换
	$$X_{*}=X-E(x),$$
	这个变换称为{\heiti 随机变量的中心化}.
\end{definition}
由期望和方差的性质推得:
$$E(X_{*})=E\left[X-E(X)\right]=E(X)-E\left[E(X)\right]=0,$$
$$\mathrm{Var}(X_{*})=\mathrm{Var}\left[X-E(X)\right]=\mathrm{Var}(X).$$
即中心化后的随机变量,期望为$0$,方差不变.

中心化的性质解释:
\begin{enumerate}[(1)]
	\item 期望归零化:中心化随机变量将其中心点(期望点)平移至原点,使其分布不偏左也不偏右,其期望为零.
	\item 分布波动不变性:平移不影响波动的分布程度,方差不变.
\end{enumerate}
\begin{definition}[随机变量的标准化]
	已知$X$是任意的随机变量,当$E(X)$和$\mathrm{Var}(X)$存在且$\mathrm{Var}(X)\neq 0$时,对$X$作变换
	$$X^{*}=\frac{X-E(x)}{\sqrt{\mathrm{Var}(X)}},$$
	这个变换称为随机变量的标准化.
\end{definition}
由期望和方差的性质推得:
$$E(X^{*})=E\left[\frac{X-E(X)}{\sqrt{\mathrm{Var}(X)}}\right]=\frac{E\left[X-E(X)\right]}{E\left(\mathrm{Var}(x)\right)}=0,$$
$$\mathrm{Var}(X^{*})=\mathrm{Var}\left[\frac{X-E(X)}{\sqrt{\mathrm{Var}(X)}}\right]=\left(\frac{1}{\sqrt{\mathrm{Var}(X)}}\right)^2\mathrm{Var}(X)=1.$$
即标准化后的随机变量,期望为$0$,方差为$1$.

标准化的性质解释:
\begin{enumerate}[(1)]
	\item 期望归零化:标准化随机变量将其中心点(期望点)平移至原点,使其分布不偏左也不偏右,其期望为$0$.
	\item 分布波动归一化:标准化将随机变量的取值按照标准差等比压缩,使其分布不疏也不密,压缩改变了分布的波动程度,方差变为$1$.
\end{enumerate}
\section{常用离散分布}
\subsection{Bernoulli分布}
\begin{definition}[Bernoulli分布]
	如果$X$只取值$0$或$1$,概率分布是
	$$P(X=1)=p,\quad P(X=0)=q,\quad p+q=1,$$
	就称$X$服从{\heiti Bernoulli 分布}(Bernoulli Distribution),或称{\heiti 二点分布}或{\heiti 0-1分布}记作$X\sim b(1,p)$.
\end{definition}
任何试验,当只考虑成功与否时,就可以用Bernoulli分布的随机变量描述.
\subsection{二项分布}
\begin{definition}[二项分布]
	如果随机变量有如下的概率分布:
	$$P(X=k)=\mathrm{C}$$
\end{definition}










\end{document}