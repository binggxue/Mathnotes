\documentclass[12pt]{ctexart}
\usepackage{amsfonts,amssymb,amsmath,amsthm,geometry,color,enumerate}
\usepackage[colorlinks,linkcolor=blue,anchorcolor=blue,citecolor=green]{hyperref}
%introduce theorem environment
\theoremstyle{definition}
\newtheorem{definition}{定义}
\newtheorem{theorem}{定理}
\newtheorem{lemma}{引理}
\newtheorem{corollary}{推论}
\newtheorem{property}{性质}
\newtheorem{example}{例}
\theoremstyle{plain}
\newtheorem*{solution}{解}
\newtheorem*{remark}{注}
\geometry{a4paper,scale=0.8}

%remove the dot after the theoremstyle
%\makeatletter
%\xpatchcmd{\@thm}{\thm@headpunct{.}}{\thm@headpunct{}}{}{}
%\makeatother

%article info
\title{\vspace{-2em}\textbf{Lusin定理}\vspace{-2em}}
\date{ }
\begin{document}
	\maketitle
	
	Lusin定理刻画了可测函数与连续函数之间的关系. 为证明需要,首先定义特征函数和简单函数.
	\begin{definition}[特征函数]
		设集合$A\subset X$,定义函数
		\begin{equation*}
			\chi_A(x)=\left\{\begin{aligned}
				&1,\ x\in A,\\
				&0,\ x\in X\backslash A.
			\end{aligned}\right.
		\end{equation*}
		称$\chi_A(x)$为定义在$X$上的$A$的\textbf{特征函数}.
	\end{definition}
	\begin{definition}[简单函数]
		设定义在$D$上的$f(x)=c_i,\ x\in D_i$,$i\in I$使得至多可数个集合$D_i$满足
		$$\bigsqcup_{i\in I}D_i=D,$$
		则称$f(x)$为\textbf{简单函数}.
	\end{definition}
	\begin{remark}
		简单来说,简单函数是有有限个取值或可列个取值的函数.
	\end{remark}
	简单函数可以用特征函数的语言描述,即
	$$f(x)=\sum_{i\in I}c_i\chi_{D_i}(x),\qquad f(x)=c_i,\ x\in D_i$$
	\begin{theorem}[简单函数逼近定理]
		\ 
		
		\begin{enumerate}
			\item 若$f(x)$是$E$上的非负可测函数,则存在非负可测的简单函数渐升列:
			$$\varphi_k(x)\leqslant\varphi_{k+1}(x),\ k=1,2,\cdots,$$
			使得
			$$\lim\limits_{k\to\infty}\varphi_k(x)=f(x),\ x\in E.$$
			\item 若$f(x)$是$E$上的可测函数,则存在可测的简单函数列$\{\varphi_k(x)\}$,使得$|\varphi_k(x)|\leqslant|f(x)|$,且
			$$\lim\limits_{k\to\infty}\varphi_k(x)=f(x),\ x\in E.$$
			若$f(x)$还是有界的,则简单函数列是一致收敛的.
		\end{enumerate}
	\end{theorem}
	\begin{remark}
		为节省篇幅和突出主题,该定理证明从略.
	\end{remark}
	\begin{theorem}[Lusin定理]
		设$f(x)$是定义在$E\subset\mathbb{R}^n$上的几乎处处有限的可测函数,则对任意$\delta>0$,存在闭集$F\subset E$满足$m(E\backslash F)<\delta$,使得$f(x)$是$F$上的连续函数.
	\end{theorem}
	\begin{proof}
		不妨设$f(x)$处处有限.
		
		首先考虑$f(x)$为可测有限简单函数.即
		$$f(x)=\sum_{i=1}^{p}c_i\chi_{E_i}(x),\ x\in E=\bigsqcup_{i=1}^{p}E_i,$$
		对任意$\delta>0$,在每个$E_i$中,可以取闭集$F_i$使得
		$$m(E_i\backslash F_i)<\frac{\delta}{p},\qquad i=1,2,\cdots,p.$$ 
		对任意$x\in F_i$,$f(x)=c_i$为连续函数,又$F_i$两两不交,故$f(x)$在$F=\bigcup F_i$上连续. 显然$F$为闭集,且
		$$m(E\backslash F)=\sum_{i=1}^{p}m(E_i\backslash F_i)<\sum_{i=1}^{p}\frac{\delta}{p}<\delta.$$
		
		其次考虑$f(x)$为一般可测函数. 不妨设为有界函数. 由简单函数逼近定理,存在简单函数列$\{\varphi_k(x)\}$,在$E$上一致收敛于$f(x)$. 故对任意$\delta>0$,可作闭集$F_k$,$m(E\backslash F_k)<\dfrac{\delta}{2^k}$,使得$\varphi_k(x)$在$F_k$上连续. 令$F=\bigcap_{k=1}^{\infty}F_k$,则$F\subset E$,且
		$$m(E\backslash F)<\sum_{k=1}^{\infty}m\left(E\backslash F_k\right)<\sum_{k=1}^{\infty}\frac{\delta}{2^k}<\delta.$$
		由于每个$\varphi_k(x)$在$F$上都是连续的,由一致收敛性,$f(x)$在$F$上连续.
		
		
	\end{proof}
\end{document}