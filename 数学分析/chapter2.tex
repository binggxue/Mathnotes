\chapter{数列极限}
\section{数列极限的定义}
数列极限在我们直观上是说对于数列$\left\{a_n\right\}$,当$n$越来越大(趋近于$\infty$)时,$a_n$越来越接近于一个确定的数$a$,以至于在$n$取到$\infty$的时候。$a_n$就等于$a$了.相信读者在直观上对数列极限的概念都有或多或少的认识.然而,这种叙述不够严谨.“$n$越来越大”是多大?“趋近于$\infty$”是如何趋近的?“$a_n$越来越接近一个确定的数”是说$a_n$和$a$总是随着$n$的增大而越接近的吗?是否存在$a_n$与$a$在距离上的波动性?以上问题都是由于极限的定义不够清晰所导致的,因此,我们采用更严谨的$\varepsilon-N$语言定义数列极限.
\begin{definition}[数列极限]
	对于数列$\left\{a_n\right\}$,$a$为一个确定的数,对$\forall \varepsilon>0,\ \exists N,\ s.t.\ n>N$时,有$\lvert a_n-a\rvert<\varepsilon$,则称数列$\left\{a_n\right\}$收敛于$a$,$a$就称为数列$\left\{a_n\right\}$的极限.记作$$\lim\limits_{n\to\infty}a_n=a\mbox{或者}a_n\to a,\ (n\to \infty)$$
\end{definition}
若数列$\{a_n\}$没有极限,则称$\{a_n\}$不收敛,或称$\{a_n\}$为{\heiti 发散数列}.
\begin{example}
	证明$\lim\limits_{n\to\infty}\frac{1}{n}=0$.
\end{example}
\begin{proof}
	$\forall \varepsilon>0$,取$N=\frac{1}{\varepsilon}$,当$n>N$时,$\lvert a_n-0\rvert<\varepsilon$,所以$\lim\limits_{n\to\infty}\frac{1}{n}=0$.$\hfill\blacksquare$
\end{proof}
例题在此不过多展示,可以参考任意一本数学分析教材.
经过一部分练习,我们对数列极限的定义有了更深刻的印象.下面我们对$\varepsilon-N$语言进行进一步的解释.

$\varepsilon$是任意的正实数,也就是说它可以无限接近于$0$,那么$2\varepsilon$,$3\varepsilon$,$\frac{\varepsilon}{2}$等都是任意的正实数,同样可以无限接近于$0$,也就反映$a_n$和$a$的距离会随着$n$的增大而非常接近.这里的$N$可以看作$n$增大的一个阈值,超过这个阈值,$a_n$和$a$的距离总是小于一个限度$\varepsilon$.我们将证明某个数列极限的步骤总结如下:
\begin{enumerate}
	\item 列出前提条件:$\forall \varepsilon>0$.
	\item 取$N$,解不等式$\lvert a_n-a \rvert<\varepsilon$,得到$n>f(\varepsilon)$($f(\varepsilon)$表示不等式右边是关于$\varepsilon$的表达式),我们可以取$N=f(\varepsilon)$或者取$N=\left[f(\varepsilon)\right]+1$(取整后加1即将$f(\varepsilon)$向上取整).
	\item 将上述取到的$N$按数列极限的定义书写,证明完毕.
\end{enumerate}

我们在上述例题中看出,$N$的取值是与$\varepsilon$有关的,所以$N$也可以写成$N(\varepsilon)$,一般来说,$\varepsilon$的值越小,取到的$N$就会越大,$\varepsilon$是任意正实数,当它充分接近于$0$时,$N$就趋向于无穷.

按定义,我们也可以这样描述数列极限:
\begin{definition}[数列极限]
	对$\varepsilon>0$,数列$\left\{a_n\right\}$至多有有限项落在$a$的$\varepsilon$邻域之外,则称$a$的数列$\left\{a_n\right\}$的极限.
\end{definition}
在所有收敛数列中,有一类重要的数列,称为无穷小数列,其定义如下.
\begin{definition}[无穷小数列]
	若$\lim\limits_{n\to\infty}a_n=0$,则称$\{a_n\}$为{\heiti 无穷小数列}.
\end{definition}
由无穷小数列的定义,不难证明以下定理.
\begin{theorem}
	数列$\{a_n\}$收敛于$a$的充要条件是:$|a_n-a|$是无穷小数列.
\end{theorem}
最后我们简单介绍一下无穷大数列.
\begin{definition}[无穷大数列]
	若数列$\{a_n\}$满足:对任意正数$M>0$,总存在正整数$N$,使得当$n>N$时,有$$|a_n|>M,$$则称数列$\{a_n\}$发散于无穷大,并记作
	$$\lim\limits_{n\to\infty}a_n=\infty\mbox{或}a_n\to\infty.$$
	我们称这样的数列$\{a_n\}$为{\heiti 无穷大数列}.
\end{definition}
无穷大数列虽然不收敛,但却有一定的变化趋势.关于无穷小数列和无穷大数列的关系,我们有以下的定理.
\begin{theorem}
	对于数列$\{x_n\},\ \forall n\in \mathbb{Z}_+,\ x_n\neq 0$,则数列$\{x_n\}$为无穷小数列$\iff$ $\{\frac{1}{x_n}\}$为无穷大数列.
\end{theorem}
\begin{proof}
	充分性\qquad $\lim\limits_{n\to\infty}x_n=0$,即$\forall \varepsilon>0,\ \exists N,\ s.t.\ \forall n>N,\ |x_n|<\varepsilon$,则$\frac{1}{x_n}>\frac{1}{\varepsilon}$,取$M=\frac{1}{\varepsilon}$,由于$\varepsilon$的任意性,$M$可以取到任意正数.所以有$\frac{1}{x_n}>M$,即$\frac{1}{x_n}$为无穷大数列.
	
	必要性\qquad $\lim\limits_{n\to\infty}\frac{1}{x_n}=\infty$,即$\forall M>0,\ \exists N,\ s.t.\ \forall n>N,\ |\frac{1}{x_n}|>M$,则$|x_n|<\frac{1}{M}$,取$\varepsilon=\frac{1}{M}$,由于$M$的任意性,$\varepsilon$可以取到任意正数.所以有$|x_n|=|x_n-0|<\varepsilon$,即$x_n$为无穷小数列.
	$\hfill\blacksquare$
\end{proof}
\section{收敛数列的性质}
收敛数列有如下一些重要性质.
\begin{theorem}[唯一性]
	若数列$\left\{a_n\right\}$收敛,则它只有一个极限.
\end{theorem}
\begin{proof}
	设$a$是收敛数列$\left\{a_n\right\}$的一个极限,对$\forall b\neq a$,取$\varepsilon_0=\frac{1}{2}|b-a|$,则$U(b,\varepsilon_0)$中至多有数列$\left\{a_n\right\}$的有限个项,则$b$不是数列$\left\{a_n\right\}$的极限.这就证明了收敛数列极限的唯一性.$\hfill\blacksquare$
\end{proof}
一个收敛数列一般含有无穷多个数,而它的极限只是一个数,我们单凭这一个数就能精确地估计出几乎全体项的大小.以下收敛数列的一些性质,大都基于这一事实.
\begin{theorem}[有界性]
	收敛数列必有界.
\end{theorem}
\begin{proof}
	设数列$\left\{a_n\right\}$收敛,$\lim\limits_{n\to\infty}a_n=a$,取$\varepsilon=1$,则由数列极限的定义,$\exists N\in\mathbb{Z}_+,\ s.t.\ \forall n>N$,有
	
	\begin{center}
		$\lvert a_n-a\rvert<1$,即$a-1<a_n<a+1$.
	\end{center}
	
	记$$M=\max\{|a_1|,|a_2|,\cdots,|a_N|,|a-1|,|a+1|\},$$
	则对$\forall n\in\mathbb{Z}_+$都有$|a_n|<M$.
	
	即数列$\left\{a_n\right\}$有界.$\hfill\blacksquare$
\end{proof}
\begin{remark}
	有界性是数列收敛的必要条件,而非充分条件.即有界不一定收敛,收敛一定有界.
\end{remark}
\begin{theorem}[保号性]
	若$\lim\limits_{n\to\infty}a_n=a>0$(或$<0$),则对任何$a'\in(0,a)$(或$a'\in (a,0)$),存在正数$N$,使得当$n>N$时,有$a_n>a'$(或$a_n<a'$).
\end{theorem}
\begin{proof}
	对于$a>0$的情形,取$\varepsilon=a-a'(>0)$,则存在正数$N$,使得当$n>N$时,有$a_n>a-\varepsilon=a'$,同理可证$a<0$的情形.$\hfill\blacksquare$
\end{proof}
\begin{corollary}[保号性推论]
	设$\lim\limits_{n\to\infty}a_n=a,\ \lim\limits_{n\to\infty}b_n=b,\ a<b$,则存在$N$,使得当$n>N$时,有$a_n<b_n.$
\end{corollary}
\begin{proof}
	因为$a<\frac{a+b}{2}<b$,所以由保号性,存在$N_1$,当$n>N_1$时,有$$a_n<\frac{a+b}{2};$$存在$N_2$,当$n>N_2$时,有$$b_n>\frac{a+b}{2}.$$取$N=\max\{N_1,N_2\}$,当$n>N$时,有$$a_n<b_n.$$$\hfill\blacksquare$
\end{proof}
\begin{theorem}[保序性]
	设$\{a_n\}$和$\{b_n\}$都是收敛数列,若存在正数$N_0$,使得当$n>N_0$时,有$a_n\leqslant b_n$,则$\lim\limits_{n\to\infty}a_n\leqslant \lim\limits_{n\to\infty}b_n$.
\end{theorem}
\begin{proof}
	设$\lim\limits_{n\to\infty}a_n=a,\ \lim\limits_{n\to\infty}b_n=b$.对于任意$\varepsilon>0$,分别存在正数$N_1$和$N_2$,当$n>N_1$时,有$$a-\varepsilon<a_n$$
	当$n>N_2$时,有$$b_n<b+\varepsilon$$
	取$N=\max\{N_0,N_1,N_2\}$,当$n>N$时,有$$a-\varepsilon<a_n\leqslant n-b_n<b+\varepsilon,$$由此得到$a<b+2\varepsilon$,由$\varepsilon$的任意性得$a\leqslant b$,即$\lim\limits_{n\to\infty}a_n\leqslant \lim\limits_{n\to\infty}b_n$.$\hfill\blacksquare$
\end{proof}
\begin{theorem}[迫敛性]
	设收敛数列$\{a_n\},\ \{b_n\}$都以$a$为极限,数列$\{c_n\}$满足:存在正数$N_0$,当$n>N_0$时,有$$a_n\leqslant c_n\leqslant b_n,$$则数列$\{c_n\}$收敛,且$\lim\limits_{n\to\infty}c_n=a$
\end{theorem}
\begin{proof}
	$\lim\limits_{n\to\infty}a_n=\lim\limits_{n\to\infty}b_n=a$,则对$\forall\varepsilon>0,\ \exists N_1,\ N_2>0,\ s.t.$
	$$a-\varepsilon<a_n;$$
	$$b_n<a+\varepsilon.$$
	取$N=\max\{N_0,N_1,N_2\}$,当$n>N$时,有
	$$a-\varepsilon	<a_n\leqslant c_n\leqslant b_n<a+\varepsilon$$
	从而有$|c_n-a|<\varepsilon$,即$\lim\limits_{n\to\infty}c_n=a.$$\hfill\blacksquare$
\end{proof}
\begin{theorem}[四则运算法则]
	若$\{a_n\}$与$\{b_n\}$为收敛数列,则$\{a_n\pm b_n\},\ \{a_n\cdot b_n\}$也都是收敛数列,且$$\lim\limits_{n\to\infty}(a_n\pm b_n)=\lim\limits_{n\to\infty}a_n\pm \lim\limits_{n\to\infty}b_n,$$
	$$\lim\limits_{n\to\infty}(a_n\cdot b_n)=\lim\limits_{n\to\infty}a_n\cdot \lim\limits_{n\to\infty}b_n.$$
	特别当$\{b_n\}$为常数$c$时,有
	$$\lim\limits_{n\to\infty}(a_n+c)=\lim\limits_{n\to\infty}a_n+c,\ \lim\limits_{n\to\infty}ca_n=c\lim\limits_{n\to\infty}a_n.$$
	若再假设$b_n\neq 0$及$\lim\limits_{n\to\infty}b_n\neq 0$,则$\{\frac{a_n}{b_n}\}$也是收敛数列,且有
	$$\lim\limits_{n\to\infty}\frac{a_n}{b_n}=\frac{\lim\limits_{n\to\infty}a_n}{\lim\limits_{n\to\infty}b_n}.$$
\end{theorem}
\begin{proof}
	由于$a_n-b_n=a_n+(-1)b_n$,$\frac{a_n}{b_n}=a_n\cdot \frac{1}{b_n}$,因此我们只需要证明关于和、积、倒数运算的结论即可.
	
	设$\lim\limits_{n\to\infty}a_n=a,\ \lim\limits_{n\to\infty}b_n=b$,则对$\forall\varepsilon>0$,分别存在正数$N_1$和$N_2$,使得
	$$|a_n-a|<\varepsilon,\mbox{当}n>N_1,$$
	$$|b_n-b|<\varepsilon,\mbox{当}n>N_2.$$
	取$N=\max\{N_1,N_2\}$,则当$n>N$时上述两不等式同时成立,从而有
	\begin{enumerate}
		\item 	$|(a_n+b_n)-(a+b)|\leqslant |a_n-a|+|b_n-b|<2\varepsilon$
		,这就证得$\lim\limits_{n\to\infty}(a_n+b_n)=\lim\limits_{n\to\infty}a_n+\lim\limits_{n\to\infty}b_n.$
		\item	$|a_nb_n-ab|=|(a_n-a)b_n+a(b_n-b)|\leqslant |b_n||a_n-a|+|a||b_n-b|.$
		由收敛数列的有界性可知,存在一正数$M$,使得$|b_n|<M$,则有
		$$|a_nb_n-ab|\leqslant |b_n||a_n-a|+|a||b_n-b|<(M+|a|)\varepsilon.$$
		由$\varepsilon$的任意性,这就证得$\lim\limits_{n\to\infty}a_nb_n=ab.$
		\item  由于$\lim\limits_{n\to\infty}b_n=b\neq 0$,根据收敛数列的保号性,存在正数$N_3$,使得当$n>N_3$时,有$|b_n|>\frac{1}{2}|b|$.取$N'=\max\{N_2,N_3\}$,则当$n>N'$时,有
		$$\left|\frac{1}{b_n}-\frac{1}{b}\right|=\frac{|b_n-b|}{|b_n b|}<\frac{2|b_n-b|}{b^2}<\frac{2\varepsilon}{b^2}.$$
		由$\varepsilon$的任意性,这就证得$\lim\limits_{n\to\infty}\frac{1}{b_n}=\frac{1}{b}.$
		$\hfill\blacksquare$
	\end{enumerate}
\end{proof}
\begin{definition}[子列]
	设$\{a_n\}$为数列,$\{n_k\}$为正整数集$\mathbb{Z}_+$的无限子集,且$n_1<n_2<\cdots <n_k<\cdots$,则数列
	$$a_{n_1},a_{n_2},\cdots,a_{n_k},\cdots$$
	称为数列$\{a_n\}$的一个子列,记为$\{a_{n_k}\}$.
\end{definition}
\begin{theorem}
	数列$\{a_n\}$收敛的充要条件是:$\{a_n\}$的任何子列都收敛.
\end{theorem}
\begin{proof}
	充分性\qquad 因为$\{a_n\}$也是自身的一个子列,所以结论是显然的.
	
	必要性\qquad 设$\lim\limits_{n\to \infty}a_n=a$,$\{a_{n_k}\}$是$\{a_n\}$的任一子列.则对$\forall \varepsilon>0,\ \exists N>0,\ s.t.\ \forall k>N$时,有$|a_k-a|<\varepsilon$.又因为$n_k\geqslant k$,故$k>N$时,有$|a_{n_k}-a|<\varepsilon$.这就证明了任一子列$\{a_{n_k}\}$收敛(且与$\{a_n\}$有相同的极限).
	$\hfill\blacksquare$
\end{proof}
\section{Stolz定理}


\section{数列极限存在的条件}
在研究比较复杂的数列极限问题时,通常先考虑数列极限的存在性问题,若有极限,再考虑数列的计算问题.在实际应用中,解决的数列极限的存在性问题,即使极限值的计算较为困难,但由于当$n$充分大时,$a_n$能充分接近其极限,故可用$a_n$来作为极限的近似值,本节将重点讨论极限的存在性问题.
\subsection{数列极限的存在性定理}
\begin{definition}[数列的单调性]
	\begin{enumerate}
		\item 若数列$a_n$满足$$a_n\leqslant a_{n+1},\forall n\in \mathbb{Z}_+,$$则称该数列是{\heiti 递增}的,其中,若有$$a_n<a_{n+1}$$成立,则称该数列是{\heiti 严格递增}的;
		\item 若数列$a_n$满足$$a_n\geqslant a_{n+1},\forall n\in \mathbb{Z}_+,$$则称该数列是{\heiti 递减}的,其中,若有$$a_n>a_{n+1}$$成立,则称该数列是{\heiti 严格递减}的.
		\item 递增数列和递减数列统称为{\heiti 单调}数列(无论是否严格).
	\end{enumerate}
\end{definition}
\begin{theorem}[单调有界收敛定理]
	在实数系中,有界的单调数列必有极限.
\end{theorem}
\begin{proof}
	不妨设$\left\{a_n\right\}$为有上界的递增数列,由确界原理,$\left\{a_n\right\}$必有上确界,设$a=\sup \left\{a_n\right\}$,根据上确界的定义,
	
	$\forall \varepsilon>0,\ \exists a_N \ s.t.\ a_N>a-\varepsilon$
	
	由$\left\{a_n\right\}$的递增性,当$n\geqslant N$时,有
	$$a-\varepsilon<a_N\leqslant a_n$$
	
	又因为$$a_n\leqslant a<a+\varepsilon$$
	
	故$$a-\varepsilon<a_n<a+\varepsilon$$
	
	即$${\lim_{n \to +\infty}a_n}=a$$
	$\hfill\blacksquare$
\end{proof}

\begin{theorem}[致密性定理]
	任何有界数列必定有收敛的子列.
\end{theorem}
要证明此定理,可以先证明以下引理.
\begin{lemma}\label{zilie}
	任何数列都存在单调子列.
\end{lemma}
\begin{proof}
	设数列为$\left\{a_n\right\}$,下面分两种情况讨论:
	\begin{enumerate}
		\item 若$\forall k\in \mathbb{Z}_+$,$\left\{a_{k+n}\right\}$都有最大项,记$\left\{a_{1+n}\right\}$的最大项为$a_{n_1}$,则$a_{{n_1}+n}$也有最大项,记作$a_{n_2}$,显然有$a_{n_1}\geqslant a_{n_2}$,同理,有$$a_{n_2}\geqslant a_{n_3}$$
		$$.........$$
		由此得到一个单调递减的子列$\left\{a_{n_k}\right\}$
		\item 若至少存在一个正整数$k$,使得$\left\{a_{k+n}\right\}$没有最大项,先取$n_1=k+1$,总存在$a_{n_1}$后面的项$a_{n_2}$($n_2>n_1$)使得$$a_{n_2}>a_{n_1}$$,同理,总存在$a_{n_2}$后面的项$a_{n_3}$($n_3>n_2$)使得$$a_{n_3}>a_{n_2}$$
		$$.........$$
		由此得到一个严格递增的子列$\left\{a_{n_k}\right\}$
	\end{enumerate}
	
	综上,命题得证.$\hfill\blacksquare$
\end{proof}
下面是对致密性定理的证明:
\begin{proof}
	设数列$\left\{a_n\right\}$有界,由引理\ref{zilie},数列$\left\{a_n\right\}$存在单调且有界的子列,由单调有界收敛定理得出该子列是收敛的.$\hfill\blacksquare$
\end{proof}
\begin{theorem}[柯西(Cauchy)收敛准则]
	数列$\left\{a_n\right\}$收敛的充要条件是:\par 
	$\forall \varepsilon>0,\ \exists N\in \mathbb{Z}_+,\ s.t.\ n,\ m>N$时,有
	$$\lvert a_n - a_m \rvert<\varepsilon$$
\end{theorem}
单调有界只是数列收敛的充分条件,而柯西收敛准则给出了数列收敛的充要条件.
\begin{proof}
	{\heiti 必要性}\qquad 设$\lim\limits_{n \to +\infty}a_n=A$,则$\forall \varepsilon>0,\ \exists N\in \mathbb{Z}_+\ s.t.\ n,\ m>N$时,有$$\lvert a_n-A\rvert <\frac{\varepsilon}{2},\ \lvert a_n-A\rvert <\frac{\varepsilon}{2}$$
	
	因而$$\lvert a_n-a_m\rvert \leqslant \lvert a_n-A\rvert + \lvert a_m-A\rvert=\varepsilon$$
	
	{\heiti 充分性}\qquad 先证明该数列必定有界.取$\varepsilon=1$,因为$\left\{a_n\right\}$满足柯西收敛准则的条件,所以$\exists N_0,\ \forall n>N_0$,有
	$$\lvert a_n-a_{N_0+1}\rvert <1$$
	
	取$M=\max\left\{\lvert a_1 \rvert,\ \lvert a_2 \rvert,\ \cdot\cdot\cdot\,\ \lvert a_{N_0} \rvert,\ \lvert a_{N_0+1} \rvert+1\right\}$,则对一切$n$,成立$$\lvert a_n \rvert\leqslant M$$
	
	由致密性原理,在$\left\{a_n\right\}$中必有收敛子列$$\lim_{k \to +\infty}a_{n_k}=\xi$$
	
	由条件,$\forall \varepsilon>0,\ \exists N$,当$n,\ m>N$时,有$$\lvert a_n-a_m\rvert <\frac{\varepsilon}{2}$$
	
	在上式中取$a_m=a_{n_k}$,其中$k$充分大,满足$n_k>N$,并且令$k \to \infty$,于是得到
	$$\lvert a_n-\xi \rvert\leqslant \frac{\varepsilon}{2}< \varepsilon $$
	
	即数列$\left\{a_n\right\}$收敛.$\hfill\blacksquare$
\end{proof}
\subsection{自然常数与Euler常数}
\begin{example}
	证明极限$\lim\limits_{n\to\infty}(1+\frac{1}{n})^n$存在.
\end{example}
\begin{proof}
	设$a_n=(1+\frac{1}{n})^n,\ n=1,2,\cdots .$由二项式定理,
	
	\begin{flalign*}
		a_n&=(1+\frac{1}{n})^n\\
		&=1+C_n^1\frac{1}{n}+\cdots+C_n^k\frac{1}{n^k}+\cdots+C_n^n\frac{1}{n}\\
		&=1+1+\frac{n(n+1)}{2!}\frac{1}{n^2}+\cdots+\frac{n(n-1)\cdots(n-k+1)}{k!}\frac{1}{n^k}+\cdots+\frac{1}{n^n}\\
		&=2+\frac{1}{2!}(1-\frac{1}{n})+\cdots+\frac{1}{k!}(1-\frac{1}{n})(1-\frac{2}{n})\cdots(1-\frac{k-1}{n})+\cdots+\\
		&\quad\frac{1}{n!}(1-\frac{1}{n})(1-\frac{2}{n})\cdots(1-\frac{n-1}{n})\\
		&<2+\frac{1}{2!}(1-\frac{1}{n+1})+\cdots+\frac{1}{k!}(1-\frac{1}{n+1})(1-\frac{2}{n+1})\cdots(1-\frac{k-1}{n+1})+\cdots+\\
		&\quad\frac{1}{(n+1)!}(1-\frac{1}{n+1})(1-\frac{2}{n+1})\cdots(1-\frac{n}{n+1})\\
		&=a_{n+1},
	\end{flalign*}
	
	故$\{a_n\}$是严格递增的.由上式可推得
	\begin{align*}
		a_n
		&<2+\frac{1}{2!}+\cdots+\frac{1}{k!}+\cdots+\frac{1}{n!}\\
		&<2+\frac{1}{1\cdot2}+\cdots+\frac{1}{(k-1)k}+\cdots+\frac{1}{(n-1)n}\\
		&=2+(1-\frac{1}{2})+\cdots+(\frac{1}{k-1}-\frac{1}{k})+\cdots+(\frac{1}{n-1}-\frac{1}{n})\\
		&=3-\frac{1}{n}<3.
	\end{align*}
	这表明$a_n$是有界的.由单调有界收敛定理可知$\lim\limits_{n\to\infty}(1+\frac{1}{n})^n$存在.$\hfill\blacksquare$
\end{proof}
\begin{remark}
	通常用拉丁字母e代表该数列的极限,即
	$$\lim\limits_{n\to\infty}(1+\frac{1}{n})^n=\text{e}.$$
	它是一个无理数(待证),其前几位数字是
	$$\text{e} \approx 2.718281828459045.$$
\end{remark}
\begin{example}
	证明$\lim\limits_{n\to\infty}(1+\frac{1}{n})^{n+1}$严格单调递减趋于e.
\end{example}
\begin{proof}
	$\lim\limits_{n\to\infty}(1+\frac{1}{n})^{n+1}=\lim\limits_{n\to\infty}(1+\frac{1}{n})^n(1+\frac{1}{n})=\lim\limits_{n\to\infty}\text{e}(1+\frac{1}{n})=\text{e}.$
	设$a_n=(1+\frac{1}{n})^{n+1}$,下面只需证明$\{a_n\}$单调递减,即$$a_n>a_{n+1}.$$
	
	由均值不等式,
	\begin{align*}
		\frac{1}{a_n}=(\frac{n}{n+1})^{n+1}
		=1\cdot \underbrace{\frac{n}{n+1}\cdot\cdots\cdot\frac{n}{n+1}}_{n+1\text{个}}<(\frac{n+1}{n+2})^{n+2}=\frac{1}{a_{n+1}}
	\end{align*}
	
	故$a_n>a_{n+1}$.$\hfill\blacksquare$
\end{proof}
由上面两道例题,我们可以得出如下不等式:
\begin{equation}
	(1+\frac{1}{n})^n<\text{e}<(1+\frac{1}{n})^{n+1}.\ n=1,2,\cdots
\end{equation}

分别对两个不等号取对数,有
$$n\ln(1+\frac{1}{n})<1;$$
$$(n+1)\ln(1+\frac{1}{n})>1$$

得
\begin{equation}{\label{equ:xln}}
	\frac{1}{n+1}<\ln(1+\frac{1}{n})<\frac{1}{n}.\ n=1,2,\cdots
\end{equation}

将$n=1,2,\cdots,n$依次代入,得
$$\frac{1}{2}<\ln\frac{2}{1}<\frac{1}{1}$$
$$\frac{1}{3}<\ln\frac{3}{2}<\frac{1}{2}$$
$$\cdots$$
$$\frac{1}{n+1}<\ln\frac{n+1}{n}<\frac{1}{n}$$

将上述各式相加,得
\begin{equation}{\label{tiaohe}}
	\frac{1}{2}+\frac{1}{3}+\cdots+\frac{1}{n+1}<\ln(n+1)<1+\frac{1}{2}+\cdots+\frac{1}{n}.\ n=1,2,\cdots
\end{equation}

以上三个不等式都非常有用.
\begin{proposition}
	设数列$$\widetilde{e_n}=1+\frac{1}{1!}+\frac{1}{2!}+\cdots+\frac{1}{n!},\ n=1,2,\cdots$$
	则$\lim\limits_{n\to\infty}\widetilde{e_n}=\text{e}.$
\end{proposition}
\begin{proof}
	在对$\lim\limits_{n\to\infty}(1+\frac{1}{n})^n=\text{e}$的证明过程中已知$\widetilde{e_n}$是有界的,显然$\widetilde{e_n}$是单调递增的,有单调有界收敛定理可知$\{\widetilde{e_n}\}$收敛.
	
	(i)由于
	$$e_n=(1+\frac{1}{n})^n\leqslant1+\frac{1}{1!}+\frac{1}{2!}+\cdots+\frac{1}{n!}=\widetilde{e_n},$$
	由极限的保序性可知
	$\lim\limits_{n\to\infty}e_n\leqslant\lim\limits_{n\to\infty}\widetilde{e_n}.$
	
	(ii)给定$m\leqslant n$,则
	$$e_n=\sum_{i=0}^{n}\frac{1}{i!}(1-\frac{1}{n})(1-\frac{2}{n})\cdots(1-\frac{i-1}{n})\geqslant\sum_{i=0}^{m}\frac{1}{i!}(1-\frac{1}{n})(1-\frac{2}{n})\cdots(1-\frac{i-1}{n}).$$
	令$n\to\infty$,得
	$$\lim\limits_{n\to \infty}e_n\geqslant\sum_{i=0}^{m}\frac{1}{i}$$
	令$m\to\infty$,得$\lim\limits_{n\to\infty}e_n\geqslant\lim\limits_{n\to\infty}\widetilde{e_n}.$
	
	综合(i)和(ii)可知$\lim\limits_{n\to\infty}\widetilde{e_n}=\lim\limits_{n\to\infty}e_n=\text{e}.$
	$\hfill\blacksquare$
\end{proof}
在实际应用中,我们常用数列$\{\widetilde{e_n}\}$来计算e的值,这是因为$\{\widetilde{e_n}\}$收敛得更快.
\begin{theorem}
	自然常数e是一个无理数.
\end{theorem}

\begin{proof}
	反证法\qquad 假设e是一个有理数,则$\text{e}=\frac{p}{q},\ p,q\in\mathbb{Z}_+$,因为$2<\text{e}<3$,所以e不是整数,所以有$q\geqslant 2$.
	
	\begin{align*}
		\widetilde{e}_{n+m}-\widetilde{e_n}
		&=\frac{1}{(n+1)!}+\frac{1}{(n+2)!}+\cdots+\frac{1}{(n+m)!}\\
		&=\frac{1}{(n+1)!}\left[1+\frac{1}{n+2}+\cdots+\frac{1}{(n+2)\cdots (n+m)}\right]\\
		&<\frac{1}{(n+1)!}\left[1+\frac{1}{n+2}+\cdots+\left(\frac{1}{n+2}\right)^{m-1}\right]\\
		&=\frac{1}{(n+1)!}\cdot \frac{1-\left(\frac{1}{n+2}\right)^m}{1-\frac{1}{n+2}}.
	\end{align*}
	
	令$m\to\infty$,则
	
	\begin{align*}
		\text{e}-\widetilde{e_n}
		&=\frac{1}{(n+1)!}\cdot \frac{1-\left(\frac{1}{n+2}\right)^m}{1-\frac{1}{n+2}}\\
		&\leqslant\frac{1}{(n+1)!}\cdot \frac{1}{1-\frac{1}{n+2}}\\
		&<\frac{1}{(n+1)!}\cdot \frac{1}{1-\frac{1}{n+1}}\\
		&=\frac{1}{(n+1)}\cdot\frac{n+1}{n}\\
		&=\frac{1}{n!n}.
	\end{align*}
	
	则
	\begin{align}
		q!(\text{e}-\widetilde{e_n})&\leqslant \frac{1}{q}\leqslant\frac{1}{2}\notin \mathbb{Z}\\
		\nonumber
		q!(\text{e}-\widetilde{e_n})
		&=q!\left[\frac{p}{q}-\left(1+\frac{1}{1!}+\frac{1}{2!}+\cdots+\frac{1}{q!}\right)\right]\\
		&=(q-1)!p-(q!+2\cdot3\cdots q+3\cdot4\cdots q+\cdots+1)\in\mathbb{Z}
	\end{align}
	
	这里出现矛盾,假设不成立,因此e是无理数.$\hfill\blacksquare$
\end{proof}
\begin{proposition}
	有两个数列$\{x_n\}$,$\{y_n\}$,
	$$x_n=1+\frac{1}{2}+\cdots+\frac{1}{n}-\ln(n+1)$$
	$$y_n=1+\frac{1}{2}+\cdots+\frac{1}{n}-\ln n$$
	则$\{x_n\}$和$\{y_n\}$收敛到同一实数.
\end{proposition}
\begin{proof}
	(i)先证明$\{x_n\}$收敛,由不等式\ref{equ:xln},
	\begin{align*}
		x_{n+1}-x_n&=\frac{1}{n+1}-\ln(n+2)+\ln(n+1)\\
		&=\frac{1}{n+1}-\ln\left(1+\frac{1}{n+1}\right)>0
	\end{align*}
	所以$\{x_n\}$严格递增.
	
	由不等式\ref{tiaohe},
	$$1+\frac{1}{2}+\cdots+\frac{1}{n}-\ln(n+1)<1-\frac{1}{n+1}<1$$
	所以$\{x_n\}$有界.
	
	由单调有界收敛定理,$\{x_n\}$收敛.
	
	由于$y_{n+1}=1+\frac{1}{2}+\cdots+\frac{1}{n}+\frac{1}{n+1}-\ln(n+1)=x_n+\frac{1}{n+1}$,所以$y_n$也收敛.
	$$\lim\limits_{n\to\infty}y_n=\lim\limits_{n\to\infty}y_{n+1}=\lim\limits_{n\to\infty}x_n+\lim\limits_{n\to\infty}\frac{1}{n+1}=\lim\limits_{n\to\infty}x_n.$$
	$\hfill\blacksquare$
\end{proof}
\begin{remark}
	以上数列的极限称为{\heiti Euler常数},记作$\gamma$,其前几位数字为
	$$\gamma=0.5772156649\dots$$
	需要注意,Euler常数虽然极有可能是无理数,但至今尚未证明其无理性.
\end{remark}
\section{上极限和下极限}
\subsection{上极限和下极限的定义}
\begin{definition}[实数系的扩充]
	定义扩充后的实数系$\widetilde{\mathbb{R}}=\mathbb{R}\cup \pm\infty$.
\end{definition}
本节内容基于扩充后的实数系讨论.
\begin{definition}[数列的聚点]
	若数$a$的任一邻域含有数列$\{x_n\}$中的无限多个项,则称$a$为数列$\{x_n\}$的一个聚点.
\end{definition}
\begin{remark}
	数列(或点列)的聚点定义与实数理论中关于数集(或点集)的聚点定义是有区别的.当把点列看作点集时,点列中对应于相同数值的项,只能作为一个点来看待,如点列$\left\{\sin\frac{n\pi}{4}\right\}$作为点集来看待时,它仅含有五个点,即
	$$\left\{\sin\frac{n\pi}{4}\right\}=\left\{-1,-\frac{\sqrt{2}}{2},0,\frac{\sqrt{2}}{2},1\right\}$$
	按点集聚点的定义,这个有限集没有聚点.而我们在点列聚点的定义中只考虑项,只要在一点的任意小邻域内聚集了无穷多个项(不论其数值是否相同),该点就称为点列的聚点.所以点列(数列)的聚点实际上就是其收敛子列的极限.
\end{remark}
\begin{theorem}[存在性]
	任何数列(点列)至少有一个聚点.
\end{theorem}
\begin{definition}[上极限与下极限]
	设数列$\{x_n\}$的聚点(极限点)组成的集合为$E$,定义$\sup E$为数列$\{x_n\}$的{\heiti 上极限},记作$\limsup\limits_{n\to\infty}x_n$或$\varlimsup\limits_{n\to\infty}x_n$,
	定义$\inf E$为数列$\{x_n\}$的{\heiti 下极限},记作$\liminf\limits_{n\to\infty}x_n$或$\varliminf\limits_{n\to\infty}x_n$.
\end{definition}
对于无界数列,其只有一个聚点$+\infty$或$-\infty$,则$$\limsup\limits_{n\to\infty}x_n=\liminf\limits_{n\to\infty}x_n=+\infty(\text{或}-\infty)$$

由于$E\neq\varnothing$,因此任一数列都存在上极限和下极限,这一点使得上极限和下极限比一般的极限更具“应用优势”.

一般来说,求数列的上极限和下极限没有固定的简单方法,对于极限点只有有限个的情况,我们可以分别求出这有限个极限点从而直接得出上极限和下极限.
\begin{proposition}
	$\limsup\limits_{n\to\infty}x_n\in E$,$\liminf\limits_{n\to\infty}x_n\in E$
\end{proposition}
上述命题说明了上极限和下极限分别是数列的最大聚点和最小聚点.
\begin{proof}
	只证明上极限的情况.
	
	(i)$\limsup\limits_{n\to\infty}x_n=+\infty$,则$E$无上界,$\{a_n\}$无上界,故$\{a_n\}$一定存在以$+\infty$为极限的子列,因此$+\infty$也是一个极限点,即$+\infty\in E$;
	
	(ii)$\limsup\limits_{n\to\infty}x_n=-\infty$,则$E=\{-\infty\}$,故$-\infty\in E$;
	
	(iii)$\limsup\limits_{n\to\infty}x_n=a$,即$\{x_n\}$有界,$\exists M>0\ s.t.\ $
	$$-M<x_n<M.$$
	
	取$\left[a_1,b_1\right]=\left[-M,M\right]$,将区间$\left[a_{k-1},b_{k-1}\right]$等分为两个子区间,若右边一个含有$\{x_n\}$的无穷多个项,则取它为$\left[a_k,b_k\right]$,否则取左边的子区间为$\left[a_k,b_k\right]$.($k=2,3,\cdots$)
	
	这样的选取方法既保证了每次选出的$\left[a_k,b_k\right]$都含有$\{x_n\}$中的无限多个项,同时在$\left[a_k,b_k\right]$的右边却至多有$\{x_n\}$的有限个项,于是由区间套$\{\left[a_k,b_k\right]\}$所确定的点列$\{x_n\}$的聚点必是$\{x_n\}$的最大聚点.
	
	类似地,只要把每次优先挑选右边一个子区间改为优先挑选左边一个,就能证明最小聚点的存在性.$\hfill\blacksquare$
\end{proof}
\begin{theorem}
	$$\liminf\limits_{n\to\infty}x_n\leqslant\limsup\limits_{n\to\infty}x_n$$
	等号成立当且仅当数列$\{a_n\}$有极限.	
\end{theorem}

上极限和下极限也可用$\varepsilon-N$语言刻画.
\begin{theorem}[上下极限的$\varepsilon-N$定义]
	设数列$\{a_n\}$,令
	$$E=\{a\in\mathbb{R}:\forall\varepsilon>0,\exists N>0,\text{当}n>N\text{时},a_n<a+\varepsilon\};$$
	$$F=\{a\in\mathbb{R}:\forall\varepsilon>0,\exists N>0,\text{当}n>N\text{时},a_n>a-\varepsilon\}.$$
	则$\limsup\limits_{n\to\infty}a_n=\inf E$,$\liminf\limits_{n\to\infty}a_n=\sup F.$
\end{theorem}
\begin{proof}
	只需证明上极限的情况.设$\limsup\limits_{n\to\infty}a_n=L.$
	
	(i)证明$L\geqslant\inf E$.\quad 用反证法,假设$L<\inf E$,即$L\notin E$,即存在$\varepsilon>0$使得对于任意$N>0$,都存在$n>N$使得$a_n\geqslant L+\varepsilon$,这表明$(L+\varepsilon,+\infty)$中有$\{a_n\}$的无穷多项,由Weierstrass定理可知$(L+\varepsilon,+\infty)$中一定有极限点$L'>L$,出现矛盾.因此$L\in E$,这表明$L\geqslant \inf E$.
	
	(ii)证明$L\leqslant\inf E$.\quad 用反证法,假设$L>\inf E$,即$\exists a\in E$满足$a<L$,存在$\varepsilon>0$使得$a+\varepsilon<L$,由于$a\in E$,故$\exists N>0$使得当$n>N$时,$a_n<a+\varepsilon$.这表明$(a+\varepsilon,+\infty)$中有$\{a_n\}$的有限项,因此$L$不是极限点,出现矛盾.于是$a\geqslant L$,这表明$L\leqslant\inf E$.
	
	综上可知$\limsup\limits_{n\to\infty}a_n=\inf E.$ $\hfill\blacksquare$
\end{proof}
\begin{remark}
	为了让上极限是$+\infty$下极限是$-\infty$的情况也能用上面的语言刻画,可以令
	$$E=\{a\in\widetilde{\mathbb{R}}:\forall x>a,\exists N>0,\text{当}n>N\text{时},a_n<x\},$$
	$$F=\{a\in\widetilde{\mathbb{R}}:\forall x<a,\exists N>0,\text{当}n>N\text{时},a_n>x\}.$$
	则$\limsup\limits_{n\to\infty}a_n=\inf E,\liminf\limits_{n\to\infty}a_n=\sup F.$
\end{remark}
\begin{proposition}[Stolz定理的推广形式]
	设数列$\{a_n\}$和$\{b_n\}$.若$\{b_n\}$严格递增,且$\lim\limits_{n\to\infty}b_n=+\infty$,则
	$$\liminf\limits_{n\to\infty}\frac{a_n-a_{n-1}}{b_n-b_{n-1}}\leqslant\liminf\limits_{n\to\infty}\frac{a_n}{b_n}\leqslant\limsup\limits_{n\to\infty}\frac{a_n}{b_n}\leqslant\limsup\limits_{n\to\infty}\frac{a_n-a_{n-1}}{b_n-b_{n-1}}.$$
\end{proposition}
\begin{proof}
	中间的不等号显然成立.则由对称性,我们只需证右边的不等号成立即可.
	
	令
	$$L=\limsup\limits_{n\to\infty}\dfrac{a_n-a_{n-1}}{b_n-b_{n-1}}.$$
	
	当$L=+\infty$时命题显然成立.当$L<+\infty$时,$\forall\varepsilon>0$,设$l=L+\varepsilon$.则$\exists N\in\mathbb{N}_+$,当$n\geqslant N$时,有
	$$\frac{a_n-a_{n-1}}{b_n-b_{n-1}}<l.$$
	因此有
	$$\frac{a_N-a_{N-1}}{b_N-b_{N-1}}<l,\ \frac{a_{N+1}-a_N}{b_{N+1}-b_N}<l,\cdots,\frac{a_n-a_{n-1}}{b_n-b_{n-1}}<l,$$
	则
	\begin{equation}{\label{rof}}
		\frac{a_n-a_{N-1}}{b_n-b_{N-1}}=\frac{\frac{a_n}{b_n}-\frac{a_{N-1}}{b_n}}{1-\frac{b_{N-1}}{b_n}}<l,
	\end{equation}
	即$$\frac{a_n}{b_n}-\frac{a_{N-1}}{b_n}<l(1-\frac{b_{N-1}}{b_n}).$$
	两边取上极限,有
	$$\limsup\limits_{n\to \infty}\frac{a_n}{b_n}<l=L+\varepsilon,$$
	即
	$$\limsup\limits_{n\to \infty}\frac{a_n}{b_n}\leqslant L=\limsup\limits_{n \to \infty}\frac{a_n-a_{n-1}}{b_n-b_{n-1}}.$$
	左边的不等号证明方法同理.$\hfill\blacksquare$
\end{proof}
\begin{remark}
	由不等式$$\min\{\frac{a_1}{b_1},\frac{a_2}{b_2},\cdots,\frac{a_n}{b_n}\}\leqslant\frac{a_1+a_2+\cdots+a_n}{b_1+b_2+\cdots+b_n}\leqslant \max\{\frac{a_1}{b_1},\frac{a_2}{b_2},\cdots,\frac{a_n}{b_n}\}$$
	得到式\ref{rof}.
\end{remark}
现在我们已经给出了上极限和下极限的两种定义,它们比较直观,都没有与数列极限“直接挂钩”,因此在处理某些问题时并不便利,下面给出了第三种方式来定义上极限和下极限.

之前讲到,研究数列的极限并不关心前面的有限项,即去掉前面的有限项并不会改变数列的极限(如果有极限的话),因此研究数列的“最终趋势”时,也不需要关心前面的有限项.受此启发,对于数列$\{a_n\}$,我们可以定义这样一个数列$\{L_n\}$:
$$L_n\coloneqq \sup\limits_{k\geqslant n}\{a_k\}=\sup\{a_n,a_{n+1},\cdots\},\quad n=1,2,\cdots.$$
由于
$$\{a_k|k\geqslant n+1\}\subset\{a_k|k\geqslant n\},\quad n=1,2,\cdots.$$
因此数列$\{L_n\}$是单调递减的.同理定义数列$\{l_n\}$:
$$l_n\coloneqq \inf\limits_{k\geqslant n}\{a_k\}=\inf\{a_n,a_{n+1},\cdots\},\quad n=1,2,\cdots.$$
且$\{l_n\}$是单调递增的.于是可知数列$\{L_n\}$和$\{l_n\}$都有极限.不难想到$\{L_n\}$和$\{l_n\}$的极限分别是数列$\{a_n\}$的上极限和下极限.
\begin{theorem}
	设数列$\{a_n\}$,则
	
	(1)\ $\limsup\limits_{n\to\infty}a_n=\lim\limits_{n\to\infty}\sup\limits_{k\geqslant n}\{a_n\}$;
	
	(2)\ $\liminf\limits_{n\to\infty}a_n=\lim\limits_{n\to\infty}\inf\limits_{k\geqslant n}\{a_n\}$.
\end{theorem}
\begin{proof}
	只证明(1)即可.令
	$$L_n=\sup\limits_{k\geqslant n}\{a_k\},\quad L=\limsup\limits_{n\to\infty}a_n.$$
	只需证明$\lim\limits_{n\to\infty}L_n=L.$
	
	(i)当$L=+\infty$时,$\{a_n\}$存在一个以$+\infty$为极限的子列.因此
	$$L_n=\sup\limits_{k\geqslant n}\{a_k\}=+\infty.$$
	故$\lim\limits_{n\to\infty}L_n=L.$
	
	(ii)当$L=-\infty$时,$\limsup\limits_{n \to \infty}a_n=-\infty$,则
	$$\lim\limits_{n\to\infty}a_n=-\infty.$$
	对$\forall M>0,\ \exists N\in \mathbb{N}_+$使得当$n\geqslant N$时,$a_n<-M$.则
	$$L_N=\sup\limits_{k\geqslant N}a_k<-M.$$
	又因为$\{L_N\}$单调递减,故当$n\geqslant N$时,有
	$$L_n\leqslant L_N<-M.$$
	即$\lim\limits_{n\to\infty}L_n=-\infty$.
	
	(iii)当$L\in\mathbb{R}$时,任取$\{a_n\}$的一个极限点$l$,对于给定的$n$,选取$i\geqslant n$,则$k_i\geqslant i\geqslant n$,于是
	$$a_{k_i}\leqslant \sup\{a_n,a_{n+1},\cdots,a_{k_i},\cdots\}=L_n,\quad n=1,2,\cdots$$
	令$i\to\infty$得$l\leqslant L_n$,再令$n\to\infty$,得$l\leqslant \lim\limits_{n\to\infty}L_n$,于是$L\leqslant \lim\limits_{n\to\infty}L_n$.
	
	由上下极限的第二种定义,$\forall \varepsilon>0,\ \exists N\in\mathbb{N}_+$使得当$n>N$时有$a_n<L+\varepsilon.$则
	$$L_n<L+\varepsilon,$$
	当$n\to\infty$时,有
	$$\lim\limits_{n\to\infty}L_n\leqslant L.$$
	因此有$L=\lim\limits_{n \to \infty}L_n$,即
	$$\limsup\limits_{n \to \infty}a_n=\lim\limits_{n\to\infty}\sup\limits_{k\geqslant n}\{a_n\}.$$
	$\hfill\blacksquare$
\end{proof}
\begin{proposition}
	设数列$\{a_n\}$,则
	
	(1)$\limsup\limits_{n \to \infty}(-a_n)=-\liminf\limits_{n \to \infty}a_n$;
	
	(2)$\liminf\limits_{n \to \infty}(-a_n)=-\limsup\limits_{n \to \infty}a_n$.
\end{proposition}
\begin{proof}
	因为有$\sup(-E)=-\inf(E)$,$\inf(-E)=-\sup(E)$,故
	$$\sup\limits_{k\geqslant n}\{-a_k\}=-\inf\limits_{k\geqslant n}\{a_k\},\inf\limits_{k\geqslant n}\{-a_k\}=-\sup\limits_{k\geqslant n}\{a_k\}$$
	取$n\to\infty$即得结论.$\hfill\blacksquare$
\end{proof}
\subsection{上下极限的性质}
\begin{theorem}[保序性]
	设数列$\{a_n\},\{b_n\}$,若$\exists N>0$,当$n>N$时,有$a_n\leqslant b_n$,则
	
	(1)$\limsup\limits_{n\to\infty}a_n\leqslant\limsup\limits_{n\to\infty}b_n,$
	
	(2)$\liminf\limits_{n\to\infty}a_n\leqslant\liminf\limits_{n\to\infty}b_n.$
	
	特别地,若$\alpha,\beta$为常数,$\exists N>0$,当$n>N$时,有$\alpha\leqslant a_n\leqslant \beta$,则$$\alpha\leqslant\liminf\limits_{n\to\infty}x_n\leqslant\limsup\limits_{n\to\infty}x_n\leqslant\beta.$$
\end{theorem}
\begin{proof}
	只需证明(1).记$\limsup\limits_{n\to\infty}a_n=A,\ \limsup\limits_{n\to\infty}b_n=B.$
	
	(i)当$B=+\infty$或$A=-\infty$时,命题显然成立;
	
	(ii)当$A=+\infty$时,$A$中存在以$+\infty$为极限的子列,因为若$\exists N>0$,当$n>N$时,有$a_n\leqslant b_n$,所以$B$中也存在以$+\infty$为极限的子列,因此$A=B$,类似可证$B=-\infty$时$A=B$;
	
	(iii)当$A,B\in\mathbb{R}$时,用反证法,假设$A>B$,则存在$\varepsilon>0$使得$B<B+\varepsilon<A$,由上极限的$\varepsilon-N$定义,存在$N_1$,当$n>N_1$时,有$b_n<B+\varepsilon$,取$N_0=\max\{N,N_1\}$,当$n>N_0$时,有
	$$a_n\leqslant b_n<B+\varepsilon<A$$,则在$(B+\varepsilon,+\infty)$之间只有$\{a_n\}$的有限项,因此$A$不可能是$\{a_n\}$的极限点,这与$A$是$\{a_n\}$的上极限矛盾.因此$A\leqslant B.$
	
	类似可证下极限的保序性.$\hfill\blacksquare$
\end{proof}
\begin{theorem}[上极限的次可加性与下极限的超可加性]
	设数列$\{a_n\}$,$\{b_n\}$,则
	\begin{enumerate}
		\item $\liminf\limits_{n \to \infty}a_n+\liminf\limits_{n \to \infty}b_n\leqslant\liminf\limits_{n \to \infty}(a_n+b_n)\leqslant\liminf\limits_{n \to \infty}a_n+\limsup\limits_{n \to \infty}b_n$;
		\item $\liminf\limits_{n \to \infty}a_n+\limsup\limits_{n \to \infty}b_n\leqslant\limsup\limits_{n \to \infty}(a_n+b_n)\leqslant\limsup\limits_{n \to \infty}a_n+\limsup\limits_{n \to \infty}a_n$.
	\end{enumerate}
\end{theorem}
\begin{remark}
	设定义在$A$上的映射$f$,$\forall a,b\in A$.
	\begin{enumerate}
		\item 若$f(a+b)=f(a)+f(b)$,则称$f$满足{\heiti 可加性}(additivity);
		\item 若$f(a+b)\geqslant f(a)+f(b)$,则称$f$满足{\heiti 超可加性}(superadditivity);
		\item 若$f(a+b)\leqslant f(a)+f(b)$,则称$f$满足{\heiti 次可加性}(subadditivity).
	\end{enumerate}
\end{remark}
\begin{proof}
	只需证明第1个不等式.对任意$l\geqslant n$,有$\inf\limits_{k\geqslant n}a_k\leqslant a_l$,$\inf\limits_{k\geqslant n}b_k\leqslant b_l$,由$l$的任意性,有
	$$\inf\limits_{k\geqslant n}a_k+\inf\limits_{k\geqslant n}b_k\leqslant \inf\limits_{k\geqslant n}(a_k+b_k).$$
	则$$\inf\limits_{k\geqslant n}a_k\geqslant\inf\limits_{k\geqslant n}(a_k+b_k)+\inf\limits_{k\geqslant n}(-b_k)=\inf\limits_{k\geqslant n}(a_k+b_k)-\sup\limits_{k\geqslant n}b_k.$$
	即$$\inf\limits_{k\geqslant n}(a_k+b_k)\leqslant\inf\limits_{k\geqslant n}a_k+\sup\limits_{k\geqslant n}b_k.$$
	当$n\to\infty$时,有
	$$\liminf\limits_{n \to \infty}a_k+\liminf\limits_{n \to \infty}b_k\leqslant\liminf\limits_{n \to \infty}(a_k+b_k)\leqslant\liminf\limits_{n \to \infty}a_k+\limsup\limits_{n \to \infty}b_k.$$
	第2个不等式类似可证.$\hfill\blacksquare$
\end{proof}
\begin{remark}
	特别地,当$\lim\limits_{n\to\infty}a_n=a$时,有
	\begin{enumerate}
		\item $\liminf\limits_{n\to\infty}(a_n+b_n)=a+\liminf\limits_{n \to \infty}b_n$;
		\item $\limsup\limits_{n\to\infty}(a_n+b_n)=a+\limsup\limits_{n \to \infty}b_n$.
	\end{enumerate}
	利用上述定理即可证明,不再赘述.
\end{remark}

与上极限的次可加性和下极限的超可加性类似,我们还有以下结论.
\begin{theorem}
	设非负数列$\{a_n\}$,$\{b_n\}$,则
	\begin{enumerate}
		\item $\left(\liminf\limits_{n \to \infty}a_n\right)\left(\liminf\limits_{n \to \infty}b_n\right)\leqslant\liminf\limits_{n \to \infty}\left(a_nb_n\right)\leqslant\left(\liminf\limits_{n \to \infty}a_n\right)\left(\limsup\limits_{n \to \infty}b_n\right)$;
		\item $\left(\liminf\limits_{n \to \infty}a_n\right)\left(\limsup\limits_{n \to \infty}b_n\right)\leqslant\liminf\limits_{n \to \infty}\left(a_nb_n\right)\leqslant\left(\limsup\limits_{n \to \infty}a_n\right)\left(\limsup\limits_{n \to \infty}b_n\right)$.
	\end{enumerate}
\end{theorem}
\begin{proof}
	只需证明第1个不等式.对任意$l\geqslant n$,有$\inf\limits_{k\geqslant n}a_k\leqslant a_l$,$\inf\limits_{k\geqslant n}b_k\leqslant b_l$,由于$\{a_n\},\{b_n\}$非负和$l$的任意性,有
	$$\left(\inf\limits_{k\geqslant n}a_k\right)\left(\inf\limits_{k\geqslant n}b_k\right)\leqslant\inf\limits_{k\geqslant n}(a_kb_k).$$
	对于任意$\varepsilon>0$,存在$m\geqslant n$,使得$a_m<\varepsilon+\inf\limits_{k\geqslant n}a_k$,由于$b_m\leqslant\sup\limits_{k\geqslant n}b_k$,且$\{a_n\},\{b_k\}$非负,因此
	$$\left(\varepsilon+\inf\limits_{k\geqslant n}a_k\right)\left(\inf\limits_{k\geqslant n}b_k\right)> a_mb_m\geqslant \inf\limits_{k\geqslant n}(a_kb_k).$$
	由$\varepsilon$的任意性,有
	$$\left(\varepsilon+\inf\limits_{k\geqslant n}a_k\right)\left(\inf\limits_{k\geqslant n}b_k\right)\geqslant \inf\limits_{k\geqslant n}(a_kb_k).$$
	于是得到不等式
	$$\left(\inf\limits_{k\geqslant n}a_k\right)\left(\inf\limits_{k\geqslant n}b_k\right)\leqslant\inf\limits_{k\geqslant n}(a_kb_k)\leqslant\left(\varepsilon+\inf\limits_{k\geqslant n}a_k\right)\left(\inf\limits_{k\geqslant n}b_k\right).$$
	$n\to\infty$时即得要证结论.第2个不等式类似可证.$\hfill\blacksquare$
\end{proof}
\begin{remark}
	特别地,当$\lim\limits_{n\to\infty}a_n=a$时,有
	\begin{enumerate}
		\item $\liminf\limits_{n\to\infty}(a_nb_n)=a\liminf\limits_{n \to \infty}b_n$;
		\item $\limsup\limits_{n\to\infty}(a_nb_n)=a\limsup\limits_{n \to \infty}b_n$.
	\end{enumerate}
\end{remark}









