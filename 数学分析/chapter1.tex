\chapter{实数理论}

数学分析研究的基本对象是定义在实数集上的函数.
\section{实数的定义}%或许可以改成“有理数的扩充”
\subsection{建立实数的原则}
\begin{definition}[数域]\label{def:field}
	设$P$是由一些复数组成的集合,其中包括0和1,如果$P$中任意两个数的和、差、积、商(除数不为0)仍为$P$中的数,则称$P$为一个{\heiti 数域}.
\end{definition}
由定义\ref{def:field}可知,数域对加、减、乘、除(除数不为0)四则运算具有封闭性,即结果仍在数域本身中。例如,全体有理数所构成的集合$\mathbb{Q}$是一个数域,称为有理数域.此外,常见的数域还有复数域$\mathbb{C}$,读者可自行验证.
\begin{example}
	证明全体有理数所构成的集合$\mathbb{Q}$是一个数域.
	\begin{proof}
		对$\forall a=\frac{p}{q},\ b=\frac{s}{t}$,其中$p,q,s,t\in \mathbb{Z}$且$st\neq0$,则由有理数的定义知,$a,b\in \mathbb{Q}$
		
		显然$0,1\in\mathbb{Q}$,
		
		$a+b=\frac{pt+qs}{qt}\in \mathbb{Q}$
		
		$a-b=\frac{pt-qs}{qt}\in \mathbb{Q}$
		
		$a\cdot b=\frac{ps}{qt}\in \mathbb{Q}$
		
		$\frac{a}{b}=\frac{pt}{qs}\in \mathbb{Q}$
		
		故$\mathbb{Q}$是一个数域.$\hfill\blacksquare$
	\end{proof}
\end{example}
\begin{definition}[阿基米德有序域]
	集合$F$构成一个{\heiti 阿基米德有序域},是说它满足以下三个条件:
	\begin{enumerate}
		\item $F$是域\qquad 在$F$中定义了加法“$+$”和乘法“$\cdot$”两种运算,使得对于$F$中任意元素$a$,$b$,$c$成立:\par
		加法的结合律:\ $(a+b)+c=a+(b+c)$;\par
		加法的交换律:\ $a+b=b+a$;\par
		乘法的结合律:\
		$(a\cdot b)\cdot c=a\cdot (b\cdot c)$;\par
		乘法的交换律:\ $a\cdot b=b\cdot a$;\par
		乘法关于加法的分配律:
		$(a+b)\cdot c=a\cdot c+b\cdot c$;\par
		$F$中对加法存在{\heiti 零元素}和{\heiti 负元素}(即存在加法的逆运算减法);
		
		$F$中对乘法存在{\heiti 单位元素}和{\heiti 逆元素}(即存在乘法的逆运算除法).
		\item $F$是有序域\qquad
		在$F$中定义了{\heiti 序}关系“\textless”具有如下{\heiti 全序}的性质:\par
		传递性:$\forall a,b,c \in F$,若$a<b$,$b<c$,则$a<c$;\par
		三歧性:$\forall a,b \in F$,$a>b$,$a<b$,$a=b$三者必居其一,也只居其一;\par
		加法保序性:$\forall a,b,c \in F$,若$a<b$,则$a+c<b+c$;\par
		乘法保序性:$\forall a,b,c \in F$,若$a<b$,则$ac<bc$\ (c>0).
		\item $F$中元素满足阿基米德性\qquad 对$F$中两个正元素$a$,$b$,必存在自然数$n$,使得$na>b$.
	\end{enumerate}
\end{definition}
\subsection{Dedekind分割}
设$A/B$是有理数域$\mathbb{Q}$上的一个分割,即把$\mathbb{Q}$中的元素分为$A$、$B$两个集合,使得$\forall a \in A,\ b \in B$有$a<b$成立.则从逻辑上分为下列四种情况:
\begin{enumerate}
	\item $A$有最大值$a_0$,$B$有最小值$b_0$;
	\item $A$有最大值$a_0$,$B$无最小值;
	\item $A$无最大值,$B$有最小值$b_0$;
	\item $A$无最大值,$B$无最小值.
\end{enumerate}	

而对于第1种情况,取$\frac{a_0+b_0}{2}\in \mathbb{Q}$,它将不属于集合$A$、$B$中的任何一个.\par 
对于第4种情况,说明分割到的数不在$\mathbb{Q}$内(我们将这种划分称为{\heiti 无端划分}),因此这是我们通过“切割”构造出来的“新数”.将所有这样切割出来的新数与原来的$\mathbb{Q}$取并集,并设新的集合为$\mathbb{R}$.将$\mathbb{R}$中的元素再次进行分割,设$A'/B'$为$\mathbb{R}$上的一个分割,则$\forall a \in A',\ b \in B'$有$a<b$成立.同理,在排除上述的第1种情况后,对$\mathbb{R}$的分割分为以下三种情况:
\begin{enumerate}
	\item $A'$有最大值$a_0$,$B'$无最小值;
	\item $A'$无最大值,$B'$有最小值$b_0$;
	\item $A'$无最大值,$B'$无最小值.
\end{enumerate}	

由此,我们给出Dedekind定理.
\begin{theorem}[Dedekind定理]
	设$A'$、$B'$是$\mathbb{R}$的两个子集,且满足:
	\begin{enumerate}
		\item $A'$和$B'$均不为空集;
		\item $A'\cup B'=\mathbb{R}$;
		\item $\forall a\in A',\ b\in B'$有$a<b$.
	\end{enumerate}
	则或者$A'$有最大元素,或者$B'$有最小元素.
\end{theorem}
Dedikind定理指出,我们在上述对$\mathbb{R}$进行分割时,只会出现第1种或第2种情况,而不可能$A'$无最大值且$B'$也无最小值。
\begin{proof}
	设$A$是$A'$中所有有理数的集合,设$B$是$B'$中所有有理数的集合,则$A/B$有三种情况:
	\begin{enumerate}
		\item $A$有最大值$a_0$,$B$无最小值;
		\item $A$无最大值,$B$有最小值$b_0$;
		\item $A$无最大值,$B$无最小值;
	\end{enumerate}	
	
	对于第1种情况,设$a'\in A'$,使得$a'>a_0$,则在$(a_0,a')$之间必定存在有理数$a$,这与$A$有最大值$a_0$矛盾.因此$a_0$也是$A'$的最大值;
	
	同理,对于第2种情况,我们可以得到$b_0$也是$B'$的最小值;
	
	对于第3种情况,设$c$是由$A/B$得到的无理数,则$a_0<c<b_0$,可知$c$或者是$A'$中的元素,或者是$B'$中的元素.不妨设$c\in A'$,可以证明c为$A'$的最大元素,因为如果存在$a$是$A$的最大元素,那么区间$(c,a)$之间必定存在有理数大于$a_0$,这与$A$的最大值是$a_0$矛盾.
	
	以上我们就证明了或者$A'$有最大元素,或者$B'$有最小元素.$\hfill\blacksquare$
\end{proof}
\subsection{实数的公理化定义}
由前两节的理论基础,我们可以定义{\heiti 实数}构成一个阿基米德有序域,且满足Dedekind定理.实数分为有理数和无理数,其中无理数由有理数的无端划分产生.
\section{确界原理}
确界原理是极限理论的基础.
\subsection{确界的定义}
\begin{definition}
	设$S$为$\mathbb{R}$中的一个数集,若存在数$M(L)$,使得对一切$x\in S$,都有$x\leqslant M(x\geqslant L)$,则称$S$为{\heiti 有上界(下界)的数集},数$M(L)$称为$S$的一个{\heiti 上界(下界)}.
\end{definition}
\begin{definition}[上确界]
	设$S$是$\mathbb{R}$中的一个数集,若数$\eta$满足:
	
	(i)$\eta$是$S$的上界;
	
	(ii)$\forall \alpha < \eta,\ \exists x_0\in S,\ s.t.x_0>\alpha$,即$\eta$又是$S$的最小上界,\\
	则称$\eta$为数集$S$的{\heiti 上确界},记作$\eta=\sup S$
\end{definition}
\begin{definition}[下确界]
	设$S$是$\mathbb{R}$中的一个数集,若数$\xi$满足:
	
	(i)$\xi$是$S$的下界;
	
	(ii)$\forall \beta > \xi,\ \exists x_0\in S,\ s.t.x_0<\beta$,即$\xi$又是$S$的最小下界,\\
	则称$\xi$为数集$S$的{\heiti 下确界},记作$\xi=\inf S$
\end{definition}
上(下)确界也可以由$\varepsilon$语言定义.
\begin{definition}[上确界的$\varepsilon$语言定义]
	设$S$为$\mathbb{R}$中的一个数集,若S的一个上界$M$,$\forall \varepsilon>0$,$\exists a\in S$,s.t.\ $a>M-\varepsilon$,则数$M$称为$S$的一个{\heiti 上确界}.
\end{definition}
\begin{definition}[下确界的$\varepsilon$语言定义]
	设$S$为$\mathbb{R}$中的一个数集,若S的一个下界$L$,$\forall \varepsilon>0$,$\exists b\in S$,s.t.\ $b<L+\varepsilon$,则数$L$称为$S$的一个{\heiti 下确界}.
\end{definition}
上确界和下确界统称为确界.
\subsection{确界原理及其证明}
\begin{theorem}[确界原理]
	设$S$为$\mathbb{R}$中的一个数集,若$S$有上界,则必有上确界;若$S$有下界,则必有下确界.
\end{theorem}
\begin{proof}
	设数集$S$有上界,下面证明$S$有上确界.
	
	设$B$是数集$S$所有上界组成的集合,记$A=\mathbb{R}\textbackslash B$.若$B$有最小元素,则$B$的最小元素$b_0$即为$S$的上确界.
	
	设$x\in A$,$x$不是$S$的上界,则$\exists t\in S,\ s.t.\ x<t$,取$x'=\frac{x+t}{2}$,则$x'>x$,因此对$A$中任何一个元素$x$,都有$x'>x$存在,即$A$没有最大元素.由Dedekind定理,$B$一定有最小元素,即$S$必有上确界.$\hfill\blacksquare$
\end{proof}
\begin{example}
	利用确界原理证明Dedekind定理,即证明二者的等价关系.
\end{example}
\begin{proof}
	设$\mathbb{R}$上任意一个Dedekind分割为$A/B$,易知$B$中的每个元素都是$A$的一个上界.由确界原理,$A$一定有上确界.设$m=\sup A$,
	
	若$m\in B$,则$A$中无最大元素,假设$B$中无最小元素,则$\exists m'\in B\ s.t.\ m'<m$,由于$m=\sup A$,推出$m'\in A$,这与$m'\in B$矛盾.故$B$中有最小元素.
	
	若$m\in A$,则$A$中有最大元素$m$,假设$B$中有最小元素$n$,则$\frac{m+n}{2}$不属于$A$和$B$,这与$A\cup B=\mathbb{R}$是矛盾的,故$B$中无最小元素.
	
	于是我们就证明了Dedekind定理.$\hfill\blacksquare$
\end{proof}
\section{实数的完备性}
本节内容是基于第二章中数列极限的前提下展开的,建议在学完第二章后进行学习.数列极限的定义与基本性质在此不再赘述.

\subsection{关于实数集完备性的基本定理}
前面我们已经学习了Dedekind分割与确界原理,从不同的角度反映了实数集的特性,通常称为{\heiti 实数的完备性}或{\heiti 实数的连续性}公理.下面我们将介绍另外的几个实数的完备性公理.

\begin{theorem}[单调有界收敛定理]
	在实数系中,有界的单调数列必有极限.
\end{theorem}
\begin{proof}
	不妨设$\left\{a_n\right\}$为有上界的递增数列,由确界原理,$\left\{a_n\right\}$必有上确界,设$a=\sup \left\{a_n\right\}$,根据上确界的定义,
	
	$\forall \varepsilon>0,\ \exists a_N \ s.t.\ a_N>a-\varepsilon$
	
	由$\left\{a_n\right\}$的递增性,当$n\geqslant N$时,有
	$$a-\varepsilon<a_N\leqslant a_n$$
	
	又因为$$a_n\leqslant a<a+\varepsilon$$
	
	故$$a-\varepsilon<a_n<a+\varepsilon$$
	
	即$${\lim_{n \to +\infty}a_n}=a$$
	$\hfill\blacksquare$
\end{proof}

\begin{theorem}[致密性定理]
	任何有界数列必定有收敛的子列.
\end{theorem}
要证明此定理,可以先证明以下引理.
\begin{lemma}\label{zilie}
	任何数列都存在单调子列.
\end{lemma}
\begin{proof}
	设数列为$\left\{a_n\right\}$,下面分两种情况讨论:
	\begin{enumerate}
		\item 若$\forall k\in \mathbb{Z}_+$,$\left\{a_{k+n}\right\}$都有最大项,记$\left\{a_{1+n}\right\}$的最大项为$a_{n_1}$,则$a_{{n_1}+n}$也有最大项,记作$a_{n_2}$,显然有$a_{n_1}\geqslant a_{n_2}$,同理,有$$a_{n_2}\geqslant a_{n_3}$$
		$$.........$$
		由此得到一个单调递减的子列$\left\{a_{n_k}\right\}$
		\item 若至少存在一个正整数$k$,使得$\left\{a_{k+n}\right\}$没有最大项,先取$n_1=k+1$,总存在$a_{n_1}$后面的项$a_{n_2}$($n_2>n_1$)使得$$a_{n_2}>a_{n_1}$$,同理,总存在$a_{n_2}$后面的项$a_{n_3}$($n_3>n_2$)使得$$a_{n_3}>a_{n_2}$$
		$$.........$$
		由此得到一个严格递增的子列$\left\{a_{n_k}\right\}$
	\end{enumerate}
	
	综上,命题得证.$\hfill\blacksquare$
\end{proof}
下面是对致密性定理的证明:
\begin{proof}
	设数列$\left\{a_n\right\}$有界,由引理\ref{zilie},数列$\left\{a_n\right\}$存在单调且有界的子列,由单调有界收敛定理得出该子列是收敛的.$\hfill\blacksquare$
\end{proof}
\begin{theorem}[柯西(Cauchy)收敛准则]
	数列$\left\{a_n\right\}$收敛的充要条件是:\par 
	$\forall \varepsilon>0,\ \exists N\in \mathbb{Z}_+,\ s.t.\ n,\ m>N$时,有
	$$\lvert a_n - a_m \rvert<\varepsilon$$
\end{theorem}
单调有界只是数列收敛的充分条件,而柯西收敛准则给出了数列收敛的充要条件.
\begin{proof}
	{\heiti 必要性}\qquad 设$\lim\limits_{n \to +\infty}a_n=A$,则$\forall \varepsilon>0,\ \exists N\in \mathbb{Z}_+\ s.t.\ n,\ m>N$时,有$$\lvert a_n-A\rvert <\frac{\varepsilon}{2},\ \lvert a_n-A\rvert <\frac{\varepsilon}{2}$$
	
	因而$$\lvert a_n-a_m\rvert \leqslant \lvert a_n-A\rvert + \lvert a_m-A\rvert=\varepsilon$$
	
	{\heiti 充分性}\qquad 先证明该数列必定有界.取$\varepsilon=1$,因为$\left\{a_n\right\}$满足柯西收敛准则的条件,所以$\exists N_0,\ \forall n>N_0$,有
	$$\lvert a_n-a_{N_0+1}\rvert <1$$
	
	取$M=\max\left\{\lvert a_1 \rvert,\ \lvert a_2 \rvert,\ \cdot\cdot\cdot\,\ \lvert a_{N_0} \rvert,\ \lvert a_{N_0+1} \rvert+1\right\}$,则对一切$n$,成立$$\lvert a_n \rvert\leqslant M$$
	
	由致密性原理,在$\left\{a_n\right\}$中必有收敛子列$$\lim_{k \to +\infty}a_{n_k}=\xi$$
	
	由条件,$\forall \varepsilon>0,\ \exists N$,当$n,\ m>N$时,有$$\lvert a_n-a_m\rvert <\frac{\varepsilon}{2}$$
	
	在上式中取$a_m=a_{n_k}$,其中$k$充分大,满足$n_k>N$,并且令$k \to \infty$,于是得到
	$$\lvert a_n-\xi \rvert\leqslant \frac{\varepsilon}{2}< \varepsilon $$
	
	即数列$\left\{a_n\right\}$收敛.$\hfill\blacksquare$
\end{proof}
\begin{definition}[闭区间套]
	设闭区间列$\left\{\left[a_n,b_n\right]\right\}$具有如下性质:
	\begin{enumerate}
		\item $\left[a_n,b_n\right]\supset \left[a_n+1,b_n+1\right],\ n=1,2,\cdots$;
		\item $\lim\limits_{n \to +\infty}(b_n-a_n)=0$.
	\end{enumerate}
	则称$\left\{\left[a_n,b_n\right]\right\}$为{\heiti 闭区间套},或简称{\heiti 区间套}.
\end{definition}

由性质1,构成闭区间套的闭区间列是前一个套着后一个的,即各闭区间端点满足如下不等式:
\begin{equation}\label{chuan}
	a_1\leqslant a_2\leqslant \cdots\leqslant a_n\leqslant\cdots\leqslant b_n\leqslant\cdots\leqslant b_2\leqslant b_1.
\end{equation}
\begin{theorem}[闭区间套定理]
	若$\left\{\left[a_n,b_n\right]\right\}$是一个闭区间套,则在实数系中存在唯一的一点$\xi$,使得$\xi\in\left[a_n,b_n\right],\ n=1,2,\cdots$,即$$a_n\leqslant\xi\leqslant b_n,\ n=1,2,\cdots.$$
\end{theorem}
\begin{proof}
	由式\ref{chuan}可以看出,数列$\left\{a_n\right\}$是递增数列且有界,$\left\{b_n\right\}$是递减数列且有界,由单调有界收敛定理,可知$\left\{a_n\right\}$和$\left\{b_n\right\}$都收敛.设$\lim\limits_{n\to \infty}a_n=\xi$,由闭区间套的第2条性质,得$\lim\limits_{n\to \infty}b_n=\xi.$
	
	$\left\{a_n\right\}$是递增数列,有$a_n\leqslant\xi,\ n=1,2,\cdots;$
	
	$\left\{b_n\right\}$是递减数列,有$b_n\geqslant\xi,\ n=1,2,\cdots.$\\
	所以有$a_n\leqslant\xi\leqslant b_n,\ n=1,2,\cdots.$\\
	下面证明$\xi$的唯一性:\\
	假设存在$\xi'$满足$a_n\leqslant\xi'\leqslant b_n,\ n=1,2,\cdots.$,则\\
	$$\lvert\xi'-\xi\rvert\leqslant b_n-a_n,\ n=1,2,\cdots,$$\\
	由闭区间套的第2条性质,有
	$$\lvert\xi'-\xi\rvert\leqslant \lim\limits_{n\to\infty}(b_n-a_n)=0,\ n=1,2,\cdots,$$\\
	故$\xi'=\xi$.$\hfill\blacksquare$
\end{proof}
\begin{definition}
	设$S$为数轴上的点集,$H$为开区间的集合(即$H$的每一个元素都是形如$(\alpha,\beta)$的开区间).若$S$中的任何一点都含在$H$中至少一个开区间内,则称$H$为$S$的一个{\heiti 开覆盖},或称$H$覆盖$S$.若$H$中开区间的个数是无限(有限)的,则称$H$为$S$的一个{\heiti 无限开覆盖(有限开覆盖)}.若存在$S$的开覆盖$H'\subseteq H$,则称$H'$是$H$的{\heiti 子覆盖},特别地,当$H'$中含有的开区间的个数为有限个时,称$H'$为$H$的{\heiti 有限子覆盖}.
\end{definition}
\begin{theorem}[Heine-Borel有限覆盖定理]
	设$H$是闭区间$\left[a,b\right]$的一个(无限)开覆盖,则从$H$中能选出有限个开区间来覆盖$\left[a,b\right]$.
	
	即:有限闭区间的任一开覆盖都存在一个有限子覆盖.
\end{theorem}
\begin{proof}
	设$H$是闭区间$\left[a,b\right]$的一个开覆盖,定义集合
	$$S=\left\{x|x\in \left(a,b \right],\ \mbox{且}\left[a,x\right]\mbox{存在开覆盖}H\mbox{的一个有限子覆盖} \right\}.$$
	
	因为$H$是$\left[a,b\right]$的一个开覆盖,所以存在一个区间$I_0\in H$使得$a\in I_0$,则存在$x_0\in I_0$满足$x_0>a$,所以$S\neq\varnothing$.显然$b$是$S$的一个上界,由确界原理,$S$一定有上确界.设$M=\sup S\leqslant b$,下面证明$M=b$:
	
	反证法\qquad 假设$M<b$,则$M\in \left(a,b \right]$,$\left[a,M\right]$存在$H$的一个有限子覆盖.假设开区间$I_1$包含$M$,则存在$\delta >0$使得$(M-\delta,M+\delta)\subseteq I_1$,因为$M$是$S$的上确界,所以$M-\delta\in S$,记$\left[a,M-\delta\right]$的有限开覆盖为$H'$,则$\left[a,M+\delta\right]$也有有限开覆盖$H'\cup I_1$,得$M+\delta\in S$这与$M$是$S$的上确界矛盾.所以$M=b$,即$\left[a,b\right]$的开覆盖$H$存在一个有限子覆盖.$\hfill\blacksquare$
\end{proof}
\begin{remark}
	法国数学家Borel于1895年第一次陈述并证明了现代形式的Heine-Borel定理.此定理只对有限闭区间成立,而对开区间则不一定成立.例如开区间集合$$\left\{(\frac{1}{n+1},1)\right\},\ (n=1,2,\cdots)$$构成了开区间$(0,1)$的开覆盖,但不能从中选出有限个开区间覆盖住$(0,1)$.
\end{remark}
从上面的讨论我们发现,如果从数轴上取下一段“紧致无缝”的集合(含端点),那么就可以从它的任意开覆盖中取出一个有限子覆盖,否则就不行.这表明我们找到了一个刻画实数集完备性的新方法,我们形象地将这个性质称为“紧致性”.
\begin{definition}[紧致集]
	设集合$E\in\mathbb{R}$,若集合$E$的任一开覆盖都存在一个有限子覆盖,则称$E$为$\mathbb{R}$上的一个{\heiti 紧致集}.
\end{definition}
\begin{remark}
	紧致集也称紧集,是一个重要的拓扑概念.
\end{remark}
\begin{remark}
	以后我们会将以上条件称为"Heine-Borel条件".
\end{remark}
我们可以重新表述Heine-Borel有限覆盖定理.
\begin{theorem}[Heine-Borel有限覆盖定理]
	$\mathbb{R}$中的任一有限闭区间都是紧致集.
\end{theorem}
\begin{definition}[邻域]
	设$a\in\mathbb{R},\ \delta>0$,将满足$\lvert x-a\rvert<\delta$的全体$x$的集合称为{\heiti $a$的$\delta$邻域},记作$U(a,\delta)$.将满足$0<\lvert x-a\rvert<\delta$的全体$x$的集合称为{\heiti $a$的$\delta$去心邻域},记作$\mathring{U}(a,\delta)$.
\end{definition}
显然,邻域与去心邻域的区别在于去心邻域不包含中心点$a$.
\begin{definition}[聚点]
	设$S$是数轴上的点集,$\xi$是一个定点(可以在$S$中也可以不在$S$中),若$\xi$的任一邻域中都含有$S$中无穷多个点,则称$\xi$为$S$的一个聚点.
\end{definition}
聚点的另一定义如下:
\begin{definition}
	若存在各项互异的收敛数列$\left\{x_n\right\}\subset S$,则其极限$\lim\limits_{n\to \infty}x_n=\xi$是$S$的一个聚点.
\end{definition}
\begin{theorem}[Weierstrass聚点定理]
	实轴上任一有界无限点集$S$至少有一个聚点.
\end{theorem}
由聚点的等价定义,该定理也可叙述为:{\heiti 有界数列必有收敛子列},即致密性定理.
\subsection{实数集完备性定理的等价关系}
通过前面的学习,我们共有以下8个基本定理来叙述实数的完备性:
\begin{enumerate}
	\item Dedekind定理;
	\item 确界原理;
	\item 单调有界收敛定理;
	\item 致密性定理;
	\item 柯西收敛准则;
	\item 闭区间套定理;
	\item Heine-Borel有限覆盖定理;
	\item Weierstrass聚点定理.
\end{enumerate}
可以证明,这8个基本定理都是等价的.(证明会在后续修正时给出)