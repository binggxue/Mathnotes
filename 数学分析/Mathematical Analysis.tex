\documentclass[lang=cn,10pt]{elegantbook}

\usepackage{amsfonts,amssymb,mathtools,derivative,fixdif,hyperref}

\title{数学分析讲义}
\subtitle{Mathematical Analysis}

\author{薛冰}
\date{\today}

\extrainfo{衣带渐宽终不悔,为伊消得人憔悴.}

%table of content depth,目录显示的深度
\setcounter{tocdepth}{2} 

% 修改封面的颜色带
\definecolor{customcolor}{RGB}{32,178,170}
\colorlet{coverlinecolor}{customcolor}

\cover{Cover.png}
%\addbibresource[location=local]{reference.bib} % 参考文献,不要删除
\begin{document}
	\maketitle
	
\chapter*{序}
数学分析是数学学习的基础课程之一。这份讲义是基于华东师范大学数学系编写的《数学分析》(第五版上下册)编写而成。我们依次介绍了实数理论、极限与连续、一元函数微分学、一元函数积分学、无穷级数、多元函数微分学、含参量积分、曲线曲面积分、重积分等内容。对书上的定理作了总结与补充,以整体的观点、层层递进的逻辑关系进行叙述。读者可能会在初次阅读时感到晦涩、难懂,这是因为此讲义的编写顺序并不符合我们的认知顺序。事实上,在数学史中,我们是先创造了微积分,后来因数学大厦的地基不够充实,从而发展了连续性理论、极限理论以及实数理论,这与我们的叙述方向刚好是相反的。由此可见,在历史的发展中,人们对数学的认知是由模糊与感觉转变成越来越精细的定义与逻辑的推理。

在数学的学习中,我们要找出各个分支的相同点与不同点。例如,对于上极限和下极限的知识,我们有集合的上极限和下极限,数列的上极限和下极限,函数的上极限和下极限,它们都是普遍存在的,并通过夹逼的方式来定义具体的极限。我们还可以从这里推广、延伸,例如利用定义Darboux上和与Darboux下和的方式来定义Darboux积分,在测度论中Lebesgue定义了外测度和内测度来定义Lebesgue测度。它们的思想都是相通的,值得我们细细体会。又例如,在反常积分部分,我们分别研究了无穷积分和瑕积分的敛散性,实际上,我们都是从定义、非负函数的判别法则和一般函数的判别法则出发的,给出了Cauchy准则、比较原则、Dirichlet判别法和Abel判别法。事实上,在数项级数、函数项级数敛散性的研究中也有类似的判别法,我们可以整体地看待这些法则。此外,还有很多很多相通的实例不胜枚举。
\newpage
	
\tableofcontents
\mainmatter
	
\chapter*{预备知识}
\addcontentsline{toc}{chapter}{预备知识}
此章节中,我们给出有关命题逻辑与朴素集合论的相关知识,这是在描述现代数学理论中必不可少的.
\section{命题}
\subsection{命题的概念}
\begin{definition}[命题]
	可以判断真假的陈述句称为{\heiti 命题}(proposition).其中判断为真的语句叫{\heiti 真命题}(true proposition),判断为假的语句叫{\heiti 假命题}(false proposition).
\end{definition}
一般地,命题可以写作“若$p$,则$q$”的形式.我们称$p$为命题的条件,$q$为命题的结论.
\subsection{逻辑联结词}
逻辑联结词分为合取词(与)、析取词(或)和否定词(非).
\begin{definition}[与]
	对于两个条件$p$,$q$,我们将$p$,$q$同时成立的称为$p${\heiti 与}$q$或$p${\heiti 且}$q$,记作$p\wedge q$.
\end{definition}

\begin{definition}[或]
	对于两个条件$p$,$q$,我们将$p$,$q$中至少有一个成立的称为$p${\heiti 或}$q$,记作$p\vee q$.
\end{definition}

\begin{definition}[非]
	对于条件$p$,我们将不满足条件$p$的条件称为{\heiti 非}$p$,记作$\neg p$.
\end{definition}

我们将不含逻辑联结词的命题称为{\heiti 简单命题},将由简单命题和逻辑联结词组成的命题称为{\heiti 复合命题}.
\subsection{命题变换与真值判断}
设命题$P:\text{若}p,\text{则}q$.

我们定义$P$的逆命题、否命题、逆否命题如下:

逆命题:若$q$,则$p$;

否命题:若$\neg p$,则$\neg q$;

逆否命题:若$\neg q$,则$\neg p$.

我们称变换前的命题$P$为原命题,那么我们有以下结论:

{\heiti 原命题为真,逆命题和否命题不一定为真,但逆否命题一定为真.}
\subsection{蕴含关系与条件语句}
设命题$P:\text{若}p,\text{则}q$.

我们可以将其记作$p\rightarrow q$,称作“$p$蕴含$q$”.我们称$p$是$q$的{\heiti 充分条件},即在$p$成立的前提下,可以充分地说明$q$是成立的;称$q$是$p$的{\heiti 必要条件},即$q$的成立不一定能说明$p$是成立的,$p$的成立可能还受其他条件制约,但$q$成立是必要的,如果$q$不成立,那么$p$也是不成立的(这就是$P$的逆否命题,它与$P$是等价的).

由上面的分析我们得出,若$p\rightarrow q$,则$\neg q\rightarrow \neg p$.同样地,若$\neg q\rightarrow \neg p$,则$p\rightarrow q$.我们可以看出$P$与$P$的逆否命题是等价的,记作
$$P\iff P\text{的逆否命题}.$$
我们可以看出,若$p\iff q$,则$p\rightarrow q$且$q\rightarrow p$,$p$既是$q$的充分条件又是$q$的必要条件,我们称$p$为$q$的{\heiti 充分必要条件},简称为{\heiti 充要条件}.根据定义,互为充要条件的两个条件是等价的.
\section{逻辑量词}
逻辑量词分为全称量词和存在量词.
\begin{definition}[全称量词]
	在语句中含有短语“所有”、“每一个”、“任何一个”、“任意一个”“一切”等都是在指定范围内,表示整体或全部的含义,这样的词叫作{\heiti 全称量词}.记作“$\forall$”,读作“任意”.
\end{definition}
\begin{definition}[存在量词]
	短语“有些”、“至少有一个”、“有一个”、“存在”等都有表示个别或一部分的含义,这样的词叫作{\heiti 存在量词}.记作“$\exists$”,读作“存在”.
\end{definition}
命题涉及“任意”和“存在”这两个逻辑量词时,它们的否定说法是把“任意”改成“存在”,把“存在”改成“任意”.
\section{集合的基本概念}
集合是数学中所谓原始概念之一,不能用别的概念加以定义.目前我们只需掌握其朴素的说法.对于那些对集合的理论有进一步需求的读者,例如打算研究集合论本身或者打算研究数理逻辑的读者,建议去研读有关公理集合论的专著.

我们把在一定特征范围内的个体构成的集体称为{\heiti 集合}(set).集合也常称为{\heiti 集}、{\heiti 族}或{\heiti 类}.组成集合的每个个体都是集合的{\heiti 元素}.通常,我们用大写字母$A,B,C,\cdots$来表示集合,小写字母$a,b,c,\cdots$来表示集合中的元素.
\subsection{集合的表示}
常用集合的表示方法有{\heiti 列举法}和{\heiti 描述法}.顾名思义,列举法就是将集合中的元素一一列举出来,例如
$$A=\{a,b,c,\cdots\}.$$
描述法是将元素的共同特征归纳在一起,例如
$$A=\{x|x\text{满足条件}P\}.$$
其中竖线“$|$”也可用冒号“$:$”、分号“$;$”表示.集合中的元素具有确定性、互异性、无序性.

习惯上,我们用空心大写字母来表示数集.例如
$\mathbb{N}$表示{\heiti 自然数集},$\mathbb{Z}$表示{\heiti 整数集},$\mathbb{N}^+$或$\mathbb{R}^+$表示{\heiti 正整数集},$\mathbb{Q}$表示有理数集,$\mathbb{R}$表示{\heiti 实数集},$\mathbb{C}$表示{\heiti 复数集}.

设$A$是一个集合,若$a$是$A$的元素,记作$a\in A$,读作{\heiti $a$属于$A$};如果$a$不是$A$的元素,记作$a\notin A$,读作{\heiti $a$不属于$A$}.
\subsection{集合的包含与相等}
\begin{definition}[包含与子集]
	对任意$x\in A$,都有$x\in B$,则称{\heiti $A$包含于$B$}或{\heiti $B$包含$A$},记作$A\subset B$或$B\supset A$.称$A$是$B$的{\heiti 子集}(subset),若存在$y\in B$,使得$y\notin A$,则称$A$为$B$的{\heiti 真子集}(proper subset).记作$A\subsetneqq B$或$B\supsetneqq A$.
\end{definition}
\begin{definition}[集合的相等]
	若集合$A$中的元素和集合$B$中的元素完全相同,则称$A$和$B$相等,记作$A=B$.
\end{definition}
不难看出,$A=B$的充要条件是$A\subset B$且$B\subset A$.
\section{集合的运算}
从给定的一些集合出发,我们可以通过所谓“集合的运算”作出一些新的集合.
\subsection{集合的基本运算}
下面我们将介绍集合的并、交、差、补的运算.
\begin{definition}[并集]
	设$A,B$是任意两个集合.由一切属于$A$或属于$B$的元素组成的集合$C$,称为$A$和$B$的{\heiti 并集}或{\heiti 和集},简称为{\heiti 并}或{\heiti 和},记为$C=A\cup B$.即
	$$A\cup B=\{x|x\in A\text{或}x\in B\}.$$
\end{definition}
并集的概念可以推广到任意多个集合的情形.
\begin{definition}[推广的并集]
	设有一族集合$\{A_\alpha|\alpha\in\Lambda\}$,其中$\alpha$是在固定指标集$\Lambda$中变化的指标;则由一切$A_\alpha(\alpha\in\Lambda)$的所有元素组成的集合称为这族集合的并集或和集,记为$\bigcup\limits_{\alpha\in\Lambda}A_\alpha$,即
	$$\bigcup_{\alpha\in\Lambda}A_\alpha=\{x|\text{存在某个}\alpha\in\Lambda,\text{使}x\in A_\alpha\}.$$
\end{definition}
习惯上,当$\Lambda=\{1,2,\cdots,k\}$为有限集时,$A=\bigcup\limits_{\alpha\in\Lambda}A_\alpha$写成$A=\bigcup\limits_{n=1}^{k}A_\alpha$,而$A=\bigcup\limits_{n\in\mathbb{N}^+}A_\alpha$写成$\bigcup\limits_{n=1}^{\infty}A_\alpha$.
\begin{definition}[交集]
	设$A,B$是任意两个集合.由一切属于$A$且属于$B$的元素组成的集合$C$,称为$A$和$B$的{\heiti 交集}或{\heiti 积集},简称为{\heiti 交}或{\heiti 积},记为$C=A\cap B$.即
	$$A\cap B=\{x|x\in A\text{且}x\in B\}.$$
\end{definition}
交集的概念也可以推广到任意多个集合的情形.
\begin{definition}[推广的交集]
	设有一族集合$\{A_\alpha|\alpha\in\Lambda\}$,其中$\alpha$是在固定指标集$\Lambda$中变化的指标;则由一切同时属于每个$A_\alpha(\alpha\in\Lambda)$的元素组成的集合称为这族集合的交集或积集,记为$\bigcap\limits_{\alpha\in\Lambda}A_\alpha$,即
	$$\bigcap_{\alpha\in\Lambda}A_\alpha=\{x|\text{对任意}\alpha\in\Lambda,\text{有}x\in A_\alpha\}.$$
\end{definition}
习惯上,当$\Lambda=\{1,2,\cdots,k\}$为有限集时,$A=\bigcap\limits_{\alpha\in\Lambda}A_\alpha$写成$A=\bigcap\limits_{n=1}^{k}A_\alpha$,而$A=\bigcap\limits_{n\in\mathbb{N}^+}A_\alpha$写成$\bigcap\limits_{n=1}^{\infty}A_\alpha$.

关于集合的并和交显然有下面的事实.
\begin{theorem}
	\begin{enumerate}[(1)]
		\item (交换律)$A\cup B=B\cup A,\ A\cap B=B\cap A$.
		\item (结合律)$A\cup(B\cup C)=(A\cup B)\cup C$,$A\cap(B\cap C)=(A\cap B)\cap C$.
		\item (分配律)$A\cap (B\cup C)=(A\cap B)\cup(A\cap C)$,$A\cap(\bigcup\limits_{\alpha\in\Lambda}B_\alpha)=\bigcup\limits_{\alpha\in\Lambda}(A\cap B_\alpha)$.
		\item $A\cup A=A,\ A\cap A=A$.
	\end{enumerate}
\end{theorem}
\begin{definition}[差集]
	若$A$和$B$是集合,称$A\backslash B=\{x|x\in A\text{且}x\notin B\}$为$A$和$B$的{\heiti 差集}.
\end{definition}
\begin{definition}[补集]
	当我们讨论的集合都是某一个大集合$S$(称为全集)的子集时,我们称$S\backslash A$为$A$的{\heiti 补集}或{\heiti 余集},并记$S\backslash A=A^c$.
\end{definition}
当全集确定时,显然$A\backslash B=A\cap B^c$.因此研究差集运算可通过研究补集运算来实现.此外,在集合论中处理差集或补集运算式时常用到以下公式.
\begin{theorem}[De\ Morgan公式]
	若$\{A_\alpha|\alpha\in \Lambda\}$是一族集合,则
	\begin{enumerate}[(1)]
		\item $(\bigcup\limits_{\alpha\in\Lambda}A_\alpha)^c=\bigcap\limits_{\alpha\in\Lambda}A_\alpha^c$;
		\item $(\bigcap\limits_{\alpha\in\Lambda}A_\alpha)^c=\bigcup\limits_{\alpha\in\Lambda}A_\alpha^c$.
	\end{enumerate}
\end{theorem}
\begin{proof}
	只需证(1).设$x\in (\bigcup\limits_{\alpha\in\Lambda}A_{\alpha})^c$,则$x\notin \bigcup\limits_{\alpha\in\Lambda}A_\alpha$,因此对任意$\alpha\in\Lambda,\ x\notin A_\alpha$,即对任意$\alpha\in\Lambda,\ x\in A_{\alpha}^c$,从而$x\in \bigcap\limits_{\alpha\in\Lambda}A_{\alpha}^c$.反之,设$x\in\bigcap\limits_{\alpha\in\Lambda}A_{\alpha}^c$,则对任意$\alpha\in\Lambda,\ x\in A_{\alpha}^c$,即对任意$\alpha\in\Lambda,\ x\notin A_{\alpha}$,则$x\notin\bigcup\limits_{\alpha\in\Lambda}A_{\alpha}$,从而$x\in(\bigcup\limits_{\alpha\in\Lambda}A_\alpha)^c$.综合可得$(\bigcup\limits_{\alpha\in\Lambda}A_\alpha)^c=\bigcap\limits_{\alpha\in\Lambda}A_\alpha^c$.
	
	对于(2),只需对等式两边取补集,并用$A_{\alpha}^c$代替$A_\alpha$即可转化为(1).$\hfill\blacksquare$
\end{proof}
{\heiti 我们注意到与“存在”相对应的是并集运算,与“任意”相对应的是交集运算.}数学分析中的很多定义、命题涉及“任意”和“存在”这两个逻辑量词,它们的否定说法是把“任意”改为“存在”,把“存在”改成“任意”.在集合论中,De\ Morgan公式很好地反映了这种论述的合理性.
\subsection{集列的上极限和下极限}
我们将一列集合$A_1,A_2,\cdots,A_n,\cdots$称为{\heiti 集列}.记作$\{A_n\}$.
\begin{definition}[上极限]
	设$A_1,A_2,\cdots,A_n,\cdots$是任意一集列.由属于上述集列中无限多个集合的那种元素的全体所组成的集合称为这一集列的{\heiti 上极限},记为$\varlimsup\limits_{n\to\infty}A_n$或$\limsup\limits_{n\to\infty}A_n$.它可表示为
	$$\varlimsup\limits_{n\to\infty}A_n=\{x|\text{存在无穷多个}A_n,\ \text{使}x\in A_n\}.$$
\end{definition}
不难证明:
$$\varlimsup\limits_{n\to\infty}A_n=\{x|\forall N>0,\ \exists n>N,\ s.t.\ x\in A_n\}.$$
\begin{definition}[下极限]
	设$A_1,A_2,\cdots,A_n,\cdots$是任意一集列.除有限个下标外,属于集列中每个集合的元素全体所组成的集合称为这一集列的{\heiti 下极限},记为$\varliminf\limits_{n\to\infty}A_n$或$\liminf\limits_{n\to\infty}A_n$.它可表示为
	$$\varliminf\limits_{n\to\infty}A_n=\{x|\text{当}n\text{充分大以后都有}x\in A_n\}.$$
\end{definition}
不难证明:
$$\varliminf\limits_{n\to\infty}A_n=\{x|\exists N>0,\ \forall n>N,\ s.t.\ x\in A_n\}.$$
根据上极限和下极限的定义,可知上极限中的元素是频繁出现的,不一定在$n$充分大的时候一定存在,比如有可能是振荡的,有可能只在偶数项上出现等等.而下极限中的元素在$n$充分大的时候都是存在的,也就是说除去有限项下标后元素是一直存在的,自此我们可以看出,下极限定义的条件比上极限更严格,所以下极限是包含于上极限的,即
$$\varliminf\limits_{n\to\infty}A_n\subset\varlimsup\limits_{n\to\infty}A_n$$

上、下极限还可以用交集和并集表示.
\begin{theorem}
	\begin{enumerate}[(1)]
		\item $\varlimsup\limits_{n\to\infty}A_n=\bigcap\limits_{n=1}^{\infty}\bigcup\limits_{m=n}^{\infty}A_m$;
		\item $\varliminf\limits_{n\to\infty}A_n=\bigcup\limits_{n=1}^{\infty}\bigcap\limits_{m=n}^{\infty}A_m$.
	\end{enumerate}
\end{theorem}
\begin{proof}
	只需证明(1).
	
	记$A=\varlimsup\limits_{n\to\infty}A_n=\{x|\forall N>0,\ \exists n>N,\ s.t.\ x\in A_n\}$,$B=\bigcap\limits_{n=1}^{\infty}\bigcup\limits_{m=n}^{\infty}A_m$.设$x\in A$,则对任意取定的$n$,总有$m>n$,使$x\in A_m$,即对任何$n$,总有$x\in\bigcup\limits_{m=n}^{\infty}A_m$,故$x\in B$.
	
	反之,设$x\in B$,则对任意的$N>0$,总有$x\in \bigcup\limits_{m=N+1}^{\infty}A_m$,即总存在$m>N$,有$x\in A_m$,所以$x\in A$,因此$A=B$,即$\varlimsup\limits_{n\to\infty}A_n=\bigcap\limits_{n=1}^{\infty}\bigcup\limits_{m=n}^{\infty}A_m$.$\hfill\blacksquare$
\end{proof}
如果$\varlimsup\limits_{n\to\infty}A_n=\varliminf\limits_{n\to\infty}A_n$,则称集列$\{A_n\}$收敛,记$\lim\limits_{n\to\infty}A_n=\varlimsup\limits_{n\to\infty}A_n=\varliminf\limits_{n\to\infty}A_n$称为集列$\{A_n\}$的极限.
\subsection{单调集列}
\begin{definition}[单调集列]
	如果集列$\{A_n\}$满足$A_n\subset A_{n+1},\ n=1,2,\cdots$,则称$\{A_n\}$为增列;如果集列$\{A_n\}$满足$A_n\supset A_{n+1},\ n=1,2,\cdots$,则称$\{A_n\}$为减列.增列和减列统称为{\heiti 单调集列}.
\end{definition}
容易证明:{\heiti 单调集列是收敛的}.若$\{A_n\}$为增列,则$\lim\limits_{n\to\infty}A_n=\bigcup\limits_{n=1}^{\infty}A_n$;若$\{A_n\}$为减列,则$\lim\limits_{n\to\infty}A_n=\bigcap\limits_{n=1}^{\infty}A_n$.

现在我们可以从单调集列的角度来定义上下极限.

对于任意的一个集列$\{A_n\}$,我们构造一个新的集列$B_n=\bigcup\limits_{m=n}^{\infty}A_m$,显然$\{B_n\}$是减的(不一定严格).则
$$\lim\limits_{n\to\infty}B_n=\bigcap\limits_{n=1}^{\infty}\bigcup\limits_{m=n}^{\infty}A_m.$$
我们将集列$\{B_n\}$的极限定义为$\{A_n\}$的上极限.类似地,我们还可以给出下极限的定义,这里不再赘述.
\subsection{集合的Cartesian积}
\begin{definition}[Cartesian积]
	若$A_i(i=1,2,\cdots,n)$是集合,则$A=\{(x_1,x_2,\cdots,x_n)|x_i\in A,\ i=1,2,\cdots,n\}$称为$A_i(i=1,2,\cdots,n)$的{\heiti Cartesian积}或{\heiti 直积},记为
	$$\prod_{i=1}^{n}A_i\text{或}A_1\times A_2\times\cdots\times A_n.$$
\end{definition}
类似地,
$$\prod_{i=1}^{\infty}A_i=A_1\times A_2\times\cdots=\{(x_1,x_2,\cdots)|x_i\in A,\ i=1,2,\cdots\}.$$
我们应当注意,集合的Cartesian积依赖于预先给定集合的次序.一般说来,集合$X$和集合$Y$的Cartesian积$X\times Y$完全不同于集合$Y$和集合$X$的Cartesian积$Y\times X$.特别地,若$A_i=A(i=1,2,\cdots)$,则
$$\prod_{i=1}^{n}A_i=A^n,\ \prod_{i=1}^{\infty}A_i=A^{\infty}.$$
\section{关系与等价关系}
\begin{definition}[关系]
	设$X,Y$是两个集合.如果$R\subset X\times Y$,则称$R$是从$X$到$Y$的一个{\heiti 关系}.
\end{definition}
\begin{definition}[值域]
	设$R$是从集合$X$到集合$Y$的一个关系,即$R\subset X\times Y$.如果$(x,y)\in R$,则我们称$x$与$y$是{\heiti $R$-相关的},并且记作$xRy$.如果$A\subset X$,则$Y$的子集
	$$\{y\in Y|\text{存在}x\in A,\ s.t.\ xRy\}$$
	称为{\heiti 集合$A$对于关系$R$而言的像集},或者简单地称为集合$A$的{\heiti 像集}或集合$A$的{\heiti $R$-像},并且记作$R(A)$.我们称$R(X)$为关系$R$的{\heiti 值域}.
\end{definition}
关系的概念十分广泛.函数、映射、等价、序、运算等概念都是关系的特例.这里有两个特别简单的从集合$X$到集合$Y$的特例,一个是$X\times Y$本身,另一个是空集$\varnothing$.
\begin{definition}[定义域]
	设$R$是从集合$X$到集合$Y$的一个关系,即$R\subset X\times Y$.这时Cartesian积$Y\times X$的子集
	$$\{(y,x)\in Y\times X|xRy\}$$
	是从集合$Y$到集合$X$的一个关系,我们称它为关系$R$的{\heiti 逆},记作$R^{-1}$.如果$B\subset Y$,则$X$的子集$R^{-1}(B)$是集合$B$的{\heiti $R^{-1}$-像},我们也常称它为集合$B$对于关系$R$而言的{\heiti 原像},或者集合$B$的{\heiti $R$-原像}.关系$R^{-1}$的值域$R^{-1}(Y)$也成为关系$R$的{\heiti 定义域}.
\end{definition}
\begin{definition}[复合关系]
	设$R$是从集合$X$到集合$Y$的一个关系,$S$是从集合$Y$到集合$Z$的一个关系.集合
	$$\{(x,z)\in(X,Z)|\text{存在}y\in Y,\ s.t.\ xRy\text{且}ySz\}$$
	是Cartesian积$X\times Z$的一个子集,即从集合$X$到集合$Z$的一个关系,此关系称为关系$R$与关系$S$的{\heiti 复合}或{\heiti 积},记作$S\circ R$.
\end{definition}
\begin{theorem}
	设$R$是从集合$X$到集合$Y$的一个关系,$S$是从集合$Y$到集合$Z$的一个关系,$T$是从集合$Z$到集合$U$的一个关系,则
	\begin{enumerate}[(1)]
		\item $(R^{-1})^{-1}=R$;
		\item $(S\circ R)^{-1}=R^{-1}\circ S^{-1}$;
		\item $T\circ(S\circ R)=(T\circ S)\circ R$.
	\end{enumerate}
\end{theorem}
\begin{theorem}
	设$R$是从集合$X$到集合$Y$的一个关系,$S$是从集合$Y$到集合$Z$的一个关系,则对于$X$的任意两个子集$A$和$B$,我们有
	\begin{enumerate}[(1)]
		\item $R(A\cup B)=R(A)\cup R(B)$;
		\item $R(A\cap B)=R(A)\cap R(B)$;
		\item $(S\circ R)(A)=S(R(A))$.
	\end{enumerate}
\end{theorem}
以上定理都可由定义直接验证,这里不再赘述.
\begin{definition}
	设$X$是一个集合,从集合$X$到集合$X$的一个关系将简称为集合$X$中的一个关系.集合$X$中的关系$\{(x,x)|x\in X\}$称为{\heiti 恒同关系}、{\heiti 恒同}或者{\heiti 对角线},记作$\Delta(X)$.
\end{definition}
\begin{definition}[等价关系]
	设$R$是集合$X$中的一个关系.如果$\Delta(X)\subset R$,即对任何$x\in X$有$xRx$,则称关系$R$为{\heiti 自反的};如果$R=R^{-1}$,即对任何$x,y\in X$,若$xRy$则$yRx$,则称关系$R$为{\heiti 对称的};如果$R\cap R^{-1}=\varnothing$,即对任何$x,y\in X$,$xRy$和$yRx$不能同时成立,则称关系$R$是{\heiti 反称的};如果$R\circ R\subset R$,即对任何$x,y,z\in X$,若$xRy,yRz$,则$xRz$,则称关系$R$是{\heiti 传递的}.
	
	集合$X$中的一个关系如果同时是自反、对称和传递的,则称为集合$X$中的一个{\heiti 等价关系}.
\end{definition}
等价关系的概念在此简单提出,之后在更深入的学习中将展开讨论,在此不再深入探讨.
\section{映射}
\begin{definition}[映射]
	设$F$是从集合$X$到集合$Y$的一个关系.如果对于每一个$x\in X$存在唯一的$y\in Y$使得$xFy$,则称关系$F$是从集合$X$到集合$Y$的一个{\heiti 映射},记作$F:X\rightarrow Y$.
\end{definition}
换言之,关系$F$是一个映射,如果对于每一个$x\in X$,满足
\begin{enumerate}[(1)]
	\item 存在$y\in Y$,使得$xFy$;
	\item 如果对于$y_1,y_2\in Y$有$xFy_1$和$xFy_2$,则$y_1=y_2$.
\end{enumerate}
\begin{definition}[像与原像]
	设$X,Y$是两个集合,$F:X\rightarrow Y$.对于每一个$x\in X$,使得$xFy$的唯一的那个$y\in Y$称为$x$的{\heiti 像}或{\heiti 值},记作$F(x)$;对于每一个$y\in Y$,若$x\in X$使得$xFy$(即$y$是$x$的像),则称$x$是$y$的一个{\heiti 原像}.(注意:$y\in Y$可以没有原像,也可以有不止一个原像.)
\end{definition}
由于映射本身就是一种特殊的关系,因此关系的定义域、值域、逆、复合等概念自然也是映射的概念,我们不再重复.下面给出单射、满射和双射的概念.
\begin{definition}
	设$X$和$Y$是两个集合,$f:X\rightarrow Y$.如果$X$中不同的点的像是$Y$中不同的点,那么称$f$是一个{\heiti 单射};如果$Y$中的每一个点都有原像,那么称$f$是一个{\heiti 满射};如果$f$既是一个单射又是一个满射,则称$f$为{\heiti 双射},又称{\heiti 一一映射}.
\end{definition}
用数学语言表示就是:
\begin{enumerate}[(1)]
	\item 单射:对任何$x_1,x_2\in X$,若$x_1\neq x_2$,则$f(x_1)\neq f(x_2)$;
	\item 满射:对任何$y\in Y$,存在$x\in X$使得$f(x)=y$.
\end{enumerate}

易见,集合$X$中的恒同关系$\Delta(X)$是从$X$到$X$的一个双射,我们也常称之为(集合$X$上的){\heiti 恒同映射}或{\heiti 恒同},有时也称之为{\heiti 单位映射},记作$i_X$或$i:X\rightarrow X$.显然,对任何$x\in X$,有$i_X(x)=x$,即恒同映射把每一个点映为这个点自身.

由于下面这个定理,双射也称为{\heiti 可逆映射}.
\begin{theorem}
	设$X$和$Y$是两个集合,$f:X\rightarrow Y$.如果$f$是双射,则$f^{-1}$是从$Y$到$X$的双射,即$f^{-1}:Y\rightarrow X$.且
	$$f^{-1}\circ f=i_X,\qquad f\circ f^{-1}=i_Y.$$
\end{theorem}
该定理的证明参见熊金城编著的《点集拓扑讲义》第13页,在此不再展开.
\section{对等与基数}
\begin{definition}[对等]
	若$A,B$是非空集合,且存在双射$\varphi:A\rightarrow B$,则称$A$与$B${\heiti 对等},记为$A\sim B$,规定$\varnothing\sim \varnothing$.
\end{definition}
\begin{example}
	我们可以给出有限集合的一个不依赖于元素个数概念的定义:集合$A$称为有限集合,如果$A=\varnothing$或$A$和正整数的某一截段$\{1,2,\cdots,n\}$对等.
\end{example}
\begin{example}\label{zo}
	$\{\text{正整数全体}\}\sim \{\text{正偶数全体}\}$.这只需令$\varphi(x)=2x,x\in\mathbb{Z}^+$即可.
\end{example}
\begin{example}\label{real}
	区间$(0,1)$和全体实数$\mathbb{R}$对等,只需对$x\in(0,1)$,令$\varphi(x)=\tan\bigg(\pi x-\dfrac{\pi}{2}\bigg)$.
\end{example}

例\ref{zo}和例\ref{real}表明,一个无限集可以和它的一个真子集对等(可以证明,这一性质正是无限集的特征,常用来作为无限集的定义).这一性质对有限集来说显然不能成立.由此可以看到无限集和有限集之间的深刻差异.此外,例\ref{real}还表明,无限长的“线段”并不比有限长的线段有“更多的点”.

对等关系显然有以下性质:
\begin{theorem}
	对任何集合$A,B,C$,均有
	\begin{enumerate}
		\item 自反性:$A\sim A$;
		\item 对称性:$A\sim B$,则$B\sim A$;
		\item 传递性:$A\sim B,\ B\sim C$,则$A\sim C$.
	\end{enumerate}
\end{theorem}
\begin{definition}[基数]
	若$A$和$B$对等,则称它们有相同的{\heiti 基数}或{\heiti 势}.记为$|A|=|B|$.
\end{definition}
\begin{definition}
	设$A,B$是两个集合,如果$A$不与$B$对等,但存在$B$的真子集$B'$,有$A\sim B'$,则称$A$比$B$有{\heiti 较小的基数}(或$B$比$A$有{\heiti 更大的基数}),记作$|A|<|B|$(或$|B|>|A|$).
\end{definition}

下面,我们提出问题:任给两个集合$A,B$,在
$$|A|<|B|,\quad |A|=|B|,\quad |A|>|B|$$
中是否必有一个成立且只有一个成立呢?回答是肯定的.但是第一个问题的论证较为复杂,不能在此讨论,我们仅简单给出第二个问题的定理.
\begin{theorem}[Bernstein定理]
	设$A,B$是两个非空集合.如果$A$对等于$B$的一个子集,$B$又对等于$A$的一个子集,那么$A\sim B$.
\end{theorem}
利用基数的说法就是:若$|A|\leqslant|B|,\ |B|\leqslant |A|$,则$|A|=|B|$.定理的证明过程不再展开讨论.
\section{可数集}
\begin{definition}[可数集]
	凡和全体正整数所成集合$\mathbb{Z}^+$对等的集合都称为{\heiti 可数集}或{\heiti 可列集}.
\end{definition}
由于$\mathbb{Z}^+$可按大小顺序排成一无穷序列$1,2,\cdots,n,\cdots$,因此,一个集合$A$是可数集合的充要条件为:$A$可以排成一个无穷序列
$$a_1,a_2,\cdots,a_n,\cdots .$$

可数集合是无限集合,那么它在一般无限集合中处于什么地位呢?
\begin{theorem}
	任何无限集合都至少包含一个可数子集.
\end{theorem}
\begin{proof}
	设$M$是一个无限集,因$M\neq\varnothing$,总可以从$M$中取出一个元素记为$e_1$,由于$M$是无限集,因此$M\backslash\{e_1\}\neq\varnothing$,于是又可以从$M\backslash\{e_1\}$中取出一个元素$e_2$,显然$e_2\in M$且$e_2\neq e_1$.以此类推,我们可以取出$n$个这样的互异元素$e_1,e_2,\cdots,e_n$.由归纳法,我们还可以从$M\backslash\{e_1,e_2,\cdots,e_n\}$取出$e_{n+1}$.这样由归纳法,我们就找到$M$的一个无限子集$\{e_1,e_2,\cdots,e_n,\cdots\}$.它显然是一个可数集.$\hfill\blacksquare$
\end{proof}
由上述定理我们知道:{\heiti 可数集在所有无限集中有最小的基数}.
\begin{theorem}\label{2}
	可数集合的任何无限子集必为可数集合,从而可数集合的任何子集或者是有限集或者是可数集.
\end{theorem}
证明思路:利用Bernstein定理来夹逼.
\begin{theorem}
	设$A$是可数集,$B$是有限或可数集,则$A\cup B$为可数集.
\end{theorem}
证明思路:将可数集排成无穷序列.
\begin{corollary}\label{1}
	设$A_i(i=1,2,\cdots,n)$是有限集或可数集,则$\bigcup\limits_{i=1}^{n}A_i$也是有限集或可数集.但如果至少有一个$A_i$是可数集,则$\bigcup\limits_{i=1}^{n}A_i$必为可数集.
\end{corollary}
\begin{theorem}\label{4}
	设$A_i(i=1,2,\cdots,n)$都是可数集,则$\bigcup\limits_{i=1}^{\infty}A_i$也是可数集.
\end{theorem}
\begin{proof}
	(1)先设$A_i\cap A_j=\varnothing(i\neq j)$.
	
	可将$\bigcup\limits_{i=1}^{\infty}A_i$排成
	$$\bigcup_{i=1}^{\infty}A_i=\{a_{11},a_{12},a_{21},a_{31},a_{22},a_{13},a_{14},\cdots\}.$$
	
	(2)一般情形下,令$A_1'=A_1,\ A_i'=A_i-\bigcup\limits_{j=1}^{i-1}A_j(i\geqslant 2)$,则$A_i'\cap A_j'=\varnothing(i\neq j)$,且$\bigcup\limits_{i=1}^{\infty}A_i=\bigcup\limits_{i=1}^{\infty}A_i'$
	
	易知$A_i'$都是有限集或可数集.(定理\ref{2})如果只有有限个$A_i'$不为空集,由推论\ref{1},$\bigcup\limits_{i=1}^{\infty}A_i'$为可数集(因至少$A_i'=A_i$为可数集),如果有无限多个(必为可数个)$A_i'$不为空集,则由(1),$\bigcup\limits_{i=1}^{\infty}A_i'$也是可数集,故在任何情形下,$\bigcup\limits_{i=1}^{\infty}A_i$都是可数集.$\hfill\blacksquare$
\end{proof}
我们用$\aleph_0$(读作“阿列夫零”)表示可数集的基数.则当$A_i$均为可数集合时,推论\ref{1}可简记为
$$n\cdot \aleph_0=\underbrace{\aleph_0+\aleph_0+\cdots+\aleph_0}_{n\text{个}}=\aleph_0.$$

定理\ref{4}的结论可简记为
$$\aleph_0\cdot \aleph_0=\underbrace{\aleph_0+\aleph_0+\cdots+\aleph_0+\cdots}_{\text{可数个}}=\aleph_0.$$
\begin{theorem}
	有理数全体成一可数集合.
\end{theorem}
\begin{proof}
	设$A_i=\biggl\{\dfrac{1}{i},\dfrac{2}{i},\dfrac{3}{i},\cdots\biggr\}(i=1,2,3,\cdots)$,则$A_i$是可数集,于是由定理\ref{4}知全体正有理数成一可数集$\mathbb{Q}^+=\bigcup\limits_{i=1}^{\infty}A_i$,因正负有理数通过$\varphi(r)=-r$一一对应,故全体负有理数成一可数集$\mathbb{Q}^-$,又$\mathbb{Q}=\mathbb{Q}^+\cup\mathbb{Q}^-\cup \{0\}$,故由推论\ref{1}知$\mathbb{Q}$为可数集.$\hfill\blacksquare$
\end{proof}

应该注意,有理数在实数中是处处稠密的,即在数轴上任何小区间中都有有理数存在(并且有无穷多个).尽管如此,全体有理数还只不过是一个和稀疏分布着的正整数全体成为一一对应的可数集.这个表面看来令人难以置信的事实,正是Cantor创立集合论,向“无限”进军的一个重要成果,它是人类理论思维的又一胜利.

用有理数集的可数性和稠密性可推断出一些重要的结论.
\begin{example}
	设集合$A$中元素都是直线上的开区间,满足条件:若开区间$K,J\in A,\ K\neq J$,则$K\cap J=\varnothing$.则$A$是可数集或有限集.
\end{example}
\begin{proof}
	作映射$\varphi:A\rightarrow\mathbb{Q}$.设$K\in A$,由于$\mathbb{Q}$在直线上稠密,任取$r\in K\cap \mathbb{Q}$,定义$\varphi(K)=r$.由于任意$K,J\in A,\ K\neq J$,有$K\cap J=\varnothing$,因此$\varphi$是$A$到$\mathbb{Q}$内的单射,于是$A\sim \varphi(A)\subset \mathbb{Q}$,所以$|A|\leqslant|\mathbb{Q}|=\aleph_0$,即$A$是可数集或有限集.$\hfill\blacksquare$
\end{proof}
\begin{theorem}\label{6}
	设$A_i(i=1,2,\cdots)$是可数集,则$\prod\limits_{i=1}^{n}A_i$是可数集.
\end{theorem}
\begin{proof}
	用归纳法证明.显然$i=1$时结论成立.设$i=n-1$时结论成立,则$\prod\limits_{i=1}^{n-1}A_i$是可数集.$A_i$可数,可设$A_i=\{x_1,x_2,\cdots,x_k,\cdots\}$.记$\hat{A_k}=\prod\limits_{i=1}^{n-1}A_i\times\{x_k\}$,则$\hat{A_k}\sim\prod\limits_{i=1}^{n-1}A_i(k=1,2,\cdots)$,因此$\hat{A_k}$是可数集.又$\prod\limits_{i=1}^{n}A_i=\bigcup\limits_{k=1}^{\infty}\hat{A_k}$,由定理\ref{4},$\prod\limits_{i=1}^{n}A_i$是可数集.$\hfill\blacksquare$
\end{proof}
\begin{example}
	平面上坐标为有理数的点的全体所成的集合为一可数集.
\end{example}
\begin{example}
	元素$(n_1,n_2,\cdots,n_k)$是由$k$个有理数组成的,其全体成一可数集.
\end{example}
\begin{example}
	整系数多项式
	$$a_nx^n+a_{n-1}x^{n-1}+\cdots+a_1x+a_0$$
	的全体是一可数集.
\end{example}
\begin{proof}
	对任意$n$,设$A_n$是$n$次整系数多项式的全体组成的集合,则$A_n=\{a_nx^n+a_{n-1}x^{n-1}+\cdots+a_1x+a_0\}\sim\mathbb{Z}_0\times\underbrace{\mathbb{Z}\times\cdots\times\mathbb{Z}}_{n\text{个}}$,其中$\mathbb{Z}_0=\mathbb{Z}\backslash\{0\}$和$\mathbb{Z}$都是可数集,因此由定理\ref{6},$A_n$是可数集.从而整系数多项式的全体组成的集合$\bigcup\limits_{n=0}^{\infty}A_n$也是可数集.$\hfill\blacksquare$
\end{proof}
每个多项式只有有限个根,所以得下面的定理.
\begin{theorem}
	代数数的全体成一可数集.
\end{theorem}
\begin{remark}
	代数数指的是整系数多项式的根.
\end{remark}
\section{不可数集}
我们将不是可数集合的{\heiti 无限集合}称为{\heiti 不可数集}.
\begin{theorem}
	全体实数所成集合$\mathbb{R}$是一个不可数集合.
\end{theorem}
\begin{proof}
	由例\ref{real}知$\mathbb{R}\sim(0,1)$,我们只需证明$(0,1)$不是可数集即可.首先$(0,1)$中的每一个实数$a$都可以唯一地表示为十进位无限小数
	$$a=0.a_1a_2\cdots=\sum\limits_{n=1}^{\infty}\dfrac{a_n}{10^n}$$
	的形式,其中各$a_n$是$0,1,\cdots,9$中的一个数字,不全为$9$,且不以$0$为循环节.我们称实数的这种表示为一个正规表示.(后面在实数理论中将给出证明)
	
	现用反证法:假设$(0,1)$中的全体实数可排列成一个序列
	$$(0,1)=\{a_1,a_2,\cdots\}.$$
	将每个$a_n$表示成正规的无限小数:
	$$a_1=0.a_{11}a_{12}\cdots,$$
	$$a_2=0.a_{21}a_{22}\cdots,$$
	$$\cdots\cdots\cdots\cdots$$
	下面设法在$(0,1)$之间找一个与所有这些实数都不同的实数.利用对角线上的数字$a_{nn}$,作一个无限小数:
	$$0.b_1b_2\cdots,\text{其中}b_n=\left\{
	\begin{aligned}
		&1,\quad & a_{nn}\neq 1,\\
		&2,\quad & a_{nn}=1.
	\end{aligned}
	\right.
	$$
	则此无限小数的各位数字既不全是$9$,也不以$0$为循环节,因此必是$(0,1)$中某一实数的正规表示.但这个实数与每一个$a_n$的正规表示都不同(至少第$n$位不同),因此$(0,1)\neq\{a_1,a_2,\cdots\}$,与假设矛盾.因此$(0,1)$是不可数集.实数集$\mathbb{R}$是不可数集.$\hfill\blacksquare$
\end{proof}
\begin{remark}
	以上定理的证明思路是Cantor的对角线技巧.
\end{remark}
\begin{corollary}
	若用$\aleph$表示全体实数所成的集合$\mathbb{R}$的基数,用$\aleph_0$表示全体正整数所成集合$\mathbb{Z}^+$的基数,则$\aleph>\aleph_0$.
\end{corollary}
我们称$\aleph$为{\heiti 连续基数}.称与$\mathbb{R}$等势的集合为{\heiti 连续统}.
\begin{theorem}
	任意区间$(a,b),\left[a,b\right),\left(a,b\right],(0,\infty),\left[0,\infty\right)$都是连续统.
\end{theorem}
\begin{theorem}
	设$A_1,A_2,\cdots,A_n,\cdots$是一列互不相交的连续统,则$\bigcup\limits_{n=1}^{\infty}A_i$也是连续统.
\end{theorem}
\begin{remark}
	上述定理说明{\heiti 可数个连续统的并仍是连续统}.
\end{remark}
\begin{theorem}
	设$A_1,A_2,\cdots,A_n,\cdots$是一列连续统,则$\prod\limits_{n=1}^{\infty}A_i$也是连续统.
\end{theorem}
\begin{remark}
	上述定理说明{\heiti 可数个连续统的Cartesian积仍是连续统}.
\end{remark}
\begin{theorem}
	设$M$是任意的一个集合,它的所有子集构成新的集合$\mu$,则$|\mu|>|M|$.
\end{theorem}
一般地,集合$M$的所有子集组成的集合记为$2^M$.上述定理说明了,对任意集合$M$,$|M|<|2^M|$,从而没有最大的基数.

由于可数集中元素比连续统中元素少得多,我们通常尽可能地用可数集合交、并运算代替不可数集的交、并运算.这一点,在测度论中有十分重要的应用.
\newpage

\part{一元函数微积分}

\chapter{实数理论}

数学分析研究的基本对象是定义在实数集上的函数.
\section{实数的定义}%或许可以改成“有理数的扩充”
\subsection{建立实数的原则}
\begin{definition}[数域]\label{def:field}
	设$P$是由一些复数组成的集合,其中包括0和1,如果$P$中任意两个数的和、差、积、商(除数不为0)仍为$P$中的数,则称$P$为一个{\heiti 数域}.
\end{definition}
由定义\ref{def:field}可知,数域对加、减、乘、除(除数不为0)四则运算具有封闭性,即结果仍在数域本身中。例如,全体有理数所构成的集合$\mathbb{Q}$是一个数域,称为有理数域.此外,常见的数域还有复数域$\mathbb{C}$,读者可自行验证.
\begin{example}
	证明全体有理数所构成的集合$\mathbb{Q}$是一个数域.
	\begin{proof}
		对$\forall a=\frac{p}{q},\ b=\frac{s}{t}$,其中$p,q,s,t\in \mathbb{Z}$且$st\neq0$,则由有理数的定义知,$a,b\in \mathbb{Q}$
		
		显然$0,1\in\mathbb{Q}$,
		
		$a+b=\frac{pt+qs}{qt}\in \mathbb{Q}$
		
		$a-b=\frac{pt-qs}{qt}\in \mathbb{Q}$
		
		$a\cdot b=\frac{ps}{qt}\in \mathbb{Q}$
		
		$\frac{a}{b}=\frac{pt}{qs}\in \mathbb{Q}$
		
		故$\mathbb{Q}$是一个数域.$\hfill\blacksquare$
	\end{proof}
\end{example}
\begin{definition}[阿基米德有序域]
	集合$F$构成一个{\heiti 阿基米德有序域},是说它满足以下三个条件:
	\begin{enumerate}
		\item $F$是域\qquad 在$F$中定义了加法“$+$”和乘法“$\cdot$”两种运算,使得对于$F$中任意元素$a$,$b$,$c$成立:\par
		加法的结合律:\ $(a+b)+c=a+(b+c)$;\par
		加法的交换律:\ $a+b=b+a$;\par
		乘法的结合律:\
		$(a\cdot b)\cdot c=a\cdot (b\cdot c)$;\par
		乘法的交换律:\ $a\cdot b=b\cdot a$;\par
		乘法关于加法的分配律:
		$(a+b)\cdot c=a\cdot c+b\cdot c$;\par
		$F$中对加法存在{\heiti 零元素}和{\heiti 负元素}(即存在加法的逆运算减法);
		
		$F$中对乘法存在{\heiti 单位元素}和{\heiti 逆元素}(即存在乘法的逆运算除法).
		\item $F$是有序域\qquad
		在$F$中定义了{\heiti 序}关系“\textless”具有如下{\heiti 全序}的性质:\par
		传递性:$\forall a,b,c \in F$,若$a<b$,$b<c$,则$a<c$;\par
		三歧性:$\forall a,b \in F$,$a>b$,$a<b$,$a=b$三者必居其一,也只居其一;\par
		加法保序性:$\forall a,b,c \in F$,若$a<b$,则$a+c<b+c$;\par
		乘法保序性:$\forall a,b,c \in F$,若$a<b$,则$ac<bc$\ (c>0).
		\item $F$中元素满足阿基米德性\qquad 对$F$中两个正元素$a$,$b$,必存在自然数$n$,使得$na>b$.
	\end{enumerate}
\end{definition}
\subsection{Dedekind分割}
设$A/B$是有理数域$\mathbb{Q}$上的一个分割,即把$\mathbb{Q}$中的元素分为$A$、$B$两个集合,使得$\forall a \in A,\ b \in B$有$a<b$成立.则从逻辑上分为下列四种情况:
\begin{enumerate}
	\item $A$有最大值$a_0$,$B$有最小值$b_0$;
	\item $A$有最大值$a_0$,$B$无最小值;
	\item $A$无最大值,$B$有最小值$b_0$;
	\item $A$无最大值,$B$无最小值.
\end{enumerate}	

而对于第1种情况,取$\frac{a_0+b_0}{2}\in \mathbb{Q}$,它将不属于集合$A$、$B$中的任何一个.\par 
对于第4种情况,说明分割到的数不在$\mathbb{Q}$内(我们将这种划分称为{\heiti 无端划分}),因此这是我们通过“切割”构造出来的“新数”.将所有这样切割出来的新数与原来的$\mathbb{Q}$取并集,并设新的集合为$\mathbb{R}$.将$\mathbb{R}$中的元素再次进行分割,设$A'/B'$为$\mathbb{R}$上的一个分割,则$\forall a \in A',\ b \in B'$有$a<b$成立.同理,在排除上述的第1种情况后,对$\mathbb{R}$的分割分为以下三种情况:
\begin{enumerate}
	\item $A'$有最大值$a_0$,$B'$无最小值;
	\item $A'$无最大值,$B'$有最小值$b_0$;
	\item $A'$无最大值,$B'$无最小值.
\end{enumerate}	

由此,我们给出Dedekind定理.
\begin{theorem}[Dedekind定理]
	设$A'$、$B'$是$\mathbb{R}$的两个子集,且满足:
	\begin{enumerate}
		\item $A'$和$B'$均不为空集;
		\item $A'\cup B'=\mathbb{R}$;
		\item $\forall a\in A',\ b\in B'$有$a<b$.
	\end{enumerate}
	则或者$A'$有最大元素,或者$B'$有最小元素.
\end{theorem}
Dedikind定理指出,我们在上述对$\mathbb{R}$进行分割时,只会出现第1种或第2种情况,而不可能$A'$无最大值且$B'$也无最小值。
\begin{proof}
	设$A$是$A'$中所有有理数的集合,设$B$是$B'$中所有有理数的集合,则$A/B$有三种情况:
	\begin{enumerate}
		\item $A$有最大值$a_0$,$B$无最小值;
		\item $A$无最大值,$B$有最小值$b_0$;
		\item $A$无最大值,$B$无最小值;
	\end{enumerate}	
	
	对于第1种情况,设$a'\in A'$,使得$a'>a_0$,则在$(a_0,a')$之间必定存在有理数$a$,这与$A$有最大值$a_0$矛盾.因此$a_0$也是$A'$的最大值;
	
	同理,对于第2种情况,我们可以得到$b_0$也是$B'$的最小值;
	
	对于第3种情况,设$c$是由$A/B$得到的无理数,则$a_0<c<b_0$,可知$c$或者是$A'$中的元素,或者是$B'$中的元素.不妨设$c\in A'$,可以证明c为$A'$的最大元素,因为如果存在$a$是$A$的最大元素,那么区间$(c,a)$之间必定存在有理数大于$a_0$,这与$A$的最大值是$a_0$矛盾.
	
	以上我们就证明了或者$A'$有最大元素,或者$B'$有最小元素.$\hfill\blacksquare$
\end{proof}
\subsection{实数的公理化定义}
由前两节的理论基础,我们可以定义{\heiti 实数}构成一个阿基米德有序域,且满足Dedekind定理.实数分为有理数和无理数,其中无理数由有理数的无端划分产生.
\section{确界原理}
确界原理是极限理论的基础.
\subsection{确界的定义}
\begin{definition}
	设$S$为$\mathbb{R}$中的一个数集,若存在数$M(L)$,使得对一切$x\in S$,都有$x\leqslant M(x\geqslant L)$,则称$S$为{\heiti 有上界(下界)的数集},数$M(L)$称为$S$的一个{\heiti 上界(下界)}.
\end{definition}
\begin{definition}[上确界]
	设$S$是$\mathbb{R}$中的一个数集,若数$\eta$满足:
	
	(i)$\eta$是$S$的上界;
	
	(ii)$\forall \alpha < \eta,\ \exists x_0\in S,\ s.t.x_0>\alpha$,即$\eta$又是$S$的最小上界,\\
	则称$\eta$为数集$S$的{\heiti 上确界},记作$\eta=\sup S$
\end{definition}
\begin{definition}[下确界]
	设$S$是$\mathbb{R}$中的一个数集,若数$\xi$满足:
	
	(i)$\xi$是$S$的下界;
	
	(ii)$\forall \beta > \xi,\ \exists x_0\in S,\ s.t.x_0<\beta$,即$\xi$又是$S$的最小下界,\\
	则称$\xi$为数集$S$的{\heiti 下确界},记作$\xi=\inf S$
\end{definition}
上(下)确界也可以由$\varepsilon$语言定义.
\begin{definition}[上确界的$\varepsilon$语言定义]
	设$S$为$\mathbb{R}$中的一个数集,若S的一个上界$M$,$\forall \varepsilon>0$,$\exists a\in S$,s.t.\ $a>M-\varepsilon$,则数$M$称为$S$的一个{\heiti 上确界}.
\end{definition}
\begin{definition}[下确界的$\varepsilon$语言定义]
	设$S$为$\mathbb{R}$中的一个数集,若S的一个下界$L$,$\forall \varepsilon>0$,$\exists b\in S$,s.t.\ $b<L+\varepsilon$,则数$L$称为$S$的一个{\heiti 下确界}.
\end{definition}
上确界和下确界统称为确界.
\subsection{确界原理及其证明}
\begin{theorem}[确界原理]
	设$S$为$\mathbb{R}$中的一个数集,若$S$有上界,则必有上确界;若$S$有下界,则必有下确界.
\end{theorem}
\begin{proof}
	设数集$S$有上界,下面证明$S$有上确界.
	
	设$B$是数集$S$所有上界组成的集合,记$A=\mathbb{R}\textbackslash B$.若$B$有最小元素,则$B$的最小元素$b_0$即为$S$的上确界.
	
	设$x\in A$,$x$不是$S$的上界,则$\exists t\in S,\ s.t.\ x<t$,取$x'=\frac{x+t}{2}$,则$x'>x$,因此对$A$中任何一个元素$x$,都有$x'>x$存在,即$A$没有最大元素.由Dedekind定理,$B$一定有最小元素,即$S$必有上确界.$\hfill\blacksquare$
\end{proof}
\begin{example}
	利用确界原理证明Dedekind定理,即证明二者的等价关系.
\end{example}
\begin{proof}
	设$\mathbb{R}$上任意一个Dedekind分割为$A/B$,易知$B$中的每个元素都是$A$的一个上界.由确界原理,$A$一定有上确界.设$m=\sup A$,
	
	若$m\in B$,则$A$中无最大元素,假设$B$中无最小元素,则$\exists m'\in B\ s.t.\ m'<m$,由于$m=\sup A$,推出$m'\in A$,这与$m'\in B$矛盾.故$B$中有最小元素.
	
	若$m\in A$,则$A$中有最大元素$m$,假设$B$中有最小元素$n$,则$\frac{m+n}{2}$不属于$A$和$B$,这与$A\cup B=\mathbb{R}$是矛盾的,故$B$中无最小元素.
	
	于是我们就证明了Dedekind定理.$\hfill\blacksquare$
\end{proof}
\section{实数的完备性}
本节内容是基于第二章中数列极限的前提下展开的,建议在学完第二章后进行学习.数列极限的定义与基本性质在此不再赘述.

\subsection{关于实数集完备性的基本定理}
前面我们已经学习了Dedekind分割与确界原理,从不同的角度反映了实数集的特性,通常称为{\heiti 实数的完备性}或{\heiti 实数的连续性}公理.下面我们将介绍另外的几个实数的完备性公理.

\begin{theorem}[单调有界收敛定理]
	在实数系中,有界的单调数列必有极限.
\end{theorem}
\begin{proof}
	不妨设$\left\{a_n\right\}$为有上界的递增数列,由确界原理,$\left\{a_n\right\}$必有上确界,设$a=\sup \left\{a_n\right\}$,根据上确界的定义,
	
	$\forall \varepsilon>0,\ \exists a_N \ s.t.\ a_N>a-\varepsilon$
	
	由$\left\{a_n\right\}$的递增性,当$n\geqslant N$时,有
	$$a-\varepsilon<a_N\leqslant a_n$$
	
	又因为$$a_n\leqslant a<a+\varepsilon$$
	
	故$$a-\varepsilon<a_n<a+\varepsilon$$
	
	即$${\lim_{n \to +\infty}a_n}=a$$
	$\hfill\blacksquare$
\end{proof}

\begin{theorem}[致密性定理]
	任何有界数列必定有收敛的子列.
\end{theorem}
要证明此定理,可以先证明以下引理.
\begin{lemma}\label{zilie}
	任何数列都存在单调子列.
\end{lemma}
\begin{proof}
	设数列为$\left\{a_n\right\}$,下面分两种情况讨论:
	\begin{enumerate}
		\item 若$\forall k\in \mathbb{Z}_+$,$\left\{a_{k+n}\right\}$都有最大项,记$\left\{a_{1+n}\right\}$的最大项为$a_{n_1}$,则$a_{{n_1}+n}$也有最大项,记作$a_{n_2}$,显然有$a_{n_1}\geqslant a_{n_2}$,同理,有$$a_{n_2}\geqslant a_{n_3}$$
		$$.........$$
		由此得到一个单调递减的子列$\left\{a_{n_k}\right\}$
		\item 若至少存在一个正整数$k$,使得$\left\{a_{k+n}\right\}$没有最大项,先取$n_1=k+1$,总存在$a_{n_1}$后面的项$a_{n_2}$($n_2>n_1$)使得$$a_{n_2}>a_{n_1}$$,同理,总存在$a_{n_2}$后面的项$a_{n_3}$($n_3>n_2$)使得$$a_{n_3}>a_{n_2}$$
		$$.........$$
		由此得到一个严格递增的子列$\left\{a_{n_k}\right\}$
	\end{enumerate}
	
	综上,命题得证.$\hfill\blacksquare$
\end{proof}
下面是对致密性定理的证明:
\begin{proof}
	设数列$\left\{a_n\right\}$有界,由引理\ref{zilie},数列$\left\{a_n\right\}$存在单调且有界的子列,由单调有界收敛定理得出该子列是收敛的.$\hfill\blacksquare$
\end{proof}
\begin{theorem}[柯西(Cauchy)收敛准则]
	数列$\left\{a_n\right\}$收敛的充要条件是:\par 
	$\forall \varepsilon>0,\ \exists N\in \mathbb{Z}_+,\ s.t.\ n,\ m>N$时,有
	$$\lvert a_n - a_m \rvert<\varepsilon$$
\end{theorem}
单调有界只是数列收敛的充分条件,而柯西收敛准则给出了数列收敛的充要条件.
\begin{proof}
	{\heiti 必要性}\qquad 设$\lim\limits_{n \to +\infty}a_n=A$,则$\forall \varepsilon>0,\ \exists N\in \mathbb{Z}_+\ s.t.\ n,\ m>N$时,有$$\lvert a_n-A\rvert <\frac{\varepsilon}{2},\ \lvert a_n-A\rvert <\frac{\varepsilon}{2}$$
	
	因而$$\lvert a_n-a_m\rvert \leqslant \lvert a_n-A\rvert + \lvert a_m-A\rvert=\varepsilon$$
	
	{\heiti 充分性}\qquad 先证明该数列必定有界.取$\varepsilon=1$,因为$\left\{a_n\right\}$满足柯西收敛准则的条件,所以$\exists N_0,\ \forall n>N_0$,有
	$$\lvert a_n-a_{N_0+1}\rvert <1$$
	
	取$M=\max\left\{\lvert a_1 \rvert,\ \lvert a_2 \rvert,\ \cdot\cdot\cdot\,\ \lvert a_{N_0} \rvert,\ \lvert a_{N_0+1} \rvert+1\right\}$,则对一切$n$,成立$$\lvert a_n \rvert\leqslant M$$
	
	由致密性原理,在$\left\{a_n\right\}$中必有收敛子列$$\lim_{k \to +\infty}a_{n_k}=\xi$$
	
	由条件,$\forall \varepsilon>0,\ \exists N$,当$n,\ m>N$时,有$$\lvert a_n-a_m\rvert <\frac{\varepsilon}{2}$$
	
	在上式中取$a_m=a_{n_k}$,其中$k$充分大,满足$n_k>N$,并且令$k \to \infty$,于是得到
	$$\lvert a_n-\xi \rvert\leqslant \frac{\varepsilon}{2}< \varepsilon $$
	
	即数列$\left\{a_n\right\}$收敛.$\hfill\blacksquare$
\end{proof}
\begin{definition}[闭区间套]
	设闭区间列$\left\{\left[a_n,b_n\right]\right\}$具有如下性质:
	\begin{enumerate}
		\item $\left[a_n,b_n\right]\supset \left[a_n+1,b_n+1\right],\ n=1,2,\cdots$;
		\item $\lim\limits_{n \to +\infty}(b_n-a_n)=0$.
	\end{enumerate}
	则称$\left\{\left[a_n,b_n\right]\right\}$为{\heiti 闭区间套},或简称{\heiti 区间套}.
\end{definition}

由性质1,构成闭区间套的闭区间列是前一个套着后一个的,即各闭区间端点满足如下不等式:
\begin{equation}\label{chuan}
	a_1\leqslant a_2\leqslant \cdots\leqslant a_n\leqslant\cdots\leqslant b_n\leqslant\cdots\leqslant b_2\leqslant b_1.
\end{equation}
\begin{theorem}[闭区间套定理]
	若$\left\{\left[a_n,b_n\right]\right\}$是一个闭区间套,则在实数系中存在唯一的一点$\xi$,使得$\xi\in\left[a_n,b_n\right],\ n=1,2,\cdots$,即$$a_n\leqslant\xi\leqslant b_n,\ n=1,2,\cdots.$$
\end{theorem}
\begin{proof}
	由式\ref{chuan}可以看出,数列$\left\{a_n\right\}$是递增数列且有界,$\left\{b_n\right\}$是递减数列且有界,由单调有界收敛定理,可知$\left\{a_n\right\}$和$\left\{b_n\right\}$都收敛.设$\lim\limits_{n\to \infty}a_n=\xi$,由闭区间套的第2条性质,得$\lim\limits_{n\to \infty}b_n=\xi.$
	
	$\left\{a_n\right\}$是递增数列,有$a_n\leqslant\xi,\ n=1,2,\cdots;$
	
	$\left\{b_n\right\}$是递减数列,有$b_n\geqslant\xi,\ n=1,2,\cdots.$\\
	所以有$a_n\leqslant\xi\leqslant b_n,\ n=1,2,\cdots.$\\
	下面证明$\xi$的唯一性:\\
	假设存在$\xi'$满足$a_n\leqslant\xi'\leqslant b_n,\ n=1,2,\cdots.$,则\\
	$$\lvert\xi'-\xi\rvert\leqslant b_n-a_n,\ n=1,2,\cdots,$$\\
	由闭区间套的第2条性质,有
	$$\lvert\xi'-\xi\rvert\leqslant \lim\limits_{n\to\infty}(b_n-a_n)=0,\ n=1,2,\cdots,$$\\
	故$\xi'=\xi$.$\hfill\blacksquare$
\end{proof}
\begin{definition}
	设$S$为数轴上的点集,$H$为开区间的集合(即$H$的每一个元素都是形如$(\alpha,\beta)$的开区间).若$S$中的任何一点都含在$H$中至少一个开区间内,则称$H$为$S$的一个{\heiti 开覆盖},或称$H$覆盖$S$.若$H$中开区间的个数是无限(有限)的,则称$H$为$S$的一个{\heiti 无限开覆盖(有限开覆盖)}.若存在$S$的开覆盖$H'\subseteq H$,则称$H'$是$H$的{\heiti 子覆盖},特别地,当$H'$中含有的开区间的个数为有限个时,称$H'$为$H$的{\heiti 有限子覆盖}.
\end{definition}
\begin{theorem}[Heine-Borel有限覆盖定理]
	设$H$是闭区间$\left[a,b\right]$的一个(无限)开覆盖,则从$H$中能选出有限个开区间来覆盖$\left[a,b\right]$.
	
	即:有限闭区间的任一开覆盖都存在一个有限子覆盖.
\end{theorem}
\begin{proof}
	设$H$是闭区间$\left[a,b\right]$的一个开覆盖,定义集合
	$$S=\left\{x|x\in \left(a,b \right],\ \mbox{且}\left[a,x\right]\mbox{存在开覆盖}H\mbox{的一个有限子覆盖} \right\}.$$
	
	因为$H$是$\left[a,b\right]$的一个开覆盖,所以存在一个区间$I_0\in H$使得$a\in I_0$,则存在$x_0\in I_0$满足$x_0>a$,所以$S\neq\varnothing$.显然$b$是$S$的一个上界,由确界原理,$S$一定有上确界.设$M=\sup S\leqslant b$,下面证明$M=b$:
	
	反证法\qquad 假设$M<b$,则$M\in \left(a,b \right]$,$\left[a,M\right]$存在$H$的一个有限子覆盖.假设开区间$I_1$包含$M$,则存在$\delta >0$使得$(M-\delta,M+\delta)\subseteq I_1$,因为$M$是$S$的上确界,所以$M-\delta\in S$,记$\left[a,M-\delta\right]$的有限开覆盖为$H'$,则$\left[a,M+\delta\right]$也有有限开覆盖$H'\cup I_1$,得$M+\delta\in S$这与$M$是$S$的上确界矛盾.所以$M=b$,即$\left[a,b\right]$的开覆盖$H$存在一个有限子覆盖.$\hfill\blacksquare$
\end{proof}
\begin{remark}
	法国数学家Borel于1895年第一次陈述并证明了现代形式的Heine-Borel定理.此定理只对有限闭区间成立,而对开区间则不一定成立.例如开区间集合$$\left\{(\frac{1}{n+1},1)\right\},\ (n=1,2,\cdots)$$构成了开区间$(0,1)$的开覆盖,但不能从中选出有限个开区间覆盖住$(0,1)$.
\end{remark}
从上面的讨论我们发现,如果从数轴上取下一段“紧致无缝”的集合(含端点),那么就可以从它的任意开覆盖中取出一个有限子覆盖,否则就不行.这表明我们找到了一个刻画实数集完备性的新方法,我们形象地将这个性质称为“紧致性”.
\begin{definition}[紧致集]
	设集合$E\in\mathbb{R}$,若集合$E$的任一开覆盖都存在一个有限子覆盖,则称$E$为$\mathbb{R}$上的一个{\heiti 紧致集}.
\end{definition}
\begin{remark}
	紧致集也称紧集,是一个重要的拓扑概念.
\end{remark}
\begin{remark}
	以后我们会将以上条件称为"Heine-Borel条件".
\end{remark}
我们可以重新表述Heine-Borel有限覆盖定理.
\begin{theorem}[Heine-Borel有限覆盖定理]
	$\mathbb{R}$中的任一有限闭区间都是紧致集.
\end{theorem}
\begin{definition}[邻域]
	设$a\in\mathbb{R},\ \delta>0$,将满足$\lvert x-a\rvert<\delta$的全体$x$的集合称为{\heiti $a$的$\delta$邻域},记作$U(a,\delta)$.将满足$0<\lvert x-a\rvert<\delta$的全体$x$的集合称为{\heiti $a$的$\delta$去心邻域},记作$\mathring{U}(a,\delta)$.
\end{definition}
显然,邻域与去心邻域的区别在于去心邻域不包含中心点$a$.
\begin{definition}[聚点]
	设$S$是数轴上的点集,$\xi$是一个定点(可以在$S$中也可以不在$S$中),若$\xi$的任一邻域中都含有$S$中无穷多个点,则称$\xi$为$S$的一个聚点.
\end{definition}
聚点的另一定义如下:
\begin{definition}
	若存在各项互异的收敛数列$\left\{x_n\right\}\subset S$,则其极限$\lim\limits_{n\to \infty}x_n=\xi$是$S$的一个聚点.
\end{definition}
\begin{theorem}[Weierstrass聚点定理]
	实轴上任一有界无限点集$S$至少有一个聚点.
\end{theorem}
由聚点的等价定义,该定理也可叙述为:{\heiti 有界数列必有收敛子列},即致密性定理.
\subsection{实数集完备性定理的等价关系}
通过前面的学习,我们共有以下8个基本定理来叙述实数的完备性:
\begin{enumerate}
	\item Dedekind定理;
	\item 确界原理;
	\item 单调有界收敛定理;
	\item 致密性定理;
	\item 柯西收敛准则;
	\item 闭区间套定理;
	\item Heine-Borel有限覆盖定理;
	\item Weierstrass聚点定理.
\end{enumerate}
可以证明,这8个基本定理都是等价的.(证明会在后续修正时给出)
\newpage

\chapter{数列极限}
\section{数列极限的定义}
数列极限在我们直观上是说对于数列$\left\{a_n\right\}$,当$n$越来越大(趋近于$\infty$)时,$a_n$越来越接近于一个确定的数$a$,以至于在$n$取到$\infty$的时候。$a_n$就等于$a$了.相信读者在直观上对数列极限的概念都有或多或少的认识.然而,这种叙述不够严谨.“$n$越来越大”是多大?“趋近于$\infty$”是如何趋近的?“$a_n$越来越接近一个确定的数”是说$a_n$和$a$总是随着$n$的增大而越接近的吗?是否存在$a_n$与$a$在距离上的波动性?以上问题都是由于极限的定义不够清晰所导致的,因此,我们采用更严谨的$\varepsilon-N$语言定义数列极限.
\begin{definition}[数列极限]
	对于数列$\left\{a_n\right\}$,$a$为一个确定的数,对$\forall \varepsilon>0,\ \exists N,\ s.t.\ n>N$时,有$\lvert a_n-a\rvert<\varepsilon$,则称数列$\left\{a_n\right\}$收敛于$a$,$a$就称为数列$\left\{a_n\right\}$的极限.记作$$\lim\limits_{n\to\infty}a_n=a\mbox{或者}a_n\to a,\ (n\to \infty)$$
\end{definition}
若数列$\{a_n\}$没有极限,则称$\{a_n\}$不收敛,或称$\{a_n\}$为{\heiti 发散数列}.
\begin{example}
	证明$\lim\limits_{n\to\infty}\frac{1}{n}=0$.
\end{example}
\begin{proof}
	$\forall \varepsilon>0$,取$N=\frac{1}{\varepsilon}$,当$n>N$时,$\lvert a_n-0\rvert<\varepsilon$,所以$\lim\limits_{n\to\infty}\frac{1}{n}=0$.$\hfill\blacksquare$
\end{proof}
例题在此不过多展示,可以参考任意一本数学分析教材.
经过一部分练习,我们对数列极限的定义有了更深刻的印象.下面我们对$\varepsilon-N$语言进行进一步的解释.

$\varepsilon$是任意的正实数,也就是说它可以无限接近于$0$,那么$2\varepsilon$,$3\varepsilon$,$\frac{\varepsilon}{2}$等都是任意的正实数,同样可以无限接近于$0$,也就反映$a_n$和$a$的距离会随着$n$的增大而非常接近.这里的$N$可以看作$n$增大的一个阈值,超过这个阈值,$a_n$和$a$的距离总是小于一个限度$\varepsilon$.我们将证明某个数列极限的步骤总结如下:
\begin{enumerate}
	\item 列出前提条件:$\forall \varepsilon>0$.
	\item 取$N$,解不等式$\lvert a_n-a \rvert<\varepsilon$,得到$n>f(\varepsilon)$($f(\varepsilon)$表示不等式右边是关于$\varepsilon$的表达式),我们可以取$N=f(\varepsilon)$或者取$N=\left[f(\varepsilon)\right]+1$(取整后加1即将$f(\varepsilon)$向上取整).
	\item 将上述取到的$N$按数列极限的定义书写,证明完毕.
\end{enumerate}

我们在上述例题中看出,$N$的取值是与$\varepsilon$有关的,所以$N$也可以写成$N(\varepsilon)$,一般来说,$\varepsilon$的值越小,取到的$N$就会越大,$\varepsilon$是任意正实数,当它充分接近于$0$时,$N$就趋向于无穷.

按定义,我们也可以这样描述数列极限:
\begin{definition}[数列极限]
	对$\varepsilon>0$,数列$\left\{a_n\right\}$至多有有限项落在$a$的$\varepsilon$邻域之外,则称$a$的数列$\left\{a_n\right\}$的极限.
\end{definition}
在所有收敛数列中,有一类重要的数列,称为无穷小数列,其定义如下.
\begin{definition}[无穷小数列]
	若$\lim\limits_{n\to\infty}a_n=0$,则称$\{a_n\}$为{\heiti 无穷小数列}.
\end{definition}
由无穷小数列的定义,不难证明以下定理.
\begin{theorem}
	数列$\{a_n\}$收敛于$a$的充要条件是:$|a_n-a|$是无穷小数列.
\end{theorem}
最后我们简单介绍一下无穷大数列.
\begin{definition}[无穷大数列]
	若数列$\{a_n\}$满足:对任意正数$M>0$,总存在正整数$N$,使得当$n>N$时,有$$|a_n|>M,$$则称数列$\{a_n\}$发散于无穷大,并记作
	$$\lim\limits_{n\to\infty}a_n=\infty\mbox{或}a_n\to\infty.$$
	我们称这样的数列$\{a_n\}$为{\heiti 无穷大数列}.
\end{definition}
无穷大数列虽然不收敛,但却有一定的变化趋势.关于无穷小数列和无穷大数列的关系,我们有以下的定理.
\begin{theorem}
	对于数列$\{x_n\},\ \forall n\in \mathbb{Z}_+,\ x_n\neq 0$,则数列$\{x_n\}$为无穷小数列$\iff$ $\{\frac{1}{x_n}\}$为无穷大数列.
\end{theorem}
\begin{proof}
	充分性\qquad $\lim\limits_{n\to\infty}x_n=0$,即$\forall \varepsilon>0,\ \exists N,\ s.t.\ \forall n>N,\ |x_n|<\varepsilon$,则$\frac{1}{x_n}>\frac{1}{\varepsilon}$,取$M=\frac{1}{\varepsilon}$,由于$\varepsilon$的任意性,$M$可以取到任意正数.所以有$\frac{1}{x_n}>M$,即$\frac{1}{x_n}$为无穷大数列.
	
	必要性\qquad $\lim\limits_{n\to\infty}\frac{1}{x_n}=\infty$,即$\forall M>0,\ \exists N,\ s.t.\ \forall n>N,\ |\frac{1}{x_n}|>M$,则$|x_n|<\frac{1}{M}$,取$\varepsilon=\frac{1}{M}$,由于$M$的任意性,$\varepsilon$可以取到任意正数.所以有$|x_n|=|x_n-0|<\varepsilon$,即$x_n$为无穷小数列.
	$\hfill\blacksquare$
\end{proof}
\section{收敛数列的性质}
收敛数列有如下一些重要性质.
\begin{theorem}[唯一性]
	若数列$\left\{a_n\right\}$收敛,则它只有一个极限.
\end{theorem}
\begin{proof}
	设$a$是收敛数列$\left\{a_n\right\}$的一个极限,对$\forall b\neq a$,取$\varepsilon_0=\frac{1}{2}|b-a|$,则$U(b,\varepsilon_0)$中至多有数列$\left\{a_n\right\}$的有限个项,则$b$不是数列$\left\{a_n\right\}$的极限.这就证明了收敛数列极限的唯一性.$\hfill\blacksquare$
\end{proof}
一个收敛数列一般含有无穷多个数,而它的极限只是一个数,我们单凭这一个数就能精确地估计出几乎全体项的大小.以下收敛数列的一些性质,大都基于这一事实.
\begin{theorem}[有界性]
	收敛数列必有界.
\end{theorem}
\begin{proof}
	设数列$\left\{a_n\right\}$收敛,$\lim\limits_{n\to\infty}a_n=a$,取$\varepsilon=1$,则由数列极限的定义,$\exists N\in\mathbb{Z}_+,\ s.t.\ \forall n>N$,有
	
	\begin{center}
		$\lvert a_n-a\rvert<1$,即$a-1<a_n<a+1$.
	\end{center}
	
	记$$M=\max\{|a_1|,|a_2|,\cdots,|a_N|,|a-1|,|a+1|\},$$
	则对$\forall n\in\mathbb{Z}_+$都有$|a_n|<M$.
	
	即数列$\left\{a_n\right\}$有界.$\hfill\blacksquare$
\end{proof}
\begin{remark}
	有界性是数列收敛的必要条件,而非充分条件.即有界不一定收敛,收敛一定有界.
\end{remark}
\begin{theorem}[保号性]
	若$\lim\limits_{n\to\infty}a_n=a>0$(或$<0$),则对任何$a'\in(0,a)$(或$a'\in (a,0)$),存在正数$N$,使得当$n>N$时,有$a_n>a'$(或$a_n<a'$).
\end{theorem}
\begin{proof}
	对于$a>0$的情形,取$\varepsilon=a-a'(>0)$,则存在正数$N$,使得当$n>N$时,有$a_n>a-\varepsilon=a'$,同理可证$a<0$的情形.$\hfill\blacksquare$
\end{proof}
\begin{corollary}[保号性推论]
	设$\lim\limits_{n\to\infty}a_n=a,\ \lim\limits_{n\to\infty}b_n=b,\ a<b$,则存在$N$,使得当$n>N$时,有$a_n<b_n.$
\end{corollary}
\begin{proof}
	因为$a<\frac{a+b}{2}<b$,所以由保号性,存在$N_1$,当$n>N_1$时,有$$a_n<\frac{a+b}{2};$$存在$N_2$,当$n>N_2$时,有$$b_n>\frac{a+b}{2}.$$取$N=\max\{N_1,N_2\}$,当$n>N$时,有$$a_n<b_n.$$$\hfill\blacksquare$
\end{proof}
\begin{theorem}[保序性]
	设$\{a_n\}$和$\{b_n\}$都是收敛数列,若存在正数$N_0$,使得当$n>N_0$时,有$a_n\leqslant b_n$,则$\lim\limits_{n\to\infty}a_n\leqslant \lim\limits_{n\to\infty}b_n$.
\end{theorem}
\begin{proof}
	设$\lim\limits_{n\to\infty}a_n=a,\ \lim\limits_{n\to\infty}b_n=b$.对于任意$\varepsilon>0$,分别存在正数$N_1$和$N_2$,当$n>N_1$时,有$$a-\varepsilon<a_n$$
	当$n>N_2$时,有$$b_n<b+\varepsilon$$
	取$N=\max\{N_0,N_1,N_2\}$,当$n>N$时,有$$a-\varepsilon<a_n\leqslant n-b_n<b+\varepsilon,$$由此得到$a<b+2\varepsilon$,由$\varepsilon$的任意性得$a\leqslant b$,即$\lim\limits_{n\to\infty}a_n\leqslant \lim\limits_{n\to\infty}b_n$.$\hfill\blacksquare$
\end{proof}
\begin{theorem}[迫敛性]
	设收敛数列$\{a_n\},\ \{b_n\}$都以$a$为极限,数列$\{c_n\}$满足:存在正数$N_0$,当$n>N_0$时,有$$a_n\leqslant c_n\leqslant b_n,$$则数列$\{c_n\}$收敛,且$\lim\limits_{n\to\infty}c_n=a$
\end{theorem}
\begin{proof}
	$\lim\limits_{n\to\infty}a_n=\lim\limits_{n\to\infty}b_n=a$,则对$\forall\varepsilon>0,\ \exists N_1,\ N_2>0,\ s.t.$
	$$a-\varepsilon<a_n;$$
	$$b_n<a+\varepsilon.$$
	取$N=\max\{N_0,N_1,N_2\}$,当$n>N$时,有
	$$a-\varepsilon	<a_n\leqslant c_n\leqslant b_n<a+\varepsilon$$
	从而有$|c_n-a|<\varepsilon$,即$\lim\limits_{n\to\infty}c_n=a.$$\hfill\blacksquare$
\end{proof}
\begin{theorem}[四则运算法则]
	若$\{a_n\}$与$\{b_n\}$为收敛数列,则$\{a_n\pm b_n\},\ \{a_n\cdot b_n\}$也都是收敛数列,且$$\lim\limits_{n\to\infty}(a_n\pm b_n)=\lim\limits_{n\to\infty}a_n\pm \lim\limits_{n\to\infty}b_n,$$
	$$\lim\limits_{n\to\infty}(a_n\cdot b_n)=\lim\limits_{n\to\infty}a_n\cdot \lim\limits_{n\to\infty}b_n.$$
	特别当$\{b_n\}$为常数$c$时,有
	$$\lim\limits_{n\to\infty}(a_n+c)=\lim\limits_{n\to\infty}a_n+c,\ \lim\limits_{n\to\infty}ca_n=c\lim\limits_{n\to\infty}a_n.$$
	若再假设$b_n\neq 0$及$\lim\limits_{n\to\infty}b_n\neq 0$,则$\{\frac{a_n}{b_n}\}$也是收敛数列,且有
	$$\lim\limits_{n\to\infty}\frac{a_n}{b_n}=\frac{\lim\limits_{n\to\infty}a_n}{\lim\limits_{n\to\infty}b_n}.$$
\end{theorem}
\begin{proof}
	由于$a_n-b_n=a_n+(-1)b_n$,$\frac{a_n}{b_n}=a_n\cdot \frac{1}{b_n}$,因此我们只需要证明关于和、积、倒数运算的结论即可.
	
	设$\lim\limits_{n\to\infty}a_n=a,\ \lim\limits_{n\to\infty}b_n=b$,则对$\forall\varepsilon>0$,分别存在正数$N_1$和$N_2$,使得
	$$|a_n-a|<\varepsilon,\mbox{当}n>N_1,$$
	$$|b_n-b|<\varepsilon,\mbox{当}n>N_2.$$
	取$N=\max\{N_1,N_2\}$,则当$n>N$时上述两不等式同时成立,从而有
	\begin{enumerate}
		\item 	$|(a_n+b_n)-(a+b)|\leqslant |a_n-a|+|b_n-b|<2\varepsilon$
		,这就证得$\lim\limits_{n\to\infty}(a_n+b_n)=\lim\limits_{n\to\infty}a_n+\lim\limits_{n\to\infty}b_n.$
		\item	$|a_nb_n-ab|=|(a_n-a)b_n+a(b_n-b)|\leqslant |b_n||a_n-a|+|a||b_n-b|.$
		由收敛数列的有界性可知,存在一正数$M$,使得$|b_n|<M$,则有
		$$|a_nb_n-ab|\leqslant |b_n||a_n-a|+|a||b_n-b|<(M+|a|)\varepsilon.$$
		由$\varepsilon$的任意性,这就证得$\lim\limits_{n\to\infty}a_nb_n=ab.$
		\item  由于$\lim\limits_{n\to\infty}b_n=b\neq 0$,根据收敛数列的保号性,存在正数$N_3$,使得当$n>N_3$时,有$|b_n|>\frac{1}{2}|b|$.取$N'=\max\{N_2,N_3\}$,则当$n>N'$时,有
		$$\left|\frac{1}{b_n}-\frac{1}{b}\right|=\frac{|b_n-b|}{|b_n b|}<\frac{2|b_n-b|}{b^2}<\frac{2\varepsilon}{b^2}.$$
		由$\varepsilon$的任意性,这就证得$\lim\limits_{n\to\infty}\frac{1}{b_n}=\frac{1}{b}.$
		$\hfill\blacksquare$
	\end{enumerate}
\end{proof}
\begin{definition}[子列]
	设$\{a_n\}$为数列,$\{n_k\}$为正整数集$\mathbb{Z}_+$的无限子集,且$n_1<n_2<\cdots <n_k<\cdots$,则数列
	$$a_{n_1},a_{n_2},\cdots,a_{n_k},\cdots$$
	称为数列$\{a_n\}$的一个子列,记为$\{a_{n_k}\}$.
\end{definition}
\begin{theorem}
	数列$\{a_n\}$收敛的充要条件是:$\{a_n\}$的任何子列都收敛.
\end{theorem}
\begin{proof}
	充分性\qquad 因为$\{a_n\}$也是自身的一个子列,所以结论是显然的.
	
	必要性\qquad 设$\lim\limits_{n\to \infty}a_n=a$,$\{a_{n_k}\}$是$\{a_n\}$的任一子列.则对$\forall \varepsilon>0,\ \exists N>0,\ s.t.\ \forall k>N$时,有$|a_k-a|<\varepsilon$.又因为$n_k\geqslant k$,故$k>N$时,有$|a_{n_k}-a|<\varepsilon$.这就证明了任一子列$\{a_{n_k}\}$收敛(且与$\{a_n\}$有相同的极限).
	$\hfill\blacksquare$
\end{proof}
\section{Stolz定理}


\section{数列极限存在的条件}
在研究比较复杂的数列极限问题时,通常先考虑数列极限的存在性问题,若有极限,再考虑数列的计算问题.在实际应用中,解决的数列极限的存在性问题,即使极限值的计算较为困难,但由于当$n$充分大时,$a_n$能充分接近其极限,故可用$a_n$来作为极限的近似值,本节将重点讨论极限的存在性问题.
\subsection{数列极限的存在性定理}
\begin{definition}[数列的单调性]
	\begin{enumerate}
		\item 若数列$a_n$满足$$a_n\leqslant a_{n+1},\forall n\in \mathbb{Z}_+,$$则称该数列是{\heiti 递增}的,其中,若有$$a_n<a_{n+1}$$成立,则称该数列是{\heiti 严格递增}的;
		\item 若数列$a_n$满足$$a_n\geqslant a_{n+1},\forall n\in \mathbb{Z}_+,$$则称该数列是{\heiti 递减}的,其中,若有$$a_n>a_{n+1}$$成立,则称该数列是{\heiti 严格递减}的.
		\item 递增数列和递减数列统称为{\heiti 单调}数列(无论是否严格).
	\end{enumerate}
\end{definition}
\begin{theorem}[单调有界收敛定理]
	在实数系中,有界的单调数列必有极限.
\end{theorem}
\begin{proof}
	不妨设$\left\{a_n\right\}$为有上界的递增数列,由确界原理,$\left\{a_n\right\}$必有上确界,设$a=\sup \left\{a_n\right\}$,根据上确界的定义,
	
	$\forall \varepsilon>0,\ \exists a_N \ s.t.\ a_N>a-\varepsilon$
	
	由$\left\{a_n\right\}$的递增性,当$n\geqslant N$时,有
	$$a-\varepsilon<a_N\leqslant a_n$$
	
	又因为$$a_n\leqslant a<a+\varepsilon$$
	
	故$$a-\varepsilon<a_n<a+\varepsilon$$
	
	即$${\lim_{n \to +\infty}a_n}=a$$
	$\hfill\blacksquare$
\end{proof}

\begin{theorem}[致密性定理]
	任何有界数列必定有收敛的子列.
\end{theorem}
要证明此定理,可以先证明以下引理.
\begin{lemma}\label{zilie}
	任何数列都存在单调子列.
\end{lemma}
\begin{proof}
	设数列为$\left\{a_n\right\}$,下面分两种情况讨论:
	\begin{enumerate}
		\item 若$\forall k\in \mathbb{Z}_+$,$\left\{a_{k+n}\right\}$都有最大项,记$\left\{a_{1+n}\right\}$的最大项为$a_{n_1}$,则$a_{{n_1}+n}$也有最大项,记作$a_{n_2}$,显然有$a_{n_1}\geqslant a_{n_2}$,同理,有$$a_{n_2}\geqslant a_{n_3}$$
		$$.........$$
		由此得到一个单调递减的子列$\left\{a_{n_k}\right\}$
		\item 若至少存在一个正整数$k$,使得$\left\{a_{k+n}\right\}$没有最大项,先取$n_1=k+1$,总存在$a_{n_1}$后面的项$a_{n_2}$($n_2>n_1$)使得$$a_{n_2}>a_{n_1}$$,同理,总存在$a_{n_2}$后面的项$a_{n_3}$($n_3>n_2$)使得$$a_{n_3}>a_{n_2}$$
		$$.........$$
		由此得到一个严格递增的子列$\left\{a_{n_k}\right\}$
	\end{enumerate}
	
	综上,命题得证.$\hfill\blacksquare$
\end{proof}
下面是对致密性定理的证明:
\begin{proof}
	设数列$\left\{a_n\right\}$有界,由引理\ref{zilie},数列$\left\{a_n\right\}$存在单调且有界的子列,由单调有界收敛定理得出该子列是收敛的.$\hfill\blacksquare$
\end{proof}
\begin{theorem}[柯西(Cauchy)收敛准则]
	数列$\left\{a_n\right\}$收敛的充要条件是:\par 
	$\forall \varepsilon>0,\ \exists N\in \mathbb{Z}_+,\ s.t.\ n,\ m>N$时,有
	$$\lvert a_n - a_m \rvert<\varepsilon$$
\end{theorem}
单调有界只是数列收敛的充分条件,而柯西收敛准则给出了数列收敛的充要条件.
\begin{proof}
	{\heiti 必要性}\qquad 设$\lim\limits_{n \to +\infty}a_n=A$,则$\forall \varepsilon>0,\ \exists N\in \mathbb{Z}_+\ s.t.\ n,\ m>N$时,有$$\lvert a_n-A\rvert <\frac{\varepsilon}{2},\ \lvert a_n-A\rvert <\frac{\varepsilon}{2}$$
	
	因而$$\lvert a_n-a_m\rvert \leqslant \lvert a_n-A\rvert + \lvert a_m-A\rvert=\varepsilon$$
	
	{\heiti 充分性}\qquad 先证明该数列必定有界.取$\varepsilon=1$,因为$\left\{a_n\right\}$满足柯西收敛准则的条件,所以$\exists N_0,\ \forall n>N_0$,有
	$$\lvert a_n-a_{N_0+1}\rvert <1$$
	
	取$M=\max\left\{\lvert a_1 \rvert,\ \lvert a_2 \rvert,\ \cdot\cdot\cdot\,\ \lvert a_{N_0} \rvert,\ \lvert a_{N_0+1} \rvert+1\right\}$,则对一切$n$,成立$$\lvert a_n \rvert\leqslant M$$
	
	由致密性原理,在$\left\{a_n\right\}$中必有收敛子列$$\lim_{k \to +\infty}a_{n_k}=\xi$$
	
	由条件,$\forall \varepsilon>0,\ \exists N$,当$n,\ m>N$时,有$$\lvert a_n-a_m\rvert <\frac{\varepsilon}{2}$$
	
	在上式中取$a_m=a_{n_k}$,其中$k$充分大,满足$n_k>N$,并且令$k \to \infty$,于是得到
	$$\lvert a_n-\xi \rvert\leqslant \frac{\varepsilon}{2}< \varepsilon $$
	
	即数列$\left\{a_n\right\}$收敛.$\hfill\blacksquare$
\end{proof}
\subsection{自然常数与Euler常数}
\begin{example}
	证明极限$\lim\limits_{n\to\infty}(1+\frac{1}{n})^n$存在.
\end{example}
\begin{proof}
	设$a_n=(1+\frac{1}{n})^n,\ n=1,2,\cdots .$由二项式定理,
	
	\begin{flalign*}
		a_n&=(1+\frac{1}{n})^n\\
		&=1+C_n^1\frac{1}{n}+\cdots+C_n^k\frac{1}{n^k}+\cdots+C_n^n\frac{1}{n}\\
		&=1+1+\frac{n(n+1)}{2!}\frac{1}{n^2}+\cdots+\frac{n(n-1)\cdots(n-k+1)}{k!}\frac{1}{n^k}+\cdots+\frac{1}{n^n}\\
		&=2+\frac{1}{2!}(1-\frac{1}{n})+\cdots+\frac{1}{k!}(1-\frac{1}{n})(1-\frac{2}{n})\cdots(1-\frac{k-1}{n})+\cdots+\\
		&\quad\frac{1}{n!}(1-\frac{1}{n})(1-\frac{2}{n})\cdots(1-\frac{n-1}{n})\\
		&<2+\frac{1}{2!}(1-\frac{1}{n+1})+\cdots+\frac{1}{k!}(1-\frac{1}{n+1})(1-\frac{2}{n+1})\cdots(1-\frac{k-1}{n+1})+\cdots+\\
		&\quad\frac{1}{(n+1)!}(1-\frac{1}{n+1})(1-\frac{2}{n+1})\cdots(1-\frac{n}{n+1})\\
		&=a_{n+1},
	\end{flalign*}
	
	故$\{a_n\}$是严格递增的.由上式可推得
	\begin{align*}
		a_n
		&<2+\frac{1}{2!}+\cdots+\frac{1}{k!}+\cdots+\frac{1}{n!}\\
		&<2+\frac{1}{1\cdot2}+\cdots+\frac{1}{(k-1)k}+\cdots+\frac{1}{(n-1)n}\\
		&=2+(1-\frac{1}{2})+\cdots+(\frac{1}{k-1}-\frac{1}{k})+\cdots+(\frac{1}{n-1}-\frac{1}{n})\\
		&=3-\frac{1}{n}<3.
	\end{align*}
	这表明$a_n$是有界的.由单调有界收敛定理可知$\lim\limits_{n\to\infty}(1+\frac{1}{n})^n$存在.$\hfill\blacksquare$
\end{proof}
\begin{remark}
	通常用拉丁字母e代表该数列的极限,即
	$$\lim\limits_{n\to\infty}(1+\frac{1}{n})^n=\text{e}.$$
	它是一个无理数(待证),其前几位数字是
	$$\text{e} \approx 2.718281828459045.$$
\end{remark}
\begin{example}
	证明$\lim\limits_{n\to\infty}(1+\frac{1}{n})^{n+1}$严格单调递减趋于e.
\end{example}
\begin{proof}
	$\lim\limits_{n\to\infty}(1+\frac{1}{n})^{n+1}=\lim\limits_{n\to\infty}(1+\frac{1}{n})^n(1+\frac{1}{n})=\lim\limits_{n\to\infty}\text{e}(1+\frac{1}{n})=\text{e}.$
	设$a_n=(1+\frac{1}{n})^{n+1}$,下面只需证明$\{a_n\}$单调递减,即$$a_n>a_{n+1}.$$
	
	由均值不等式,
	\begin{align*}
		\frac{1}{a_n}=(\frac{n}{n+1})^{n+1}
		=1\cdot \underbrace{\frac{n}{n+1}\cdot\cdots\cdot\frac{n}{n+1}}_{n+1\text{个}}<(\frac{n+1}{n+2})^{n+2}=\frac{1}{a_{n+1}}
	\end{align*}
	
	故$a_n>a_{n+1}$.$\hfill\blacksquare$
\end{proof}
由上面两道例题,我们可以得出如下不等式:
\begin{equation}
	(1+\frac{1}{n})^n<\text{e}<(1+\frac{1}{n})^{n+1}.\ n=1,2,\cdots
\end{equation}

分别对两个不等号取对数,有
$$n\ln(1+\frac{1}{n})<1;$$
$$(n+1)\ln(1+\frac{1}{n})>1$$

得
\begin{equation}{\label{equ:xln}}
	\frac{1}{n+1}<\ln(1+\frac{1}{n})<\frac{1}{n}.\ n=1,2,\cdots
\end{equation}

将$n=1,2,\cdots,n$依次代入,得
$$\frac{1}{2}<\ln\frac{2}{1}<\frac{1}{1}$$
$$\frac{1}{3}<\ln\frac{3}{2}<\frac{1}{2}$$
$$\cdots$$
$$\frac{1}{n+1}<\ln\frac{n+1}{n}<\frac{1}{n}$$

将上述各式相加,得
\begin{equation}{\label{tiaohe}}
	\frac{1}{2}+\frac{1}{3}+\cdots+\frac{1}{n+1}<\ln(n+1)<1+\frac{1}{2}+\cdots+\frac{1}{n}.\ n=1,2,\cdots
\end{equation}

以上三个不等式都非常有用.
\begin{proposition}
	设数列$$\widetilde{e_n}=1+\frac{1}{1!}+\frac{1}{2!}+\cdots+\frac{1}{n!},\ n=1,2,\cdots$$
	则$\lim\limits_{n\to\infty}\widetilde{e_n}=\text{e}.$
\end{proposition}
\begin{proof}
	在对$\lim\limits_{n\to\infty}(1+\frac{1}{n})^n=\text{e}$的证明过程中已知$\widetilde{e_n}$是有界的,显然$\widetilde{e_n}$是单调递增的,有单调有界收敛定理可知$\{\widetilde{e_n}\}$收敛.
	
	(i)由于
	$$e_n=(1+\frac{1}{n})^n\leqslant1+\frac{1}{1!}+\frac{1}{2!}+\cdots+\frac{1}{n!}=\widetilde{e_n},$$
	由极限的保序性可知
	$\lim\limits_{n\to\infty}e_n\leqslant\lim\limits_{n\to\infty}\widetilde{e_n}.$
	
	(ii)给定$m\leqslant n$,则
	$$e_n=\sum_{i=0}^{n}\frac{1}{i!}(1-\frac{1}{n})(1-\frac{2}{n})\cdots(1-\frac{i-1}{n})\geqslant\sum_{i=0}^{m}\frac{1}{i!}(1-\frac{1}{n})(1-\frac{2}{n})\cdots(1-\frac{i-1}{n}).$$
	令$n\to\infty$,得
	$$\lim\limits_{n\to \infty}e_n\geqslant\sum_{i=0}^{m}\frac{1}{i}$$
	令$m\to\infty$,得$\lim\limits_{n\to\infty}e_n\geqslant\lim\limits_{n\to\infty}\widetilde{e_n}.$
	
	综合(i)和(ii)可知$\lim\limits_{n\to\infty}\widetilde{e_n}=\lim\limits_{n\to\infty}e_n=\text{e}.$
	$\hfill\blacksquare$
\end{proof}
在实际应用中,我们常用数列$\{\widetilde{e_n}\}$来计算e的值,这是因为$\{\widetilde{e_n}\}$收敛得更快.
\begin{theorem}
	自然常数e是一个无理数.
\end{theorem}

\begin{proof}
	反证法\qquad 假设e是一个有理数,则$\text{e}=\frac{p}{q},\ p,q\in\mathbb{Z}_+$,因为$2<\text{e}<3$,所以e不是整数,所以有$q\geqslant 2$.
	
	\begin{align*}
		\widetilde{e}_{n+m}-\widetilde{e_n}
		&=\frac{1}{(n+1)!}+\frac{1}{(n+2)!}+\cdots+\frac{1}{(n+m)!}\\
		&=\frac{1}{(n+1)!}\left[1+\frac{1}{n+2}+\cdots+\frac{1}{(n+2)\cdots (n+m)}\right]\\
		&<\frac{1}{(n+1)!}\left[1+\frac{1}{n+2}+\cdots+\left(\frac{1}{n+2}\right)^{m-1}\right]\\
		&=\frac{1}{(n+1)!}\cdot \frac{1-\left(\frac{1}{n+2}\right)^m}{1-\frac{1}{n+2}}.
	\end{align*}
	
	令$m\to\infty$,则
	
	\begin{align*}
		\text{e}-\widetilde{e_n}
		&=\frac{1}{(n+1)!}\cdot \frac{1-\left(\frac{1}{n+2}\right)^m}{1-\frac{1}{n+2}}\\
		&\leqslant\frac{1}{(n+1)!}\cdot \frac{1}{1-\frac{1}{n+2}}\\
		&<\frac{1}{(n+1)!}\cdot \frac{1}{1-\frac{1}{n+1}}\\
		&=\frac{1}{(n+1)}\cdot\frac{n+1}{n}\\
		&=\frac{1}{n!n}.
	\end{align*}
	
	则
	\begin{align}
		q!(\text{e}-\widetilde{e_n})&\leqslant \frac{1}{q}\leqslant\frac{1}{2}\notin \mathbb{Z}\\
		\nonumber
		q!(\text{e}-\widetilde{e_n})
		&=q!\left[\frac{p}{q}-\left(1+\frac{1}{1!}+\frac{1}{2!}+\cdots+\frac{1}{q!}\right)\right]\\
		&=(q-1)!p-(q!+2\cdot3\cdots q+3\cdot4\cdots q+\cdots+1)\in\mathbb{Z}
	\end{align}
	
	这里出现矛盾,假设不成立,因此e是无理数.$\hfill\blacksquare$
\end{proof}
\begin{proposition}
	有两个数列$\{x_n\}$,$\{y_n\}$,
	$$x_n=1+\frac{1}{2}+\cdots+\frac{1}{n}-\ln(n+1)$$
	$$y_n=1+\frac{1}{2}+\cdots+\frac{1}{n}-\ln n$$
	则$\{x_n\}$和$\{y_n\}$收敛到同一实数.
\end{proposition}
\begin{proof}
	(i)先证明$\{x_n\}$收敛,由不等式\ref{equ:xln},
	\begin{align*}
		x_{n+1}-x_n&=\frac{1}{n+1}-\ln(n+2)+\ln(n+1)\\
		&=\frac{1}{n+1}-\ln\left(1+\frac{1}{n+1}\right)>0
	\end{align*}
	所以$\{x_n\}$严格递增.
	
	由不等式\ref{tiaohe},
	$$1+\frac{1}{2}+\cdots+\frac{1}{n}-\ln(n+1)<1-\frac{1}{n+1}<1$$
	所以$\{x_n\}$有界.
	
	由单调有界收敛定理,$\{x_n\}$收敛.
	
	由于$y_{n+1}=1+\frac{1}{2}+\cdots+\frac{1}{n}+\frac{1}{n+1}-\ln(n+1)=x_n+\frac{1}{n+1}$,所以$y_n$也收敛.
	$$\lim\limits_{n\to\infty}y_n=\lim\limits_{n\to\infty}y_{n+1}=\lim\limits_{n\to\infty}x_n+\lim\limits_{n\to\infty}\frac{1}{n+1}=\lim\limits_{n\to\infty}x_n.$$
	$\hfill\blacksquare$
\end{proof}
\begin{remark}
	以上数列的极限称为{\heiti Euler常数},记作$\gamma$,其前几位数字为
	$$\gamma=0.5772156649\dots$$
	需要注意,Euler常数虽然极有可能是无理数,但至今尚未证明其无理性.
\end{remark}
\section{上极限和下极限}
\subsection{上极限和下极限的定义}
\begin{definition}[实数系的扩充]
	定义扩充后的实数系$\widetilde{\mathbb{R}}=\mathbb{R}\cup \pm\infty$.
\end{definition}
本节内容基于扩充后的实数系讨论.
\begin{definition}[数列的聚点]
	若数$a$的任一邻域含有数列$\{x_n\}$中的无限多个项,则称$a$为数列$\{x_n\}$的一个聚点.
\end{definition}
\begin{remark}
	数列(或点列)的聚点定义与实数理论中关于数集(或点集)的聚点定义是有区别的.当把点列看作点集时,点列中对应于相同数值的项,只能作为一个点来看待,如点列$\left\{\sin\frac{n\pi}{4}\right\}$作为点集来看待时,它仅含有五个点,即
	$$\left\{\sin\frac{n\pi}{4}\right\}=\left\{-1,-\frac{\sqrt{2}}{2},0,\frac{\sqrt{2}}{2},1\right\}$$
	按点集聚点的定义,这个有限集没有聚点.而我们在点列聚点的定义中只考虑项,只要在一点的任意小邻域内聚集了无穷多个项(不论其数值是否相同),该点就称为点列的聚点.所以点列(数列)的聚点实际上就是其收敛子列的极限.
\end{remark}
\begin{theorem}[存在性]
	任何数列(点列)至少有一个聚点.
\end{theorem}
\begin{definition}[上极限与下极限]
	设数列$\{x_n\}$的聚点(极限点)组成的集合为$E$,定义$\sup E$为数列$\{x_n\}$的{\heiti 上极限},记作$\limsup\limits_{n\to\infty}x_n$或$\varlimsup\limits_{n\to\infty}x_n$,
	定义$\inf E$为数列$\{x_n\}$的{\heiti 下极限},记作$\liminf\limits_{n\to\infty}x_n$或$\varliminf\limits_{n\to\infty}x_n$.
\end{definition}
对于无界数列,其只有一个聚点$+\infty$或$-\infty$,则$$\limsup\limits_{n\to\infty}x_n=\liminf\limits_{n\to\infty}x_n=+\infty(\text{或}-\infty)$$

由于$E\neq\varnothing$,因此任一数列都存在上极限和下极限,这一点使得上极限和下极限比一般的极限更具“应用优势”.

一般来说,求数列的上极限和下极限没有固定的简单方法,对于极限点只有有限个的情况,我们可以分别求出这有限个极限点从而直接得出上极限和下极限.
\begin{proposition}
	$\limsup\limits_{n\to\infty}x_n\in E$,$\liminf\limits_{n\to\infty}x_n\in E$
\end{proposition}
上述命题说明了上极限和下极限分别是数列的最大聚点和最小聚点.
\begin{proof}
	只证明上极限的情况.
	
	(i)$\limsup\limits_{n\to\infty}x_n=+\infty$,则$E$无上界,$\{a_n\}$无上界,故$\{a_n\}$一定存在以$+\infty$为极限的子列,因此$+\infty$也是一个极限点,即$+\infty\in E$;
	
	(ii)$\limsup\limits_{n\to\infty}x_n=-\infty$,则$E=\{-\infty\}$,故$-\infty\in E$;
	
	(iii)$\limsup\limits_{n\to\infty}x_n=a$,即$\{x_n\}$有界,$\exists M>0\ s.t.\ $
	$$-M<x_n<M.$$
	
	取$\left[a_1,b_1\right]=\left[-M,M\right]$,将区间$\left[a_{k-1},b_{k-1}\right]$等分为两个子区间,若右边一个含有$\{x_n\}$的无穷多个项,则取它为$\left[a_k,b_k\right]$,否则取左边的子区间为$\left[a_k,b_k\right]$.($k=2,3,\cdots$)
	
	这样的选取方法既保证了每次选出的$\left[a_k,b_k\right]$都含有$\{x_n\}$中的无限多个项,同时在$\left[a_k,b_k\right]$的右边却至多有$\{x_n\}$的有限个项,于是由区间套$\{\left[a_k,b_k\right]\}$所确定的点列$\{x_n\}$的聚点必是$\{x_n\}$的最大聚点.
	
	类似地,只要把每次优先挑选右边一个子区间改为优先挑选左边一个,就能证明最小聚点的存在性.$\hfill\blacksquare$
\end{proof}
\begin{theorem}
	$$\liminf\limits_{n\to\infty}x_n\leqslant\limsup\limits_{n\to\infty}x_n$$
	等号成立当且仅当数列$\{a_n\}$有极限.	
\end{theorem}

上极限和下极限也可用$\varepsilon-N$语言刻画.
\begin{theorem}[上下极限的$\varepsilon-N$定义]
	设数列$\{a_n\}$,令
	$$E=\{a\in\mathbb{R}:\forall\varepsilon>0,\exists N>0,\text{当}n>N\text{时},a_n<a+\varepsilon\};$$
	$$F=\{a\in\mathbb{R}:\forall\varepsilon>0,\exists N>0,\text{当}n>N\text{时},a_n>a-\varepsilon\}.$$
	则$\limsup\limits_{n\to\infty}a_n=\inf E$,$\liminf\limits_{n\to\infty}a_n=\sup F.$
\end{theorem}
\begin{proof}
	只需证明上极限的情况.设$\limsup\limits_{n\to\infty}a_n=L.$
	
	(i)证明$L\geqslant\inf E$.\quad 用反证法,假设$L<\inf E$,即$L\notin E$,即存在$\varepsilon>0$使得对于任意$N>0$,都存在$n>N$使得$a_n\geqslant L+\varepsilon$,这表明$(L+\varepsilon,+\infty)$中有$\{a_n\}$的无穷多项,由Weierstrass定理可知$(L+\varepsilon,+\infty)$中一定有极限点$L'>L$,出现矛盾.因此$L\in E$,这表明$L\geqslant \inf E$.
	
	(ii)证明$L\leqslant\inf E$.\quad 用反证法,假设$L>\inf E$,即$\exists a\in E$满足$a<L$,存在$\varepsilon>0$使得$a+\varepsilon<L$,由于$a\in E$,故$\exists N>0$使得当$n>N$时,$a_n<a+\varepsilon$.这表明$(a+\varepsilon,+\infty)$中有$\{a_n\}$的有限项,因此$L$不是极限点,出现矛盾.于是$a\geqslant L$,这表明$L\leqslant\inf E$.
	
	综上可知$\limsup\limits_{n\to\infty}a_n=\inf E.$ $\hfill\blacksquare$
\end{proof}
\begin{remark}
	为了让上极限是$+\infty$下极限是$-\infty$的情况也能用上面的语言刻画,可以令
	$$E=\{a\in\widetilde{\mathbb{R}}:\forall x>a,\exists N>0,\text{当}n>N\text{时},a_n<x\},$$
	$$F=\{a\in\widetilde{\mathbb{R}}:\forall x<a,\exists N>0,\text{当}n>N\text{时},a_n>x\}.$$
	则$\limsup\limits_{n\to\infty}a_n=\inf E,\liminf\limits_{n\to\infty}a_n=\sup F.$
\end{remark}
\begin{proposition}[Stolz定理的推广形式]
	设数列$\{a_n\}$和$\{b_n\}$.若$\{b_n\}$严格递增,且$\lim\limits_{n\to\infty}b_n=+\infty$,则
	$$\liminf\limits_{n\to\infty}\frac{a_n-a_{n-1}}{b_n-b_{n-1}}\leqslant\liminf\limits_{n\to\infty}\frac{a_n}{b_n}\leqslant\limsup\limits_{n\to\infty}\frac{a_n}{b_n}\leqslant\limsup\limits_{n\to\infty}\frac{a_n-a_{n-1}}{b_n-b_{n-1}}.$$
\end{proposition}
\begin{proof}
	中间的不等号显然成立.则由对称性,我们只需证右边的不等号成立即可.
	
	令
	$$L=\limsup\limits_{n\to\infty}\dfrac{a_n-a_{n-1}}{b_n-b_{n-1}}.$$
	
	当$L=+\infty$时命题显然成立.当$L<+\infty$时,$\forall\varepsilon>0$,设$l=L+\varepsilon$.则$\exists N\in\mathbb{N}_+$,当$n\geqslant N$时,有
	$$\frac{a_n-a_{n-1}}{b_n-b_{n-1}}<l.$$
	因此有
	$$\frac{a_N-a_{N-1}}{b_N-b_{N-1}}<l,\ \frac{a_{N+1}-a_N}{b_{N+1}-b_N}<l,\cdots,\frac{a_n-a_{n-1}}{b_n-b_{n-1}}<l,$$
	则
	\begin{equation}{\label{rof}}
		\frac{a_n-a_{N-1}}{b_n-b_{N-1}}=\frac{\frac{a_n}{b_n}-\frac{a_{N-1}}{b_n}}{1-\frac{b_{N-1}}{b_n}}<l,
	\end{equation}
	即$$\frac{a_n}{b_n}-\frac{a_{N-1}}{b_n}<l(1-\frac{b_{N-1}}{b_n}).$$
	两边取上极限,有
	$$\limsup\limits_{n\to \infty}\frac{a_n}{b_n}<l=L+\varepsilon,$$
	即
	$$\limsup\limits_{n\to \infty}\frac{a_n}{b_n}\leqslant L=\limsup\limits_{n \to \infty}\frac{a_n-a_{n-1}}{b_n-b_{n-1}}.$$
	左边的不等号证明方法同理.$\hfill\blacksquare$
\end{proof}
\begin{remark}
	由不等式$$\min\{\frac{a_1}{b_1},\frac{a_2}{b_2},\cdots,\frac{a_n}{b_n}\}\leqslant\frac{a_1+a_2+\cdots+a_n}{b_1+b_2+\cdots+b_n}\leqslant \max\{\frac{a_1}{b_1},\frac{a_2}{b_2},\cdots,\frac{a_n}{b_n}\}$$
	得到式\ref{rof}.
\end{remark}
现在我们已经给出了上极限和下极限的两种定义,它们比较直观,都没有与数列极限“直接挂钩”,因此在处理某些问题时并不便利,下面给出了第三种方式来定义上极限和下极限.

之前讲到,研究数列的极限并不关心前面的有限项,即去掉前面的有限项并不会改变数列的极限(如果有极限的话),因此研究数列的“最终趋势”时,也不需要关心前面的有限项.受此启发,对于数列$\{a_n\}$,我们可以定义这样一个数列$\{L_n\}$:
$$L_n\coloneqq \sup\limits_{k\geqslant n}\{a_k\}=\sup\{a_n,a_{n+1},\cdots\},\quad n=1,2,\cdots.$$
由于
$$\{a_k|k\geqslant n+1\}\subset\{a_k|k\geqslant n\},\quad n=1,2,\cdots.$$
因此数列$\{L_n\}$是单调递减的.同理定义数列$\{l_n\}$:
$$l_n\coloneqq \inf\limits_{k\geqslant n}\{a_k\}=\inf\{a_n,a_{n+1},\cdots\},\quad n=1,2,\cdots.$$
且$\{l_n\}$是单调递增的.于是可知数列$\{L_n\}$和$\{l_n\}$都有极限.不难想到$\{L_n\}$和$\{l_n\}$的极限分别是数列$\{a_n\}$的上极限和下极限.
\begin{theorem}
	设数列$\{a_n\}$,则
	
	(1)\ $\limsup\limits_{n\to\infty}a_n=\lim\limits_{n\to\infty}\sup\limits_{k\geqslant n}\{a_n\}$;
	
	(2)\ $\liminf\limits_{n\to\infty}a_n=\lim\limits_{n\to\infty}\inf\limits_{k\geqslant n}\{a_n\}$.
\end{theorem}
\begin{proof}
	只证明(1)即可.令
	$$L_n=\sup\limits_{k\geqslant n}\{a_k\},\quad L=\limsup\limits_{n\to\infty}a_n.$$
	只需证明$\lim\limits_{n\to\infty}L_n=L.$
	
	(i)当$L=+\infty$时,$\{a_n\}$存在一个以$+\infty$为极限的子列.因此
	$$L_n=\sup\limits_{k\geqslant n}\{a_k\}=+\infty.$$
	故$\lim\limits_{n\to\infty}L_n=L.$
	
	(ii)当$L=-\infty$时,$\limsup\limits_{n \to \infty}a_n=-\infty$,则
	$$\lim\limits_{n\to\infty}a_n=-\infty.$$
	对$\forall M>0,\ \exists N\in \mathbb{N}_+$使得当$n\geqslant N$时,$a_n<-M$.则
	$$L_N=\sup\limits_{k\geqslant N}a_k<-M.$$
	又因为$\{L_N\}$单调递减,故当$n\geqslant N$时,有
	$$L_n\leqslant L_N<-M.$$
	即$\lim\limits_{n\to\infty}L_n=-\infty$.
	
	(iii)当$L\in\mathbb{R}$时,任取$\{a_n\}$的一个极限点$l$,对于给定的$n$,选取$i\geqslant n$,则$k_i\geqslant i\geqslant n$,于是
	$$a_{k_i}\leqslant \sup\{a_n,a_{n+1},\cdots,a_{k_i},\cdots\}=L_n,\quad n=1,2,\cdots$$
	令$i\to\infty$得$l\leqslant L_n$,再令$n\to\infty$,得$l\leqslant \lim\limits_{n\to\infty}L_n$,于是$L\leqslant \lim\limits_{n\to\infty}L_n$.
	
	由上下极限的第二种定义,$\forall \varepsilon>0,\ \exists N\in\mathbb{N}_+$使得当$n>N$时有$a_n<L+\varepsilon.$则
	$$L_n<L+\varepsilon,$$
	当$n\to\infty$时,有
	$$\lim\limits_{n\to\infty}L_n\leqslant L.$$
	因此有$L=\lim\limits_{n \to \infty}L_n$,即
	$$\limsup\limits_{n \to \infty}a_n=\lim\limits_{n\to\infty}\sup\limits_{k\geqslant n}\{a_n\}.$$
	$\hfill\blacksquare$
\end{proof}
\begin{proposition}
	设数列$\{a_n\}$,则
	
	(1)$\limsup\limits_{n \to \infty}(-a_n)=-\liminf\limits_{n \to \infty}a_n$;
	
	(2)$\liminf\limits_{n \to \infty}(-a_n)=-\limsup\limits_{n \to \infty}a_n$.
\end{proposition}
\begin{proof}
	因为有$\sup(-E)=-\inf(E)$,$\inf(-E)=-\sup(E)$,故
	$$\sup\limits_{k\geqslant n}\{-a_k\}=-\inf\limits_{k\geqslant n}\{a_k\},\inf\limits_{k\geqslant n}\{-a_k\}=-\sup\limits_{k\geqslant n}\{a_k\}$$
	取$n\to\infty$即得结论.$\hfill\blacksquare$
\end{proof}
\subsection{上下极限的性质}
\begin{theorem}[保序性]
	设数列$\{a_n\},\{b_n\}$,若$\exists N>0$,当$n>N$时,有$a_n\leqslant b_n$,则
	
	(1)$\limsup\limits_{n\to\infty}a_n\leqslant\limsup\limits_{n\to\infty}b_n,$
	
	(2)$\liminf\limits_{n\to\infty}a_n\leqslant\liminf\limits_{n\to\infty}b_n.$
	
	特别地,若$\alpha,\beta$为常数,$\exists N>0$,当$n>N$时,有$\alpha\leqslant a_n\leqslant \beta$,则$$\alpha\leqslant\liminf\limits_{n\to\infty}x_n\leqslant\limsup\limits_{n\to\infty}x_n\leqslant\beta.$$
\end{theorem}
\begin{proof}
	只需证明(1).记$\limsup\limits_{n\to\infty}a_n=A,\ \limsup\limits_{n\to\infty}b_n=B.$
	
	(i)当$B=+\infty$或$A=-\infty$时,命题显然成立;
	
	(ii)当$A=+\infty$时,$A$中存在以$+\infty$为极限的子列,因为若$\exists N>0$,当$n>N$时,有$a_n\leqslant b_n$,所以$B$中也存在以$+\infty$为极限的子列,因此$A=B$,类似可证$B=-\infty$时$A=B$;
	
	(iii)当$A,B\in\mathbb{R}$时,用反证法,假设$A>B$,则存在$\varepsilon>0$使得$B<B+\varepsilon<A$,由上极限的$\varepsilon-N$定义,存在$N_1$,当$n>N_1$时,有$b_n<B+\varepsilon$,取$N_0=\max\{N,N_1\}$,当$n>N_0$时,有
	$$a_n\leqslant b_n<B+\varepsilon<A$$,则在$(B+\varepsilon,+\infty)$之间只有$\{a_n\}$的有限项,因此$A$不可能是$\{a_n\}$的极限点,这与$A$是$\{a_n\}$的上极限矛盾.因此$A\leqslant B.$
	
	类似可证下极限的保序性.$\hfill\blacksquare$
\end{proof}
\begin{theorem}[上极限的次可加性与下极限的超可加性]
	设数列$\{a_n\}$,$\{b_n\}$,则
	\begin{enumerate}
		\item $\liminf\limits_{n \to \infty}a_n+\liminf\limits_{n \to \infty}b_n\leqslant\liminf\limits_{n \to \infty}(a_n+b_n)\leqslant\liminf\limits_{n \to \infty}a_n+\limsup\limits_{n \to \infty}b_n$;
		\item $\liminf\limits_{n \to \infty}a_n+\limsup\limits_{n \to \infty}b_n\leqslant\limsup\limits_{n \to \infty}(a_n+b_n)\leqslant\limsup\limits_{n \to \infty}a_n+\limsup\limits_{n \to \infty}a_n$.
	\end{enumerate}
\end{theorem}
\begin{remark}
	设定义在$A$上的映射$f$,$\forall a,b\in A$.
	\begin{enumerate}
		\item 若$f(a+b)=f(a)+f(b)$,则称$f$满足{\heiti 可加性}(additivity);
		\item 若$f(a+b)\geqslant f(a)+f(b)$,则称$f$满足{\heiti 超可加性}(superadditivity);
		\item 若$f(a+b)\leqslant f(a)+f(b)$,则称$f$满足{\heiti 次可加性}(subadditivity).
	\end{enumerate}
\end{remark}
\begin{proof}
	只需证明第1个不等式.对任意$l\geqslant n$,有$\inf\limits_{k\geqslant n}a_k\leqslant a_l$,$\inf\limits_{k\geqslant n}b_k\leqslant b_l$,由$l$的任意性,有
	$$\inf\limits_{k\geqslant n}a_k+\inf\limits_{k\geqslant n}b_k\leqslant \inf\limits_{k\geqslant n}(a_k+b_k).$$
	则$$\inf\limits_{k\geqslant n}a_k\geqslant\inf\limits_{k\geqslant n}(a_k+b_k)+\inf\limits_{k\geqslant n}(-b_k)=\inf\limits_{k\geqslant n}(a_k+b_k)-\sup\limits_{k\geqslant n}b_k.$$
	即$$\inf\limits_{k\geqslant n}(a_k+b_k)\leqslant\inf\limits_{k\geqslant n}a_k+\sup\limits_{k\geqslant n}b_k.$$
	当$n\to\infty$时,有
	$$\liminf\limits_{n \to \infty}a_k+\liminf\limits_{n \to \infty}b_k\leqslant\liminf\limits_{n \to \infty}(a_k+b_k)\leqslant\liminf\limits_{n \to \infty}a_k+\limsup\limits_{n \to \infty}b_k.$$
	第2个不等式类似可证.$\hfill\blacksquare$
\end{proof}
\begin{remark}
	特别地,当$\lim\limits_{n\to\infty}a_n=a$时,有
	\begin{enumerate}
		\item $\liminf\limits_{n\to\infty}(a_n+b_n)=a+\liminf\limits_{n \to \infty}b_n$;
		\item $\limsup\limits_{n\to\infty}(a_n+b_n)=a+\limsup\limits_{n \to \infty}b_n$.
	\end{enumerate}
	利用上述定理即可证明,不再赘述.
\end{remark}

与上极限的次可加性和下极限的超可加性类似,我们还有以下结论.
\begin{theorem}
	设非负数列$\{a_n\}$,$\{b_n\}$,则
	\begin{enumerate}
		\item $\left(\liminf\limits_{n \to \infty}a_n\right)\left(\liminf\limits_{n \to \infty}b_n\right)\leqslant\liminf\limits_{n \to \infty}\left(a_nb_n\right)\leqslant\left(\liminf\limits_{n \to \infty}a_n\right)\left(\limsup\limits_{n \to \infty}b_n\right)$;
		\item $\left(\liminf\limits_{n \to \infty}a_n\right)\left(\limsup\limits_{n \to \infty}b_n\right)\leqslant\liminf\limits_{n \to \infty}\left(a_nb_n\right)\leqslant\left(\limsup\limits_{n \to \infty}a_n\right)\left(\limsup\limits_{n \to \infty}b_n\right)$.
	\end{enumerate}
\end{theorem}
\begin{proof}
	只需证明第1个不等式.对任意$l\geqslant n$,有$\inf\limits_{k\geqslant n}a_k\leqslant a_l$,$\inf\limits_{k\geqslant n}b_k\leqslant b_l$,由于$\{a_n\},\{b_n\}$非负和$l$的任意性,有
	$$\left(\inf\limits_{k\geqslant n}a_k\right)\left(\inf\limits_{k\geqslant n}b_k\right)\leqslant\inf\limits_{k\geqslant n}(a_kb_k).$$
	对于任意$\varepsilon>0$,存在$m\geqslant n$,使得$a_m<\varepsilon+\inf\limits_{k\geqslant n}a_k$,由于$b_m\leqslant\sup\limits_{k\geqslant n}b_k$,且$\{a_n\},\{b_k\}$非负,因此
	$$\left(\varepsilon+\inf\limits_{k\geqslant n}a_k\right)\left(\inf\limits_{k\geqslant n}b_k\right)> a_mb_m\geqslant \inf\limits_{k\geqslant n}(a_kb_k).$$
	由$\varepsilon$的任意性,有
	$$\left(\varepsilon+\inf\limits_{k\geqslant n}a_k\right)\left(\inf\limits_{k\geqslant n}b_k\right)\geqslant \inf\limits_{k\geqslant n}(a_kb_k).$$
	于是得到不等式
	$$\left(\inf\limits_{k\geqslant n}a_k\right)\left(\inf\limits_{k\geqslant n}b_k\right)\leqslant\inf\limits_{k\geqslant n}(a_kb_k)\leqslant\left(\varepsilon+\inf\limits_{k\geqslant n}a_k\right)\left(\inf\limits_{k\geqslant n}b_k\right).$$
	$n\to\infty$时即得要证结论.第2个不等式类似可证.$\hfill\blacksquare$
\end{proof}
\begin{remark}
	特别地,当$\lim\limits_{n\to\infty}a_n=a$时,有
	\begin{enumerate}
		\item $\liminf\limits_{n\to\infty}(a_nb_n)=a\liminf\limits_{n \to \infty}b_n$;
		\item $\limsup\limits_{n\to\infty}(a_nb_n)=a\limsup\limits_{n \to \infty}b_n$.
	\end{enumerate}
\end{remark}










\newpage

\chapter{函数极限}
\section{函数的概念}
关于函数概念,在中学数学中我们已经有了初步的了解,本节仅对此作一定的复习与补充.
\subsection{函数的定义}
\begin{definition}[函数]
	给定两个实数集$D$和$M$,若有对应法则$f$,使对每一个$x\in D$,都有唯一的$y\in M$与它相对应,则称$f$是定义在数集$D$上的{\heiti 函数}(function),记作
	$$f:D\to M,$$
	$$x\mapsto y.$$
	数集$D$称为函数$f$的{\heiti 定义域}(domain of definition),$x$所对应的$y$称为$f$在点$x$的函数值,常记为$f(x)$.全体函数值的集合
	$$f(D)=\{y|y=f(x),x\in D\}(\subset M)$$
	称为函数$f$的{\heiti 值域}(range).
\end{definition}
关于函数的定义,做如下说明:
\begin{enumerate}
	\item 习惯上我们称上述函数定义中的$x$为{\heiti 自变量}(independent variable),$y$为{\heiti 因变量}(dependent variable).
	\item 用数学运算式子表示函数时,我们将使该运算式子有意义的自变量值的全体称为{\heiti 存在域}(domain of existence).
	\item 在函数定义中,对每一个$x\in D$,只能有唯一的一个$y$值与它对应,这样定义的函数称为{\heiti 单值函数}(single-valued function).若同一个$x$值可以对应多于一个的$y$值,则称这种函数为{\heiti 多值函数}(multi-valued function).在本书范围内,我们只讨论单值函数.
\end{enumerate}
\subsection{复合函数}
\begin{definition}[复合函数]
	
	设有两函数
	$$y=f(u)$$
	$$u=g(x)$$
	
	对每一个$x$都通过函数$g$对应了唯一的$u$,而$u$又通过函数$f$对应了唯一的$y$.这就确定了一个以$x$为自变量,$y$为因变量的函数,记作$y=f(g(x))$或$y=(f\circ g)(x)$,称为{\heiti 复合函数}(composite function).$u$称为中间变量,函数$f$和$g$的复合运算可以简单写为$f\circ g$.
\end{definition}
\subsection{初等函数}
我们已经熟悉的基本初等函数有常量函数、幂函数、指数函数、对数函数、三角函数、反三角函数六类.
\begin{definition}[初等函数]
	由基本初等函数经过\textbf{有限次}四则运算与复合运算所得到的函数,统称为{\heiti 初等函数}(elementary function).
\end{definition}
\section{函数的性质}
\subsection{有界性}
\begin{definition}[上界和下界]
	设$f$是定义在$D$上的函数.
	
	若存在数$M$,使得$\forall x\in D$,有
	$$f(x)\leqslant M,$$
	则称$f$在$D$上有上界,$M$为$f$在$D$上的一个{\heiti 上界}(upper bound).
	
	若存在数$L$,使得$\forall x\in D$,
	$$f(x)\geqslant M,$$
	则称$f$在$D$上有下界,$L$为$f$在$D$上的一个{\heiti 下界}(lower bound).
\end{definition}
\begin{definition}[有界函数]
	设$f$是定义在$D$上的函数.若存在正数$M$,使得$\forall x\in D$,有
	$$|f(x)|\leqslant M,$$
	则称$f$是$D$上的{\heiti 有界函数}(bounded function).
\end{definition}
根据定义,$f$有界的充要条件是$f$既有上界也有下界.
\subsection{单调性}
\begin{definition}[单调函数]
	设$f$为定义在$D$上的函数.若对任何$x_1,x_2\in D$,当$x_1<x_2$时,总有
	
	(i)$f(x_1)\leqslant f(x_2)$,则称$f$为$D$上的{\heiti 增函数}(increasing function),特别当成立严格不等式$f(x_1)<f(x_2)$时,称$f$为$D$上的{\heiti 严格增函数}(strictly increasing function);
	
	(ii)$f(x_1)\geqslant f(x_2)$,则称$f$为$D$上的{\heiti 减函数}(decreasing function),特别当成立严格不等式$f(x_1)>f(x_2)$时,称$f$为$D$上的{\heiti 严格减函数}(strictly decreasing function).
\end{definition}
\subsection{奇偶性}
\begin{definition}[奇偶函数]
	设$D$是对称于原点的数集,$f$为定义在$D$上的函数.
	
	若对每一个$x\in D$,有
	$$f(-x)=-f(x)$$
	则称$f$是$D$上的{\heiti 奇函数}(odd function);
	
	若对每一个$x\in D$,有
	$$f(-x)=f(x)$$
	则称$f$是$D$上的{\heiti 偶函数}(even function).
\end{definition}
\subsection{周期性}
\begin{definition}[周期函数]
	设$f$为定义在$D$上的函数.若存在$\sigma>0$,使得对一切$x\in D$,$x\pm \sigma\in D$,有$f(x\pm\sigma)=f(x)$,则称$f$为{\heiti 周期函数}(periodic function),$\sigma$称为$f$的一个{\heiti 周期}(period).在$f(x)$所有的正周期中,最小的正周期就称为$f$的{\heiti 最小正周期}.
\end{definition}
\section{函数极限的定义}
\subsection{$x$趋于$\infty$时函数的极限}
\begin{definition}
	设$f$为定义在$\left[a,+\infty \right)$上的函数,$A$为定数.若对任给的$\varepsilon>0$,存在正数$M(\geqslant a)$,使得当$x>M$时,有
	$$|f(x)-A|<\varepsilon,$$
	则称{\heiti 函数$f$当$x$趋于$+\infty$时以$A$为极限},记作
	$$\lim\limits_{x\to +\infty}f(x)=A\text{或}f(x)\to A(x\to +\infty).$$
\end{definition}
定义中正数$M$的作用与数列定义中的$N$相类似,表明$x$充分大的程度;但这里考虑的是比$M$大的所有实数$x$,而不仅仅是正整数$n$.因此,当$x\to +\infty$时函数$f$以$A$为极限意味着:$A$的任意小邻域内必含有$f$在$+\infty$的某邻域上的全部函数值.

现在我们设$f$为定义在$U(-\infty)$或$U(\infty)$上的函数,当$x\to -\infty$或$x\to\infty$时,若函数值$f(x)$能无限地接近某定数$A$,则称$f$当$x\to -\infty$或$x\to\infty$时以$A$为极限,分别记作
$$\lim\limits_{x\to -\infty}f(x)=A\text{或}f(x)\to A(x\to -\infty);$$
$$\lim\limits_{x\to \infty}f(x)=A\text{或}f(x)\to A(x\to \infty).$$
这两种函数极限的精确定义与上述定义相仿,只需把上述定义中的“$x>M$”分别改为“$x<-M$”或"$|x|>M$"即可.

不难发现,若$f$是定义在$U(\infty)$上的函数,则
$$\lim\limits_{x\to\infty}f(x)=A\iff \lim\limits_{x\to +\infty}f(x)=\lim\limits_{x\to -\infty}f(x)=A.$$
\subsection{$x$趋于$x_0$时函数的极限}
\begin{definition}[函数极限的$\varepsilon-\delta$定义]
	设函数$f$在点$x_0$的某个空心邻域$\mathring{U}(x_0;\delta')$内有定义,$A$为定数.若对任给的$\varepsilon>0$,存在正数$\delta(<\delta')$,使得当$0<|x-x_0|<\delta$时,有
	$$|f(x)-A|<\varepsilon,$$
	则称{\heiti 函数$f$当$x$趋于$x_0$时以$A$为极限},记作
	$$\lim\limits_{x\to x_0}f(x)=A\text{或}f(x)\to A(x\to x_0).$$
\end{definition}
\subsection{单侧极限}
有些函数在其定义域上的某些点左侧与右侧的解析式不同(如分段函数定义域上的某些点),或函数只在某些点的一侧有定义(如定义域的区间端点处),这时函数在那些点上的极限只能单侧地给出定义.
\begin{definition}
	设函数$f$在$\mathring{U}_+(x_0;\delta')$内有定义,$A$为定数.若对任给的$\varepsilon>0$,存在正数$\delta(<\delta')$,使得当$x_0<x<x_0+\delta$时,有
	$$|f(x)-A|<\varepsilon,$$
	则称$A$为函数$f$当$x$趋于$x_0^+$时的右极限,记作
	$$\lim\limits_{x\to x_0^+}f(x)=A\text{或}f(x)\to A(x\to x_0^+).$$
\end{definition}
类似地,我们可以定义函数的左极限.

右极限和左极限统称为单侧极限.$f$在点$x_0$处的右极限和左极限 又分别记为
$$f(x_0+0)=\lim\limits_{x\to x_0^+}f(x)\text{与}f(x_0-0)=\lim\limits_{x\to x_0^-}f(x).$$

关于函数极限与单侧极限的关系有如下定理.
\begin{theorem}
	$$\lim\limits_{x\to x_0}f(x)=A\iff \lim\limits_{x\to x_0^+}f(x)=\lim\limits_{x\to x_0^-}f(x)=A.$$
\end{theorem}
\section{函数极限的性质}
在上一节中,我们引入了下面六种类型的极限:
\begin{enumerate}
	\item $\lim\limits_{x\to\infty}f(x);$
	\item $\lim\limits_{x\to+\infty}f(x);$
	\item $\lim\limits_{x\to-\infty}f(x);$
	\item $\lim\limits_{x\to x_0}f(x);$
	\item $\lim\limits_{x\to x_0^+}f(x);$
	\item $\lim\limits_{x\to x_0^-}f(x).$
\end{enumerate}

它们具有与数列极限相类似的一些性质,下面以第4种类型的极限为代表来叙述并证明这些性质.至于其他类型极限的性质及证明,只要相应地做些修改即可.
\begin{theorem}[唯一性]
	若极限$\lim\limits_{x\to x_0}f(x)$存在,则此极限是唯一的.
\end{theorem}
\begin{proof}
	设$A,B$都是$f$当$x\to x_0$时的极限,则对任给的$\varepsilon>0$,分别存在正数$\delta_1$和$\delta_2$,使得当$0<|x-x_0|<\delta_1$时,有
	$$|f(x)-A|<\varepsilon,$$
	当$0<|x-x_0|<\delta_2$时,有
	$$|f(x)-B|<\varepsilon.$$
	取$\delta=\min\{\delta_1,\delta_2\}$,则当$0<|x-x_0|<\delta$时,上述两式同时成立,故有
	\begin{align*}
		|A-B|&=|(f(x)-A)-(f(x)-B)|\\
		&\leqslant|f(x)-A|+|f(x)-B|<2\varepsilon.
	\end{align*}
	由$\varepsilon$的任意性得$A=B$.这就证明了极限的唯一性.$\hfill\blacksquare$
\end{proof}
\begin{theorem}[局部有界性]
	若$\lim\limits_{x\to x_0}f(x)$存在,则$f$在$x_0$的某空心邻域$\mathring{U}(x_0)$上有界.
\end{theorem}
\begin{proof}
	设$\lim\limits_{x\to x_0}f(x)=A$.取$\varepsilon=1$,则存在$\delta>0$,使得对一切$x\in \mathring{U}(x_0;\delta)$,有
	$$|f(x)-A|<1\Rightarrow|f(x)|<|A|+1.$$
	这就证明了$f$在$\mathring{U}(x_0)$上有界.$\hfill\blacksquare$
\end{proof}
\begin{theorem}[局部保号性]
	若$\lim\limits_{x\to x_0}f(x)=A>0$,则对任何正数$r<A$,存在$\mathring{U}(x_0)$,使得对一切$x\in \mathring{U}(x_0)$,有
	$$f(x)>r>0.$$
	$A<0$的情况类似.
\end{theorem}
\begin{proof}
	设$A>0$,对任何$r\in (0,A)$,取$\varepsilon=A-r$,则存在$\delta>0$,使得对一切$x\in \mathring{U}(x_0;\delta)$,有
	$$f(x)>A-\varepsilon=r,$$
	这就证得结论.对于$A<0$的情形可类似证明.
\end{proof}
\begin{theorem}[保序性]
	设$\lim\limits_{x\to x_0}f(x)$和$\lim\limits_{x\to x_0}g(x)$都存在,且在某邻域$\mathring{U}(x_0;\delta')$上有$f(x)\leqslant g(x)$,则
	$$\lim\limits_{x\to x_0}f(x)\leqslant \lim\limits_{x\to x_0}g(x).$$
\end{theorem}
\begin{proof}
	设$\lim\limits_{x\to x_0}f(x)=A$,$\lim\limits_{x\to x_0}g(x)=B$,则对任意$\varepsilon>0$,分别存在正数$\delta_1$和$\delta_2$,使得当$0<|x-x_0|<\delta_1$时,有
	$$A-\varepsilon<f(x),$$
	当$0<|x-x_0|<\delta_2$时,有
	$$g(x)<B+\varepsilon.$$
	令$\delta=\min\{\delta',\delta_1,\delta_2\}$,则当$0<|x-x_0|<\delta$时,有
	$$A-\varepsilon<f(x)\leqslant g(x)<B+\varepsilon,$$
	从而$A<B+2\varepsilon$.由$\varepsilon$的任意性推出$A\leqslant B$,即
	$$\lim\limits_{x\to x_0}f(x)\leqslant \lim\limits_{x\to x_0}g(x).$$
	$\hfill\blacksquare$
\end{proof}
\begin{theorem}[迫敛性]
	设$\lim\limits_{x\to x_0}f(x)=\lim\limits_{x\to x_0}g(x)=A$,且在某$\mathring{U}(x_0;\delta')$上有
	$$f(x)\leqslant h(x)\leqslant g(x),$$
	则$\lim\limits_{x\to x_0}h(x)=A$.
\end{theorem}
\begin{proof}
	对任意$\varepsilon>0$,分别存在正数$\delta_1$和$\delta_2$,使得当$0<|x-x_0|<\delta_1$时,有
	$$A-\varepsilon<f(x),$$
	当$0<|x-x_0|<\delta_2$时,有
	$$g(x)<A+\varepsilon.$$
	令$\delta=\min\{\delta',\delta_1,\delta_2\}$,则当$0<|x-x_0|<\delta$时,有
	$$A-\varepsilon<f(x)\leqslant h(x)\leqslant g(x)<A+\varepsilon,$$
	由此得$|h(x)-A|<\varepsilon$,所以$\lim\limits_{x\to x_0}h(x)=A$.$\hfill\blacksquare$
\end{proof}
\begin{theorem}[四则运算法则]
	若极限$\lim\limits_{x\to x_0}f(x)$与$\lim\limits_{x\to x_0}g(x)$都存在,则函数$f\pm g$,$f\cdot g$当$x\to x_0$时极限也存在,且
	$$\lim\limits_{x\to x_0}\left[f(x)\pm g(x)\right]=\lim\limits_{x\to x_0}f(x)\pm \lim\limits_{x\to x_0}g(x);$$
	$$\lim\limits_{x\to x_0}\left[f(x)g(x)\right]=\lim\limits_{x\to x_0}f(x)\cdot \lim\limits_{x\to x_0}g(x);$$
	又若$\lim\limits_{x\to x_0}g(x)\neq 0$,则$f/g$当$x\to x_0$时极限存在,且
	$$\lim\limits_{x\to x_0}\frac{f(x)}{g(x)}=\frac{\lim\limits_{x\to x_0}f(x)}{\lim\limits_{x\to x_0}g(x)}.$$
\end{theorem}
这个定理的证明类似于数列极限的四则运算法则,不再赘述.

\section{函数极限存在的条件}
\begin{theorem}[Heine归结原理]
	设函数$f(x)$在$\mathring{U}(x_0)$上有定义.则$\lim\limits_{x\to x_0}f(x)=A$的充要条件是$\mathring{U}(x_0)$中的任一趋于$x_0$的数列$\{x_n\}$都有$\lim\limits_{n\to \infty}f(x_n)=A$.
\end{theorem}
\begin{proof}
	必要性\qquad 若$\lim\limits_{x\to x_0}f(x)=A$,则$\forall \varepsilon>0,\ \exists\delta>0\ s.t.\text{当}x\in \mathring{U}(x_0,\delta)\text{时,}f(x)\in \mathring{U}(A,\varepsilon)$.\\
	若$\lim\limits_{n\to\infty}x_n=x_0\text{,则}\exists N>0\text{,当}n\in N\text{时,都有}x_n\in \mathring{U}(x_0,\delta).\text{则}\lim\limits_{n\to\infty}f(x_n)=A.$
	
	充分性\qquad 假设$\lim\limits_{x\to x_0}f(x)\neq A$,则$\exists \varepsilon_0>0$,对$\forall\delta>0$,都存在一个$x\in \mathring{U}(x,\delta)$使$f(x')\notin \mathring{U}(A,\varepsilon_0)$.我们依次取$\delta=\delta_0,\frac{\delta_0}{2},\cdots,\frac{\delta_0}{n},\cdots$,则存在相应的点$x_1,x_2,\cdots,x_n,\cdots$,使得
	$$\{x_n\}\subset\mathring{U}(x_0;\delta_0)\qquad\text{而}\qquad\lim\limits_{n\to\infty}f(x_n)\neq A.$$
	$\hfill\blacksquare$
\end{proof}
\begin{remark}
	在求证充分性时,直接进行求证涉及数列$\{x_n\}$的任意性,我们不能将所有的$\{x_n\}$列举出来,由于逆否命题和原命题等价,因此我们只需证明其逆否命题,即证明
	$$\text{若}\lim\limits_{x\to x_0}f(x)\neq A\text{,则}\mathring{U}(x_0)\text{中存在趋于}x_0\text{数列}\{x_n\}\text{使}\lim\limits_{n\to\infty}f(x_n)\neq A.$$
	关键在于我们构造了这样一个趋于$x_0$的数列$\{x_n\}$.
\end{remark}
\begin{remark}
	若可找到一个以$x_0$为极限的数列$\{x_n\}$,使$\lim\limits_{n\to \infty}f(x_n)$不存在,或找到两个都以$x_0$为极限的数列$\{x_n'\}$与$\{x_n''\}$,使$\lim\limits_{n\to\infty}f(x_n')$与$\lim\limits_{n\to\infty}f(x_n'')$都存在而不相等,则$\lim\limits_{x\to x_0}f(x)$不存在.
\end{remark}
\begin{example}
	证明极限$\lim\limits_{x\to 0}\sin\frac{1}{x}$不存在.
\end{example}
\begin{proof}
	设$x_n'=\frac{1}{n\pi},\ x_n''=\frac{1}{2n\pi+\frac{\pi}{2}}(n=1,2,\cdots)$,则显然有
	$$x_n'\to 0,\ x_n''\to 0(n\to \infty),$$
	$$\sin\frac{1}{x_n'}=0\to 0,\ \sin\frac{1}{x_n''}=1\to 1(n\to \infty).$$
	故由归结原理即得结论.$\hfill\blacksquare$
\end{proof}
\begin{theorem}[Cauchy准则]
	设函数$f$在$\mathring{U}(x_0,\delta')$上有定义.$\lim\limits_{x\to x_0}f(x)$存在的充要条件是:任给$\varepsilon>0$,存在正数$\delta(<\delta')$,使得对任何$x',x''\in\mathring{U}(x_0,\delta)$,有$|f(x')-f(x'')|<\varepsilon$.
\end{theorem}
\begin{proof}
	必要性\qquad 设$\lim\limits_{x\to x_0}f(x)=A$,则对任给的$\varepsilon>0$,存在正数$\delta(<\delta')$,使得对任何$x\in \mathring{U}(x_0,\delta)$,有$|f(x)-A|<\frac{\varepsilon}{2}$.于是对任何$x',x''\in\mathring{U}(x_0,\delta)$,有
	$$|f(x')-f(x'')|\leqslant|f(x')-A|+|f(x'')-A|<\frac{\varepsilon}{2}+\frac{\varepsilon}{2}=\varepsilon.$$
	充分性\qquad 设数列$\{x_n\}\subset \mathring{U}(x_0;\delta)$且$\lim\limits_{n\to \infty}x_n=x_0$.按假设,对任给的$\varepsilon>0$,存在正数$\delta(<\delta')$,使得对任何$x',x''\in\mathring{U}(x_0,\delta)$,有$|f(x')-f(x'')|<\varepsilon$.由于$x_n\to x_0(n\to \infty)$,对上述的$\delta>0,\ \exists N>0$,当$n,m>N$时,有$x_n,x_m\in \mathring{U}(x_0;\delta)$,从而有
	$$|f(x_n)-f(x_m)|<\varepsilon.$$
	于是,按数列的柯西收敛准则,数列$f(x_n)$的极限存在,由Heine归结原则,$\lim\limits_{x\to x_0}f(x)$也存在.$\hfill\blacksquare$
\end{proof}
按照函数极限的柯西准则,我们能写出极限$\lim\limits_{x\to x_0}f(x)$不存在的充要条件:存在$\varepsilon_0>0$,对任何$\delta>0$,总可找到$x',x''\in\mathring{U}(x_0,\delta)$,使得$|f(x')-f(x'')|\geqslant\varepsilon_0$.
\section{两个重要极限}
\subsection{$\lim\limits_{x\to 0}\dfrac{\sin x}{x}=1$}
\begin{proof}
	在中学数学的学习中,我们已经知道如下不等式:
	$$\sin x<x<\tan x(0<x<\frac{\pi}{2}),$$
	除以$\sin x$,得到
	$$1<\frac{x}{\sin x}<\frac{1}{\cos x},$$
	由此得
	$$\cos x<\frac{\sin x}{x}<1.$$
	易知$\cos x$和$\frac{\sin x}{x}$都是偶函数,因此当$-\frac{\pi}{2}<x<0$时上式也成立.由$\lim\limits_{x\to 0}\cos x=1$及函数极限的迫敛性,可得
	$$\lim\limits_{x\to 0}\frac{\sin x}{x}=1.$$
	$\hfill\blacksquare$
\end{proof}
\subsection{$\lim\limits_{x\to\infty}(1+\frac{1}{x})^x=\text{e}$}
\begin{proof}
	所求证的极限等价于同时成立以下两个极限:
	$$\lim\limits_{x\to +\infty}(1+\frac{1}{x})^x=\text{e},$$
	$$\lim\limits_{x\to -\infty}(1+\frac{1}{x})^x=\text{e}.$$
	
	先利用数列极限$\lim\limits_{n\to \infty}(1+\frac{1}{n})^n=\text{e}$证明第一个极限成立.
	
	因为$\lim\limits_{n\to \infty}(1+\frac{1}{n+1})^n=\lim\limits_{n\to \infty}(1+\frac{1}{n})^{n+1}=\text{e}$,所以对$\forall\varepsilon>0,\ \exists N\in \mathbb{Z}_+$,当$n>N$时,有
	$$\text{e}-\varepsilon<(1+\frac{1}{n+1})^n<(1+\frac{1}{n})^{n+1}<\text{e}+\varepsilon.$$
	取$X=N$,当$x>X$时,令$n=\left[x\right]$,那么
	$$(1+\frac{1}{n+1})^n<(1+\frac{1}{x})^x<(1+\frac{1}{n})^{n+1}.$$
	所以有
	$$\text{e}-\varepsilon<(1+\frac{1}{x})^x<\text{e}+\varepsilon.$$
	这就证明了$\lim\limits_{x\to +\infty}(1+\frac{1}{x})^x=\text{e}.$
	
	\hspace*{\fill}
	
	下面证明第二个极限成立.为此做代换$x=-y$,则
	$$(1+\frac{1}{x})^x=(1-\frac{1}{y})^{-y}=(1+\frac{1}{y-1})^y,$$
	且当$x\to -\infty$时$y\to +\infty$,从而有
	$$\lim\limits_{x\to -\infty}(1+\frac{1}{x})^x=\lim\limits_{y\to +\infty}(1+\frac{1}{y-1})^{y-1}\cdot (1+\frac{1}{y-1})=\text{e}.$$
	$\hfill\blacksquare$
\end{proof}
\begin{remark}
	以后还常用到e的另一种极限形式:
	$$\lim\limits_{\alpha\to 0}(a+\alpha)^{\frac{1}{\alpha}}=\text{e}.$$
	事实上,令$\alpha=\frac{1}{x}$,则$x\to\infty\iff\alpha\to 0$,所以
	$$\text{e}=\lim\limits_{x\to\infty}(1+\frac{1}{x})^x=\lim\limits_{\alpha\to 0}(a+\alpha)^{\frac{1}{\alpha}}.$$
\end{remark}
\section{无穷小量与无穷大量}
\subsection{无穷小量}
\begin{definition}
	设函数$f$在某$\mathring{U}(x_0)$上有定义.若
	$$\lim\limits_{x\to x_0}f(x)=0,$$
	则称$f$为当$x\to x_0$时的{\heiti 无穷小量}(infinitesimal).
	若函数$g$在某$\mathring{U}(x_0)$上有界,则称$g$为当$x\to x_0$时的{\heiti 有界量}(bounded).
\end{definition}

类似地,我们可以定义$x\to x_0^+$,$x\to x_0^-$,$x\to +\infty$,$x\to -\infty$以及$x\to \infty$时的无穷小量和有界量.特别地,任何无穷小量也必都是有界量.

由函数极限、无穷小量和有界量的定义可以立刻推得以下性质:
\begin{enumerate}
	\item 两个(相同类型的)无穷小量之和、差、积仍为无穷小量.
	\item 无穷小量与有界量的乘积为无穷小量.
	\item $\lim\limits_{x\to x_0}f(x)=A\iff f(x)-A$是当$x\to x_0$时的无穷小量.
\end{enumerate}
\subsection{无穷小量阶的比较}
无穷小量是以$0$为极限的函数,而不同的无穷小量收敛于$0$的速度有快有慢,为此,我们考察两个无穷小量的比,以便对它们的收敛速度做出判断.

设当$x\to x_0$时,$f$和$g$均为无穷小量.
\begin{definition}[不同阶无穷小量]
	若$\lim\limits_{x\to x_0}\frac{f(x)}{g(x)}=0$,则称当$x\to x_0$时$f$为$g$的{\heiti 高阶无穷小量}(infinitemal of higher order),或称$g$为$f$的{\heiti 低阶无穷小量}(infinitemal of lower order),记作
	$$f(x)=o(g(x))(x\to x_0).$$
	特别地,$f$为当$x\to x_0$时的无穷小量记作
	$$f(x)=o(1)(x\to x_0).$$
\end{definition}
\begin{definition}[同阶无穷小量]
	若存在正数$K$和$L$,使得在某$\mathring{U}(x_0)$上有
	$$K\leqslant\left|\frac{f(x)}{g(x)}\right|\leqslant L,$$
	则称$f$与$g$为当$x\to x_0$时的{\heiti 同阶无穷小量}(infinitemal of the same order).特别当
	$$\lim\limits_{x\to x_0}\frac{f(x)}{g(x)}=c\neq 0$$
	时,$f$和$g$必为同阶无穷小量.
	
	若无穷小量$f$与$g$满足关系式
	$$\left|\frac{f(x)}{g(x)}\right|\leqslant L,$$
	则记作
	$$f(x)=O(g(x))(x\to x_0)$$
	特别地,若$f$在某$\mathring{U}(x_0)$上有界,则记为
	$$f(x)=O(1)(x\to x_0).$$
\end{definition}
\begin{remark}
	这里的等式$f(x)=o(g(x))(x\to x_0)$与$f(x)=O(g(x))(x\to x_0)$等,与通常的等式的含义是不同的.这里等式左边是一个函数,右边是一个函数类(函数的集合),而中间的等号的含义是“属于”.例如$f(x)=o(g(x))(x\to x_0)$时,
	$$o(g(x)=\left\{f\big|\lim\limits_{x\to x_0}\frac{f(x)}{g(x)}=0\right\}.$$
\end{remark}
\begin{definition}[等价无穷小量]
	若$\lim\limits_{x\to x_0}\frac{f(x)}{g(x)}=1$,则称$f$与$g$是当$x\to x_0$时的{\heiti 等价无穷小量}(equivalent infinitesimal).记作
	$$f(x)\sim g(x)\quad (x\to x_0).$$
\end{definition}
可以看出,等价无穷小量是同阶无穷小量的一种特殊情况.

以上讨论了两个无穷小量阶的比较.但应指出,并不是任何两个无穷小量都能进行这种阶的比较,只有在两个无穷小量的比是有界量时才能进行阶的比较.

下面的定理显示了等价无穷小量在求极限问题中的作用.
\begin{theorem}
	设函数$f,g,h$在$\mathring{U}(x_0)$上有定义,且有
	$$f(x)\sim g(x)\quad (x\to x_0).$$
	(i)若$\lim\limits_{x\to x_0}f(x)h(x)=A$,则$\lim\limits_{x\to x_0}g(x)h(x)=A$;\\
	(ii)若$\lim\limits_{x\to x_0}\frac{h(x)}{f(x)}=B$,则$\lim\limits_{x\to x_0}\frac{h(x)}{g(x)}=B$.
\end{theorem}
\begin{proof}
	(i)$\lim\limits_{x\to x_0}g(x)h(x)=\lim\limits_{x\to x_0}\dfrac{g(x)}{f(x)}\cdot \lim\limits_{x\to x_0}f(x)h(x)=1\cdot A=A.$
	
	\hspace*{\fill}
	
	(ii)$\lim\limits_{x\to x_0}\dfrac{h(x)}{g(x)}=\lim\limits_{x\to x_0}\dfrac{f(x)}{g(x)}\cdot \lim\limits_{x\to x_0}\dfrac{h(x)}{f(x)}=1\cdot B=B.$
\end{proof}

\hspace*{\fill}

下面给出一些等价无穷小量,供读者证明与参考.
\begin{enumerate}
	\item $x\sim \sin x\sim \tan x\sim \arcsin x\sim \arctan x\sim \text{e}^x-1\sim \ln (1+x)\qquad(x\to 0);$
	\item $1-\cos x\sim \frac{1}{2}x^2\qquad(x\to 0);$
	\item $(1+x)^\alpha-1\sim \alpha x\qquad(x\to 0)(\alpha\text{为非零实数});$
	\item $\alpha^x-1\sim x\ln \alpha\qquad(x\to 0)(\alpha>0\text{且}\alpha\neq 1).$
\end{enumerate}

\hspace*{\fill}

\begin{example}
	求$\lim\limits_{x\to 0}\dfrac{\sqrt{1+x^2}-1}{1-\cos x}.$
\end{example}

\hspace*{\fill}

\begin{solution}
	由于$\sqrt{1+x}-1\sim \frac{1}{2}x$,$1-\cos x\sim \frac{1}{2}x^2$,
	故
	
	\hspace*{\fill}
	$$\lim\limits_{x\to 0}\frac{\sqrt{1+x^2}-1}{1-\cos x}=\dfrac{\frac{1}{2}x^2}{\frac{1}{2}x^2}=1.$$
	$\hfill\blacksquare$
\end{solution}
\begin{remark}
	在利用等价无穷小量代换求极限时,应注意只有对所求极限式中相乘除的因式才能用等价无穷小量来替代.
\end{remark}
\subsection{无穷大量}
\begin{definition}
	设函数$f$在某$\mathring{U}(x_0,\delta')$上有定义.若对任给的$M>0$,存在$\delta>0$,使得当$x\in\mathring{U}(x_0;\delta)(\delta<\delta')$时,有
	$|f(x)|>M$,则称$f$当$x\to x_0$时有{\heiti 非正常极限$\infty$},记作
	$$\lim\limits_{x\to x_0}f(x)=\infty.$$
	对于$f(x)>M$或$f(x)<-M$,我们称$f$当$x\to x_0$时有非正常极限$+\infty$或$-\infty$,记作
	$$\lim\limits_{x\to x_0}f(x)=+\infty\text{或}\lim\limits_{x\to x_0}f(x)=-\infty$$
\end{definition}
\begin{definition}[无穷大量]
	以$\infty,+\infty$或$-\infty$为极限的函数或数列都称为{\heiti 无穷大量}(infinity).
\end{definition}
\begin{remark}
	无穷大量是无界的,但无无界的不一定是无穷大量,比如在$+\infty$和$\-\infty$中振荡的无界函数不是无穷大量.
\end{remark}
对两个无穷大量也可定义高阶无穷大量、同阶无穷大量的概念.由于对无穷大量的研究可以归结到对无穷小量的讨论,因此在此不再详述高阶无穷大量、同阶无穷大量的概念.以下定理展示了无穷小量和无穷大量之间的关系.
\begin{theorem}
	设$f,g$在$\mathring{U}(x_0)$上有定义且不等于$0$.
	
	(i)若$f$为$x\to x_0$时的无穷小量,则$\dfrac{1}{f}$为$x\to x_0$时的无穷大量.
	
	(ii)若$g$为$x\to x_0$时的无穷大量,则$\dfrac{1}{g}$为$x\to x_0$时的无穷小量.
\end{theorem}
\begin{proof}
	仅对$f,g>0$讨论即可,其余情形类似.
	
	(i)$\lim\limits_{x\to x_0}f(x)=0$,则$\forall\varepsilon>0,\ \exists\delta>0$\ s.t.\ $x\in \mathring{U}(x_0;\delta)$时,
	$$f(x)\in\mathring{U}(x_0;\varepsilon),$$
	
	则$\dfrac{1}{f}\in(\frac{1}{\varepsilon},+\infty)$.
	
	令$M=\frac{1}{\varepsilon}$,则$\exists\delta>0$\ s.t.\ $x\in \mathring{U}(x_0;\delta)$时,
	$$\frac{1}{f}>M.$$
	
	故$\lim\limits_{x\to x_0}\dfrac{1}{f}=+\infty.$
	
	(ii)$\lim\limits_{x\to x_0}g(x)=+\infty$,则$\forall M>0,\ \exists\delta>0$\ s.t.\ $x\in \mathring{U}(x_0;\delta)$时,
	$$g(x)>M.$$
	
	令$\varepsilon=\frac{1}{M}$,则$\exists\delta>0$\ s.t.\ $x\in \mathring{U}(x_0;\delta)$时,
	$$\frac{1}{g}\in (0,\varepsilon).$$
	
	故$\lim\limits_{x\to x_0}\dfrac{1}{g}=0.$
	$\hfill\blacksquare$
\end{proof}
\section{曲线的渐近线}
作为函数极限的一个应用,我们讨论曲线的渐近线问题.
\begin{definition}[渐近线]
	若曲线$C$上的动点$P$沿着曲线无限地远离原点时,点$P$与某定直线$L$的距离趋于$0$,则称直线$L$为曲线$C$的{\heiti 渐近线}(asymptote).
	
	渐近线分为{\heiti 斜渐近线}(oblique asymptote)和{\heiti 垂直渐近线}(vertical asymptote).
\end{definition}
假设曲线$y=f(x)$有斜渐近线$y=kx+b$,则当$x\to \infty$时,有
$$\lim\limits_{x\to \infty}\left[f(x)-(kx+b)\right]=0,$$
即
\begin{equation}{\label{b}}
	\lim\limits_{x\to \infty}\left[f(x)-kx\right]=b
\end{equation}
又由
$$\lim\limits_{x\to \infty}\left[\frac{f(x)}{x}-k\right]=\lim\limits_{x\to \infty}\frac{1}{x}\left[f(x)-kx\right]=0\cdot b=0,$$
得
\begin{equation}{\label{k}}
	\lim\limits_{x\to \infty}\frac{f(x)}{x}=k
\end{equation}

由上面的讨论可知,如果曲线$y=f(x)$有斜渐近线$y=kx+b$,则$k$和$b$可分别由式\ref{k}和式\ref{b}确定.

若函数$f$满足$$\lim\limits_{x\to x_0(\text{或}x_0^\pm)}f(x)=\infty,$$
则按渐近线的定义可知,曲线$y=f(x)$有垂直渐近线$x=x_0$.
\section{部分函数定义补叙}
\section{函数的上极限和下极限}
在研究数列时我们介绍了上极限和下极限,并非每个数列都有极限,但是任何数列都有上极限和下极限.类似地,我们引入函数的上极限和下极限.

设函数$f$在$x_0$附近有定义且有界.根据Heine归结原则,若$\lim\limits_{x\to x_0}f(x)=a$,则对于任一趋于$x_0$的数列$\{x_n\}\subseteq \mathring{U}(x_0)$,都满足$f(x_n)\to a$.若$\lim\limits_{x\to x_0}$不存在,我们也可以取一个趋于$x_0$的数列$\{x_n\}\subseteq U(x_0;\delta)$,对应地可以得到数列$f(x_n)$.由于$f(x)$在$x_0$附近有界,故$f(x_n)$有界.由Bolzano-Weierstrass定理可知,$f(x_n)$存在一个收敛子列$f(x_{k_n})$.而$\{x_{k_n}\}$是$\{x_n\}$的一个子列,由于$x_n\to x_0$,故$x_{k_n}\to x_0$.对于在$x_0$附近上无界的函数,则可以找到一个趋于$x_0$的数列$\{x_n\}\subseteq U(x_0;\delta)$使得$f(x_n)\to\pm\infty$.

以上讨论表明,对于在$x_0$附近有定义的函数,总存在一个趋于$x_0$的数列$\{x_n\}\subseteq \mathring{U}(x_0)$使得$\lim\limits_{n\to\infty}f(x_n)=l$,其中$l\in\widetilde{\mathbb{R}}$.于是我们可以用这样的$l$组成的集合的上确界和下确界来定义函数的上极限和下极限.
\begin{definition}
	设函数$f$在$\mathring{U}(x_0)$有定义.令
	$$E=\left\{a\in \widetilde{\mathbb{R}}|\text{存在}x_n\in\mathring{U}(x_0;\delta),\ x_n\to x_0\text{时}f(x_n)\to a\right\}.$$
	令
	$$\limsup\limits_{x\to x_0}f(x)\coloneqq\sup E,\ \liminf\limits_{x\to x_0}f(x)\coloneqq\inf E.$$
	我们称$\limsup\limits_{x\to x_0}f(x)$为$f(x)$的{\heiti 上极限}(limit superior),称$\liminf\limits_{x\to x_0}f(x)$为$f(x)$的{\heiti 下极限}(limit inferior).
\end{definition}
\begin{remark}
	对其他的极限过程,上极限和下极限也可类似定义.
\end{remark}
\begin{proposition}
	设函数$f$在$x_0$的一个去心邻域$\mathring{U}(x_0;\delta)$内有定义.令
	$$E=\left\{a\in \widetilde{\mathbb{R}}|\text{存在}x_n\in\mathring{U}(x_0;\delta),\ x_n\to x_0\text{时}f(x_n)\to a\right\}.$$
	则$\limsup\limits_{x\to x_0}f(x),\ \liminf\limits_{x\to x_0}f(x)\in E$.
\end{proposition}
\begin{proof}
	只证明上极限的情况即可.
	
	(i)若$\limsup\limits_{x\to x_0}f(x)=+\infty$,则$E$无上界,因此对于任一$n\in\mathbb{N}_+$都存在$l_n\in E$使得$l_n>n$.即对于任一$n\in\mathbb{N}_+$都存在$x_n$满足$0<|x_n-x_0|<1/n$且$f(x_n)>n$.令$n\to \infty$,则$x_n\to x_0$且$f(x_n)\to +\infty$.于是可知$+\infty\in E$.
	
	(ii)若$\limsup\limits_{x\to x_0}f(x)=-\infty$,则$E=\{-\infty\}$,因此$-\infty\in E$.
	
	(iii)若$\limsup\limits_{x\to x_0}f(x)=a\in\mathbb{R}$,由于$a=\sup E$,故对于任一$n\in \mathbb{N}_+$都存在$l_n\in E$使得
	$$a-\frac{1}{n}<l_n<a+\frac{1}{n}.$$
	因此对于任一$n\in \mathbb{N}_+$都存在$x_n$满足
	$$0<|x_n-x_0|<\frac{1}{n},\qquad a-\frac{1}{n}<f(x_n)<a+\frac{1}{n}.$$
	令$n\to \infty$,则$x_n\to x_0$且$f(x_n)\to a$,于是可知$a\in E$.
	$\hfill\blacksquare$
\end{proof}
\begin{theorem}
	设函数在某邻域$\mathring{U}(x_0)$上有定义.则
	$$\liminf\limits_{x\to x_0}f(x)\leqslant \limsup\limits_{x\to x_0}f(x).$$
	等号成立当且仅当$f(x)$在$x\to x_0$时有极限,即
	$$\liminf\limits_{x\to x_0}f(x)= \limsup\limits_{x\to x_0}f(x)=\lim\limits_{x\to x_0}f(x).$$
\end{theorem}
与数列上极限和下极限类似,我们也可以用$\varepsilon-\delta$语言来刻画函数的上极限和下极限.
\begin{theorem}
	设函数$f$在$x_0$的一个去心邻域$\mathring{U}(x_0;\delta')$内有定义.令
	$$E=\{a\in\mathbb{R}:\forall\varepsilon>0,\exists \delta>0(\delta<\delta'),\text{当}x\in\mathring{U}(x_0;\delta)\text{时},f(x)<a+\varepsilon\};$$
	$$F=\{a\in\mathbb{R}:\forall\varepsilon>0,\exists \delta>0(\delta<\delta'),\text{当}x\in\mathring{U}(x_0;\delta)\text{时},f(x)>a-\varepsilon\};$$
	则$\limsup\limits_{x\to x_0}f(x)=\inf E,\ \liminf\limits_{x\to x_0}f(x)=\sup F$.
\end{theorem}
\begin{proof}
	只需证明上极限的情况.令$\limsup\limits_{x\to x_0}f(x)=L$.
	
	(i)当$f(x)$在$x_0$的任一去心邻域中都无上界时,$L=+\infty$,此时$E=\{+\infty\}$,因此$\inf E=L$.
	
	(ii)当$f(x)$在$x_0$的一个去心邻域中有界时.
	
	证明$L\geqslant \inf E$.只需证明$L\in E$.假设$L\notin E$,即存在$\varepsilon>0$使得$x_0$的任一去心邻域内都存在$x$满足$f(x)\geqslant L+\varepsilon$.这表明$\mathring{U}(x_0)$中存在数列$x_n\to x_0$使得$f(x_n)\to l>L$,这与$L=\limsup\limits_{x\to x_0}f(x)$矛盾.于是可知$L\in E$.
	
	证明$L\leqslant\inf E$.任取$a\in E$,假设$a<L$,则存在$\varepsilon>0$使得$a+\varepsilon<L$,此时存在$\delta$使得当$x\in \mathring{U}(x_0;\delta)$时,$f(x)<a+\varepsilon$,这表明不存在数列$\{x_n\}\in\mathring{U}(x_0;\delta)$使得$f(x_n)\to L$,出现矛盾.于是可知$a\geqslant L$,即$L\leqslant\inf E$.
	
	综上所述,得$\limsup\limits_{x\to x_0}f(x)=\inf E$.$\hfill\blacksquare$
\end{proof}
\begin{remark}
	为了让上极限是$+\infty$下极限是$-\infty$的情况也能用上面的语言刻画,可以令
	$$E=\{a\in\widetilde{\mathbb{R}}:\forall m>a,\exists \delta>0,\text{当}x\in\mathring{U}(x_0;\delta)\text{时},f(x)<m\},$$
	$$F=\{a\in\widetilde{\mathbb{R}}:\forall m<a,\exists \delta>0,\text{当}x\in\mathring{U}(x_0;\delta)\text{时},f(x)>m\}.$$
	则$\limsup\limits_{n\to\infty}f(x)=\inf E,\liminf\limits_{n\to\infty}f(x)=\sup F.$
\end{remark}
类似地,我们给出函数上极限和下极限的第三种定义.
\begin{theorem}
	设函数$f$在$x_0$的某去心邻域内有定义.则
	
	(1)$\limsup\limits_{x\to x_0}f(x)=\lim\limits_{\delta\to 0^+}\left(\sup\limits_{x\in \mathring{U}(x_0;\delta)}f(x)\right)$;
	
	(2)$\liminf\limits_{x\to x_0}f(x)=\lim\limits_{\delta\to 0^+}\left(\inf\limits_{x\in \mathring{U}(x_0;\delta)}f(x)\right)$.
\end{theorem}
\begin{proof}
	只证明上极限的情况即可.令
	$$\varphi(\delta)=\sup\limits_{x\in \mathring{U}(x_0;\delta)}f(x),\quad L=\limsup\limits_{x\to x_0}f(x)$$
	容易验证$\varphi(\delta)$单调递增.
	
	(i)当$L=+\infty$时,在$x_0$的一个去心邻域中存在一个数列$x_n\to x_0$使得$f(x_n)\to +\infty$.因此$\varphi(\delta)=\sup\limits_{x\in \mathring{U}(x_0;\delta)}f(x)=+\infty$.于是可知$\lim\limits_{\delta\to 0^+}\varphi(\delta)=+\infty$.
	
	(ii)当$L=-\infty$时,对于$x_0$的去心邻域中任一极限为$x_0$的数列$x_n$,都有$f(x_n)\to -\infty$,即$\lim\limits_{x\to x_0}f(x)=-\infty$.故对于任一$M>0$都存在$\delta_1>0$使得当$x\in\mathring{U}(x_0;\delta_1)$时$f(x)<-M$,因此
	$$\varphi(\delta_1)=\sup\limits_{x\in \mathring{U}(x_0;\delta)}f(x)<-M.$$
	由于$\varphi(\delta)$单调递增,因此当$\delta\in(0,\delta_1)$时$\varphi(\delta)\leqslant\varphi(\delta_1)<-E$.这表明$\lim\limits_{\delta\to 0}\varphi(\delta)=-\infty$.
	
	(iii)当$L\in\mathbb{R}$时,任取
	$$l\in E=\{a\in\mathbb{R}|\exists x_n\in\mathring{U}(x_0;\delta),\ x_n\to x_0\ s.t.\ f(x_n)\to a\}.$$
	则在$x_0$的一个去心邻域中存在一个数列$x_n\to x_0$使得$f(x_n)\to l$.取$f(x_n)$的一个子列$f(x_{k_n})$使得$\{x_{k_n}\}\in\mathring{U}(x_0;\delta)$.于是
	$$f(x_{k_n})\leqslant \sup\limits_{x\in \mathring{U}(x_0;\delta)}f(x)=\varphi(\delta),\quad i=1,2,\cdots.$$
	令$i\to\infty$得$l\leqslant\varphi(\delta)$.再令$\delta\to 0^+$得$l\leqslant\lim\limits_{\delta\to 0^+}\varphi(\delta)$.于是$L\leqslant\lim\limits_{\delta\to 0^+}\varphi(\delta)$.
	
	另一方面,由函数上极限和下极限的$\varepsilon-\delta$定义可知,对任意$\varepsilon>0$都存在$\delta_2$>0使得当$x\in\mathring{U}(x_0;\delta_2)$时$f(x)<L+\varepsilon$.因此$\varphi(\delta_2)<L+\varepsilon$.由于$\varphi(\delta)$单调递增,因此当$\delta\in(0,\delta_2)$时,$\varphi(\delta)\leqslant\varphi(\delta_2)\leqslant L+\varepsilon$.令$\delta\to 0^+$,则
	$$\lim\limits_{\delta\to 0^+}\varphi(\delta)\leqslant L+\varepsilon.$$
	令$\varepsilon\to 0$,则
	$$L\geqslant\lim\limits_{\delta\to 0^+}\varphi(\delta).$$
	于是可知$L=\lim\limits_{\delta\to 0^+}\varphi(\delta)$.
	$\hfill\blacksquare$
\end{proof}
函数的上极限和下极限也有保序性.
\begin{theorem}[保序性]
	设函数$f(x)$和$g(x)$在$\mathring{U}(x_0;\delta)$中满足$f(x)\leqslant g(x)$,则
	
	(1)$\limsup\limits_{x\to x_0}f(x)\leqslant\limsup\limits_{x\to x_0}g(x),$
	
	(2)$\liminf\limits_{x\to x_0}f(x)\leqslant\liminf\limits_{x\to x_0}g(x).$
\end{theorem}
\begin{proof}
	只需证明(1).记$\limsup\limits_{x\to x_0}f(x)=A,\ \limsup\limits_{x\to x_0}g(x)=B.$
	
	(i)当$B=+\infty$或$A=-\infty$时,命题显然成立;
	
	(ii)当$A=+\infty$时,在$\mathring(x_0;\delta)$中存在数列$x_n\to x_0$使得$\lim\limits_{n\to \infty}g(x_{k_n})=+\infty.$由于在$\mathring{U}(x_0;\delta)$中满足$f(x)\leqslant g(x)$,故$\{x_n\}$中存在一个子列$\{x_{k_n}\}$使得$\lim\limits_{n\to\infty}g(x_{k_n})=+\infty$.
	$$\lim\limits_{n\to\infty}f(x_{k_n})=A\leqslant\lim\limits_{n\to\infty}g(x_{k_n})\leqslant B.$$
	于是可知$A=B$,类似可证$B=-\infty$时$A=B$.
	
	(iii)当$A,B\in\mathbb{R}$时,在$\mathring{U}(x_0;\delta)$中存在数列$x_n\to x_0$使得$f(x_n)\to A$.由于在$\mathring{U}(x_0;\delta)$中满足$f(x)\leqslant g(x)$,故$\{x_n\}$中存在一个子列$\{x_{k_n}\}$使得
	$$\lim\limits_{n\to\infty}g(x_{k_n})\geqslant A=\lim\limits_{n\to\infty}f(x_{k_n}).$$
	于是可知$A\leqslant B$.$\hfill\blacksquare$
\end{proof}


\newpage

\chapter{函数的连续性}
连续函数是数学分析中着重讨论的一类函数.
\section{连续与间断}
\subsection{函数在一点的连续性}
\begin{definition}
	设函数$f$在某$U(x_0)$上有定义.若
	$$\lim\limits_{x\to x_0}f(x)=f(x_0),$$
	则称$f${\heiti 在点$x_0$连续}.
\end{definition}
记$\Delta x=x-x_0$,称为$x$在点$x_0$处的增量.设$y_0=f(x_0)$,相应的函数$y$在点$x_0$处的增量记为
$$\Delta y=f(x)-f(x_0)=f(x_0+\Delta x)-f(x_0)=y-y_0.$$
则函数在一点处连续的定义等价如下.
\begin{definition}
	设函数$f$在某$U(x_0)$上有定义.若
	$$\lim\limits_{\Delta x\to 0}\Delta y=0,$$
	则称$f${\heiti 在点$x_0$连续}.
\end{definition}
由于函数在一点处的连续性是通过极限来定义的,因而也可直接用$\varepsilon-\delta$语言来叙述.
\begin{definition}
	若对任给的$\varepsilon>0$,存在$\delta>0$,使得当$|x-x_0|<\delta$时,有
	$$|f(x)-f(x_0)|<\varepsilon,$$
	则称函数$f$在点$x_0$连续.
\end{definition}
\begin{definition}
	设函数$f$在某$U_+(x_0)$上有定义.若
	$$\lim\limits_{x\to x_0^+}f(x)=f(x_0),$$
	则称$f$在点$x_0${\heiti 右连续}.
	
	类似地,我们可以定义左连续.
\end{definition}
\begin{theorem}
	函数$f$在点$x_0$连续的充要条件是:$f$在$x_0$点既是右连续,又是左连续.
\end{theorem}

函数$f$在$x_0$处连续,意味着$f$在$x_0$处有极限,且极限值为$x_0$.而函数在$x_0$处有极限意味着它在这一点的上极限和下极限相等,且上极限和下极限都等于$f(x_0)$.从这个角度也可以刻画函数的逐点连续.为了叙述方便,我们给出函数振幅的概念.
\begin{definition}[函数在区间上的振幅]
	设区间$I$上的函数$f$.令
	$$\omega(I)\coloneqq\sup f(I)-\inf f(I),$$
	我们称$\omega(I)$为$f$在$I$上的{\heiti 振幅}(amplitude).
\end{definition}
振幅还有一种常用的等价定义.
\begin{proposition}
	设区间$I$上的函数$f$.令
	$$\omega=\sup\left[f(x_1)-f(x_2)\right],\quad(\forall x_1,x_2\in I),$$
	则$\omega=\omega(I)$.
\end{proposition}
\begin{proof}
	令$M=\sup f(I)$,$m=\inf f(I)$,只需证明$\omega=M-m$.
	
	(i)对于任意$x_1,x_2\in I$,都有
	$$m\leqslant f(x_1)\leqslant M,\qquad m\leqslant f(x_2)\leqslant M.$$
	因此
	$$|f(x_1)-f(x_2)|\leqslant M-m.$$
	于是可知$\omega\leqslant M-m.$
	
	(ii)对于任一$\varepsilon>0$都存在$x_1,x_2\in I$使得
	$$f(x_1)>M-\frac{\varepsilon}{2},\qquad f(x_2)<m+\frac{\varepsilon}{2}.$$
	因此
	$$|f(x_1)-f(x_2)|\geqslant f(x_1)-f(x_2)>M-m-\varepsilon.$$
	于是可知$\omega\geqslant M-m$.
	
	综上,有$\omega=M-m=\omega(I).$
	$\hfill\blacksquare$
\end{proof}
\begin{definition}[函数在一点的振幅]\label{def:amplitude}
	设区间$I$上的函数$f$,令
	$$\omega(x_0)\coloneqq\lim\limits_{\delta\to 0^+}\omega\left[U(x_0;\delta)\right]=\lim\limits_{\delta\to 0^+}\left[\sup\limits_{x\in U(x_0;\delta)}f(x)-\inf\limits_{x\in U(x_0;\delta)}f(x)\right].$$
	我们称$\omega(x_0)$为$f$在点$x_0$处的{\heiti 振幅}(amplitude).
\end{definition}
\begin{remark}
$\sup\limits_{x\in U(x_0;\delta)}f(x)$单调递减,而$\inf\limits_{x\in U(x_0;\delta)}f(x)$单调递增.因此$\sup\limits_{x\in U(x_0;\delta)}f(x)-\inf\limits_{x\in U(x_0;\delta)}f(x)$单调递增.又因为
$$\sup\limits_{x\in U(x_0;\delta)}f(x)-\inf\limits_{x\in U(x_0;\delta)}f(x)\geqslant 0,$$
因此定义式右侧有极限,这说明定义是合理的.
\end{remark}
用振幅的观点也可以刻画函数的逐点连续.
\begin{theorem}\label{amplitude}
	函数$f$在$x_0$处连续当且仅当$f$在$x_0$处的振幅$\omega(x_0)=0$.
\end{theorem}
\begin{proof}
	必要性\qquad 若$f$在$x_0$处连续,则对于任一$\varepsilon>0$存在$\delta>0$使得对于任意$x_1,x_2\in U(x_0;\delta)$都有
	$$|f(x_1)-f(x_0)|<\frac{\varepsilon}{2},\quad |f(x_2)-f(x_0)|<\frac{\varepsilon}{2}.$$
	因此
	$$|f(x_1)-f(x_2)|\leqslant|f(x_1)-f(x_0)|+|f(x_2)-f(x_0)|<\frac{\varepsilon}{2}+\frac{\varepsilon}{2}=\varepsilon.$$
	因此$\omega\left[U(x_0;\delta)\right]\leqslant\varepsilon$.令$\delta\to 0^+$得$0\leqslant\omega(x_0)\leqslant\varepsilon$.令$\varepsilon\to 0$即得$\omega(x_0)=0$.
	
	充分性\qquad 若$\omega(x_0)=0$,则
	$$\lim\limits_{\delta\to 0^+}\omega\left[U(x_0;\delta)\right]=0.$$
	因此对于任一$\varepsilon>0$,存在$\delta>0$使得$\omega\left[U(x_0;\delta)\right]<\varepsilon$.因此对于任一$x\in U(x_0;\delta)$都有
	$$|f(x)-f(x_0)|\leqslant\omega\left[U(x_0;\delta)\right]<\varepsilon.$$
	这表明$\lim\limits_{x\to x_0}f(x)=f(x_0)$.于是$f$在$x_0$处连续.$\hfill\blacksquare$
\end{proof}
不难看出函数在一点的振幅和函数在一点的上极限和下极限的定义只差了$x_0$这“一点”.
\begin{theorem}
	设函数$f$在$x_0$附近有定义.则
	$$\lim\limits_{\delta\to 0^+}\left[\sup\limits_{x\in\mathring{U}(x_0;\delta)}f(x)\right]=\lim\limits_{\delta\to 0^+}\left[\inf\limits_{x\in\mathring{U}(x_0;\delta)}f(x)\right]=f(x_0)\iff \lim\limits_{\delta\to 0^+}\left[\sup\limits_{x\in U(x_0;\delta)}f(x)\right]=\lim\limits_{\delta\to 0^+}\left[\inf\limits_{x\in U(x_0;\delta)}f(x)\right].$$
\end{theorem}
\begin{proof}
	由定理\ref{amplitude}可知
	\begin{align*}
		&\lim\limits_{\delta\to 0^+}\left[\sup\limits_{x\in\mathring{U}(x_0;\delta)}f(x)\right]=\lim\limits_{\delta\to 0^+}\left[\inf\limits_{x\in\mathring{U}(x_0;\delta)}f(x)\right]=f(x_0)\iff\limsup\limits_{x\to x_0}f(x)=\liminf\limits_{x\to x_0}f(x)=f(x_0)\\
		&\iff\lim\limits_{x\to x_0}f(x)=f(x_0)\iff f\text{在}x_0\text{处连续}\iff\omega(x_0)=0\\
		&\iff\lim\limits_{\delta\to 0^+}\left[\sup\limits_{x\in U(x_0;\delta)}f(x)-\inf\limits_{x\in U(x_0;\delta)}f(x)\right]=0\\
		&\iff\lim\limits_{\delta\to 0^+}\left[\sup\limits_{x\in U(x_0;\delta)}f(x)\right]=\lim\limits_{\delta\to 0^+}\left[\inf\limits_{x\in U(x_0;\delta)}f(x)\right].
	\end{align*}
	$\hfill\blacksquare$
\end{proof}
\subsection{区间上的连续函数}
\begin{definition}
	若函数$f$在区间$I$上的每一点都连续,则称$f$为$I$上的{\heiti 连续函数}(continuous function).对于闭区间或半开半闭区间的端点,函数在这些点上连续是指左连续或右连续.
\end{definition}
后面我们将证明任何初等函数在其定义区间上为连续函数.同时,也存在着在其定义区间上每一点都不连续的函数.
\subsection{间断点及其分类}
\begin{definition}[间断点]
	设函数在某$\mathring{U}(x_0)$上有定义.若$f$在点$x_0$无定义,或$f$在点$x_0$有定义而不连续,则称$x_0$为$f$的{\heiti 间断点}(point of discontinuity).
\end{definition}
我们对函数的间断点做如下分类.
\begin{definition}[可去间断点]
	若
	$$\lim\limits_{x\to x_0}f(x)=A,$$
	而$f$在$x_0$无定义或有定义但$f(x_0)\neq A$,则称$x_0$为$f$的可去间断点.
\end{definition}
\begin{definition}[跳跃间断点]
	若函数$f$在点$x_0$处的左右极限都存在,但
	$$\lim\limits_{x\to x_0^+}f(x)\neq \lim\limits_{x\to x_0^-}f(x),$$
	则称$x_0$为$f$的跳跃间断点.
\end{definition}
可去间断点和跳跃间断点统称为{\heiti 第一类间断点}.

函数的所有其他形式的间断点,即使得函数至少有一侧极限不存在的那些点,称为{\heiti 第二类间断点}.

前面介绍了用振幅的观点来刻画函数的连续点,类似地也可以用振幅的观点刻画不连续点.
\begin{theorem}
	$x_0$是函数$f$的一个间断点当且仅当$\omega(x_0)>0$.
\end{theorem}
设$I$上的函数$f$,把$f$的间断点的集合记作$D(f)$.用振幅的大小可以把不连续点归类.令
$$D_\delta\coloneqq\{x\in I|\omega(x)\geqslant\delta\}.$$
\begin{proposition}\label{prop:jianduan}
	设$I$上的函数$f$.则
	$$D(f)=\bigcup_{n=1}^\infty D_{1/n}.$$
\end{proposition}
\begin{proof}
	显然$D(f)\supseteq\displaystyle\bigcup_{n=1}^{\infty}D_{1/n}$,因此只需证明$D(f)\subseteq\displaystyle\bigcup_{n=1}^{\infty}D_{1/n}$.任取$x_0\in D(f)$,则$x_0$是$f$的一个间断点,即$\omega(x_0)>0$.因此存在$m$使得$\omega(x_0)\geqslant\frac{1}{m}$,这表明$x_0\in D_{1/m}$.因此$D(f)\subseteq\displaystyle\bigcup_{n=1}^{\infty}D_{1/n}$.于是可知
	$$D(f)=\bigcup_{n=1}^{\infty}D_{1/n}.$$
	$\hfill\blacksquare$
\end{proof}
\section{连续函数的性质}
\subsection{连续函数的局部性质}
\begin{theorem}[局部有界性]
	若函数$f$在点$x_0$连续,则$f$在某$U(x_0)$上有界.
\end{theorem}
\begin{theorem}[局部保号性]
	若函数$f$在点$x_0$连续,且$f(x_0)>0$,则对任何正数$r<f(x_0)$,存在某$U(x_0)$,使得对一切$x\in U(x_0)$,有
	$$f(x)>r.$$
\end{theorem}
\begin{theorem}[四则运算]
	若函数$f$和$g$在点$x_0$连续,则$f\pm g,f\cdot g,f/g$在有意义的情况下也都在点$x_0$连续.
\end{theorem}
以上三个定理的证明,都可从函数极限的有关定理直接推得.
\begin{theorem}[复合函数的连续性]
	若函数$f$在点$x_0$连续,$g$在点$u_0$连续,$u_0=f(x_0)$,则复合函数$g\circ f$在点$x_0$连续. 
\end{theorem}
\begin{proof}
	由于$g$在$u_0$连续,对任给的$\varepsilon>0$,存在$\delta_1>0$,使得当$|u-u_0|<\delta_1$时,有
	$$|g(u)-g(u_0)|<\varepsilon.$$
	又由$u_0=f(x_0)$及$u=f(x)$在点$x_0$连续,故对上述$\delta_1>0$,存在$\delta>0$,使得当$|x-x_0|<\delta$时,有$|u-u_0|=|f(x)-f(x_0)|<\delta_1$.由此,对任给的$\varepsilon>0$,存在$\delta>0$,当$|x-x_0|<\delta$时,有
	$$|g(f(x))-g(f(x_0))|<\varepsilon.$$
	这就证明了$g\circ f$在点$x_0$处连续.$\hfill\blacksquare$
\end{proof}
\begin{remark}
	根据连续性的定义,上述定理的结论可表示为
	$$\lim\limits_{x\to x_0}g(f(x))=g(\lim\limits_{x\to x_0}f(x))=g(f(x_0)).$$
\end{remark}
\subsection{反函数的连续性}
\begin{theorem}
	若函数$f$在$\left[a,b\right]$上{\heiti 严格单调并连续},则反函数$f^{-1}$在其定义域$\left[f(a),f(b)\right]$或$\left[f(b),f(a)\right]$上连续.
\end{theorem}
\begin{proof}
	不妨设$f$在$\left[a,b\right]$上严格递增.此时$f$的值域即$f^{-1}$的定义域为$\left[f(a),f(b)\right]$.任取$y_0\in(f(a),f(b))$,设$x_0=f^{-1}(y_0)$,则$x_0\in(a,b)$.于是对任给的$\varepsilon>0$,可在$(a,b)$上$x_0$的两侧各取异于$x_0$的点$x_1,x_2(x_1<x_0<x_2)$,使它们与$x_0$的距离小于$\varepsilon$.
	
	设与$x_1,x_2$对应的函数值分别为$y_1,y_2$,由$f$的严格递增性可知$y_1<y_0<y_2$.令
	$$\delta=\min\{y_2-y_0,y_0-y_1\},$$
	则当$y\in\mathring{U}(y_0;\delta)$时,对应的$x=f^{-1}(y)$的值都落在$x_1$与$x_2$之间,故有
	$$|f^{-1}(y)-f^{-1}(y_0)|=|x-x_0|<\varepsilon,$$
	这就证明了$f^{-1}$在$y_0$处连续,由$y_0$的任意性可知$f^{-1}$在$(f(a),f(b))$上连续.
	
	类似可证$f^{-1}$在其定义区间的端点$f(a)$与$f(b)$上分别为右连续和左连续,因此$f^{-1}$在$\left[f(a),f(b)\right]$上连续.$\hfill\blacksquare$
\end{proof}
\subsection{闭区间上连续函数的性质}
\begin{definition}[函数的最值]
	设$f$是定义在数集$D$上的函数.若存在$x_0\in D$,使得对一切$x\in D$,有
	$$f(x_0)\geqslant f(x)$$
	则称$f$在$D$上有最大值,并称$f(x_0)$为$f$在$D$上的最大值.
	
	若存在$x_0\in D$,使得对一切$x\in D$,有
	$$f(x_0)\leqslant f(x)$$
	则称$f$在$D$上有最小值,并称$f(x_0)$为$f$在$D$上的最小值.
\end{definition}
\begin{theorem}[有界性定理]
	若函数$f(x)$在闭区间$\left[a,b\right]$上连续,那么$f(x)$在闭区间$\left[a,b\right]$上有界.
\end{theorem}
\begin{proof}
	只需证明有上界的情况.用反证法.
	
	假设$f(x)$无上界,则存在$x_n\in\left[a,b\right]$,使得
	$$f(x_n)>n\qquad n=1,2,\cdots.$$
	由此得$\lim\limits_{n\to \infty}f(x_n)=+\infty$.
	
	另一方面,由于$\{x_n\}$是有界数列,由致密性原理,$\{x_n\}$有收敛的子列$\{x_{n_k}\}$,设$\lim\limits_{k\to\infty}x_{n_k}=x_0$.
	
	由于
	$$a\leqslant x_{n_k}\leqslant b,$$
	由极限的不等式性质推得
	$$a\leqslant x_0\leqslant b,$$
	故$f(x)$在$x_0$处连续.由归结原则,
	$$+\infty=\lim\limits_{n\to\infty}f(x_n)=\lim\limits_{k\to\infty}f(x_{n_k})=\lim\limits_{x\to x_0}f(x)=f(x_0).$$
	这与$f(x)$在$x_0$处连续矛盾.$\hfill\blacksquare$
\end{proof}
\begin{theorem}[最大、最小值定理]
	若函数$f(x)$在闭区间$\left[a,b\right]$上连续,则$f(x)$在$\left[a,b\right]$上有最大值与最小值.
\end{theorem}
\begin{proof}
	由有界性定理和确界原理,$f(x)$存在上确界
	$$\sup\limits_{x\in\left[a,b\right]}f(x)=M.$$
	下面证明:存在$\xi\in\left[a,b\right]$,使$f(\xi)=M$.用反证法,
	假设对一切$x\in\left[a,b\right]$,都有$f(x)<M$.令
	$$g(x)=\frac{1}{M-f(x)},\qquad x\in \left[a,b\right].$$
	显然$g(x)$在$\left[a,b\right]$上连续,且取正值,故$g$在$\left[a,b\right]$上有上界,记为$G$.则有
	$$0<g(x)=\frac{1}{M-f(x)}\leqslant G,\qquad x\in \left[a,b\right].$$
	从而推得
	$$f(x)\leqslant M-\frac{1}{G},\qquad x\in\left[a,b\right].$$
	这与$M$是$f(\left[a,b\right])$的上确界矛盾.故存在$\xi\in\left[a,b\right]$,使$f(\xi)=M$.
	
	同理可证$f$在$\left[a,b\right]$上有最小值.$\hfill\blacksquare$
\end{proof}
\begin{theorem}[介值定理]
	设函数$f(x)$在闭区间$\left[a,b\right]$上连续,且$f(a)\neq f(b)$.若$\mu$为介于$f(a)$和$f(b)$之间的任何实数,则至少存在一点$x_0\in(a,b)$,使得$f(x_0)=\mu.$
\end{theorem}
\begin{corollary}[根的存在定理]
	若函数在闭区间$\left[a,b\right]$上连续,且$f(a)f(b)<0$,则至少存在一点$x_0\in(a,b)$,使得$f(x_0)=0$.
\end{corollary}
要证介值定理,可以将其转化为证明根的存在定理.
\begin{proof}
	不妨设$f(a)<\mu<f(b)$.令$g(x)=f(x)-\mu$,则$g$也是$\left[a,b\right]$上的连续函数,且$g(a)<0,g(b)>0.$于是定理的结论转化为:存在$x_0\in\left[a,b\right]$,使得$f(x_0)=0$.
	
	设集合
	$$E=\{x|g(x)<0,x\in\left[a,b\right]\}.$$
	
	显然$E$为非空有界数集,故由确界原理,存在上确界$x_0\in\sup E.$另一方面,因为$g(a)<0,g(b)>0$,由连续函数的局部保号性,存在$\delta>0$,使得
	$$g(x)<0,\qquad x\in\left[a,a+\delta\right);$$
	$$g(x)>0,\qquad x\in\left(b-\delta,b\right].$$
	由此易见$x_0\neq a,x_0\neq b$,即$x_0\in(a,b)$.
	
	下证$g(x_0)=0$.用反证法,假设$g(x_0)\neq 0$,则$g(x_0)<0$.由局部保号性,存在$U(x_0;\eta)(\subset(a,b))$,使在其上$g(x)<0$,特别有$g(x_0+\frac{\eta}{2})\in E.$但这与$x_0=\sup E$相矛盾,故必有$g(x_0)=0$.$\hfill\blacksquare$
\end{proof}
\subsection{一致连续性}
函数$f$在区间上连续,是指$f$在该区间上的每一点都连续.下面讨论的一致连续性反映了函数在区间上更强的连续性.在连续性的定义中,对于给定的$\varepsilon>0$,在不同的点$x_0$处,相应的$\delta$不一定相同.我们提出这样的问题:对于任意的$\varepsilon>0$,是否存在适用于一切$x_0\in E$的$\delta$,使得只要$$x,x_0\in E,\qquad |x-x_0|<\delta,$$
就有$$|f(x)-f(x_0)|<\varepsilon?$$
由此我们引出一致连续的定义.
\begin{definition}[一致连续]
	设$E$是$\mathbb{R}$的一个子集,函数$f$在$E$上有定义.若对任意$\varepsilon>0$,存在$\delta>0$,使得只要
	$$x_1,x_2\in E,\qquad |x_1-x_2|<\delta,$$
	就有$$|f(x_1)-f(x_2)|<\varepsilon$$
	那么我们就说函数$f$在集合$E$上是{\heiti 一致连续}的.
\end{definition}
函数$f$在区间$I$上连续时,$\delta$的取值与$\varepsilon$和$x$都有关,因此我们写$\delta=\delta(\varepsilon,x)$表示$\delta$与$\varepsilon$和$x$的依赖关系.如果能做到$\delta$只与$\varepsilon$有关,而与$x$无关,或者说存在一个适合所有$x$的公共的$\delta=\delta(\varepsilon)$,那么函数不仅在$I$上连续,而且是一致连续.

一般地,函数在某区间上连续并不一定能推出一致连续,但对于闭区间却可以推出.于是我们有以下定理.
\begin{theorem}[Heine-Cantor一致连续性定理]
	若函数在闭区间$\left[a,b\right]$上连续,则$f$在$\left[a,b\right]$上一致连续.
\end{theorem}
\begin{proof}
	用反证法\qquad 假设存在$\varepsilon_0>0$,对任意$\delta>0$,存在点列$\{x_n\},\{y_n\}\in \left[a,b\right]$,使得$|x_n-y_n|<\delta$时,
	$$|f(x_1-f(x_2))|\geqslant\varepsilon_0.$$
	由于$\delta$的任意性,有
	$$\lim\limits_{n\to\infty}|x_n-y_n|=0.$$
	又$\{x_n\},\{y_n\}$有界,由致密性原理,存在收敛子列$\{x_{n_k}\},\{y_{n_k}\}$使得
	$$\lim\limits_{k\to\infty}x_{n_k}=\lim\limits_{k\to\infty}y_{n_k}=x_0.$$
	由极限的不等式性质可推知$a\leqslant x_0\leqslant b$.故$f(x)$在点$x_0$处连续.由归结原则,
	$$\lim\limits_{k\to\infty}|f(x_{n_k})-f(y_{n_k})|=0<\varepsilon_0.$$
	与假设矛盾.$\hfill\blacksquare$
\end{proof}
\begin{remark}
	闭区间确保了$x_0\in\left[a,b\right]$,如果是开区间,当$x_{n_k}$收敛到端点处时上述结论将不成立.
\end{remark}
如果把以上定理的条件放宽,可以得到以下结论,证明方法完全一样.
\begin{proposition}\label{prop:biji}
	设闭区间$I$上的函数$f$.若$D(f)$存在一个开覆盖$\{(\alpha_i,\beta_i)|i=1,2,\cdots\}$.令
	$$K=I\bigg\backslash\bigcup_{n=1}^{\infty}(\alpha_i,\beta_i).$$
	则对于任一$\varepsilon>0$都存在$\delta>0$使得当$x\in K,\ y\in I$且$|x-y|<\delta$时有$|f(x)-f(y)|<\varepsilon$.
\end{proposition}
\begin{proof}
	用反证法.假设结论不成立,则存在$\varepsilon_0>0$对于任一$n\in\mathbb{N}_+$都存在$x_n\in K,\ y_n\in I$使得
	$$|x_n-y_n|<\frac{1}{n},\quad |f(x_n)-f(y_n)|\geqslant\varepsilon_0.$$
	由于$\{x_n\}\subseteq K\subseteq I$,故$\{x_n\}$有界,由Weierstrass定理可知$\{x_n\}$有一个子列$x_{k_n}\to \xi\in K$.由于
	$$|y_{k_n}-\xi|\leqslant|y_{k_n}-x_{k_n}|+|x_{k_n}-\xi|<\frac{1}{k_n}+|x_{k_n}-\xi|\leqslant\frac{1}{n}+|x_{k_n}-\xi|.$$
	因此$y_{k_n}\to\xi$.由于$f$在$\xi$处连续,故
	$$\lim\limits_{n\to\infty}f(x_{k_n})=\lim\limits_{n\to\infty}f(y_{k_n})=f(\xi).$$
	则
	$$\lim\limits_{k\to\infty}|f(x_{n_k})-f(y_{n_k})|=0<\varepsilon_0.$$
	这与$|f(x_n)-f(y_n)|\geqslant\varepsilon_0$矛盾.于是可知命题成立.$\hfill\blacksquare$
\end{proof}
由一致连续性定理的证明,我们认识到可以用数列来描述一致连续性.
\begin{theorem}[一致连续性的数列式描述]
	设$E$是$\mathbb{R}$的一个子集,函数$f$在$E$上有定义.则$f$在$E$上一致连续的充要条件是:对任何满足条件$$\lim\limits_{n\to\infty}(x_n-y_n)=0$$
	的数列$\{x_n\}\in E$,都有$$\lim\limits_{n\to\infty}(f(x_n)-f(y_n))=0.$$
\end{theorem}
\section{初等函数的连续性}
由于初等函数由基本初等函数经过有限次四则运算和有限次复合运算得到,而前面已经证明了连续函数关于四则运算和复合运算的性质,因此我们只需讨论基本初等函数的连续性即可.
\begin{proposition}[常量函数]
	函数$y=c$\quad ($c$是常数)是连续的.
\end{proposition}
\begin{proof}
	$\forall\varepsilon>0$,取任意的$x_0$,在任意的$\mathring{U}(x_0)$中,都有$f(x)\in\mathring{U}(c)$,因此这个命题成立. $\hfill\blacksquare$
\end{proof}
\begin{proposition}[指数函数]
函数$f(x)=a^x$在$\mathbb{R}$上连续.
\end{proposition}
\begin{proof}
先证明$\lim\limits_{x\to 0}a^x=1.$
对任意$\varepsilon>0$,存在$\delta>0$使得
$$1-\varepsilon<a^{-\delta}<a^{\delta}<1+\varepsilon.$$
当$x\in\mathring{U}(0;\delta)$时,从而有
$$1-\varepsilon<a^{-\delta}<a^x<a^{\delta}<1+\varepsilon.$$
故$\lim\limits_{x\to 0}a^x=1.$
则对任意$x_0\in\mathbb{R}$,有
$$\lim\limits_{x\to x_0}a^{x-x_0}=1.$$
由$\varepsilon$的任意性,有
$$-\varepsilon a^{-x_0}<a^{x-x_0}-1<\varepsilon a^{-x_0}.$$
即$$\lim\limits_{x\to x_0}a^x=a^x_0.$$
$\hfill\blacksquare$
\end{proof}
\begin{proposition}[三角函数]
函数$f(x)=\sin x$和$f(x)=\cos x$在$\mathbb{R}$上连续.
\end{proposition}
\begin{proof}
对任意$\varepsilon>0$,取$\delta=\varepsilon$,则当$|x-x_0|<\delta$时,有
$$|\sin x-\sin x_0|=\big|2\cos \frac{x+x_0}{2}\sin\frac{x-x_0}{2}\big|\leqslant2\big|\sin\frac{x-x_0}{2}\big|\leqslant|x-x_0|<\varepsilon.$$
$$|\cos x-\cos x_0|=\big|-2\sin \frac{x+x_0}{2}\sin\frac{x-x_0}{2}\big|\leqslant2\big|\sin\frac{x-x_0}{2}\big|\leqslant|x-x_0|<\varepsilon.$$
$\hfill\blacksquare$
\end{proof}
由于反函数的连续性,可知对数函数和反三角函数也具有相应的连续性.对于幂函数$x^\alpha$,其可写成$x^\alpha=\text{e}^{\alpha\ln x}$从而看作函数$\text{e}^u$和$u=\alpha\ln x$的复合函数.从而推知幂函数的连续性.

以上我们完成了基本初等函数连续性的证明,于是有下述定理.
\begin{theorem}
任何初等函数都是在其定义区间的连续函数.
\end{theorem}
\newpage

\chapter{导数与微分}
\section{导数与微分的概念}
\subsection{导数与微分的定义}
\begin{definition}[导数]
	设函数$y=f(x)$在点$x_0$的某邻域内有定义,若极限
	\begin{equation}{\label{def:derivative}}
		\lim\limits_{x\to x_0}\frac{f(x)-f(x_0)}{x-x_0}
	\end{equation}
	存在,则称函数$f${\heiti 在点$x_0$可导},并称该极限为函数$f${\heiti 在点$x_0$的导数},记作$f'(x)$.
\end{definition}
令$x=x_0+\Delta x$,$\Delta y=f(x)-f(x_0)=f(x_0+\Delta x)-f(x_0)$,则式\ref{def:derivative}可改写为
\begin{equation}{\label{def':derivative}}
	\lim\limits_{\Delta x\to 0}\frac{\Delta y}{\Delta x}=\lim\limits_{\Delta x\to 0}\frac{f(x_0+\Delta x)-f(x_0)}{\Delta x}=f'(x).
\end{equation}

所以导数是函数增量$\Delta y$与自变量增量$\Delta x$之比$\dfrac{\Delta y}{\Delta x}$的极限.这个增量比称为函数关于自变量的平均变化率(又称{\heiti 差商}),而导数$f'(x)$则为$f$在$x_0$处关于$x$的变化率(又称{\heiti 微商}).

若式\ref{def:derivative}或式\ref{def':derivative}不存在,则称$f${\heiti 在点$x_0$不可导}.
\begin{definition}[可微]
	设函数$y=f(x)$在点$x_0$的某邻域$U(x_0)$上有定义.当给$x_0$一个增量$\Delta x,x_0+\Delta x\in U(x_0)$时,相应地得到函数的增量为
	$$\Delta y=f(x_0+\Delta x)-f(x_0).$$
	
	如果存在常数$A$,使得$\Delta y$能表示成
	\begin{equation}{\label{def:differential}}
		\Delta y=A\Delta x+o(x),
	\end{equation}
	则称函数$f$在点$x_0${\heiti 可微}
\end{definition}
\begin{theorem}
	函数$f$在点$x_0$处可微的充要条件是$f$在点$x_0$处可导.
\end{theorem}
\begin{proof}
	必要性\qquad 如果
	$$f(x_0+\Delta x)-f(x_0)=A\Delta x+o(\Delta x),$$
	那么
	$$\frac{f(x_0+\Delta x)-f(x_0)}{\Delta x}=A+\frac{o(\Delta x)}{\Delta x},$$
	因而$f(x)$在$x$点可导:
	$$f'(x)=\lim\limits_{\Delta x\to 0}\frac{f(x_0+\Delta x)-f(x_0)}{\Delta x}=A.$$
	这就证明了$f$在点$x_0$可导且导数等于$A$.
	充分性\qquad 如果存在极限
	$$\lim\limits_{\Delta x\to 0}\frac{f(x_0+\Delta x)-f(x_0)}{\Delta x}=f'(x),$$
	那么当$\Delta x\to 0$时,
	$$\alpha(h)=\frac{f(x_0+\Delta x)-f(x_0)}{\Delta x}-f'(x)\to 0,$$
	并且有
	$$f(x+\Delta x)-f(x)=f'(x)\Delta x+\alpha(\Delta x)\Delta x.$$
	这就是说
	$$f(x+\Delta x)-f(x)=f'(x)\Delta x+o(\Delta x).$$
	$\hfill\blacksquare$
\end{proof}
\begin{remark}
	上述定理说明可导与可微是等价的.因此在现阶段可导与可微可以当作同义词使用,求导数的方法也称为微分法.之后的论述中我们将侧重讨论导数,因为微分的相关概念在此都可以相类比.
\end{remark}
\begin{definition}[微分]
	设函数$y=f(x)$在点$x_0$处可微.我们引入记号
	$$\odif{x}\coloneqq \Delta x,$$
	$$\odif{y}\coloneqq f'(x_0)\odif{x}=f'(x_0)\Delta x,$$
	并把$\odif{y}$叫做函数$y=f(x)$的{\heiti 微分}.
\end{definition}
\begin{remark}
	关于微分的意义,从上面的讨论我们已经得知:
	\begin{enumerate}
		\item 从几何的角度来看,微分$\odif{y}=f'(x)\odif{x}$正好是切线函数的增量.
		\item 从代数的角度来看,微分$\odif{y}=f'(x)\odif{x}$是增量$\Delta y$的线性主部,$\odif{y}$与$\Delta y$仅仅相差一个高阶的无穷小量$o(\Delta x)$,因而当$\Delta x$充分小时,可以用$\odif{y}$作为$\Delta y$的近似值.这一事实是微分的许多实际应用的基础.
	\end{enumerate}
\end{remark}
\subsection{单侧导数}
\begin{definition}[单侧导数]
	设函数$y=f(x)$在点$x_0$的某右邻域$\left[x_0,x_0+\delta\right)$上有定义,若右极限
	$$\lim\limits_{\Delta x\to 0^+}\frac{\Delta y}{\Delta x}=\lim\limits_{\Delta x\to 0^+}\frac{f(x_0+\Delta x)-f(x_0)}{\Delta x}\qquad(0<\Delta x<\delta)$$
	存在,则称该极限值为$f$在点$x_0$的{\heiti 右导数},记作$f'_+(x_0)$.
	
	类似地,我们可以定义左导数
	$$f'_-(x_0)=\lim\limits_{\Delta x\to 0^-}\frac{f(x_0+\Delta x)-f(x_0)}{\Delta x}.$$
	
	右导数和左导数统称为{\heiti 单侧导数}.
\end{definition}
\begin{theorem}
	若函数$y=f(x)$在点$x-0$的某邻域上有定义,则$f'(x_0)$存在($f(x)$在点$x_0$处可导)的充要条件是$f'_+(x_0)=f'_-(x_0).$
\end{theorem}
\subsection{导函数}
\begin{definition}[导函数]
	若函数$f$在区间$I$上每一点都可导(对区间端点,仅考虑相应的单侧导数),则称$f$为$I$上的可导函数.此时对每一个$x\in I$,都有$f$的一个导数$f'(x)$(或单侧导数)与之对应.这样就定义了一个在$I$上的函数,称为$f$在$I$上的{\heiti 导函数},也简称为{\heiti 导数}.记作$f',y'$或$\dfrac{\odif{y}}{\odif{x}}$,即
	$$f'(x)=\lim\limits_{\Delta x\to 0}\frac{\Delta y}{\Delta x}=\lim\limits_{\Delta x\to 0}\frac{f(x_0+\Delta x)-f(x_0)}{\Delta x},\qquad x\in I.$$
\end{definition}
\subsection{导数的几何意义}
我们已经知道$f(x)$在点$x=x_0$的切线斜率$k$,正是割线斜率在$x\to x_0$时的极限,即
$$k=\lim\limits_{x\to x_0}\frac{f(x)-f(x_0)}{x-x_0}.$$
由导数的定义,$k=f'(x)$,所以曲线$y=f(x)$在点$(x_0,y_0)$的切线方程是
$$y-y_0=f'(x_0)(x-x_0).$$
\begin{definition}[极值点]
	若函数$f$在点$x_0$的某邻域$U(x_0)$上对一切$x\in U(x_0)$有
	$$f(x_0)\geqslant f(x)$$
	则称函数$f$在点$x_0$取得{\heiti 极大值},称点$x_0$为{\heiti 极大值点};
	
	若有
	$$f(x_0)\leqslant f(x)$$
	则称函数$f$在点$x_0$取得{\heiti 极小值},称点$x_0$为{\heiti 极小值点}.
	
	极大值点、极小值点统称为{\heiti 极值点}.
\end{definition}
\begin{proposition}{\label{prooffermat}}
	若$f'_+(x_0)>0$,则存在$\delta>0$,对任何$x\in(x_0,x_0+\delta)$,有
	$$f(x_0)<f(x).$$
\end{proposition}
\begin{proof}
	因为
	$$f'_+(x_0)=\lim\limits_{x\to x_0^+}\frac{f(x)-f(x_0)}{x-x_0}>0,$$
	所以由保号性可知,存在正数$\delta$,对一切$x\in(x_0,x_0+\delta)$,有
	$$\frac{f(x)-f(x_0)}{x-x_0}>0,$$
	从而不难推得,当$0<x-x_0<\delta$时,有$f(x_0)<f(x).$$\hfill\blacksquare$
\end{proof}
\begin{remark}
	用类似的方法可讨论$f'_+(x_0)<0$,$f'_-(x_0)>0$和$f'_-(x_0)<0$的情况.
\end{remark}
\begin{remark}
	由上述命题,我们可以得出:若$f'(x_0)$存在且不为零,则$x_0$不是$f(x)$的极值点.
\end{remark}
这样我们就得到了著名的Fermat定理.
\begin{theorem}[Fermat定理]
	设函数$f$在点$x_0$的某邻域上有定义,且在点$x_0$可导.若点$x_0$为$f$的极值点,则必有
	$$f'(x_0)=0.$$
\end{theorem}
Fermat定理的几何意义:若函数$f(x)$在极值点$x=x_0$可导,那么在该点的切线平行于$x$轴.
我们称满足方程$f'(x)=0$的点为{\heiti 稳定点}或{\heiti 驻点}.
需要注意的是,稳定点不一定是极值点(如$x^3$当$x=0$时),极值点也不一定是稳定点(如$|x|$当$x=0$时).
\section{求导法则}
\subsection{导数的四则运算}
\begin{theorem}[加减法公式]
	若函数$u(x)$,$v(x)$可导,则$u(x)\pm v(x)$也可导,且
	$$\left[u(x)\pm v(x)\right]'=u'(x)\pm v'(x).$$
\end{theorem}
\begin{proof}
	\begin{align*}
		\left[u(x)\pm v(x)\right]'&=\lim\limits_{\Delta x\to 0}\frac{\left[u(x+\Delta x)\pm v(x+\Delta x)\right]-\left[u(x)\pm v(x)\right]}{\Delta x}\\
		&=\lim\limits_{\Delta x\to 0}\frac{\left[u(x+\Delta x)-u(x)\right]\pm \left[v(x+\Delta x)-v(x)\right]}{\Delta x}\\
		&=u'(x)+v'(x).
	\end{align*}
	$\hfill\blacksquare$
\end{proof}
\begin{theorem}[乘法公式]
	若函数$u(x)$,$v(x)$可导,则$u(x)v(x)$也可导,且
	$$\left[u(x)v(x)\right]'=u'(x)v(x)+u(x)v'(x).$$
\end{theorem}
\begin{proof}
	\begin{align*}
		\left[u(x)v(x)\right]'&=\lim\limits_{\Delta x\to 0}\frac{u(x+\Delta x)v(x+\Delta x)-u(x)v(x)}{\Delta x}\\
		&=\lim\limits_{\Delta x\to 0}\frac{u(x+\Delta x)v(x+\Delta x)-u(x)v(x+\Delta x)+u(x)v(x+\Delta x)-u(x)v(x)}{\Delta x}\\
		&=\lim\limits_{\Delta x\to 0}\frac{u(x+\Delta x)-u(x)}{\Delta x}v(x+\Delta x)+u(x+\Delta x)\lim\limits_{\Delta x\to 0}\frac{v(x+\Delta x)-v(x)}{\Delta x}\\
		&=u'(x)v(x)+u(x)v'(x).
	\end{align*}
	$\hfill\blacksquare$
\end{proof}
\begin{remark}
	第二行用了“添项减项”的方法,这是数学分析中经常遇到的一种方法.
\end{remark}
\begin{remark}
	利用数学归纳法可以将这个法则推广到任意有限个函数乘积的情形.例如
	$$(uvw)'=u'vw+uv'w+uvw'.$$
\end{remark}
\begin{corollary}
	若函数$v(x)$可导,$c$为常数,则
	$$\left[cv(x)\right]'=cv'(x).$$
\end{corollary}
\begin{theorem}[除法公式]
	若函数$u(x)$,$v(x)$可导,且$v(x)\neq 0$,则$\dfrac{u(x)}{v(x)}$也可导,且
	$$\frac{u(x)}{v(x)}=\frac{u'(x)v(x)-u(x)v'(x)}{\left[v(x)\right]^2}.$$
\end{theorem}
\begin{proof}
	\begin{align*}
		\big(\frac{u(x)}{v(x)}\big)'&=\lim\limits_{\Delta x\to 0}\frac{\frac{u(x+\Delta x)}{v(x+\Delta x)}-\frac{u(x)}{v(x)}}{\Delta x}\\
		&=\lim\limits_{\Delta x\to 0}\frac{u(x+\Delta x)v(x)-u(x)v(x+\Delta x)}{v(x+\Delta x)v(x)\Delta x}\\
		&=\lim\limits_{\Delta x\to 0}\frac{u(x+\Delta x)v(x)-u(x)v(x)+u(x)v(x)-u(x)v(x+\Delta x)}{v(x+\Delta x)v(x)\Delta x}\\
		&=\lim\limits_{\Delta x\to 0}\frac{\left[u(x+\Delta x)-u(x)\right]}{\Delta x}\frac{v(x)}{v(x+\Delta x)v(x)}-\lim\limits_{\Delta x\to 0}\frac{\left[v(x+\Delta x)-v(x)\right]}{\Delta x}\frac{u(x)}{v(x+\Delta x)v(x)}\\
		&=\frac{u'(x)v(x)-u(x)v'(x)}{\left[v(x)\right]^2}.
	\end{align*}
	$\hfill\blacksquare$
\end{proof}
\subsection{复合函数的导数}
为证明复合函数的求导公式,我们先证明一个引理.
\begin{lemma}
	$f(x)$在点$x_0$可导的充要条件是:在$x_0$的某邻域$U(x_0)$上,存在一个在点$x_0$连续的函数$H(x)$,使得
	$$f(x)-f(x_0)=H(x)(x-x_0),$$
	从而$f'(x)=H(x_0).$
\end{lemma}
\begin{proof}
	必要性\qquad 设$f(x)$在点$x_0$可导,令
	\begin{equation*}
		H(x)=\left\{
		\begin{aligned}
			&\frac{f(x)-f(x_0)}{x-x_0}, & & x\in \mathring{U}(x_0),\\
			&f'(x_0), & & x=x_0,
		\end{aligned}
		\right.
	\end{equation*}
	则因
	$$\lim\limits_{x\to x_0}H(x)=\lim\limits_{x\to x_0}\frac{f(x)-f(x_0)}{x-x_0}=f'(x_0)=H(x_0),$$
	所以$H(x)$在点$x_0$连续,且$f(x)-f(x_0)=H(x)(x-x_0),\ x\in U(x_0).$
	
	充分性\qquad 设存在$H(x_0),\ x\in U(x_0)$,它在点$x_0$连续,且
	$$f(x)-f(x_0)=H(x)(x-x_0),\ x\in U(x_0).$$
	因存在极限
	$$\lim\limits_{x\to x_0}\frac{f(x)-f(x_0)}{x-x_0}=\lim\limits_{x\to x_0}H(x)=H(x_0),$$
	所以$f(x)$在点$x_0$可导,且$f'(x_0)=H(x_0)$.$\hfill\blacksquare$
\end{proof}
\begin{remark}
	引理说明了点$x_0$是函数$g(x)=\dfrac{f(x)-f(x_0)}{x-x_0}$可去间断点的充要条件是$f(x)$在点$x_0$可导.这个结论可以推广到向量函数的导数.
\end{remark}
\begin{theorem}
	设$u=\varphi(x)$在点$x_0$可导,$y=f(u)$在点$u_0=\varphi(x_0)$可导,则复合函数$f\circ \varphi$在点$x_0$可导,且
	$$(f\circ \varphi)'(x_0)=f'(u_0)\varphi'(x_0)=f'(\varphi(x_0))\varphi'(x_0).$$
\end{theorem}
\begin{proof}
	由$f(u)$在点$u_0$可导,由引理的必要性部分,存在一个在点$u_0$连续的函数$F(u)$,使得$f'(u_0)=F(u_0)$,且
	$$f(u)-f(u_0)=F(u)(u-u_0),\ u\in U(u_0).$$
	又由$u=\varphi(x)$在点$x_0$可导,同理存在一个在点$x_0$连续的函数$\varPhi(x)$,使得$\varphi'(x_0)=\varPhi(x_0)$,且
	$$\varphi(x)-\varphi(x_0)=\varPhi(x)(x-x_0),\ x\in U(x_0).$$
	于是就有
	\begin{align*}
		f(\varphi(x))-f(\varphi(x_0))&=F(\varphi(x))(\varphi(x)-\varphi(x_0))\\
		&=F(\varphi(x))\varPhi(x)(x-x_0).
	\end{align*}
	因为$\varphi,\varPhi$在点$x_0$连续,$F$在点$u_0=\varphi(x_0)$连续,因此$H(x)=F(\varphi(x))\varPhi(x)$在点$x_0$连续.由引理的充分性部分证得$f\circ \varphi$在点$x_0$可导,且
	$$(f\circ\varphi)'(x_0)=H(x_0)=F(\varphi(x_0))\varPhi(x_0)=f'(u_0)\varphi'(x_0).$$
	$\hfill\blacksquare$
\end{proof}
\begin{remark}
	复合函数的求导公式也称为{\heiti 链式法则}(chainrule),函数$y=f(u),u=\varphi(x)$的复合函数在点$x$的求导公式一般也写作
	$$\frac{\odif{y}}{\odif{x}}=\frac{\odif{y}}{\odif{u}}\cdot\frac{\odif{u}}{\odif{x}}.$$
\end{remark}
\subsection{反函数的导数}
\begin{theorem}
	设$y=f(x)$为$x=\varphi(y)$的反函数,若$\varphi(y)$在点$y_0$的某邻域上连续,严格单调且$\varphi'(y_0)\neq 0$,则$f(x)$在点$x_0(=\varphi(y_0))$可导,且
	$$f'(x_0)=\frac{1}{\varphi'(y_0)}.$$
\end{theorem}
\begin{proof}
	设$\Delta x=\varphi(y_0+\Delta y)-\varphi(y_0),\ \Delta y=f(x_0+\Delta x)-f(x_0)$.因为$\varphi$在$y_0$的某邻域上连续且严格单调,故$f=\varphi^{-1}$在$x_0$的某邻域上连续且严格单调.从而当且仅当$\Delta y=0$时$\Delta x=0$,并且当且仅当$\Delta y\to 0$时$\Delta x\to 0$.由$\varphi'(y_0)\neq 0$,可得
	$$f'(x)=\lim\limits_{\Delta x\to 0}\frac{\Delta y}{\Delta x}=\lim\limits_{\Delta y\to 0}\frac{\Delta y}{\Delta x}=\frac{1}{\lim\limits_{\Delta y\to 0}\frac{\Delta x}{\Delta y}}=\frac{1}{\varphi'(y_0)}.$$
	$\hfill\blacksquare$
\end{proof}
\subsection{参变量函数的导数}
一般地,设有参数表达式
\begin{equation*}
	\left\{
	\begin{aligned}
		&x=\varphi(t)\\
		&y=\psi(t)
	\end{aligned}
	\qquad t\in J
	\right.
\end{equation*}
其中函数$\varphi$在区间$J$上严格单调并且连续,函数$\psi$在区间$J$上连续.我们可以把$t$表示为$x$的连续函数
$$t=\varphi^{-1}(x),\ x\in I=\varphi(J),$$
于是$y$表示为$x$的连续函数
$$y=\psi(\varphi^{-1}(x)),\ x\in I.$$
如果函数$\varphi$和$\psi$都在区间$J$的内点$t_0$处可导,并且$\varphi'(t_0)\neq 0$,那么由复合函数与反函数的求导法则可知函数$y=\psi\circ\varphi^{-1}$在$x_0=\varphi(t_0)$处可导,并且有
\begin{align*}
	(\psi\circ\varphi^{-1})'(x_0)&=\psi'(\varphi^{-1}(x_0))(\varphi^{-1})(x_0)\\
	&=\psi'(\varphi^{-1}(x_0))\frac{1}{\varphi'(\varphi^{-1}(x_0))}\\
	&=\frac{\psi'(t_0)}{\varphi'(t_0)}.
\end{align*}
以上我们得到
$$\frac{\odif{y}}{\odif{x}}=\frac{\odif{y}/\odif{t}}{\odif{x}/\odif{t}}=\frac{\psi'(t)}{\varphi'(t)}.$$
\subsection{初等函数导数公式}
由于初等函数由基本初等函数经过有限次四则运算和复合运算得到,因此我们只需考虑基本初等函数的导数公式.对于六类基本初等函数,我们只需讨论常量函数、对数函数和正余弦函数的导数,其余的基本初等函数的导数公式都可由这三类函数的导数公式和四则运算、复合函数与反函数的求导法则得出.
\begin{proposition}[常量函数]
	$$(c)'=0\qquad(c\text{为常数}).$$
\end{proposition}
\begin{proposition}[对数函数]
	$$(\log_ax)'=\frac{1}{x\ln a}\qquad(a>0\text{且}a\neq 1).$$
\end{proposition}
\begin{proof}
	\begin{align*}
		(\log_ax)'
		&=\lim\limits_{\Delta x\to 0}\dfrac{\log_a(x+\Delta x)-\log_ax}{\Delta x}\\
		&=\lim\limits_{\Delta x\to 0}\dfrac{\ln(1+\frac{\Delta x}{x})}{\Delta x\ln a}\\
		&=\lim\limits_{\Delta x\to 0}\dfrac{\frac{\Delta x}{x}}{\Delta x\ln a}\\
		&=\dfrac{1}{x\ln a}.
	\end{align*}
	$\hfill\blacksquare$
\end{proof}
\begin{proposition}[正余弦函数]
	$$(\sin x)'=\cos x,$$
	$$(\cos x)'=-\sin x.$$
\end{proposition}
\begin{proof}
	\begin{align*}
		(\sin x)'
		&=\lim\limits_{\Delta x\to 0}\dfrac{\sin(x+\Delta x)-\sin x}{\Delta x}\\
		&=\lim\limits_{\Delta x\to 0}\dfrac{\sin x\cos\Delta x+\cos x\sin\Delta x-\sin x}{\Delta x}\\
		&=\lim\limits_{\Delta x\to 0}\dfrac{\sin x(\cos\Delta x-1)}{\Delta x}+\lim\limits_{\Delta x\to 0}\dfrac{\sin\Delta x}{\Delta x}\cdot\cos x\\
		&=\lim\limits_{\Delta x\to 0}\dfrac{\sin x\cdot(-\frac{1}{2}\Delta x^2)}{\Delta x}+\cos x\\
		&=\cos x.
	\end{align*}
	\begin{align*}
		(\cos x)'
		&=\lim\limits_{\Delta x\to 0}\dfrac{\cos(x+\Delta x)-\cos x}{\Delta x}\\
		&=\lim\limits_{\Delta x\to 0}\dfrac{\cos x\cos\Delta x-\sin x\sin\Delta x-\cos x}{\Delta x}\\
		&=\lim\limits_{\Delta x\to 0}\dfrac{\cos x(\cos\Delta x-1)}{\Delta x}-\lim\limits_{\Delta x\to 0}\dfrac{\sin\Delta x}{\Delta x}\cdot\sin x\\
		&=\lim\limits_{\Delta x\to 0}\dfrac{\cos x\cdot(-\frac{1}{2}\Delta x^2)}{\Delta x}-\sin x\\
		&=-\sin x.
	\end{align*}
	$\hfill\blacksquare$
\end{proof}
指数函数可由对数函数和反函数的导数公式求得,幂函数可由指数函数、对数函数通过复合函数求导公式求得,反三角函数可由三角函数和反函数的导数公式求得,在此不展开详述,而是给出相应的导数公式.

综上所述,我们有
\begin{proposition}[基本初等函数的导数公式]
	\iffalse
	$$(c)'=0\qquad(c\text{为常数}).$$
	$$(x^\alpha)'=\alpha x^{\alpha-1}\qquad(\alpha\text{为任意实数}).$$
	$$(a^x)'=a^x\ln a\qquad(a>0\text{且}a\neq 1).$$
	$$(\log_ax)'=\frac{1}{x\ln a}\qquad(a>0\text{且}a\neq 1).$$
	$$(\sin x)'=\cos x.$$
	$$(\cos x)'=-\sin x.$$
	$$(\tan x)'=\sec^2x.$$
	$$(\cot x)'=-\csc^2x.$$
	$$(\sec x)'=\sec x\tan x.$$
	$$(\csc x)'=-\csc x\cot x.$$
	$$(\arcsin x)'=\frac{1}{\sqrt{1-x^2}}.$$
	$$(\arccos x)'=-\frac{1}{\sqrt{1-x^2}}.$$
	$$(\arctan x)'=\frac{1}{1+x^2}.$$
	$$(\text{arccot} x)'=-\frac{1}{1+x^2}.$$
	特别地,
	$$(e^x)'=e^x.$$
	$$(\ln x)'=\frac{1}{x}$$
	\fi
	\begin{align*}
		&(c)'=0\qquad(c\text{为常数}).\\
		&(x^\alpha)'=\alpha x^{\alpha-1}\qquad(\alpha\text{为任意实数}).\\
		&(a^x)'=a^x\ln a\qquad(a>0\text{且}a\neq 1).\\
		&(\log_ax)'=\frac{1}{x\ln a}\qquad(a>0\text{且}a\neq 1).\\
		&(\sin x)'=\cos x.\\
		&(\cos x)'=-\sin x.\\
		&(\tan x)'=\sec^2x.\\
		&(\cot x)'=-\csc^2x.\\
		&(\sec x)'=\sec x\tan x.\\
		&(\csc x)'=-\csc x\cot x.\\
		&(\arcsin x)'=\frac{1}{\sqrt{1-x^2}}.\\
		&(\arccos x)'=-\frac{1}{\sqrt{1-x^2}}.\\
		&(\arctan x)'=\frac{1}{1+x^2}.\\
		&(\text{arccot} x)'=-\frac{1}{1+x^2}.\\
		&\text{特别地,}\\
		&(e^x)'=e^x.\\
		&(\ln x)'=\frac{1}{x}.&
	\end{align*}
\end{proposition}
\section{高阶导数}
\subsection{高阶导数的定义}
\begin{definition}[二阶导数]
	若函数$f$的导函数$f'$在点$x_0$可导,则称$f'$在点$x_0$的导数为$f$在点$x_0$的{\heiti 二阶导数},记作$f''(x_0)$,即
	$$\lim\limits_{x\to x_0}\frac{f'(x)-f'(x_0)}{x-x_0}=f''(x_0),$$
	同时称$f$在点$x_0$为{\heiti 二阶可导}.
	
	若$f$在区间$I$上的每一点都二阶可导,则得到一个定义在$I$上的函数,这个函数称为$f$的二阶导函数,记作$f''(x),\ x\in I$,或者简单记作$f''$.
\end{definition}
由上述过程,我们可以继续推出三阶导数、四阶导数以及$n$阶导数的定义.
\begin{definition}[n阶导数]
	一般地,可由$f$的$n-1$阶导数定义$f$的$n$阶导数.即若$f$在区间$I$上的每一点都$(n-1)$阶可导,则得到一个定义在$I$上的函数,这个函数称为$f$的{\heiti $n$阶导函数}.记作
	$$f^{(n)},y^{(n)}\text{或}\frac{\d^ny}{\odif{x}^n}.$$
\end{definition}
\subsection{高阶导数的运算法则}
\begin{theorem}[加减法则]
	$$\left[u\pm v\right]^{(n)}=u^{(n)}+v^{(n)}.$$
\end{theorem}
这里是显然的,不再证明.
\begin{theorem}[乘法法则]
	$$(uv)^{(n)}=\sum_{k=0}^{n}C_n^ku^{(n-k)}v^{(k)}.$$
\end{theorem}
\begin{remark}
	这里可通过数学归纳法证明.该公式又被称为Leibnitz公式.
\end{remark}
\newpage

\chapter{微分中值定理及其应用}
在这一章里,我们要讨论怎样由导数$f'$的已知性质来推断函数$f$所应具有的性质.微分中值定理(包括Rolle定理、Lagrange定理、Cauchy定理、Taylor定理)正是进行这一讨论的有效工具.
\section{Lagrange定理}
本节首先介绍Lagrange定理以及它的预备定理——Rolle定理.

根据微分的定义,如果函数$f(x)$在点$x_0$可微,那么当$x\to x_0$时,有
$$f(x)-f(x_0)=f'(x_0)(x-x_0)+o(x-x_0).$$

以上公式称为{\heiti 无穷小增量公式},它反映了$x\to x_0$时函数值的变化情况.显然,这是一个局部的增量公式.我们由此思考:如果函数$f(x)$在闭区间$\left[a,b\right]$上连续,在开区间$(a,b)$上可导,是否也有一个刻画函数“整体上”的增量公式?也就是说是否存在一点$\xi$,对于任意的$x_1,x_2\in\left[a,b\right]$,有
$$f(x_1)-f(x_2)=f'(\xi)(x_1-x_2)?$$

从几何上看,我们要研究的问题就是是否存在$\xi$使得$f(x)$在$(\xi,f(\xi))$处的切线与经过$(x_1,f(x_1))$和$(x_2,f(x_2))$的直线平行.这个满足条件的$\xi$就称为{\heiti 中值}(mean value).我们关心中值的存在性,而并不关心它的具体位置.我们有以下Lagrange中值定理.
\begin{theorem}[Lagrange中值定理]
	若函数$f$满足:
	
	(i)$f$在闭区间$\left[a,b\right]$上连续;
	
	(ii)$f$在开区间$(a,b)$上可导,\\
	则在$(a,b)$上至少存在一点$\xi$使得
	\begin{equation}{\label{equ:lagrange}}
		f'(\xi)=\frac{f(b)-f(a)}{b-a}.
	\end{equation}
\end{theorem}
要证明此定理,我们可以先证明它的一个特殊情形——Rolle定理.
\begin{theorem}[Rolle定理]
	若函数$f$满足:
	
	(i)$f$在闭区间$\left[a,b\right]$上连续;
	
	(ii)$f$在开区间$(a,b)$上可导;
	
	(iii)$f(a)=f(b)$,\\
	则在$(a,b)$上至少存在一点$\xi$,使得
	$$f'(\xi)=0.$$
\end{theorem}
\begin{proof}
	因为$f$在闭区间$\left[a,b\right]$上连续,由最值定理,$f(x)$都最大值和最小值,分别用$M$和$m$来表示,下面分两种情况讨论:
	\begin{enumerate}
		\item $M=m$,则$f$在$\left[a,b\right]$上必为常数,从而结论显然成立.
		\item $M>m$,则因$f(a)=f(b)$,使得最大值$M$与最小值$m$至少有一个在$(a,b)$上的某点处取到,从而$\xi$是$f$的极值点.由Fermat定理,有
		$$f'(\xi)=0.$$
	\end{enumerate}
	$\hfill\blacksquare$
\end{proof}
下面是对Lagrange定理的证明.
\begin{proof}
	作辅助函数
	$$F(x)=f(x)-f(a)-\frac{f(b)-f(a)}{b-a}(x-a).$$
	显然$F(a)=F(b)$,且$F$满足Rolle定理的另两个条件,因此存在$\xi\in(a,b)$使
	$$F'(\xi)=f'(\xi)-\frac{f(b)-f(a)}{b-a}=0,$$
	即$$f'(\xi)=\frac{f(b)-f(a)}{b-a}.$$
	$\hfill\blacksquare$
\end{proof}

Lagrange定理的几何意义:在满足定理条件的曲线$y=f(x)$上至少存在一点$P(\xi,f(\xi))$,使得该曲线在该点处的切线平行于曲线两端点的连线$AB$.我们在证明中引入的辅助函数$F(x)$,正是曲线$y=f(x)$与直线$AB(y=f(a)+\frac{f(b)-f(a)}{b-a}(x-a))$之差.


Lagrange定理中的公式\ref{equ:lagrange}称为{\heiti Lagrange公式}.其还有几种等价形式如下:
\begin{align}
	f(b)-f(a)=f'(\xi)(b-a),\qquad a<\xi<b;\\
	\label{eq1}	f(b)-f(a)=f'(a+\theta(b-a))(b-a),\qquad 0<\theta<1;\\
	\label{eq2}	f(a+h)-f(a)=f'(a+\theta h)h,\qquad 0<\theta<1.
\end{align}

值得注意的是,Lagrange公式无论对于$a<b$还是$a>b$都成立,而$\xi$是介于$a$与$b$之间的一个确定的数.而\ref{eq1}和\ref{eq2}两式的特点在于把中值点$\xi$表示成了$a+\theta(b-a)$,使得不论$a,b$为何值,$\theta$总可为小于$1$的某一正数.
\begin{corollary}
	若函数$f$在区间$I$上可导,且$f'(x)\equiv 0,\ x\in I$,则$f$为$I$上的一个常量函数.
\end{corollary}
\begin{proof}
	任取两点$x_1,x_2\in I(\text{不妨设}x_1<x_2)$,在区间$\left[x_1,x_2\right]$上应用Lagrange定理,存在$\xi\in(x_1,x_2)\subset I$,使得
	$$f(x_2)-f(x_1)=f'(\xi)(x_2-x_1)=0.$$
	这就证得$f$在区间$I$上任何两点之值相等.
\end{proof}
进一步,我们有以下推论.
\begin{corollary}
	若函数$f$和$g$均在区间$I$上可导,且$f'(x)\equiv g'(x),\ x\in I$,则在区间$I$上$f(x)$与$g(x)$只相差某一常数,即
	$$f(x)=g(x)+c\ (c\text{为某一常数}).$$
\end{corollary}
\begin{corollary}[导数极限定理]
	设函数$f$在点$x_0$的某邻域$U(x_0)$上连续,在$\mathring{U}(x_0)$上可导,且极限$\lim\limits_{x\to x_0}f'(x)$存在,则$f$在点$x_0$可导,且
	$$f'(x_0)=\lim\limits_{x\to x_0}f'(x).$$
\end{corollary}
\begin{proof}
	分别按左、右导数证明结论成立.
	
	任取$x\in \mathring{U}_+(x_0)$,$f(x)$在$\left[x_0,x\right]$上满足Lagrange定理的条件,则存在$\xi\in(x_0,x)$,使得
	$$\frac{f(x)-f(x_0)}{x-x_0)}=f'(\xi).$$
	由于$x_0<\xi<x$,因此当$x\to x_0^+$时,有$\xi\to x_0^+$,对上式取极限,得
	$$\lim\limits_{x\to x_0^+}\frac{f(x)-f(x_0)}{x-x_0}=\lim\limits_{x\to x_0^+}f'(\xi)=f'(x_0+0).$$
	同理可得$f'_-(x_0)=f'(x_0-0).$
	
	因为$\lim\limits_{x\to x_0}f'(x)=k$存在,所以$f'(x_0+0)=f'(x_0-0)=k$,从而$f'_+(x_0)=f'_-(x_0)=k$,即$f'(x_0)=k$.
	$\hfill\blacksquare$
\end{proof}
\begin{remark}
	证明过程中要注意:$f'_+(x_0)=\lim\limits_{x\to x_0^+}\dfrac{f(x)-f(x_0)}{x-x_0}$,指的是$f(x)$在$x_0$处的左导数;而$f'(x_0+0)=\lim\limits_{x\to x_0}f'(x)$,指的是导函数$f'(x)$趋近于$x_0$时的左极限.该定理的本质说明了函数的左右导数等于导函数的左右极限,因此函数在某一点的导数等于其导函数在该点的极限.符合定理条件的$f(x)$的导函数在$x_0$点连续.导数极限定理适合于求分段函数在分段点处的导数.
\end{remark}
\section{Cauchy中值定理与L'Hospital法则}
\subsection{Cauchy中值定理}
Cauchy中值定理是形式更一般的微分中值定理,定理内容如下.
\begin{theorem}[Cauchy中值定理]
	设函数$f$和$g$满足:
	\begin{enumerate}
		\item 在闭区间$\left[a,b\right]$上都连续;
		\item 在开区间$(a,b)$上都可导;
		\item $f'(x)$和$g'(x)$不同时为零;
		\item $g(a)\neq g(b)$,
	\end{enumerate}
	则存在$\xi\in(a,b)$,使得
	$$\frac{f'(\xi)}{g'(\xi)}=\frac{f(b)-f(a)}{g(b)-g(a)}.$$
\end{theorem}
\begin{proof}
	作辅助函数
	$$F(x)=f(x)-f(a)-\frac{f(b)-f(a)}{g(b)-g(a)}(g(x)-g(a)).$$
	容易看出$F$在$\left[a,b\right]$上满足Rolle定理条件,故存在$\xi\in(a,b)$,使得
	$$F'(\xi)=f'(\xi)-\frac{f(b)-f(a)}{g(b)-g(a)}g'(\xi)=0.$$
	因为$g'(\xi)\neq 0$(否则由上式$f'(\xi)$也为零),所以可把上式改写成Cauchy定理的结论.
	$\hfill\blacksquare$
\end{proof}
\hspace*{\fill}

Cauchy中值定理与Lagrange定理的几何意义相类似.只是现在要把$f,g$写作以$x$为参量的参量方程
\begin{equation*}
	\left\{
	\begin{aligned}
		u=g(x),\\
		v=f(x),
	\end{aligned}
	\right.
\end{equation*}
它在$uOv$平面表示一段曲线.由于$\dfrac{f(b)-f(a)}{b-a}$表示连接该曲线两端的弦$AB$的斜率,而$$\frac{f'(\xi)}{g'(\xi)}=\frac{\d v}{\d u}\bigg|_{x=\xi}$$则表示该曲线上与$x=\xi$相对应的一点$(g(\xi),f(\xi))$处的切线的斜率,因此Cauchy中值定理的结论表明上述切线与弦$AB$互相平行.
\subsection{L'Hospital法则}
我们在极限理论中学习无穷小(大)量阶的比较时,已经遇到过两个无穷小量或两个无穷大量之比的极限.由于这种极限可能存在也可能不存在,因此我们将两个无穷小量或两个无穷大量之比的极限统称为{\heiti 不定式极限},分别记为$\dfrac{0}{0}$型或$\dfrac{\infty}{\infty}$型的不定式极限.下面我们将以导数研究不定式极限,这个方法通常称为{\heiti L'Hospital法则}.

{\heiti 1.\ $\dfrac{0}{0}$型不定式极限}
\begin{theorem}
	若函数$f$和$g$满足:
	\begin{enumerate}
		\item $\lim\limits_{x\to x_0}f(x)=\lim\limits_{x\to x_0}g(x)=0$;
		\item 在点$x_0$的某空心邻域$\mathring{U}(x_0)$上两者都可导,且$g'(x)\neq 0$;
		\item $\lim\limits_{x\to x_0}\frac{f'(x)}{g'(x)}=A$($A$为扩充后的任一实数),
	\end{enumerate}
	则
	$$\lim\limits_{x\to x_0}\frac{f(x)}{g(x)}=\lim\limits_{x\to x_0}\frac{f'(x)}{g'(x)}=A.$$
\end{theorem}
\begin{proof}
	补充定义$f(x_0)=g(x_0)=0$,使得$f$与$g$在点$x_0$连续.任取$x\in \mathring{U}(x_0)$,在区间$\left[x_0,x\right]$上应用Cauchy中值定理,有
	$$\frac{f(x)-f(x_0)}{g(x)-g(x_0)}=\frac{f'(\xi)}{g'(\xi)}\qquad x_0<\xi<x,$$即
	$$\frac{f(x)}{g(x)}=\frac{f'(\xi)}{g'(\xi)}.$$
	当令$x\to x_0$时,也有$\xi\to x_0$,故得
	$$\lim\limits_{x\to x_0}\frac{f(x)}{g(x)}=\lim\limits_{x\to x_0}\frac{f'(\xi)}{g'(\xi)}=\lim\limits_{x\to x_0}\frac{f'(x)}{g'(x)}=A.$$
	$\hfill\blacksquare$
\end{proof}
\begin{remark}
	对于其它极限过程,只要相应地修正条件2中的邻域即可得到同样的结论.
\end{remark}
{\heiti 2.\ $\dfrac{\bullet}{\infty}$型不定式极限}
\begin{theorem}
	若函数$f$和$g$满足:
	\begin{enumerate}
		\item 在$x_0$的某个右邻域$\mathring{U}(x_0)$上二者可导,且$g'(x)\neq 0$;
		\item $\lim\limits_{x\to x_0^+}g(x)=\infty$;
		\item $\lim\limits_{x\to x_0^+}\dfrac{f'(x)}{g'(x)}=A$($A$为扩充后的任一实数),
	\end{enumerate}
	则$$\lim\limits_{x\to x_0^+}\frac{f(x)}{g(x)}=A.$$
\end{theorem}
\begin{proof}
	先设$A\in \mathbb{R}$,则对任意$\varepsilon>0$,存在$\delta>0$,当$x\in(x_0,x_0+\delta)$时,有
	$$A-\varepsilon<\frac{f'(x)}{g'(x)}<A+\varepsilon.$$
	对于$(x,c)\subseteq (x_0,x_0+\delta)$,由Cauchy中值定理可知,存在$\xi\in(x,c)$使得
	$$\frac{f(x)-f(c)}{g(x)-g(c)}=\frac{f'(\xi)}{g'(\xi)}.$$
	因此
	$$A-\varepsilon<\frac{f(x)-f(c)}{g(x)-g(c)}<A+\varepsilon.$$
	即
	\begin{equation}{\label{proofhospital}}
		A-\varepsilon<\frac{\frac{f(x)}{g(x)}-\frac{f(c)}{g(x)}}{1-\frac{g(c)}{g(x)}}<A+\varepsilon.
	\end{equation}
	由于$\lim\limits_{x\to x_0^+}g(x)=\infty$,固定$c$,令不等式\ref{proofhospital}的右边的$x\to x_0^+$取上极限得
	$$\limsup\limits_{x\to x_0^+}\frac{f(x)}{g(x)}\leqslant A+\varepsilon.$$
	令$\varepsilon\to 0$,得
	$$\limsup\limits_{x\to x_0^+}\frac{f(x)}{g(x)}\leqslant A.$$
	同理,固定$c$,对不等式\ref{proofhospital}左边的$x\to x_0^+$取下极限得
	$$\liminf\limits_{x\to x_0^+}\frac{f(x)}{g(x)}\geqslant A-\varepsilon.$$
	令$\varepsilon\to 0$,得
	$$\liminf\limits_{x\to x_0^+}\frac{f(x)}{g(x)}\geqslant A.$$
	所以有
	$$\limsup\limits_{x\to x_0^+}\frac{f(x)}{g(x)}=\liminf\limits_{x\to x_0^+}\frac{f(x)}{g(x)}=\lim\limits_{x\to x_0^+}\frac{f(x)}{g(x)}=A.$$
	类似的可以证明$A=\pm\infty$或$\infty$的情形和其他极限过程,在此不再赘述.$\hfill\blacksquare$
\end{proof}
与Stolz定理一样,我们也有L'Hospital法则的推广形式.
\begin{theorem}
	设函数$f$和$g$在开区间$(a,b)$可导,$g(x)\neq 0(\forall x\in(a,b)\backslash\{x_0\})$,则对任意$x_0\in(a,b)$,有
	$$\liminf\limits_{x\to x_0}\frac{f'(x)}{g'(x)}\leqslant\liminf\limits_{x\to x_0}\frac{f(x)}{g(x)}\leqslant\limsup\limits_{x\to x_0}\frac{f(x)}{g(x)}\leqslant\limsup\limits_{x\to x_0}\frac{f'(x)}{g'(x)}.$$
\end{theorem}
\begin{remark}
	对其他极限过程,也有以上结论.
\end{remark}
从以上定理可看出,当$x\to x_0$时,若$\dfrac{f'(x)}{g'(x)}$存在,则可以断言$\dfrac{f(x)}{g(x)}$一定存在;反之,若$\dfrac{f'(x)}{g'(x)}$不存在,却不一定能说明$\dfrac{f(x)}{g(x)}$不存在.
\section{Taylor公式}
多项式函数是各类函数中最简单的一种,用多项式逼近函数是近似计算和理论分析的一个重要内容.
\subsection{带有Peano型余项的Taylor公式}
我们在学习导数和微分概念时已经知道,如果函数$f$在点$x_0$处可导,则有
$$f(x)=f(x_0)+f'(x_0)(x-x_0)+o(x-x_0).$$
即在点$x_0$附近,用一次多项式$f(x_0)+f'(x_0)(x-x_0)$逼近函数$f(x)$时,其误差为$(x-x_0)$的高阶无穷小量.然而在很多时候,用一次多项式逼近是不够的.下面我们研究用$n$次多项式逼近,则其误差为$o((x-x_0))$.

我们探索如下面形式的任一$n$次多项式:
$$p_n(x)=a_0+a_1(x-x_0)^1+a_2(x-x_0)^2+\cdots+a_n(x-x_0)^n=\sum_{k=0}^{n}a_k(x-x_0)^k.$$
逐次求其各阶导数,我们得到
$$p_n^{(k)}=k!a_k,\quad k=1,2,\cdots,n.$$
所以有
$$a_k=\frac{p_n^{(k)}}{k!}.$$
由此可见,多项式$p_n(x)$的各项系数由其在点$x_0$的各阶导数值所唯一确定.

对于一般函数$f$,设它在点$x_0$存在直到$n$阶的导数.由这些导数构造一个$n$次多项式
$$T_n(x)=f(x_0)+\frac{f'(x_0)}{1!}(x-x_0)+\frac{f''(x_0)}{2!}(x-x_0)^2+\cdots+\frac{f^{(n)}(x_0)}{n!}(x-x_0)^n=\sum_{k=0}^{n}\frac{f^{(k)}(x_0)}{k!}(x-x_0)^k.$$
我们将这个多项式$T_n$称为函数$f$在点$x_0$处的{\heiti Taylor多项式},$T_n(x)$的各项系数$\frac{f^{(k)}(x_0)}{k!}\ (k=1,2,\cdots,n)$称为{\heiti Taylor系数}.由上面对多项式系数的讨论,我们知道$f(x)$与其Taylor展开式$T_n(x)$在点$x_0$处有相同的函数值和相同的直至$n$阶的导数值,即
$$f^{(k)}(x_0)=T_n^{(k)}(x_0),\quad k=1,2,\cdots,n.$$
下面证明$f(x)-T_n(x)=o((x-x_0)^n)$.
\begin{proof}
	设$R_n(x)=f(x)-T_n(x),\ Q_n(x)=(x-x_0)^n$,即证
	$$\lim\limits_{x\to x_0}\frac{R_n(x)}{Q_n(x)}=0.$$
	易知
	$$R_n(x_0)=R_n'(x_0)=\cdots=R_n^{(n)}(x_0)=0,$$
	$$Q_n(x_0)=Q_n'(x_0)=\cdots=Q_n^{(n-1)}(x_0)=0,\ Q_n^{(n)}(x_0)=n!.$$
	它们满足L'Hospital法则中的“0/0”型,即
	\begin{enumerate}
		\item $\lim\limits_{x\to x_0}R_n^{(n-1)}(x)=\lim\limits_{x\to x_0}Q_n^{(n-1)}(x)=0$;
		
		\item 在点$x_0$的空心邻域$\mathring{U}(x_0)$上两者都可导,且$Q_n^{(n)}(x)=n!\neq 0$;
		
		\item $\lim\limits_{x\to x_0}\dfrac{R_n^{(n)}}{Q_n^{(n)}}=0$.
		
	\end{enumerate}
	故有
	$$\lim\limits_{x\to x_0}\frac{R_n^{(n-1)}}{Q_n^{(n-1)}}=\lim\limits_{x\to x_0}\frac{R_n^{(n)}}{Q_n^{(n)}}=0.$$
	以此类推,我们有
	$$\lim\limits_{x\to x_0}\frac{R_n}{Q_n}=0.$$
	$\hfill\blacksquare$
\end{proof}
现在我们有以下定理.
\begin{theorem}
	若函数$f$在点$x_0$存在直至$n$阶导数,则有$f(x)=T_n(x)+o((x-x_0)^n)$,即
	\begin{equation}{\label{peanotaylor}}
		f(x)=f(x_0)+f'(x_0)(x-x_0)+\frac{f''(x_0)}{2!}+\cdots+\frac{f^{(n)}}{n!}(x-x_0)^n+o((x-x_0)^n)=\sum_{k=0}^{n}\frac{f^{(k)}(x_0)}{k!}(x-x_0)^k.
	\end{equation}
\end{theorem}
我们将上述定理中的\ref{peanotaylor}式称为{\heiti Taylor公式},将$R_n(x)=f(x)-T_n(x)$称为Taylor公式的{\heiti 余项},形如$o((x-x_0)^n)$的余项称为{\heiti Peano型余项},因此\ref{peanotaylor}式又称为{\heiti 带有Peano型余项的Taylor公式}.\\
\begin{remark}
	若$f(x)$在$x_0$附近满足
	$$f(x)=p_n(x)+o((x-x_0)^n),$$
	其中$p_n(x)$为$n$阶多项式,这并不能说明$p_n(x)$必定是$f$的Taylor多项式.
\end{remark}
\begin{remark}
	满足$f(x)=p_n(x)+o((x-x_0)^n)$的多项式$p_n(x)$是唯一的.若函数$f$满足上述定理的条件,满足$f(x)=p_n(x)+o((x-x_0)^n)$的多项式$p_n(x)$只可能是$f$的Taylor多项式$T_n$.
\end{remark}
常用的Taylor公式是在$x_0=0$时的特殊形式:
$$f(x)=f(0)+f'(0)x+\frac{f''(0)}{2!}x^2+\cdots+\frac{f^{(n)}(0)}{n!}+o(x^n).$$
它称为带有Peano余项的{\heiti Maclaurin公式}.

下面给出一些常用的带有Peano余项的Maclaurin公式,读者可自行验证.
\begin{align*}
	&(1)\ \text{e}^x=1+x+\frac{x^2}{2}+\cdots+\frac{x_n}{n!}+o(x^n);\\
	&(2)\ \sin x=x-\frac{x^3}{3!}+\frac{x^5}{5!}+\cdots+(-1)^{m-1}\frac{x^{2m-1}}{(2m-1)!}+o(x^{2m});\\
	&(3)\ \cos x=1-\frac{x^2}{2!}+\frac{x^4}{4!}+\cdots+(-1)^m\frac{x^{2m}}{(2m)!}+o(x^{2m+1});\\
	&(4)\ \ln(1+x)=x-\frac{x^2}{2}+\frac{x^3}{3}+\cdots+(-1)^{n-1}\frac{x^n}{n}+o(x^n);\\
	&(5)\ (1+x)^\alpha=1+\alpha x+\frac{\alpha(\alpha-1)}{2!}+\cdots+\frac{\alpha(\alpha-1)\cdots(\alpha-n+1)}{n!}x^n+o(x^n);\\
	&(6)\ \frac{1}{1-x}=1+x+x^2+\cdots+x^n+o(x^n).&
\end{align*}

利用上述Maclaurin公式可间接求得其他一些函数的Maclaurin公式或Taylor公式,还可用来求某种类型的极限.
\subsection{带有Lagrange型余项的Taylor公式}
上面我们从微分近似出发,推广得到$n$次多项式逼近函数的Taylor公式\ref{peanotaylor}.Peano型余项是{\heiti 定性}的,下面我们将Taylor公式构造一个{\heiti 定量}形式的余项,即Lagrange余项,便于对误差进行具体的计算和估计.
\begin{theorem}[Taylor定理]
	若函数$f$在$\left[a,b\right]$上存在直至$n$阶的导函数,在$(a,b)$上存在$(n+1)$阶导函数,则对任意给定的$x,x_0\in\left[a,b\right]$,至少存在一点$\xi\in(a,b)$,使得
	\begin{equation}{\label{lagtaylor}}
		f(x)=f(x_0)+f'(x_0)(x-x_0)+\frac{f''(x_0)}{2}(x-x_0)^2+\cdots+\frac{f^{(n)}(x_0)}{n!}(x-x_0)^n+\frac{f^{(n+1)}(\xi)}{(n+1)!}(x-x_0)^{n+1}.
	\end{equation}
\end{theorem}
\begin{proof}
	作辅助函数
	$$F(t)=f(x)-\left[f(t)+f'(t)(x-t)+\cdots+\frac{f^{(n)}(t)}{n!}(x-t)^n\right],$$
	$$G(t)=(x-t)^{n+1}.$$
	所要证明的式\ref{lagtaylor}即为
	$$F(x_0)=\frac{f^{(n+1)}(\xi)}{(n+1)!}G(x_0).$$
	即证$$\frac{F(x_0)}{G(x_0)}=\frac{f^{(n+1)}(\xi)}{(n+1)!}.$$
	不妨设$x_0<x$,则$F(t)$与$G(t)$在$\left[x_0,x\right]$上连续,在$(x_0,x)$上可导,且
	$$F'(t)=-\frac{f^{(n+1)}(\xi)}{n!}(x-t)^n,$$
	$$G'(t)=-(n+1)(x-t)^n\neq 0.$$
	又因$F(x)=G(x)=0$,由Cauchy中值定理得
	$$\frac{F(x_0)}{G(x_0)}=\frac{F(x_0)-F(x)}{G(x_0)-G(x)}=\frac{F'(\xi)}{G'(\xi)}=\frac{f^{(n+1)}(\xi)}{(n+1)!},\quad \xi\in(x_0,x)\subset(a,b).$$
	$\hfill\blacksquare$
\end{proof}

式\ref{lagtaylor}同样称为{\heiti Taylor公式},它的余项为
$$R_n(x)=f(x)-T_n(x)=\frac{f^{(n+1)}(\xi)}{(n+1)!}(x-x_0)^{n+1},$$
$$\xi=x_0+\theta(x-x_0),\quad (0<\theta<1).$$
我们将该余项称为{\heiti Lagrange型余项}.因此\ref{lagtaylor}式又称为{\heiti 带有Lagrange型余项的Taylor公式}.

注意到$n=0$时,式\ref{lagtaylor}即为Lagrange中值公式
$$f(x)-f(x_0)=f'(\xi)(x-x_0).$$
所以Taylor定理可以看作Lagrange中值定理的推广.

当$x_0=0$时,得到Taylor公式
$$f(x)=f(0)+f'(0)x+\frac{f''(0)}{2!}x^2+\cdots+\frac{f^{(n)}(0)}{n!}x^n+\frac{f^{(n+1)}(\theta x)}{(n+1)!}x^{n+1}\quad (0<\theta<1).$$
上式称为带有Lagrange余项的Maclaurin公式.

下面将带有Peano余项的Maclaurin公式改写成了带有Lagrange余项的Maclaurin公式.($0<\theta<1$)
\begin{align*}
	(1)&\ \text{e}^x=1+x+\frac{x^2}{2}+\cdots+\frac{x_n}{n!}+\frac{e^{\theta x}}{(n+1)!}x^{n+1},\quad x\in(-\infty,+\infty);\\
	(2)&\ \sin x=x-\frac{x^3}{3!}+\frac{x^5}{5!}+\cdots+(-1)^{m-1}\frac{x^{2m-1}}{(2m-1)!}+(-1)^m\frac{\cos \theta x}{(2m+1)!}x^{2m+1},\quad x\in(-\infty,+\infty);\\
	(3)&\ \cos x=1-\frac{x^2}{2!}+\frac{x^4}{4!}+\cdots+(-1)^m\frac{x^{2m}}{(2m)!}+(-1)^{m+1}\frac{\cos \theta x}{(2m+2)!}x^{2m+2},\quad x\in(-\infty,+\infty);\\
	(4)&\ \ln(1+x)=x-\frac{x^2}{2}+\frac{x^3}{3}+\cdots+(-1)^{n-1}\frac{x^n}{n}+(-1)^n\frac{x^{n+1}}{(n+1)(1+\theta x)^{n+1}},\quad x>-1;\\
	(5)&\ (1+x)^\alpha=1+\alpha x+\frac{\alpha(\alpha-1)}{2!}+\cdots+\frac{\alpha(\alpha-1)\cdots(\alpha-n+1)}{n!}x^n+\\
	&\frac{\alpha(\alpha-1)\cdots(\alpha-n)}{(n+1)!}(1+\theta x)^{\alpha-n-1}x^{n+1},\quad x>-1;\\
	(6)&\ \frac{1}{1-x}=1+x+x^2+\cdots+x^n+\frac{x^{n+1}}{(1-\theta x)^{n+2}},\quad |x|<1.&
\end{align*}
\subsection{Taylor公式的应用}
\section{函数的单调性}
\begin{theorem}
	设函数$f(x)$在区间$I$上可导,则$f(x)$在$I$上递增的充要条件是$f'(x)\geqslant 0$;$f(x)$在$I$上递减的充要条件是$f'(x)\leqslant 0$.
\end{theorem}
\begin{proof}
	只需证递增的情况即可.设$f$为递增函数,则对每一$x_0\in I$,当$x\neq x_0$时,有
	$$\frac{f(x)-f(x_0)}{x-x_0)}\geqslant 0.$$
	令$x\to x_0$,即得$f'(x_0)\geqslant 0$.
	
	反之,若$f(x)$在区间$I$上恒有$f'(x)\geqslant 0$,则对任意$x_1,x_2\in I\ (\text{设}x_1<x_2)$,由Lagrange中值定理,存在$\xi\in(x_1,x_2)$,使得
	$$f(x_2)-f(x_1)=f'(\xi)(x_2-x_1)\geqslant 0$$.
	所以$f$在$I$上为递增函数.$\hfill\blacksquare$
\end{proof}
\begin{theorem}
	若函数$f$在$(a,b)$上可导,则$f$在$(a,b)$上严格递增(递减)的充要条件为
	
	(1)对$\forall x\in(a,b),\ f'(x)\geqslant 0\ (f'(x)\leqslant 0)$;
	
	(2)在$(a,b)$的任意子区间上$f'(x)\not\equiv 0$.
\end{theorem}
\begin{proof}
	用反证法.
	
	必要性\qquad (i)是显然的.对(ii),假设存在$(c,d)\subset(a,b)$,对于$\forall x\in(c,d)$,$f'(x)\equiv 0$,则对任意$x_1,x_2\in (c,d)(\text{设}x_1<x_2)$,对任意$\xi\in(x_1,x_2)$,有$f(x_2)-f(x_1)=f'(\xi)(x_2-x_1)=0$.与严格递增矛盾.
	
	充分性\qquad (i)是显然的.对(ii),假设不严格递增,则存在$(c,d)\subset(a,b)$,对于任意$x_1,x_2\in (c,d)(\text{设}x_1<x_2)$,对任意$\xi\in(c,d)$,有$f'(\xi)(x_2-x_1)=f(x_2)-f(x_1)=0$,由于$x_1\neq x_2$,故$f'(\xi)=0$,即对任意$x\in(c,d)$,$f'(x)\equiv 0$,这与(ii)矛盾.
	$\hfill\blacksquare$
\end{proof}
\begin{corollary}
	设函数在区间$I$上可微,若$f'(x)>0(<0)$,则$f$在$I$上严格递增(递减).
\end{corollary}
\begin{theorem}[Darboux定理]
	若函数$f$在$\left[a,b\right]$上可导,且$f'_+(a)\neq f'_-(b)$,$k$为介于$f'_+(a),f'_-(b)$之间任一实数,则至少存在一点$\xi\in(a,b)$,使得
	$$f'(\xi)=k.$$
\end{theorem}
\begin{proof}
	设$F=f(x)-kx$,则$F(x)$在$\left[a,b\right]$上可导,且
	$$F'_+(a)\cdot F'_-(b)=(f'_+(a)-k)(f'_-(b)-k)<0.$$
	不妨设$F'_+(a)>0,F'_-(b)<0$.由命题\ref{prooffermat},分别存在$x_1\in\mathring{U}_+(a),x_2\in\mathring{U}_-(b)$,且$x_1<x_2$,使得
	$$F(x_1)>F(a),F(x_2)>F(b).$$
	因为$F$在$\left[a,b\right]$可导,所以连续.根据最值定理,存在一点$\xi\in\left[a,b\right]$,使$F$在点$\xi$取得最大值.$\xi\neq a,b$,这就说明$\xi$是$F$的极大值点.由Fermat定理得$F'(\xi)=0$,即
	$$f'(\xi)=k.$$ $\hfill\blacksquare$
\end{proof}
\begin{remark}
	有时候称上述定理为{\heiti 导函数介值定理}.
\end{remark}
\begin{corollary}
	设函数$f(x)$在区间$I$上满足$f'(x)\neq 0$,那么$f(x)$在区间$I$上严格单调.
\end{corollary}
\section{函数的极值与最值}
函数的极值不仅在实际问题中占有重要的地位,而且也是函数性态的一个重要特征.

Fermat定理已经告诉我们,可导函数在点$x_0$取极值的必要条件是$f'(x_0)=0$.下面讨论充分条件.
\begin{theorem}[极值的第一充分条件]
	设$f$在点$x_0$连续,在某邻域$\mathring{U}(x_0;\delta)$上可导.
	
	(i)若当$x\in(x_0-\delta,x_0)$时$f'(x)\leqslant 0$,当$x\in (x_0,x_0+\delta)$时$f'(x)\geqslant 0$,则$f$在点$x_0$取得极小值;
	
	(ii)若当$x\in(x_0-\delta,x_0)$时$f'(x)\geqslant 0$,当$x\in (x_0,x_0+\delta)$时$f'(x)\leqslant 0$,则$f$在点$x_0$取得极大值.
\end{theorem}
\begin{proof}
	只需证(i)即可.由定理的条件,$f$在$(x_0-\delta,x_0)$上递减,在$(x_0,x_0+\delta)$上递增,又由$f$在点$x_0$上连续,故对任意$x\in U(x_0;\delta)$,恒有
	$$f(x)\geqslant f(x_0).$$
	即$f$在点$x_0$取得极小值.可类似证明(ii)的情况.$\hfill\blacksquare$
\end{proof}
\begin{theorem}[极值的第二充分条件]
	设$f$在$x_0$的某邻域$U(x_0;\delta)$上一阶可导,在$x_0$处二阶可导,且$f'(x_0)=0,f''(x_0)\neq 0$.
	
	(i)若$f''(x_0)<0$,则$f$在$x_0$取得极大值.
	
	(ii)若$f''(x_0)>0$,则$f$在$x_0$取得极小值.
\end{theorem}
\begin{proof}
	由条件,可得$f$在$x_0$处的二阶Taylor公式
	$$f(x)=f(x_0)+f'(x_0)(x-x_0)+\frac{1}{2!}f''(x_0)(x-x_0)^2+o((x-x_0)^2).$$
	由于$f'(x_0)=0$,因此
	$$f(x)-f(x_0)=\left[\frac{f''(x_0)}{2}+o(1)\right](x-x_0)^2.$$
	又因$f''(x_0)\neq 0$,故存在正数$\delta'\leqslant\delta$,当$x\in U(x_0;\delta')$时,$\dfrac{1}{2}f''(x_0)$和$\dfrac{1}{2}f''(x_0)+o(1)$同号.所以,当$f''(x_0)<0$时,$\dfrac{1}{2}f''(x_0)+o(1)<0$,从而对任意$x\in \mathring{U}(x_0;\delta')$,有
	$$f(x)-f(x_0)<0.$$
	即$f$在$x_0$取极大值.同理,对$f''(x_0)>0$,可得$f$在$x_0$取极小值.
	$\hfill\blacksquare$
\end{proof}
\begin{theorem}[极值的第三充分条件]
	设$f$在$x_0$的某邻域上存在直到$n-1$阶导函数,在$x_0$处$n$阶可导,且$f^{(k)}(x_0)\ (k=1,2,\cdots,n-1)$,$f^{(n)}(x_0)\neq 0$,则
	
	(i)当$n$为偶数时,$f$在$x_0$取得极值,且当$f^{(n)}(x_0)<0$时取极大值,$f^{(n)}(x_0)>0$时取极小值.
	
	(ii)当$n$为奇数时,$f$在$x_0$处不取极值.
\end{theorem}
\begin{proof}
	与极值的第二充分条件证明类似.$f$在$x_0$的$n$阶Taylor公式
	$$f(x)=f(x_0)+f'(x_0)(x-x_0)+\cdots+\frac{f^{(n)}(x_0)}{n!}(x-x_0)^{n}+o((x-x_0)^n).$$
	其中$f^{(k)}(x_0)\ (k=1,2,\cdots,n-1)$,则只剩
	$$f(x)-f(x_0)=\left[\frac{f^{(n)}(x_0)}{n!}+o(1)\right](x-x_0)^{(n)}.$$
	
	当$n$为偶数时,$(x-x_0)^n>0$,$x_0$两侧某邻域内$f(x)$同号.$f^{(n)}(x_0)<0$时,$f(x)-f(x_0)<0$.则$f(x)$在$x_0$处取极大值,同理,对$f^{(n)}(x_0)>0$,可得$f$在$x_0$取极小值.
	
	当$n$为奇数时,$(x-x_0)^n$在$x_0$的两侧异号,因此$f(x)-f(x_0)$也在$x_0$的两侧异号,故不取极值.$\hfill\blacksquare$
\end{proof}

根据闭区间上连续函数的基本性质,若函数$f$在闭区间$\left[a,b\right]$上连续,则$f$在$\left[a,b\right]$上一定有最大、最小值.这是我们求连续函数的最大、最小值的理论保证.若函数$f$的最大(小)值点$x_0$在开区间$(a,b)$上,则$x_0$必定是$f$的极大(小)值点.又若$f$在$x_0$可导,则$x_0$还是一个稳定点.所以我们只要比较$f$在区间内部的所有稳定点、不可导点和区间端点上的函数值,就能从中找到$f$在$\left[a,b\right]$上的最大值和最小值.
\section{函数的凸性与拐点}
\begin{definition}[凸函数和凹函数]
	设$f$为定义在区间$I$上的函数,若对$I$上的任意两点$x_1,x_2$和任意实数$\lambda\in(0,1)$,都有
	$$f(\lambda x_1+(1-\lambda)x_2)\leqslant\lambda f(x_1)+(1-\lambda)f(x_2),$$
	则称$f$为$I$上的{\heiti 凸函数}(convex function).反之,如果总有
	$$f(\lambda x_1+(1-\lambda)x_2)\geqslant\lambda f(x_1)+(1-\lambda)f(x_2),$$
	则称$f$为$I$上的{\heiti 凹函数}(concave function).
\end{definition}
如果将上述的不等式改为严格不等式,则相应的函数称为{\heiti 严格凸函数}和{\heiti 严格凹函数}.

容易证明:若$-f$为区间$I$上的凸函数,则$f$为区间$I$上的凹函数.因此,只需讨论凸函数的性质即可.
\begin{theorem}
	$f$为$I$上的凸函数的充要条件是:对于$I$上的任意三点$x_1<x_2<x_3$,总有
	$$\frac{f(x_2)-f(x_1)}{x_2-x_1}\leqslant\frac{f(x_3)-f(x_2)}{x_3-x_2}.$$
\end{theorem}
\begin{proof}
	必要性\qquad 记$\lambda=\dfrac{x_3-x_2}{x_2-x_1}$,则$x_2=\lambda x_1+(1-\lambda)x_3$.由$f$的凸性知道
	\begin{align*}
		f(x_2)&=f(\lambda x_1+(1-\lambda)x_3)\leqslant\lambda f(x_1)+(1-\lambda)f(x_3)\\
		&=\frac{x_3-x_2}{x_3-x_1}f(x_1)+\frac{x_2-x_1}{x_3-x_1}f(x_3),
	\end{align*}
	从而有$$(x_3-x_1)f(x_2)\leqslant(x_3-x_2)f(x_1)+(x_2-x_1)f(x_3),$$
	$$(x_3-x_2)f(x_2)+(x_2-x_1)f(x_2)\leqslant (x_3-x_2)f(x_1)+(x_2-x_1)f(x_3).$$
	整理后即得结论.
	
	充分性\qquad 在$I$上任取两点$x_1,x_3(x_1<x_3)$,在$\left[x_1,x_3\right]$上任取一点$x_2=\lambda x_1+(1-\lambda)x_3,\lambda\in(0,1)$,即$\lambda=\dfrac{x_3-x_2}{x_3-x_1}$.由必要性推导的逆过程,可得
	$$f(\lambda x_1+(1-\lambda)x_3)\leqslant\lambda f(x_1)+(1-\lambda)f(x_3),$$
	故$f$为$I$上的凸函数.$\hfill\blacksquare$
\end{proof}
\begin{remark}
	如果$f$是$I$上的严格凸函数,则定理中的“$\leqslant$”可改为“$<$”.
\end{remark}
\begin{remark}
	同理可证加强的定理
	$$\frac{f(x_2)-f(x_1)}{x_2-x_1}\leqslant\frac{f(x_3)-f(x_1)}{x_3-x_1}\leqslant\frac{f(x_3)-f(x_2)}{x_3-x_2}.$$
	如果$f$是$I$上的严格凸函数,则定理中的“$\leqslant$”亦可改为“$<$”.
\end{remark}
\begin{theorem}
	设$f$是区间$I$上的可导函数,则下述论断互相等价:
	\begin{enumerate}
		\item $f$为$I$上的凸函数;
		\item $f'$为$I$上的增函数;
		\item 对$I$上任意两点$x_1,x_2$,有
		$$f(x_2)\geqslant f(x_1)+f'(x_1)(x_2-x_1).$$
	\end{enumerate}
\end{theorem}
\begin{proof}
	$1\to 2$\qquad 任取$I$上两点$x_1,x_2(x_1<x_2)$及充分小的正数$h$.由于$x_1-h<x_1<x_2<x_2+h$,根据$f$的凸性有
	$$\frac{f(x_1)-f(x_1-h)}{h}\leqslant\frac{f(x_2)-f(x_1)}{x_2-x_1}\leqslant\frac{f(x_2+h)-f(x_2)}{h}.$$
	由$f$是可导函数,令$h\to 0$,得
	$$f'(x_1)\leq\frac{f(x_2-f(x_1))}{x_2-x_1}\leqslant f'(x_2),$$
	所以$f'$是$I$上的递增函数.
	
	$2\to 3$\qquad 在以$x_1,x_2(\text{不妨设}x_1<x_2)$为端点的区间上,由Lagrange定理和$f'$递增条件,有
	$$f(x_2)-f(x_1)=f'(\xi)(x_2-x_1)\geqslant f'(x_1)(x_2-x_1).$$
	移项后即得
	$$f(x_2)\geqslant f(x_1)+f'(x_1)(x_2-x_1).$$
	
	$3\to 1$\qquad 设$x_1,x_2$为$I$上任意两点,$x_3=\lambda x_1+(1-\lambda)x_2,\ \lambda\in(0,1)$.由论断3,并利用$x_1-x_3=(1-\lambda)(x_1-x_2)$与$x_2-x_3=\lambda(x_2-x_1)$,有
	$$f(x_1)\geqslant f(x_3)+f'(x_3)(x_1-x_3)=f(x_3)+(1-\lambda)f'(x_3)(x_1-x_2),$$
	$$f(x_2)\geqslant f(x_3)+f'(x_3)(x_2-x_3)=f(x_3)+\lambda f'(x_3)(x_2-x_1).$$
	分别用$\lambda$和$1-\lambda$乘上列两式并相加,得
	$$f(x_1)+(1-\lambda)f(x_2)\geqslant f(x_3)=f(\lambda x_1+(1-\lambda)x_2)$$
	即$f$为$I$上的凸函数.$\hfill\blacksquare$
\end{proof}
\begin{remark}
	论断3的几何意义是:曲线$y=f(x)$总是在它的任一切线的上方.这是可导凸函数的几何特征.对于凹函数也有类似结论.
\end{remark}
\begin{corollary}
	设$f$为区间$I$上的二阶可导函数,则在$I$上$f$为凸函数的充要条件是
	$$f''(x)\geqslant 0,\ x\in I.$$
	$f$为凹函数的充要条件是
	$$f''(x)\leqslant 0,\ x\in I.$$
\end{corollary}
\begin{definition}[拐点]
	设曲线$y=f(x)$在点$(x_0,f(x_0))$处有穿过曲线的切线.且在切点近旁,曲线在切线的两侧分别是严格凸和严格凹的,这时称点$(x_0,f(x_0))$为曲线$y=f(x)$的{\heiti 拐点}(inflection point).
\end{definition}
由定义可见,拐点正是凸和凹曲线的分界点.

易证下面有关拐点的定理.
\begin{theorem}
	若$f$在$x_0$二阶可导,则$(x_0,f(x_0))$是曲线拐点的必要条件是$f''(x_0)=0$.
\end{theorem}
\begin{theorem}
	设$f$在$x_0$可导,在某邻域$\mathring{U}(x_0)$上二阶可导.若在$\mathring{U}_+(x_0)$和$\mathring{U}_-(x_0)$上$f''(x_0)$的符号相反,则$(x_0,f(x_0))$为曲线的一个拐点.
\end{theorem}
然而,若$(x_0,f(x_0))$是曲线$y=f(x)$的一个拐点,$y=f(x)$在$x_0$处的导数不一定存在.例如$y=\sqrt[3]{x}$在$x=0$时的情况.
\section{方程的近似解}
\newpage

\chapter{不定积分}
\section{不定积分的概念}
正如加法有逆运算减法,乘法有逆运算除法一样,微分法也有逆运算——积分法.
\subsection{原函数与不定积分}
\begin{definition}[原函数]
	设函数$f$与$F$在区间$I$上都有定义.若
	$$F'(x)=f(x),\quad x\in I,$$
	则称$F$为$f$在区间$I$上的一个{\heiti 原函数}(primitive function).
\end{definition}
在研究原函数之前,我们需要明确原函数存在的条件与是否唯一,在这个前提下我们谈论求解原函数才是有意义的.
\begin{theorem}[原函数存在定理(待证)]
	若函数$f$在区间$I$上连续,则$f$在$I$上存在原函数$F$.
\end{theorem}
由于初等函数在其定义区间上都是连续函数,因此每个初等函数在其定义区间上都有原函数.
\begin{theorem}
	设$F$是$f$在区间$I$上的一个原函数,则
	
	(i)$F+C$也是$f$在$I$上的原函数,其中$C$为任意常量函数(常数);
	
	(ii)$f$在$I$上的任意两个原函数之间,只可能相差一个常数.
\end{theorem}
\begin{proof}
	(i)这是因为$\left[F(x)+C\right]'=F'(x)=f(x),\ x\in I$.
	
	(ii)设$F$和$G$是$f$在$I$上的任意两个原函数,则有
	$$\left[F(x)-G(x)\right]'=F'(x)-G'(x)=f(x)-f(x)=0,\ x\in I.$$
	根据Lagrange定理的推论2,有$$F(x)-G(x)\equiv C,\ x\in I.$$
	$\hfill\blacksquare$
\end{proof}
\begin{definition}[不定积分]
	函数$f$在区间$I$上的全体原函数称为$f$在$I$上的{\heiti 不定积分}(indefinite integral),记作
	$$\int f(x)\d x,$$
	其中称$\int$为积分号,$f(x)$为{\heiti 被积函数}(integrand),$f(x)\d x$为{\heiti 被积表达式},$x$为{\heiti 积分变量}.
\end{definition}
可见,不定积分与原函数是总体和个体的关系,即若$F$为$f$的一个原函数,则$f$的不定积分是一个函数族$\{F+C\}$,其中$C$为任意常数,为方便起见,记作
$$\int f(x)\d x=F(x)+C.$$
于是又有
$$\left[\int f(x)\d x\right]'=\left[F(x)+C\right]'=f(x),$$
$$d\int f(x)\d x=d\left[F(x)+C\right]=f(x)\d x.$$

不定积分的几何意义:若$F$是$f$的一个原函数,则称$y=F(x)$为$f$的一条{\heiti 积分曲线}.$f$的不定积分在几何上表示$f$的某一积分曲线沿纵轴方向任意平移所得一切积分曲线组成的曲线族.
\subsection{基本积分表}
我们把基本导数公式写成基本积分公式.
\begin{proposition}[基本积分公式]
	\begin{align*}
		&\int x^\alpha\d x=\frac{x^{\alpha +1}}{\alpha +1}+C\ (\alpha\neq -1,\ x\geqslant 0).\\
		&\int \frac{1}{x}\d x=\ln|x|+C\ (x\neq 0).\\
		&\int a^x\d x=\frac{a^x}{\ln a}+C\ (a>0,\ a\neq 1).\\
		&\int \cos x\d x=\sin x+C.\\
		&\int \sin x\d x=-\cos x+C.\\
		&\int \sec^2 x\d x=\tan x+C.\\
		&\int \csc^2 x\d x=-\cot x+C.\\
		&\int \sec x\cdot\tan x\d x=\sec x+C.\\
		&\int \csc x\cdot\cot x\d x=-\csc x+C.\\
		&\int \frac{\d x}{\sqrt{1-x^2}}=\arcsin x+C=-\arccos x+C_1.\\
		&\int \frac{\d x}{1+x^2}=\arctan x+C=-\text{arccot}x+C_1.&
	\end{align*}
\end{proposition}
我们可以从导数的线性运算法则得到不定积分的线性运算法则.
\begin{theorem}[积分的线性性]
	若函数$f_i(x)(i=1,2,\cdots)$在区间$I$上都存在原函数,$k_i$为任意常数,则
	$$\sum_{i=1}^{n}k_if_i(x)$$在$I$上也存在原函数,且当$k_i$中所有项不同时为零时,有
	$$\int\left(\sum_{i=1}^{n}k_if_i(x)\right)\d x=\sum_{i=1}^{n}k_i\int f_i(x)\d x.$$
\end{theorem}
对等式两边求导即可证得,在此不再赘述.
\section{积分方法}
上一节我们通过基本导数公式得到了一部分基本初等函数的原函数.但即使像$\ln x$,$\tan x$这样的基本初等函数我们还不知道如何求得其原函数,因此我们还需要一些求积法则.下面我们介绍换元积分法和分部积分法.
\subsection{换元积分法}
\begin{theorem}[第一换元积分法]
	设函数$f(x)$在区间$I$上有定义,$\varphi(t)$在区间$J$上可导,且$\varphi(J)\subseteq I$.如果不定积分$\int f(x)\d x=F(x)+C$在$I$上存在,则不定积分$\int f(\varphi(t))\varphi'(t)\d t$也存在,且
	$$\int f(\varphi(t))\varphi'(t)\d t=F(\varphi(t))+C.$$
\end{theorem}
\begin{proof}
	用复合函数求导法进行验证.因为对于任何$t\in J$,有
	$$\frac{\d}{\d t}(F(\varphi(t)))=F'(\varphi(t))\varphi'(t)=f(\varphi(t))\varphi'(t),$$
	所以$f(\varphi(t))\varphi'(t)$以$F(\varphi(t))$为原函数$\hfill\blacksquare$
\end{proof}
\begin{remark}
	第一换元公式也可以写成
	\begin{align*}
		\int f(\varphi(t))\varphi'(t)\d t
		&=\int f(\varphi(t))\d \varphi(t)\\
		&=\int f(x)\d x\qquad(\text{令}x=\varphi (t))\\
		&=F(x)+C
		&=F(\varphi(t))+C.
	\end{align*}
	因此第一换元法也成为{\heiti 凑微分法}
\end{remark}
\begin{theorem}[第二换元积分法]
	设函数$f(x)$在区间$I$上有定义,$\varphi(t)$在区间$J$上可导,$\varphi(J)=I$,且$x=\varphi(t)$在区间$J$上存在反函数$t=\varphi^{-1}(x),\ x\in I$.如果不定积分$\int f(x)\d x$在$I$上存在,则当$\int f(\varphi(t))\varphi'(t)\d t$在$J$上存在时,在$I$上有
	$$\int f(x)\d x=G(\varphi^{-1}(x))+C.$$
\end{theorem}
\begin{proof}
	设$\int f(x)\d x=F(x)+C.$对于任何$t\in J$,有
	\begin{align*}
		\frac{\d}{\d t}(F(\varphi(t))-G(t))&=F'(\varphi(t))\varphi'(t)-G'(t)
		&=f(\varphi(t))\varphi'(t)-f(\varphi(t))\varphi'(t)=0.
	\end{align*}
	所以存在常数$C_1$,使得$F(\varphi(t))-G(t)=C_1$对于任何$t\in J$成立,从而$G(\varphi'(x))=F(x)-C_1$对于任何$x\in I$成立.因此,对于任何$x\in I$,有
	$$\frac{\d}{\d x}(G(\varphi^{-1}(x)))=F'(x)=f(x),$$
	即$G(\varphi^{-1}(x))$为$f(x)$的原函数.$\hfill\blacksquare$
\end{proof}
\begin{remark}
	第二换元公式也可以写成
	\begin{align*}
		\int f(x)\d x&=\int f(\varphi(t))\varphi'(t)\d t\ (\text{令}x=\varphi(t))
		&=G(t)+C
		&=G(\varphi^{-1}(x))+C.(t=\varphi^{-1}(x))
	\end{align*}
	因此第二换元法也成为{\heiti 代入换元法}.
\end{remark}
\subsection{分部积分法}
由乘积求导法,可以导出分部积分法.
\begin{theorem}[分部积分法]
	若$u(x)$与$v(x)$可导,不定积分$\int u'(x)v(x)\d x$存在,则$\int u(x)v'(x)\d x$也存在,并有
	$$\int u(x)v'(x)\d x=u(x)v(x)-\int u'(x)v(x)\d x.$$
\end{theorem}
\begin{proof}
	由$$\left[u(x)v(x)\right]'=u'(x)v(x)+u(x)v'(x),$$
	得$$u(x)v'(x)=\left[u(x)v(x)\right]'-u'(x)v(x),$$
	对上式两边积分即得定理中的结论.$\hfill\blacksquare$
\end{proof}
分部积分公式也可写作
$$\int u\d v=uv-\int v\d u.$$
\section{有理函数和可化为有理函数的不定积分}
\newpage

\chapter{Riemann积分}
\section{Riemann积分的概念}
\subsection{长度、面积与Euclid空间}
虽然我们在小学时期就学习了长度、面积等相关概念,但事实上我们从未严格定义过长度、面积和体积.下面我们尝试定义这些概念.

我们可以把“长度”看作是1维实空间$\mathbb{R}$(即实数轴)的一个子集族$X$($\mathbb{R}$的每个子集不一定都有“长度”)到实数域的一个映射$M$.我们首先规定
$$M(\left[a,b\right])\coloneqq b-a.$$
其中$a\leqslant b$.这表明任何闭区间$\left[a,b\right]$的长度为$b-a$,并蕴含了数轴上任意一点的长度为零.然后我们可以列出几条公理(姑且称它们为公理):对于任意$A,B\in X$,都满足
\begin{enumerate}
	\item 非负性:$M(A)\geqslant 0$.
	\item 单调性:若$A\subseteq B$,则$M(A)\leqslant M(B)$.
	\item 可加性:若$A\cap B=\varnothing$,则$M(A\cup B)=M(A)+M(B)$.
\end{enumerate}
若集合$A$可通过平面上的正交变换(平面的正交变换即为平移、旋转、反射以及它们的乘积)变成了$B$,则称$A$和$B${\heiti 全等}或{\heiti 合同}.我们规定,$M(A)=M(B)$当且仅当$A\cong B$.

由于任意一点的长度都是零,由可加性公理可知开区间$(a,b)$的长度也是$b-a$,半开半闭区间的长度亦然.为了让整个实数轴也有长度,我们规定$M$可以取到$+\infty$.

类似地,我们可以把面积看作是2维实空间$\mathbb{R}^2$(即实平面)的一个子集族$X$到实数域$\mathbb{R}$的一个映射$M$.我们首先规定一个邻边长分别为$a$和$b$的矩形$A$的面积为$a\cdot b,\ a,b\geqslant 0$.这蕴含了线段的面积为零.以上的三条定理可以“原封不动”地来刻画面积.依次下去,还可以进一步把长度、面积的概念推广到体积以及$n$维Euclid空间$\mathbb{R}^n$中.事实上物理中的{\heiti 功}(work),{\heiti 位移}(displacement),{\heiti 冲量}(impulse)都满足以上三条公理,我们可以考虑用一个统一的概念来描述它们,这就是{\heiti 测度}(measure).今后我们会专门研究这个主题.
\subsection{Riemann积分的定义}
对于任意多边形,我们总能分成若干个三角形从而容易求得其面积.但对于曲边图形,这个方法就失效了,因此我们定义了Riemann积分.

下面我们讨论曲边梯形面积的求解.设$f$是闭区间$\left[a,b\right]$上的连续函数.我们要讨论的就是$x=a$,$x=b$,$y=f(x)$以及与$x$轴围成的曲边梯形的面积.

我们采用“{\heiti 分割,近似求和,取极限}”的方法来求解.下面给出分割的定义.
\begin{definition}[分割]
	设闭区间$\left[a,b\right]$上有$n-1$个点,依次为
	$$a=x_0<x_1<x_2<\cdots <x_{n-1}<x_n=b,$$
	它们把$\left[a,b\right]$分成$n$个小区间$\Delta_i=\left[x_{i-1},x_i\right],\ i=1,2,\cdots,n$.这些分点或这些闭子区间构成对闭区间$\left[a,b\right]$的一个{\heiti 分割},记为
	$$T=\{x_0,x_1,\cdots,x_n\}\text{或}\{\Delta_1,\Delta_2,\cdots,\Delta_n\}.$$
	小区间$\Delta_i$的长度为$\Delta x_i=x_i-x_{i-1}$,并记
	$$\| T\|\coloneqq\mathop{\max}_{1\leqslant i\leqslant n}\{\Delta x_i\},$$
	称为分割$T$的{\heiti 模}.
\end{definition}
\begin{definition}[Riemann和]
	设$f$是定义在$\left[a,b\right]$上的一个函数.对于$\left[a,b\right]$的一个分割$T=\{\Delta_1,\Delta_2,\cdots,\Delta_n\}$,任取点$\xi_i\in\Delta_i,\ i=1,2,\cdots,n$,并作和式
	$$\sum_{i=1}^{n}f(\xi_i)\Delta x_i.$$
	称此和式为函数$f$在$\left[a,b\right]$上的一个{\heiti Riemann和}(Riemann sum).
\end{definition}
显然Riemann和既与分割$T$有关,又与所选取的点集$\{\xi_i\}$有关.
\begin{definition}[Riemann积分]
	设$f$是定义在$J=\left[a,b\right]$上的一个函数,$I$是一个确定的实数.若对任意$\varepsilon>0$,总存在$\delta>0$,使得对$\left[a,b\right]$上的{\heiti 任意}分割$T$,以及在其上{\heiti 任意}选取的点集$\{\xi_i\}$,当$\Vert T\Vert<\delta$时,有
	$$\big|\sum_{i=1}^{n}f(\xi_i)\Delta x_i-I\big|<\varepsilon,$$
	即
	$$\lim\limits_{\|T\|\to 0}\sum_{i=1}^{n}f(\xi_i)\Delta x_i=I$$
	则称函数$f$在区间$\left[a,b\right]$上{\heiti Riemann可积}(在本章中也可称为可积);数$I$称为$f$在$\left[a,b\right]$上的{\heiti Riemann积分}或{\heiti 定积分},记作
	$$I=\int_{a}^{b}f(x)\d x.$$
	其中$f$称为{\heiti 被积函数}(integrand),$f(x)\d x$称为{\heiti 积分表达式}(integrand expression),$a$和$b$分别称为{\heiti 积分下限}(lower limit of integration)和{\heiti 积分上限}(upper limit of integration).
\end{definition}
\begin{remark}
	由于$f$是定义在区间$J$上的,因此在$J$上的定积分也可以写成
	$$\int_{J}f(x)\d x.$$
\end{remark}
\begin{remark}
	定积分的几何意义就是曲边梯形的面积,对于在$x$轴上方的部分,面积取正值,在$x$轴下方的部分,面积取负值.对一般的非定号的函数$f$,其定积分的定义记为积分区间上所以曲边梯形的正面积和负面积的代数和.
\end{remark}
\begin{remark}
	要区分Riemann积分定义的$\varepsilon-\delta$语言和极限定义中$\varepsilon-\delta$语言的差别.函数极限可以由Heine归结原理从而由数列极限刻画,而Riemann积分的$\|T\|$趋于$0$的过程是无法这样刻画的.这也是Riemann积分比函数极限复杂的地方.
\end{remark}
\begin{proposition}[Riemann积分的简单性质]
	设函数$f$和$g$在区间$\left[a,b\right]$上Riemann可积.
	\begin{enumerate}
		\item 保号性:若$f(x)$非负,则
		$$\int_{a}^{b}f(x)\d x\geqslant0.$$
		\item 保序性:若$f(x)\geqslant g(x)$,则
		$$\int_{a}^{b}f(x)\d x\geqslant\int_{a}^{b}g(x)\d x.$$
		\item 可加性:若$c\in(a,b)$,且$f$在$\left[a,c\right],\left[c,b\right]$上Riemann可积,则
		$$\int_{a}^{b}f(x)\d x=\int_{a}^{c}f(x)\d x+\int_{c}^{b}f(x)\d x.$$
		\item 线性性:
		$$\int_{a}^{b}\left(\lambda f(x)+g(x)\right)\d x=\lambda\int_{a}^{b}f(x)\d x+\int_{a}^{b}g(x)\d x.$$
	\end{enumerate}
\end{proposition}
\begin{remark}
	后面我们将证明若函数$f$在$\left[a,b\right]$都可积,则它在$\left[a,c\right],\left[b,c\right]$上都可积.因此性质3的可积性条件是可以去掉的.
\end{remark}
\begin{remark}
	由性质4可以看出,Riemann积分也是一个线性算子.
\end{remark}
\section{Riemann可积条件}
\subsection{Riemann可积的必要条件}
\begin{theorem}
	若函数$f$在$\left[a,b\right]$上可积,则$f$在$\left[a,b\right]$上有界.
\end{theorem}
\begin{proof}
	用反证法.如果$f$在$\left[a,b\right]$上无界,则对于闭区间$\left[a,b\right]$的任何分割$T$,函数$f$至少在一个分割区间$\left[x_{i-1},x_i\right]$上无界.这表示,可以选出点$\xi_i\in\left[x_{i-1},x_i\right]$,使$|f(\xi_i)\Delta x_i|$取任意大的值.但这样一来,只要改变点$\xi_i$在这个区间中的位置,就能使Riemann和$\displaystyle\sum_{i=1}^{n}f(\xi_i)\Delta x_i$具有任意大的绝对值.
	
	显然,由Riemann积分的定义可知,Riemann和现在不可能具有有限的极限.$\hfill\blacksquare$
\end{proof}
\begin{remark}
	该定理简述为“有界不一定可积,可积一定有界”(“可积”特指Riemann可积).有了这个Riemann可积的必要条件,我们下面只需研究有界函数即可.
\end{remark}
\subsection{Riemann可积的充要条件}
\begin{definition}[Darboux和]
	设函数$f$在$\left[a,b\right]$上有界.设$T=\left\{\Delta_i|i=1,2,\cdots,n\right\}$为对$\left[a,b\right]$的任一分割.则$f(x)$在每个小区间$\Delta_i$上有上、下确界,我们将$f(x)$在每个小区间$\Delta_i$上的上确界记作$M_i$,下确界记作$m_i$.令
	$$\overline{S}(T)=\sum_{i=1}^{n}M_i\Delta x_i,\quad \underline{S}(T)=\sum_{i=1}^{n}m_i\Delta x_i.$$
	我们把$\overline{S}(T)$和$\underline{S}(T)$分别称为$f$关于分割$T$的{\heiti Darboux上和}(upper Darboux sum)与{\heiti Darboux下和}(lower Darboux sum),简称{\heiti 上和}与{\heiti 下和}.
\end{definition}
我们在函数的连续性章节的讨论中已经给出了振幅的定义(定义\ref{def:amplitude}),再此不再赘述.

接下来,我们来研究上和与下和.显然的,根据Darboux和的定义,我们有
$$\underline{S}(T)\leqslant \sum_{i=1}^{n}f(\xi_i)\Delta x_i\leqslant \overline{S}(T).$$

类比数列(或函数)的上极限和下极限,我们可以推测上和与下和可能有以下结论:
\begin{enumerate}
	\item 所以上和组成的集合有下界,从而有下确界;所以下和组成的集合有上界,从而有上确界.我们暂且把这个下确界和上确界分别称为“上积分”与“下积分”.
	\item 在某个过程中,下和单调递增且有上界,上和单调递减且有下界,因此它们都有极限,它们的极限恰好是“上积分”和“下积分”.
	\item 任何有界函数都存在“上积分”和“下积分”.函数Riemann可积当且仅当它的“上积分”等于“下积分”.
\end{enumerate}

下面考虑在分割$T$的基础上增加分割点,我们想知道在这过程中是否会发生我们预期的事情.先从最简单的情况开始:在分割$T$上多加一个分割点$x'$,它位于$\left[x_{i-1},x_i\right]$,这样就得到了一个新的分割$T'$,它有$n+2$个分割点.这时下和$\underline{S}(T)$与$\underline{S}(T')$的差别在于$\underline{S}$中的$m_i\Delta x_i$变成了
$$(x'-x_{i-1})\inf f(\left[x_{i-1},x'\right])+(x_i-x')\inf f(\left[x',x_i\right]).$$
因此
$$\underline{S}(T')-\underline{S}(T)=(x'-x_{i-1})\inf f(\left[x_{i-1},x'\right])+(x_i-x')\inf f(\left[x',x_i\right])-m_i\Delta x_i.$$
容易知道
$$m_i\leqslant\inf f(\left[x_{i-1},x'\right])\leqslant M_i,$$
$$m_i\leqslant\inf f((\left[x',x_i\right])\leqslant M_i.$$
因此
$$\underline{S}(T')-\underline{S}(T)\geqslant m_i(x'-x_{i-1})+m_i(x_i-x')-m_i(x_i-x_{i-1})=0.$$
\begin{align*}
	\underline{S}(T')-\underline{S}(T)&\leqslant M_i(x'-x_{i-1})+M_i(x_i-x')-m_i(x_i-x_{i-1})\\
	&=(M_i-m_i)\Delta x_i=\omega_i\Delta x_i\\
	&\leqslant\omega\Delta x_i\leqslant\omega\|T\|.
\end{align*}
综上,我们得到
$$\underline{S}(T)\leqslant\underline{S}(T')\leqslant\underline{S}(T)+\omega\|T\|.$$
类似地可以得到
$$\overline{S}(T)\geqslant\overline{S}(T')\geqslant\overline{S}(T)-\omega\|T\|.$$
用数学归纳法可以证明增加$k$个分割点的情况.于是我们有以下命题.
\begin{proposition}{\label{darboux1}}
	设函数$f(x)$在区间$\left[a,b\right]$上有界.给定$\left[a,b\right]$的一个分割$T$,在此基础上增加$k$个分割点,得到新的分割$T'$.若$f(x)$在$\left[a,b\right]$上的振幅为$\omega$,则
	$$\underline{S}(T)\leqslant\underline{S}(T')\leqslant\underline{S}(T)+k\omega\|T\|,$$
	$$\overline{S}(T)\geqslant\overline{S}(T')\geqslant\overline{S}(T)-k\omega\|T\|.$$
\end{proposition}
\begin{remark}
	若分割$T$的所有分点都是$T'$的分点,则称$T'$比$T$更细,记作$T'\leqslant T$.显然这样规定的“$\geqslant$”是一个偏序关系,并不是任意两个分割都可以比较粗细.
\end{remark}
\begin{remark}
	上述命题表明,在$T$上不断增加分割点的过程中,下和单调递增,上和单调递减.
\end{remark}
\begin{proposition}
	对任意两个分割$T_1$和$T_2$,总有
	$$\underline{S}(T_1)\leqslant\overline{S}(T_2).$$
\end{proposition}
\begin{proof}
	令$T=T_1+T_2$,则由命题\ref{darboux1}可得
	$$\underline{S}(T_1)\leqslant\underline{S}(T)\leqslant\overline{S}(T)\leqslant\overline{S}(T_2).$$
	$\hfill\blacksquare$
\end{proof}

以上命题表明,在对$\left[a,b\right]$所做的两个分割中,一个分割的下和总不大于另一个分割的上和.因此对所有分割来说,所有下和有上界,从而有上确界;所有上和有下界,从而有下确界.这就证实了我们的第一个想法.于是可以定义“上积分”和“下积分”
\begin{definition}[Darboux积分]
	设函数$f$在区间$\left[a,b\right]$上有界.\\
	$f$在$\left[a,b\right]$上所有Darboux上和组成的集合的下确界称为$f$在$\left[a,b\right]$上的{\heiti Darboux上积分}(upper Darboux integral);\\
	$f$在$\left[a,b\right]$上所有Darboux下和组成的集合的上确界称为$f$在$\left[a,b\right]$上的{\heiti Darboux下积分}(lower Darboux integral),\\
	简称上积分与下积分,分别记作:
	$$\overline{\int_{a}^{b}}f(x)\d x\coloneqq\inf\limits_{T}\overline{S}(T),$$
	$$\underline{\int_{a}^{b}}f(x)\d x\coloneqq\sup\limits_{T}\underline{S}(T).$$
	当
	$$\overline{\int_{a}^{b}}f(x)\d x=\underline{\int_{a}^{b}}f(x)\d x=I\in\mathbb{R}$$
	时,我们称$f$在$\left[a,b\right]$上{\heiti Darboux可积}(Darboux integrable),称$I$是$f$在$\left[a,b\right]$上的{\heiti Darboux积分}(Darboux integral).
\end{definition}
\begin{remark}
	Darboux积分是由法国数学家Jean\ Gaston\ Darboux于1875年提出的.
\end{remark}

我们已经看到在$T$不断增加分割点的过程中,下和单调递增且有上界;上和单调递减且有下界.因此当$\|T\|\to 0$时上和与下和一定有极限,下面来证明它们的极限恰好是“上积分”与“下积分”.
\begin{theorem}[Darboux定理]
	设函数$f$在区间$\left[a,b\right]$上有界.作$\left[a,b\right]$的一个分割$T$,则
	$$\lim\limits_{\|T\|\to 0}\overline{S}(T)=\overline{\int_{a}^{b}}f(x)\d x,$$
	$$\lim\limits_{\|T\|\to 0}\underline{S}(T)=\underline{\int_{a}^{b}}f(x)\d x.$$
\end{theorem}
\begin{proof}
	只证明下积分的情况.把$f$在$\left[a,b\right]$上的下积分记作$\underline{I}$.由下积分的定义可知对于任一$\varepsilon>0$,存在$\left[a,b\right]$的一个分割$T_0$,使得
	$$\underline{S}(T_0)>\underline{I}-\frac{\varepsilon}{2}.$$
	设$T_0$一共有$l$个分割点(不含$a,b$),则对于$\left[a,b\right]$的任一分割$T$,当$\|T\|<\dfrac{\varepsilon}{2l\omega+1}$时,由命题\ref{darboux1}可知
	$$\underline{S}(T)\geqslant\underline{S}(T_0+T)\geqslant\underline{S}(T_0)-l\omega\|T\|>\underline{I}-\frac{\varepsilon}{2}-l\omega\cdot\frac{\varepsilon}{2l\omega+1}>\underline{I}-\frac{\varepsilon}{2}-\frac{\varepsilon}{2}=l-\varepsilon.$$
	因此
	$$\underline{I}-\underline{S}(T)<\varepsilon.$$
	于是可知
	$$\lim\limits_{\|T\|\to 0}\underline{S}(T)=\underline{\int_{a}^{b}}f(x)\d x.$$
	$\hfill\blacksquare$
\end{proof}
\begin{remark}
	由于$\omega$可能为$0$,为了避免繁琐的分类讨论,此处用了“$\omega+1$”.
\end{remark}
\begin{remark}
	以上定理也可作为上积分与下积分的定义.
\end{remark}
类似于数列的上极限和下极限,可以得到一系列关于上积分和下积分的简单性质.
\begin{proposition}
	设函数$f(x)$在区间$\left[a,b\right]$上有界.则
	$$\underline{\int_{a}^{b}}f(x)\d x\leqslant\overline{\int_{a}^{b}}f(x)\d x.$$
\end{proposition}
\begin{proposition}[保序性]
	设函数$f(x),g(x)$在区间$\left[a,b\right]$上有界.若$f(x\geqslant g(x)(\forall x\in\left[a,b\right]))$则
	$$\overline{\int_{a}^{b}}f(x)\d x\geqslant\overline{\int_{a}^{b}}g(x)\d x,$$
	$$\underline{\int_{a}^{b}}f(x)\d x\geqslant\underline{\int_{a}^{b}}g(x)\d x.$$
\end{proposition}
\begin{proposition}[可加性]
	设函数$f(x)$在区间$\left[a,b\right]$上有界.若$c\in(a,b)$,则
	$$\overline{\int_{a}^{b}}f(x)\d x=\overline{\int_{a}^{c}}f(x)\d x+\overline{\int_{c}^{b}}f(x)\d x,$$
	$$\underline{\int_{a}^{b}}f(x)\d x=\underline{\int_{a}^{c}}f(x)\d x+\underline{\int_{c}^{b}}f(x)\d x.$$
\end{proposition}
\begin{proposition}[下积分的超可加性与上积分的次可加性]
	设函数$f(x),g(x)$在区间$\left[a,b\right]$上有界.则
	$$\overline{\int_{a}^{b}}f(x)\d x+\overline{\int_{a}^{b}}g(x)\d x\geqslant\overline{\int_{a}^{b}}\left[f(x)+g(x)\right]\d x,$$
	$$\underline{\int_{a}^{b}}f(x)\d x+\underline{\int_{a}^{b}}g(x)\d x\leqslant\underline{\int_{a}^{b}}\left[f(x)+g(x)\right]\d x,$$
\end{proposition}
\begin{proposition}
	设函数$f(x),g(x)$在区间$\left[a,b\right]$上有界.由于任一实数$c\geqslant 0$有
	$$\overline{\int_{a}^{b}}cf(x)\d x=c\overline{\int_{a}^{b}}f(x)\d x,\quad\underline{\int_{a}^{b}}cf(x)\d x=c\underline{\int_{a}^{b}}f(x)\d x,$$
	对于任一实数$c\leqslant 0$都有
	$$\overline{\int_{a}^{b}}cf(x)\d x=c\underline{\int_{a}^{b}}f(x)\d x,\quad\underline{\int_{a}^{b}}cf(x)\d x=c\overline{\int_{a}^{b}}f(x)\d x.$$
\end{proposition}

现在终于可以来验证我们的第三个想法.
\begin{theorem}[Riemann可积的充要条件]
	设函数$f(x)$在$\left[a,b\right]$上有界,则以下4个命题等价:
	\begin{enumerate}
		\item $f(x)$在$\left[a,b\right]$上Riemann可积.
		\item 作分割$T=\{\Delta_i\}(i=1,2,\cdots,n)$.设$f(x)$在$\left[x_{i-1},x_i\right]$上的振幅为$\omega_i$.则
		$$\lim\limits_{\|T\|\to 0}\sum_{i=1}^{n}\omega_i\Delta x_i=0.$$
		\item 对于任一$\varepsilon>0$都存在$\left[a,b\right]$的一个分割$T$使得
		$$\overline{S}(T)-\underline{S}(T)<\varepsilon.$$
		\item $f(x)$在$\left[a,b\right]$上Darboux可积.
	\end{enumerate}
\end{theorem}
\begin{proof}
	\begin{enumerate}
		\item 证明$1\Rightarrow2$.若1成立,令$I=\displaystyle\int_{a}^{b}f(x)\d x$,则对于任一$\varepsilon>0$,存在$\delta>0$使得当$\|T\|<\delta$时,无论$\xi_i$在$\left[x_i-1,x_i\right]$中如何选取都有
		$$I-\frac{\varepsilon}{3}<\sum_{i=1}^{n}f(\xi_i)\Delta x_i<I+\frac{\varepsilon}{3}.$$
		因此
		$$0\leqslant\overline{S}(T)-\underline{S}(T)\leqslant\frac{2}{3}\varepsilon<\varepsilon\iff 0\leqslant\sum_{i=1}^{n}\omega_i\Delta x_i\leqslant\frac{2}{3}\varepsilon<\varepsilon.$$
		于是可知
		$$\lim\limits_{\|T\|\to 0}\sum_{i=1}^{n}\omega_i\Delta x_i=0.$$
		\item 证明$2\Rightarrow3$.若2成立,则对于任一$\varepsilon>0$都存在一个$\left[a,b\right]$的分割$T$使得
		$$\sum_{i=1}^{n}\omega_i\Delta x_i=\overline{S}(T)-\underline{S}(T)<\varepsilon.$$
		\item 证明$3\Rightarrow4$.若3成立,则对于任一$\varepsilon>0$都有
		$$0\leqslant\overline{\int_{a}^{b}}f(x)\d x-\underline{\int_{a}^{b}}f(x)\d x\leqslant\overline{S}(T)-\underline{S}(T)<\varepsilon.$$
		令$\varepsilon\to 0$得
		$$\overline{\int_{a}^{b}}f(x)=\underline{\int_{a}^{b}}f(x)\d x.$$
		\item 证明$4\Rightarrow1$.若$f(x)$在$\left[a,b\right]$上Darboux可积,则
		$$\overline{\int_{a}^{b}}f(x)\d x=\underline{\int_{a}^{b}}f(x)\d x.$$
		故对于任一分割$T$都有
		$$\underline{S}(T)\leqslant\sum_{i=1}^{n}f(\xi_i)\Delta x_i\leqslant\overline{S}(T).$$
		令上式中的$\|T\|\to 0$,由Darboux定理可知
		$$\underline{\int_{a}^{b}}f(x)\d x=\lim\limits_{\|T\|\to 0}\underline{S}(T)\leqslant\lim\limits_{\|T\|\to 0}\sum_{i=1}^{n}f(\xi_i)\Delta x_i\leqslant\lim\limits_{\|T\|\to 0}\overline{S}(T)=\overline{\int_{a}^{b}}f(x)\d x.$$
		由迫敛性,极限$\lim\limits_{\|T\|\to 0}\displaystyle\sum_{i=1}^{n}f(\xi_i)\Delta x_i$存在,于是可知$f(x)$在$\left[a,b\right]$上Riemann可积.$\hfill\blacksquare$
	\end{enumerate}
\end{proof}

以上定理表明,Riemann积分和Darboux积分是等价的,它们分别从两个角度定义了同一种积分.Riemann积分借鉴了极限的$\varepsilon-\delta$语言,而Darboux积分用了上极限和下极限的思想.Riemann积分的定义比较直观,但Darboux积分的定义“更容易操作”,因此要判断一个函数是否Riemann可积,我们一般都是去判断它是否Darboux可积.至此我们找到了Riemann可积的两个充要条件.
\begin{example}
	判断Dirichlet函数$D(x)$在$\left[0,1\right]$上是否Riemann可积.
\end{example}
\begin{solution}
	对于$\left[0,1\right]$的任一分割,由于分割内的任一小区间内都同时含有无理点和有理点,因此$D(x)$的任一上和都等于1,任一下和都等于0.于是可知
	$$\overline{\int_{a}^{b}}D(x)\d x=1,\quad \underline{\int_{a}^{b}}D(x)\d x=0.$$
	上积分与下积分不相等,故$D(x)$不Riemann可积.
	$\hfill\blacksquare$
\end{solution}
\subsection{Riemann可积的充分条件}
\begin{theorem}
	若函数$f$在$\left[a,b\right]$上单调,则$f$在$\left[a,b\right]$上可积.
\end{theorem}
\begin{proof}
	只需考虑增函数的情况.若$f(a)=f(b)$,则$f$为常值函数,显然可积.下设$f(a)<f(b)$.对$\left[a,b\right]$的任一分割$T$,$f$在$T$所属的每个小区间$\Delta_i$上的振幅为
	$$\omega_i=f(x_i)-f(x_{i-1}),$$
	于是有
	\begin{align*}
		\sum_{T}\omega_i\Delta x_i
		\leqslant\sum_{i=1}^{n}\left[f(x_i)-f(x_{i-1})\right]\|T\|
		=\left[f(b)-f(a)\right]\|T\|.
	\end{align*}
	由此可见,任给$\varepsilon>0$,只要$\|T\|\leqslant\dfrac{\varepsilon}{f(b)-f(a)}$,就有
	$$\sum_{T}\omega_i\Delta x_i<\varepsilon,$$
	所以$f$在$\left[a,b\right]$上可积.
	$\hfill\blacksquare$
\end{proof}
\begin{theorem}
	若函数$f$在$\left[a,b\right]$上连续,则$f$在$\left[a,b\right]$上可积.
\end{theorem}
\begin{proof}
	$f$在闭区间$\left[a,b\right]$上连续,则它在$\left[a,b\right]$上一致连续.对$\forall\varepsilon>0,\ \exists\delta>0$,对任意$x_1,x_2\in\left[a,b\right]$,当$|x_1-x_2|<\delta$时,有
	$$|f(x_1)-f(x_2)|<\frac{\varepsilon}{b-a}.$$
	由于连续函数必有最大、最小值,不妨设$f(s_i)=M_i=\sup\limits_{\Delta_i}f(x),\ f(t_i)=\inf\limits_{\Delta_i}f(x),\ s_i,t_i\in\left[x_{i-1},x_i\right]$,则
	\begin{align*}
		\sum_{i=1}^{n}\omega_i\Delta x_i
		=\sum_{i=1}^{n}\left[f(s_i)-f(t_i)\right]\Delta x_i
		\leqslant\frac{\varepsilon}{b-a}\sum_{i=1}^{n}\Delta x_i=\varepsilon.
	\end{align*}
	所以$f$在$\left[a,b\right]$上可积.$\hfill\blacksquare$
\end{proof}
\begin{theorem}
	若函数$f$在区间$\left[a,b\right]$上有有限个间断点,则$f$在$\left[a,b\right]$上Riemann可积.
\end{theorem}
\begin{proof}
	不失一般性,我们只需证明$f$在$\left[a,b\right]$上仅有一个间断点的情形,并假设该间断点是端点$b$.
	
	对任意$\varepsilon>0$,取$\delta'$满足$0<\delta'<\frac{\varepsilon}{2(M-m)}$,且$\delta'<b-a$,其中$M$与$m$分别为$f$在$\left[a,b\right]$上的上确界和下确界.若$M=m$,则$f$是常值函数,显然可积.下设$M>m$,记$f$在小区间$\Delta'=\left[b-\delta',b\right]$上的振幅为$\omega'$,则
	$$\omega'\delta'<(M-m)\cdot\frac{\varepsilon}{2(M-m)}=\frac{vare	}{2}.$$
	因为$f$在$\left[a,b-\delta'\right]$上连续,因此Riemann可积.存在$\left[a,b-\delta'\right]$的某个分割$T'$,使得
	$$\sum_{T'}\omega_i\Delta x_i<\frac{\varepsilon}{2}.$$
	则$T=T'\cup\Delta'$是对$\left[a,b\right]$的一个分割,对于$T$,有
	$$\sum_{T}\omega_i\Delta x_i=\sum_{T'}\omega_i\Delta x_i+\omega'\delta'<\frac{\varepsilon}{2}+\frac{\varepsilon}{2}=\varepsilon.$$
	所以$f$在$\left[a,b\right]$上可积.$\hfill\blacksquare$
\end{proof}

由上述定理我们会想这样一个问题:间断点增加到什么时候函数就会不可积?这个临界状态在哪里?也就是说我们想要弄清函数的间断点的多少与函数是否可积的联系.

为了刻画集合中点的多少,我们引入以下概念.
\begin{definition}[零测度集]
	设集合$E\subseteq\mathbb{R}$,对$\forall\varepsilon>0$,集合$E$可以被至多可数个开区间集合$\{I_k\}$覆盖,且这些开区间的长度之和
	$$\sum_{k=1}^{\infty}|I_k|\leqslant\varepsilon,$$
	则称集合$E$具有{\heiti 零测度}或者称集合$E$是{\heiti 零测度集}(null set),简称{\heiti 零测集}.
\end{definition}
以下的结论是显然的.
\begin{proposition}
	空集是零测集.
\end{proposition}
\begin{proposition}
	零测集的子集也是零测集.
\end{proposition}
\begin{proposition}
	设集合$A$是至多可数的,则$A$是一个零测集.
\end{proposition}
\begin{proof}
	不失一般性,设$A$为可数集,设
	$$A=\{a_1,a_2,\cdots,a_n,\cdots\}.$$
	对于任意给定的$\varepsilon>0$,令
	$$I_n=(a_n-\frac{\varepsilon}{2^{n+1}},a_n+\frac{\varepsilon}{2^{n+1}}),\quad n=1,2,\cdots.$$
	显然$\{I_n\}$是$A$的一个开覆盖.由于
	$$\sum_{n=1}^{\infty}|I_n|=\sum_{n=1}^{\infty}2\cdot\frac{\varepsilon}{2^{n+1}}=\varepsilon\sum_{n=1}^{\infty}\frac{1}{2^n}=\varepsilon.$$
	则由零测集定义可知$A$是一个零测集.
\end{proof}
\begin{remark}
	需要注意,可数集一定是零测集,但不可数集也可能是零测集.后续我们在《实分析》中将进一步研究.
\end{remark}
\begin{proposition}
	长度不为零的区间都不是零测集.
\end{proposition}
\begin{proof}
	不妨设开区间$(a,b),\ (a<b)$,若$\{I_n\}$是$(a,b)$的一个开覆盖,则
	$\sum_{n=1}^{\infty}|I_n|\geqslant b-a>0$
	因此$(a,b)$不是一个零测集.$\hfill\blacksquare$
\end{proof}
\begin{proposition}
	至多可数个零测集的并集仍是零测集.
\end{proposition}
\begin{proof}
	设$E=\displaystyle\bigcup\limits_{n}E_n$是数目至多可数的零测集$E_n$的并集.对于每个$E_n$,我们根据$\varepsilon>0$构造集合$E_n$的覆盖$\{I_{n,k}\}$使$\displaystyle\sum\limits_{k}|I_{n,k}|<\dfrac{\varepsilon}{2}$.
	
	因为数目至多可数的可数集的并集也是至多可数集,所以开区间$I_{n,k}(n,k\in\mathbb{N}_+)$组成集合$E$的至多可数覆盖,并且
	$$\sum_{n,k}|I_{n,k}|<\frac{\varepsilon}{2}+\frac{\varepsilon}{2^2}+\cdots+\frac{\varepsilon}{2^n}+\cdots=\varepsilon.$$
	在$\sum_{n,k}|I_{n,k}|$中,由于级数收敛,所以求和是顺序是无关紧要的.上述级数的任何部分和以$\varepsilon$为上界,因而收敛.这就证得了$E$是零测集.$\hfill\blacksquare$
\end{proof}

用零测集的观点可以给出Riemann可积的充要条件.
\begin{theorem}[Lebesgue-Vitali定理]
	设函数$f$在区间$\left[a,b\right]$上有界,则$f$在$\left[a,b\right]$上Riemann可积当且仅当$f$在$\left[a,b\right]$上的不连续点组成的集合是一个零测集.
\end{theorem}
\begin{proof}
	对$\left[a,b\right]$作分割:
	$$T:a=x_0<x_1<\cdots<x_m=b.$$
	
	(i)证明必要性.由命题\ref{prop:jianduan}可知
	$$D(f)=\bigcup_{n=1}^{\infty}D_{1/n}.$$
	若证明了对于任一$\delta>0$,$D_\delta$都是零测集,则$D_1,D_{1/2},\cdots$都是零测集,也就证明了$D(f)$是一个零测集.
	
	设$f$在$\left[a,b\right]$上Riemann可积,由Riemann可积的充要条件可知,对于任一$\varepsilon>0$,都有
	$$\sum_{i=1}^{n}\omega_i\Delta x_i<\frac{\varepsilon\delta}{2}.$$
	设$x\in D_\delta$.若$x$不是$x_0,x_1,\cdots,x_m$中的任意一个,则存在$i\in\{1,2,\cdots,m\}$使得$x\in (x_{i-1},x_i)$,因此存在$r$使得$U(x;r)\subseteq(x_{i-1},x_i)$.设$f$在$(x_{i-1},x_i)$上的振幅为$\omega_i$,则
	$$\omega_i\geqslant\omega\left[U(x;r)\right]\geqslant\omega(x)\geqslant \delta.$$
	令
	$$\Lambda=\{i|D_\delta\cap(x_{i-1},x_i)\neq\varnothing,\ i=1,2,\cdots,m\}.$$
	于是
	$$\frac{\varepsilon\delta}{2}>\sum_{i=1}^{n}\omega_i\Delta x_i\geqslant\sum_{i\in\Lambda}\omega_i\Delta x_i\geqslant\delta\sum_{i\in\Lambda}\Delta x_i.\Rightarrow\sum_{i\in\Lambda}\Delta x_i<\frac{\varepsilon}{2}.$$
	由于
	$$D_\delta\subseteq\left[\bigcup_{i\in\Lambda}(x_{i-1},x_i)\right]\cup\{x_0,x_1,\cdots,x_m\}.$$
	因此
	$$D_\delta\subseteq\left[\bigcup_{i\in\Lambda}(x_{i-1},x_i)\right]\cup\left[\bigcup_{i=0}^m(x_i-\frac{\varepsilon}{4(m+1)},x_i+\frac{\varepsilon}{4(m+1)})\right].$$
	由于
	$$\sum_{i\in\Lambda}\Delta x_i+(m+1)\frac{2\varepsilon}{4(m+1)}<\frac{\varepsilon}{2}+\frac{\varepsilon}{2}=\varepsilon.$$
	这表明$D_\delta$是一个零测集.
	
	(ii)证明充分性.设$D(f)$是一个零测集,则对于任一$\varepsilon>0$都存在$D(f)$的开覆盖$\{(\alpha_i,\beta_i)|i=1,2,\cdots\}$满足
	$$\sum_{i=1}^{\infty}(\beta_i-\alpha_i)<\frac{\varepsilon}{2\omega}.$$
	其中$\omega$是$f$在$\left[a,b\right]$上的振幅.令
	$$K=\left[a,b\right]\bigg\backslash\bigcup_{i=1}^{\infty}(\alpha_i,\beta_i).$$
	由命题\ref{prop:biji}可知,对于前面给定的$\varepsilon$,存在$\delta>0$使得当$x\in K,\ y\in\left[a,b\right]$且$|x-y|<\delta$时
	$$|f(x)-f(y)|<\frac{\varepsilon}{4(b-a)}.$$
	取分割$T$满足$\|T\|<\delta$.令
	$$\Lambda_1=\{i|K\cap(x_{i-1},x_i)\neq\varnothing,\ i=1,2,\cdots,m\},\quad \Lambda_2=\{i|K\cap(x_{i-1},x_i)=\varnothing,\ i=1,2,\cdots,m\}.$$
	则
	$$\sum_{i=1}^{n}\omega_i\Delta x_i=\sum_{i\in\Lambda_1}\omega_i\Delta x_i+\sum_{i\in\Lambda_2}\omega_i\Delta x_i.$$
	先来看$\displaystyle\sum_{i\in\Lambda_1}\omega_i\Delta x_i$的情况.由于
	\begin{align*}
		\omega_i
		&=\sup\{|f(z_1)-f(z_2)|:z_1,z_2\in\left[x_{i-1},x_i\right]\}\\
		&\leqslant\sup\{|f(z_1)-f(y_i)|+|f(z_2)-f(y_i)|:z_1,z_2\in\left[x_{i-1},x_i\right],\ y_i\in K\cap(x_{i-1},x_i)\}\\
		&\leqslant\frac{\varepsilon}{2(b-a)}.
	\end{align*}
	因此
	$$\sum_{i\in\Lambda_1}\omega_i\Delta x_i<\frac{\varepsilon}{2(b-a)}(b-a)=\frac{\varepsilon}{2}.$$
	再看$\displaystyle\sum_{i\in\Lambda_2}\omega_i\Delta x_i$的情况.由于$\omega_i\leqslant\omega$,故
	$$\sum_{i\in\Lambda_2}\omega_i\Delta x_i\leqslant\sum_{i\in\Lambda_2}\Delta x_i.$$
	显然
	$$\bigcup_{i\in\Lambda_2}(x_{i-1},x_i)\subseteq\bigcup_{i=1}^{\infty}(\alpha_i,\beta_i).$$
	故进一步有
	$$\sum_{i\in\Lambda_2}\Delta x_i\leqslant\sum_{i=1}^{\infty}(\beta_i-\alpha_i)<\frac{\varepsilon}{2\omega}.$$
	于是
	$$\sum_{i\in\Lambda_2}\omega_i\Delta x_i<\omega\frac{\varepsilon}{2\omega}=\frac{\varepsilon}{2}.$$
	于是可知
	$$\sum_{i=1}^{n}\omega_i\Delta x_i=\sum_{i\in\Lambda_1}\omega_i\Delta x_i+\sum_{i\in\Lambda_2}\omega_i\Delta x_i<\frac{\varepsilon}{2}+\frac{\varepsilon}{2}=\varepsilon.$$
	由Riemann可积的充要条件可知$f$在$\left[a,b\right]$上Riemann可积.$\hfill\blacksquare$
\end{proof}
\begin{remark}
	1907年法国数学家Henri\ Lebesgue与意大利数学家Giuseppe\ Vitali同时独立证明了以上定理.
\end{remark}
\begin{remark}
	设零测集$E_0\subseteq E$,$P$是关于$E$中元素的命题,若对$\forall x\in E\backslash E_0$,命题$P$成立,那么我们说命题$P$在$E$上{\heiti 几乎处处}(almost everywhere)成立.以上定理就可以说成:有界函数$f$在$\left[a,b\right]$上Riemann可积的充要条件是$f$在$\left[a,b\right]$上几乎处处连续.
\end{remark}

在Lebesgue定理之下,可以立刻得到一系列关于可积性的结论.
\begin{corollary}
	设函数$f$在$\left[a,b\right]$上有界.若$f$在$\left[a,b\right]$上只有至多可数个间断点,则$f$在$\left[a,b\right]$上Riemann可积.
\end{corollary}
\begin{corollary}
	设函数$f$在$\left[a,b\right]$上Riemann可积,则$|f|$也在$\left[a,b\right]$上Riemann可积.
\end{corollary}
\begin{corollary}
	设函数$f$和$g$在$\left[a,b\right]$上Riemann可积,则$fg$也在$\left[a,b\right]$上Riemann可积.
\end{corollary}
\begin{corollary}
	设函数$f$在$\left[a,b\right]$上Riemann可积.若$1/f$在$\left[a,b\right]$上有定义且有界,则$1/f$也在$\left[a,b\right]$上Riemann可积.
\end{corollary}
\begin{corollary}
	设函数$f$在$\left[a,b\right]$上Riemann可积,若$\left[a_1,b_1\right]\subseteq\left[a,b\right]$,则$f$在$\left[a_1,b_1\right]$上Riemann可积.
\end{corollary}
\begin{corollary}
	设函数$f$在$\left[a,b\right]$和$\left[b,c\right]$上都Riemann可积,则$f(x)$在$\left[a,c\right]$上Riemann可积.
\end{corollary}
\begin{corollary}
	设函数$f$在$\left[a,b\right]$上Riemann可积.若函数$g$在$\left[a,b\right]$上除去有限个点$x_1,x_2,\cdots,x_n$之外与$f$相等,则$g$也在$\left[a,b\right]$上Riemann可积.且
	$$\int_{a}^{b}f(x)\d x=\int_{a}^{b}g(x)\d x.$$
\end{corollary}
\begin{proof}
	令$h=f-g$.则$h$除$x_1,x_2,\cdots,x_n$之外的函数值都等于零.因此$D(h)\subseteq\{x_1,x_2,\cdots,x_n\}$.因此$D(h)$是一个零测集.由Lebesgue定理可知$h$在$\left[a,b\right]$上Riemann可积.因此$g$也在$\left[a,b\right]$上Riemann可积.容易知道
	$$\int_{a}^{b}h(x)\d x=0.$$
	于是可知
	$$\int_{a}^{b}f(x)\d x=\int_{a}^{b}g(x)\d x.$$
	$\hfill\blacksquare$
\end{proof}
\begin{corollary}
	单调函数的不连续点集一定是零测集.
\end{corollary}

下面来考察Thomae函数的Riemann积分.
\begin{example}
	设函数$T$满足
	\begin{equation*}
		T(x)=\left\{
		\begin{aligned}
			&1,\quad x=0\\
			&\frac{1}{q},\quad x=\frac{p}{q}\\
			&0,\quad x\in\mathbb{R}\backslash\mathbb{Q}
		\end{aligned}
		\right.
	\end{equation*}
	其中$q>0,\ p,q\in\mathbb{Z}_+$且$(p,q)=1$.则$T$在任一有限区间$\left[a,b\right]$上都Riemann可积,且
	$$\int_{a}^{b}T(x)\d x=0.$$
\end{example}
\begin{proof}
	易知函数$T$在任一无理点都连续,在任一有理点都不连续,因此$D(T)=\mathbb{Q}$是一个零测集.由Lebesgue定理可知$T$在任一有限闭区间都Riemann可积.由Riemann可积的定义可知
	$$\int_{a}^{b}T(x)\d x=0.$$
	$\hfill\blacksquare$
\end{proof}
\begin{proposition}
	设函数$f$在$\left[a,b\right]$上连续,$g$在$\left[c,d\right]$上可积.若$g(\left[c,d\right])\subseteq\left[a,b\right]$,则$f\circ g$在$\left[c,d\right]$上Riemann可积.
\end{proposition}
\begin{proof}
	设$C=\left[c,d\right]\backslash D(g)$,则$f\circ g$在$C$上连续.因此$D(f\circ g)$也是一个零测集.由Lebesgue定理可知$f\circ g$在$\left[c,d\right]$上Riemann可积.$\hfill\blacksquare$
\end{proof}
\begin{remark}
	若把$f$在$\left[a,b\right]$上连续改成$f$在$\left[a,b\right]$上Riemann可积,则结论不成立.举例说明:设
	\begin{equation*}
		f(x)=\left\{
		\begin{aligned}
			&0,\quad x=0\\
			&1,\quad x\neq 0
		\end{aligned}
		\right.
		\qquad
		T(x)=\left\{
		\begin{aligned}
			&1,\quad x=0\\
			&\frac{1}{q},\quad x=\frac{p}{q}\\
			&0,\quad x\in\mathbb{R}\backslash\mathbb{Q}
		\end{aligned}
		\right.
	\end{equation*}
	显然$f$和$T$都在$\mathbb{R}$的任一有限闭区间上Riemann可积.而
	\begin{equation*}
		f(T(x))=\left\{
		\begin{aligned}
			&1,\quad x\in\mathbb{Q}\\
			&0,\quad x\in\mathbb{R}\backslash\mathbb{Q}
		\end{aligned}
		\right.
	\end{equation*}
	这是Dirichlet函数,它在$\mathbb{R}$的任一有限闭区间上都不Riemann可积.
\end{remark}

\section{积分基本定理}
本节主要介绍两个积分中值定理和微积分学基本定理.
\begin{theorem}[积分第一中值定理]
	若$f$在$\left[a,b\right]$上连续,则至少存在一点$\xi \in\left[a,b\right]$使得
	$$\int_{a}^{b}f(x)\d x=f(\xi)(b-a).$$
\end{theorem}
\begin{proof}
	由于$f$在$\left[a,b\right]$上连续,因此存在最大值$M$和最小值$m$.由
	$$m\leqslant f(x)\leqslant M,\quad x\in\left[a,b\right],$$
	使用积分不等式性质得
	$$m(b-a)\leqslant\int_{a}^{b}f(x)\d x\leqslant M(b-a),$$
	即
	$$m\leqslant\frac{1}{b-a}\int_{a}^{b}f(x)\d x\leqslant M.$$
	由连续函数的介值定理可知,至少存在一点$\xi\in\left[a,b\right]$,使得
	$$f(\xi)=\frac{1}{b-a}\int_{a}^{b}f(x)\d x,$$
	即
	$$\int_{a}^{b}f(x)\d x=f(\xi)(b-a).$$
	$\hfill\blacksquare$
\end{proof}
\begin{remark}
	$\dfrac{1}{b-a}\displaystyle\int_{a}^{b}f(x)\d x$可以理解为$f(x)$在区间$\left[a,b\right]$上所有函数值的平均值.这是通常有限个数的算术平均值的推广.
\end{remark}
\begin{theorem}[推广的积分第一中值定理]
	若$f$和$g$都在$\left[a,b\right]$上连续,且$g(x)$在$\left[a,b\right]$上不变号,则至少存在一点$\xi\in\left[a,b\right]$,使得
	$$\int_{a}^{b}f(x)g(x)\d x=f(\xi)\int_{a}^{b}g(x)\d x.$$
\end{theorem}
\begin{proof}
	不妨设$g(x)\geqslant 0,\ x\in\left[a,b\right]$.这时有
	$$mg(x)\leqslant f(x)g(x)\leqslant Mg(x),\ x\in\left[a,b\right],$$
	其中$M,m$分别为$f$在$\left[a,b\right]$上的最大、最小值.由定积分的不等式性质,得到
	$$m\int_{a}^{b}g(x)\d x\leqslant\int_{a}^{b}f(x)g(x)\d x\leqslant M\int_{a}^{b}g(x)\d x.$$
	若$\displaystyle\int_{a}^{b}g(x)\d x=0$,则$\displaystyle\int_{a}^{b}f(x)g(x)\d x=0$,从而对任何$\xi\in\left[a,b\right]$,结论都成立.\\
	若$\displaystyle\int_{a}^{b}g(x)\d x>0$,则
	$$m\leqslant\frac{\displaystyle\int_{a}^{b}f(x)g(x)\d x}{\displaystyle\int_{a}^{b}g(x)\d x}\leqslant M.$$
	由连续函数的介值性,必至少存在一点$\xi\in\left[a,b\right]$,使得
	$$f(\xi)=\frac{\displaystyle\int_{a}^{b}f(x)g(x)\d x}{\displaystyle\int_{a}^{b}g(x)\d x}.$$
	$\hfill\blacksquare$
\end{proof}
\begin{definition}[变限积分]
	设$f$在$\left[a,b\right]$上可积,则对任何$x\in\left[a,b\right]$,$f$在$\left[a,x\right]$上也可积.设
	$$\varPhi(x)=\int_{a}^{x}f(t)\d t,\quad x\in\left[a,b\right].$$
	它是一个以积分上限$x$为自变量的函数,称为变上限的定积分.类似地,我们可定义变下限的定积分:
	$$\Psi(x)=\int_{x}^{b}f(t)\d t,\quad x\in\left[a,b\right].$$
	$\varPhi$和$\Psi$统称为{\heiti 变限积分}.
\end{definition}
由于
$$\int_{x}^{b}f(t)\d t=-\int_{a}^{x}f(t)\d t,$$
因此下面只讨论变上限积分的情形.
\begin{theorem}
	若$f$在$\left[a,b\right]$上可积,则上面的$\varPhi(x)$在$\left[a,b\right]$上连续.
\end{theorem}
\begin{proof}
	对$\left[a,b\right]$上任一确定的点$x$,只要$x+\Delta x\in\left[a,b\right]$,则
	$$\Delta \varPhi=\int_{a}^{x+\Delta x}f(t)\d t-\int_{a}^{x}f(t)\d t=\int_{x}^{x+\Delta x}f(t)\d t.$$
	因$f$在$\left[a,b\right]$上有界,设$|f(t)|\leqslant M,\ t\in\left[a,b\right]$.当$\Delta x>0$时,有
	$$|\Delta\varPhi|=\big|\int_{x}^{x+\Delta x}f(t)\d t\big|\leqslant\int_{x}^{x+\Delta x}|f(t)|\d t\leqslant M\Delta x.$$
	当$\Delta x<0$时,有$|\Delta \varPhi|\leqslant M|\Delta x|$.由此得到
	$$\lim\limits_{\Delta x\to 0}\Delta \varPhi=0.$$
	即$\varPhi$在点$x$处连续,由$x$的任意性可知$\varPhi$在$\left[a,b\right]$上处处连续.$\hfill\blacksquare$
\end{proof}
\begin{theorem}[微积分学基本定理]
	若$f$在$\left[a,b\right]$上连续,则$\varPhi(x)$在$\left[a,b\right]$上处处可导,且
	$$\varPhi'(x)=\frac{\d}{\d x}\int_{a}^{x}f(t)\d t=f(x),\quad x\in\left[a,b\right].$$
\end{theorem}
\begin{proof}
	对$\left[a,b\right]$上任一确定的$x$,当$\Delta x\neq 0$且$x+\Delta x\in\left[a,b\right]$时,由积分第一中值定理,有
	$$\frac{\Delta\varPhi}{\Delta x}=\frac{1}{\Delta x}=\frac{1}{\Delta x}\int_{x}^{x+\Delta x}f(t)\d t=f(x+\theta x),\ 0\leqslant\theta\leqslant 1.$$
	由于$f$在点$x$连续,故有
	$$\varPhi'(x)=\lim\limits_{\Delta x\to 0}\frac{\Delta \varPhi}{\Delta x}=\lim\limits_{\Delta x\to 0}f(x+\theta\Delta x)=f(x).$$
	由$x$在$\left[a,b\right]$上的任意性,证得$\varPhi$是$f$在$\left[a,b\right]$上的一个原函数.$\hfill\blacksquare$
\end{proof}
\begin{remark}
	本定理沟通了导数和定积分这两个从表面看似不相干的概念之间的内在联系;同时也证明了“连续函数必有原函数”这一基本结论,并以积分形式给出了$f$的一个原函数.可见该定理的重要意义.
\end{remark}

\begin{theorem}[积分第二中值定理]
	设函数$f$在$\left[a,b\right]$上可积.
	\begin{enumerate}
		\item 若函数$g$在$\left[a,b\right]$上减,且$g(x)\geqslant 0$,则存在$\xi\in\left[a,b\right]$,使得
		$$\int_{a}^{b}f(x)g(x)\d x=g(a)\int_{a}^{\xi}f(x)\d x.$$
		\item 若函数$g$在$\left[a,b\right]$上增,且$g(x)\leqslant 0$,则存在$\eta\in\left[a,b\right]$,使得
		$$\int_{a}^{b}f(x)g(x)\d x=g(b)\int_{\eta}^{b}f(x)\d x.$$
	\end{enumerate}
\end{theorem}
\begin{proof}
	只需证1,类似地可证2.设
	$$F(x)=\int_{a}^{x}f(t)\d t,\ x\in\left[a,b\right].$$
	由于$f$在$\left[a,b\right]$上可积,因此$f$有界,设$|f(x)|\leqslant L$.且有$F(x)$在$\left[a,b\right]$上连续,从而存在最大值$M$和最小值$m$.由于可积函数是连续的,所以下面只需证明
	$$m\leqslant\frac{1}{g(a)}\int_{a}^{b}f(x)g(x)\d x\leqslant M.$$
	即证
	$$mg(a)\leqslant\int_{a}^{b}f(x)g(x)\d x\leqslant Mg(a).$$
	
	由于$g(x)$是单调的,因此必可积,对任意$\varepsilon>0$,存在分割$T:a=x_0<x_1<\cdots<x_n=b$,使
	$$\sum_{i=1}^{n}\omega_i\Delta x_i<\frac{\varepsilon}{L}.$$
	\begin{align*}
		\int_{a}^{b}f(x)g(x)\d x
		&=\sum_{i=1}^{n}\int_{x_{i-1}}^{x_i}f(x)g(x)\d x\\
		&=\sum_{i=1}^{n}\int_{x_{i-1}}^{x_i}\left[g(x)-g(x_{i-1})\right]f(x)\d x+\sum_{i=1}^{n}g(x_{i-1})\int_{x_{i-1}}^{x_i}f(x)\d x\\
		&\leqslant\sum_{i=1}^{n}\int_{x_{i-1}}^{x_i}\left|g(x)-g(x_{i-1})\right|\cdot|f(x)|\d x+\sum_{i=1}^{n}g(x_{i-1})\left[F(x_i)-F(x_{i-1})\right]\\
		&\leqslant L\cdot\sum_{i=1}^{n}\omega_i\Delta x_i+\sum_{i=1}^{n-1}F(x_i)\left[g(x_{i-1})-g(x_i)\right]+F(b)g(x_{n-1})\\
		&<L\cdot\frac{\varepsilon}{L}+M\sum_{i=1}^{n-1}\left[g(x_{i-1})-g(x_i)\right]+Mg(x_{n-1})\\
		&=\varepsilon+Mg(a).
	\end{align*}
	同理,有
	$$\int_{a}^{b}f(x)g(x)\d x>-\varepsilon+mg(a).$$
	结合两式,
	$$-\varepsilon+mg(a)<\int_{a}^{b}f(x)g(x)\d x<Mg(a)+\varepsilon.$$
	令$\varepsilon\to 0$,得
	$$mg(a)\leqslant\int_{a}^{b}f(x)g(x)\d x\leqslant Mg(a).$$
	$\hfill\blacksquare$
\end{proof}
\begin{corollary}
	设函数$f$在$\left[a,b\right]$上可积.若$g$为单调函数,则存在$\xi\in\left[a,b\right]$,使得
	$$\int_{a}^{b}f(x)g(x)\d x=g(a)\int_{a}^{\xi}f(x)\d x+g(b)\int_{\xi}^{b}f(x)\d x.$$
\end{corollary}
\begin{proof}
	只需证$g$为递减的情况,递增的情况可类似证明.设$h(x)=g(x)-g(b)$,则$h$为非负、递减函数.由积分第二中值定理,存在$\xi\in\left[a,b\right]$,使得
	$$\int_{a}^{b}f(x)h(x)\d x=h(a)\int_{a}^{\xi}f(x)\d x.$$
	又
	$$\int_{a}^{b}f(x)h(x)\d x=\int_{a}^{b}f(x)g(x)\d x-g(b)\int_{a}^{b}f(x)\d x.$$
	化简即得结论.$\hfill\blacksquare$
\end{proof}
\begin{remark}
	积分第二中值定理及其推论是今后建立反常积分收敛判别法的工具.
\end{remark}
\section{Riemann积分的计算}
\begin{theorem}[Newton-Leibnitz公式]
	若函数$f$在$\left[a,b\right]$上Riemann可积,且存在原函数$F$,若$F$在$\left[a,b\right]$上连续,则
	$$\int_{a}^{b}f(x)\d x=F(b)=F(a).$$
	上式称为{\heiti Newton-Leibnitz公式}.
\end{theorem}
\begin{proof}
	把$\left[a,b\right]$作$n$等分
	$$a=x_0<x_1<\cdots<x_n=b.$$
	由Lagrange中值定理可知,存在$\xi_i\in(x_{i-1},x_i)(i=1,2,\cdots,n)$满足
	$$F(b)-F(a)=\sum_{i=1}^{n}\left[F(x_i)-F(x_{i-1})\right]=\sum_{i=1}^{n}F'(\xi)(x_i-x_{i-1})=\sum_{i=1}^{n}f(\xi_i)\Delta x_i.$$
	由于$f(x)$在$\left[a,b\right]$上Riemann可积,因此令上式的$n\to\infty$得
	$$F(b)-F(a)=\lim\limits_{n\to\infty}\sum_{i=1}^{n}f(\xi_i)\Delta x_i=\int_{a}^{b}f(x)\d x.$$
	$\hfill\blacksquare$
\end{proof}
\begin{remark}
	用微积分学基本定理也可以很容易地证明Newton-Leibnitz公式,在此不再赘述.
\end{remark}

对原函数的存在性有了正确的认识,就能顺利地把不定积分的换元积分法和分部积分法移植到定积分计算中来.
\begin{theorem}[换元积分法]
	若函数$f$在$\left[a,b\right]$上连续,$\varphi'$在$\left[\alpha,\beta\right]$上可积,且满足
	$$\varphi(\alpha)=a,\ \varphi(\beta)=b,\ \varphi(\left[\alpha,\beta\right])\subseteq\left[a,b\right],$$
	则有定积分换元公式:
	$$\int_{a}^{b}f(x)\d x=\int_{\alpha}^{\beta}f(\varphi(t))\varphi'(t)\d t.$$
\end{theorem}
\begin{proof}
	由于$f$在$\left[a,b\right]$上连续,因此其原函数存在.设$F$为$f$在$\left[a,b\right]$上的一个原函数,则
	$$\frac{\d}{\d t}\bigg(F(\varphi(t))\bigg)=f(\varphi(t))\varphi'(t).$$
	又$\varphi'(t)$在$\left[\alpha,\beta\right]$上可积,由Newton-Leibnitz公式,有
	$$\int_{\alpha}^{\beta}f(\varphi(t))\varphi'(t)\d t=F(\varphi(\beta))-F(\varphi(\alpha))=F(b)-F(a)=\int_{a}^{b}f(x)\d x.$$
	$\hfill\blacksquare$
\end{proof}
\begin{theorem}[分部积分法]
	若$u(x),v(x)$为$\left[a,b\right]$上的可微函数,且$u'(x)$和$v'(x)$都在$\left[a,b\right]$上可积,则有定积分分部积分公式:
	$$\int_{a}^{b}u(x)v'(x)\d x=u(x)v(x)\bigg|_{a}^{b}-\int_{a}^{b}u'(x)v(x)\d x.$$
\end{theorem}
\begin{proof}
	$$\int_{a}^{b}u(x)v'(x)\d x+\int_{a}^{b}u'(x)v(x)\d x=\int_{a}^{b}\left[u(x)v'(x)+v(x)u'(x)\right]\d x=u(x)v(x)\bigg|_{a}^{b}.$$
	$\hfill\blacksquare$
\end{proof}
\begin{remark}
	为方便,分部积分公式也可写成
	$$\int_{a}^{b}u(x)\d v(x)=u(x)v(x)\bigg|_{a}^{b}-\int_{a}^{b}v(x)\d u(x).$$
\end{remark}
\section{Taylor公式的积分型余项}
由前面的学习中我们知道,设$R_n(x)$是Taylor公式的余项,则$f(x)$在$x_0$处的Taylor公式为
$$f(x)=\sum_{k=0}^{n}\frac{f^{(k)}(x)}{k!}(x-x_0)^k+R_n(x).$$
我们已经从定性和定量的角度分别介绍了Peano型余项和Lagrange型余项,下面我们从积分的角度给出积分型余项和Cauchy型余项.

先给出一个引理,这是求积分型余项的基础.
\begin{lemma}[推广的分部积分公式]
	设函数$u(x),v(x)$在$\left[a,b\right]$上有$n+1$阶连续导函数,则
	\begin{align*}
		\int_{a}^{b}u(x)v^{(n+1)}(x)\d x=
		&\left[u(x)v^{(n)}(x)-u'(x)v^{(n-1)}(x)+\cdots+(-1)^nu^{(n)}(x)v(x)\right]_{a}^{b}+\\
		&(-1)^{n+1}\int_{a}^{b}u^{(n+1)}(x)v(x)\d x\quad(n=1,2,\cdots).
	\end{align*}
\end{lemma}

该引理可由数学归纳法证明.

设函数$f$在$x_0$的某邻域$U(x_0)$上有$n+1$阶连续导函数.对$x\in U(x_0),\ t\in\left[x_0,x\right](\text{或}\left[x,x_0\right])$,利用推广的分部积分公式,有
\begin{align*}
	&\int_{x_0}^{x}(x-t)^nf^{(n+1)}(t)\d t\\
	&=\left[(x-t)^nf^{(n)}(t)+n(x-t)^{(n-1)}(t)+\cdots+n!f(t)\right]_{x_0}^{x}+\int_{x_0}^{x}0\cdot f(t)\d t\\
	&=n!f(x)-n!\left[f(x_0)+f'(x_0)(x-x_0)+\cdots+\frac{f^{(n)}(x_0)}{n!}(x-x_0)^n\right]\\
	&=n!R_n(x).
\end{align*}
其中$R_n(x)$即为Taylor公式的{\heiti 积分型余项}.求得
$$R_n(x)=\frac{1}{n!}\int_{x_0}^{x}f^{(n+1)}(t)(x-t)^n\d t.$$

由于$f^{(n+1)}(t)$连续,$(x-t)^n$在$\left[x_0,x\right](\text{或}\left[x,x_0\right])$上保持同号,因此由推广的积分第一中值定理,可将积分型余项写作
$$R_n(x)=\frac{1}{n!}f^{(n+1)}(\xi)\int_{x_0}^{x}(x-t)^n\d t=\frac{1}{(n+1)!}f^{(n+1)}(\xi)(x-x_0)^{n+1},\ \xi=x_0+\theta(x-x_0),\ 0\leqslant\theta\leqslant 1.$$
这就是以前所熟悉的Lagrange型余项.

\hspace*{\fill}

如果直接对积分型余项应用积分第一中值定理,则
\begin{align*}
	R_n(x)
	&=\frac{1}{n!}f^{(x+1)}(\xi)(x-\xi)^n(x-x_0),\quad \xi=x_0+\theta(x-x_0),\ 0\leqslant\theta\leqslant 1.\\
	&=\frac{1}{n!}f^{(x+1)}(\xi)\left[x-x_0-\theta(x-x_0)^n\right](x-x_0)\\
	&=\frac{1}{n!}f^{(x+1)}(x_0+\theta(x-x_0))(1-\theta)^n(x-x_0)^{n+1}
\end{align*}
特别当$x_0=0$时,有
$$R_n(x)=\frac{1}{n!}f^{(n+1)}(\theta x)(1-\theta)^nx^{n+1},\quad 0\leqslant\theta\leqslant 1.$$
我们将上述两式称为Taylor公式的{\heiti Cauchy型余项}.各种形式的Taylor公式余项将在幂级数中显示它们的功用.
\newpage

\chapter{反常积分}
在讨论Riemann积分时有两个最基本的限制:积分区间的有穷性和被积函数的有界性.但在很多实际问题中要突破这些限制,考虑无穷区间上的积分或无界函数的积分,这便是本章的主题.
\section{反常积分概念}
第一类,我们考虑无穷区间上的积分.
\begin{definition}[无穷积分]
	设函数$f$定义在无穷区间$\left[a,+\infty\right)$上,且在任何有限区间$\left[a,u\right]$上可积.如果存在极限
	$$\lim\limits_{u\to +\infty}\int_{a}^{u}f(x)\d x=J,$$
	则称此极限$J$为函数$f$在$\left[a,+\infty\right)$上的{\heiti 无穷限反常积分}(简称{\heiti 无穷积分}),记作
	$$J=\int_{a}^{+\infty}f(x)\d x,$$
	并称$\displaystyle\int_{a}^{+\infty}f(x)\d x${\heiti 收敛}.如果极限不存在,则称$\displaystyle\int_{a}^{+\infty}f(x)\d x${\heiti 发散}.
\end{definition}
类似地,可定义$f$在$\left(-\infty,b\right]$上的无穷积分:
$$\int_{-\infty}^{b}f(x)\d x=\lim\limits_{u\to -\infty}\int_{u}^{b}f(x)\d x.$$

对于$f$在$(-\infty,+\infty)$上的无穷积分,用前面两种积分来定义:
$$\int_{-\infty}^{+\infty}f(x)\d x=\int_{-\infty}^{a}f(x)\d x+\int_{a}^{+\infty}f(x)\d x,\quad a\in\mathbb{R}.$$
当且仅当等号右边两个无穷积分都收敛时它才是收敛的.

可以根据函数极限的性质与定积分的性质,导出无穷积分的一些相应性质.
\begin{proposition}
	对于任意$i=1,2,\cdots,n$,若$\displaystyle\int_{a}^{+\infty}f_i(x)\d x$都收敛,$k_i$是任意常数,则$\displaystyle\int_{a}^{+\infty}\displaystyle\sum_{i=1}^{n}k_if_i(x)\d x$也收敛,且
	$$\int_{a}^{+\infty}\sum_{i=1}^{n}k_if_i(x)\d x=\sum_{i=1}^{n}k_i\int_{a}^{+\infty}f_i(x)\d x.$$
\end{proposition}
\begin{proposition}
	若$f$在任何有限区间$\left[a,u\right]$上可积,$a<b$,则$\displaystyle\int_{a}^{+\infty}f(x)\d x$与$\displaystyle\int_{b}^{+\infty}f(x)\d x$同敛态,且有
	$$\int_{a}^{+\infty}f(x)\d x=\int_{a}^{b}f(x)\d x+\int_{b}^{+\infty}f(x)\d x.$$
	其中右边第一项为Riemann积分.
\end{proposition}
第二类,我们考虑无界函数的积分.
\begin{definition}[瑕积分]
	设函数$f$定义在区间$\left(a,b\right]$上,在点$a$的任一右邻域上无界,但在任何内闭区间$\left[u,b\right]\subset\left(a,b\right]$上有界且可积.如果存在极限
	$$\lim\limits_{u\to a^+}\int_{u}^{b}f(x)\d x=J,$$
	则称此极限为无界函数$f$在$\left(a,b\right]$上的反常积分,记作
	$$J=\int_{a}^{b}f(x)\d x,$$
	其中,$f$在$a$近旁是无界的,这时我们称$a$为$f$的{\heiti 瑕点},无界函数的反常积分又称为{\heiti 瑕积分}.
\end{definition}
类似地,可定义瑕点为$b$时的瑕积分:
$$\int_{a}^{b}f(x)\d x=\lim\limits_{u\to b^-}\int_{a}^{u}f(x)\d x.$$

若$f$的瑕点$c\in(a,b)$,则定义瑕积分
$$\int_{a}^{b}f(x)\d x=\int_{a}^{c}f(x)\d x+\int_{c}^{b}f(x)\d x=\lim\limits_{u\to c^-}f(x)\d x+\lim\limits_{u\to c^+}f(x)\d x$$
当且仅当等号右边两个瑕积分都收敛时它才是收敛的.

若$a,b$两点都是$f$的瑕点,而$f$在任何$\left[u,v\right]\in\left(a,b\right)$上可积,这时定义瑕积分
$$\int_{a}^{b}f(x)\d x=\int_{a}^{c}f(x)\d x+\int_{c}^{b}f(x)\d x=\lim\limits_{u\to a^+}\int_{u}^{c}f(x)\d x+\lim\limits_{v\to b^-}\int_{c}^{v}f(x)\d x.$$
其中$c$为$(a,b)$上任一实数.当且仅当等号右边两个瑕积分都收敛时它才是收敛的.

同样可以根据函数极限的性质与定积分的性质,导出瑕积分的一些相应性质.
\begin{proposition}
	对于任意$i=1,2,\cdots,n$,设函数$f_i$的瑕点都是$x=a$,$k_i$为常数,则当瑕积分$\displaystyle\int_{a}^{b}f_i(x)\d x$都收敛时,$\displaystyle\int_{a}^{b}\displaystyle\sum_{i=1}^{n}k_if_i(x)\d x$必定收敛,且
	$$\int_{a}^{b}\sum_{i=1}^{n}k_if_i(x)\d x=\sum_{i=1}^{n}k_i\int_{a}^{b}f_i(x)\d x.$$
\end{proposition}
\begin{proposition}
	设函数$f$的瑕点为$x=a$,$c\in(a,b)$为任一常数.则瑕积分$\displaystyle\int_{a}^{b}f(x)\d x$与$\int_{a}^{c}f(x)\d x$同敛态,并有
	$$\int_{a}^{b}f(x)\d x=\int_{a}^{c}f(x)\d x+\int_{c}^{b}f(x)\d x.$$
	其中等号右边第二项为Riemann积分.
\end{proposition}
\section{无穷积分的敛散判别}
\subsection{无穷积分收敛的Cauchy准则}
由定义可知,无穷积分$\displaystyle\int_{a}^{+\infty}f(x)\d x$收敛与否,取决于函数$F(u)=\displaystyle\int_{a}^{u}f(x)\d x$在$u\to +\infty$时是否存在极限.因此可由函数极限的Cauchy准则导出无穷积分收敛的Cauchy准则.
\begin{theorem}[无穷积分收敛的Cauchy准则]
	无穷积分收敛的充要条件是:任给$\varepsilon>0$,存在$G\geqslant a$,只要$u_1,u_2>G$,便有
	$$\bigg|\int_{a}^{u_2}f(x)\d x-\int_{a}^{u_1}f(x)\d x\bigg|=\bigg|\int_{u_1}^{u_2}f(x)\d x\bigg|<\varepsilon.$$
\end{theorem}
\begin{proposition}
	若$f$在任何有限区间$\left[a,u\right]$上可积,且有$\displaystyle\int_{a}^{+\infty}|f(x)|\d x$收敛,则$\displaystyle\int_{a}^{+\infty}f(x)\d x$亦必收敛,并有
	$$\bigg|\int_{a}^{+\infty}f(x)\d x\bigg|\leqslant\int_{a}^{+\infty}|f(x)|\d x.$$
\end{proposition}
\begin{proof}
	由$\displaystyle\int_{a}^{+\infty}|f(x)|\d x$收敛,根据Cauchy准则(必要性),任给$\varepsilon>0$,存在$G\geqslant a$,当$u_2>u_1>G$时,总有
	$$\bigg|\int_{u_1}^{u_2}|f(x)|\d x\bigg|=\int_{u_1}^{u_2}|f(x)|\d x<\varepsilon.$$
	利用定积分的绝对值不等式,又有
	$$\bigg|\int_{u_1}^{u_2}f(x)\d x\bigg|\leqslant\int_{u_1}^{u_2}|f(x)|\d x<\varepsilon.$$
	再由Cauchy准则(充分性),证得$\displaystyle\int_{a}^{+\infty}f(x)\d x$收敛.
	
	又因为$\bigg|\displaystyle\int_{a}^{u}f(x)\d x\bigg|\leqslant\int_{a}^{u}|f(x)|\d x\ (u>a)$,令$u\to +\infty$,得
	$$\bigg|\int_{a}^{+\infty}f(x)\d x\bigg|\leqslant\int_{a}^{+\infty}|f(x)|\d x.$$
	$\hfill\blacksquare$
\end{proof}
\begin{remark}
	当$\displaystyle\int_{a}^{+\infty}|f(x)|\d x$收敛时,称$\displaystyle\int_{a}^{+\infty}f(x)\d x$为{\heiti 绝对收敛}.由上述命题我们知道,绝对收敛的无穷积分,其自身也一定收敛.但是,收敛的无穷积分不一定绝对收敛.我们称收敛而不绝对收敛者为{\heiti 条件收敛}.
\end{remark}
\subsection{非负函数无穷积分敛散判别}
\begin{theorem}[比较原则]
	设定义在$\left[a,+\infty\right)$上的两个非负函数$f$和$g$都在任何有限区间$\left[a,u\right]$上可积,且满足
	$$f(x)\leqslant g(x),\ x\in \left[a,+\infty\right),$$
	则

		(i)当$\displaystyle\int_{a}^{+\infty}g(x)\d x$收敛时,$\displaystyle\int_{a}^{+\infty}f(x)\d x$必收敛;
		
		(ii)当$\displaystyle\int_{a}^{+\infty}f(x)\d x$发散时,$\displaystyle\int_{a}^{+\infty}g(x)\d x$必发散.

	
\end{theorem}
\begin{proof}
	只需证明(i).由于$\displaystyle\int_{a}^{u}g(x)\d x$关于上界$u$是单调递增的,因此$\displaystyle\int_{a}^{+\infty}g(x)\d x$收敛的充要条件是$\displaystyle\int_{a}^{u}g(x)\d x$在$\left[a,+\infty\right)$上存在上界$M$.
	
	由$f(x)\leqslant g(x)$且$f,g$非负,有
	$$\int_{a}^{u}f(x)\d x\leqslant\int_{a}^{u}g(x)\d x\leqslant M.$$
	因此$\displaystyle\int_{a}^{+\infty}f(x)\d x$收敛.$\hfill\blacksquare$
\end{proof}
\begin{corollary}[比较原则的极限形式]
	若$f$和$g$都在任何有限区间$\left[a,u\right]$上可积,当$x\in\left[a,+\infty\right)$时,$f(x)\geqslant 0,\ g(x)>0$,且$\lim\limits_{x\to +\infty}\dfrac{f(x)}{g(x)}=c$,则有:
	
	(i)当$0<c<+\infty$时,$\displaystyle\int_{a}^{+\infty}f(x)\d x$与$\displaystyle\int_{a}^{+\infty}g(x)\d x$同敛态;
	
	(ii)当$c=0$时,由$\displaystyle\int_{a}^{+\infty}g(x)\d x$收敛可推知$\displaystyle\int_{a}^{+\infty}f(x)\d x$也收敛;
	
	(iii)当$c=+\infty$时,由$\displaystyle\int_{a}^{+\infty}g(x)\d x$发散可推知$\displaystyle\int_{a}^{+\infty}f(x)\d x$也发散.
\end{corollary}
特别地,如果选用$\displaystyle\int_{a}^{+\infty}\frac{\d x}{x^p}$作为比较对象,则有下面的{\heiti Cauchy判别法}及其极限形式.
\begin{corollary}[Cauchy判别法]
	设$f$定义于$\left[a,+\infty\right)\ (a>0)$,且在任何有限区间$\left[a,u\right]$上可积,则有:
	
	(i)当$0\leqslant f(x)\leqslant\dfrac{1}{x^p},\ x\in\left[a,+\infty\right)$,且$p>1$时,$\displaystyle\int_{a}^{+\infty}f(x)\d x$收敛;
	
	(ii)当$f(x)\geqslant\dfrac{1}{x^p},\ x\in\left[a,+\infty\right)$,且$p\leqslant 1$时,$\displaystyle\int_{a}^{+\infty}f(x)\d x$发散.
\end{corollary}
\begin{corollary}[Cauchy判别法的极限形式]
	设$f$是定义于$\left[a,+\infty\right)$上的非负函数,在任何有限区间$\left[a,u\right]$上可积,且
	$$\lim\limits_{x\to +\infty}x^pf(x)=\lambda.$$
	则有
	
	(i)当$p>1,\ 0\leqslant\lambda<+\infty$时,$\displaystyle\int_{a}^{+\infty}f(x)\d x$收敛;
	
	(ii)当$p\leqslant 1,\ 0<\lambda\leqslant +\infty$时,$\displaystyle\int_{a}^{+\infty}f(x)\d x$发散.
\end{corollary}
\subsection{一般无穷积分的敛散判别法}
\begin{theorem}[Dirichlet判别法]
	若$F(u)=\displaystyle\int_{a}^{u}f(x)\d x$在$\left[a,+\infty\right)$上有界,$g(x)$在$\left[a,+\infty\right)$上当$x\to +\infty$时单调趋于$0$,则$\displaystyle\int_{a}^{+\infty}f(x)g(x)\d x$收敛.
\end{theorem}
\begin{proof}
	设$\bigg|\displaystyle\int_{a}^{u}f(x)\d x\bigg|\leqslant M,\ u\in\left[a,+\infty\right)$.任给$\varepsilon>0$,由于$\lim\limits_{x\to +\infty}g(x)=0$,因此存在$G\geqslant a$,当$x>G$时,有
	$$|g(x)|<\frac{\varepsilon}{4M}.$$
	又因为$g$是单调函数,由积分第二中值定理的推论,对于任何$u_2>u_1>G$,存在$\xi\in\left[u_1,u_2\right]$,使得
	$$\int_{u_1}^{u_2}f(x)g(x)\d x=g(u_1)\int_{u_1}^{\xi}f(x)\d x+g(u_2)\int_{\xi}^{u_2}f(x)\d x.$$
	于是有
	\begin{align*}
		\bigg|\int_{u_1}^{u_2}f(x)g(x)\d x\bigg|
		&\leqslant|g(u_1)|\cdot\bigg|\int_{u_1}^{\xi}f(x)\d x\bigg|+|g(u_2)|\cdot\bigg|\int_{\xi}^{u_2}f(x)\d x\bigg|\\
		&=|g(u_1)|\cdot\bigg|\int_{a}^{\xi}f(x)\d x-\int_{u_1}^{a}f(x)\d x\bigg|+|g(u_2)|\cdot\bigg|\int_{a}^{u_2}f(x)\d x-\int_{a}^{\xi}f(x)\d x\bigg|\\
		&<\frac{\varepsilon}{4M}\cdot 2M+\frac{\varepsilon}{4M}\cdot 2M=\varepsilon.
	\end{align*}
	根据Cauchy准则,证得$\displaystyle\int_{a}^{+\infty}f(x)g(x)\d x$收敛.
	$\hfill\blacksquare$
\end{proof}
\begin{theorem}[Abel判别法]
	若$\displaystyle\int_{a}^{+\infty}f(x)\d x$收敛,$g(x)$在$\left[a,+\infty\right)$上单调有界,则$\displaystyle\int_{a}^{+\infty}f(x)g(x)\d x$收敛.
\end{theorem}
\begin{proof}
	由于$\displaystyle\int_{a}^{+\infty}f(x)\d x$收敛,即$\lim\limits_{u\to +\infty}\displaystyle\int_{a}^{u}f(x)\d x$存在,则$\displaystyle\int_{a}^{u}f(x)\d x$在$\left[0,+\infty\right)$上有界,又$g(x)$在$\left[a,+\infty\right)$上单调有界,则必有极限.设$\lim\limits_{x\to +\infty}g(x)=a$,即有
	$$\lim\limits_{x\to +\infty}\left[g(x)-a\right]=0.$$
	由Dirichlet判别法知,$\displaystyle\int_{a}^{+\infty}f(x)\left[g(x)-a\right]\d x$收敛,即
	$$\displaystyle\int_{a}^{+\infty}\left[f(x)g(x)-af(x)\right]\d x=\displaystyle\int_{a}^{+\infty}f(x)g(x)\d x-a\displaystyle\int_{a}^{+\infty}f(x)\d x.$$
	由于$\displaystyle\int_{a}^{+\infty}f(x)\d x$收敛,故$\displaystyle\int_{a}^{+\infty}f(x)g(x)\d x$收敛.$\hfill\blacksquare$
\end{proof}
\hspace*{\fill}

\begin{remark}
	上述证明利用了Dirichlet判别法,当然,这里也可利用积分第二中值定理证明,在此不再赘述.
\end{remark}
\section{瑕积分的敛散判别}
瑕积分与无穷积分的敛散判别法相类似,我们只是给出陈述,不再加以证明.
\subsection{瑕积分收敛的Cauchy准则}
\begin{theorem}[瑕积分收敛的Cauchy准则]
	瑕积分$\displaystyle\int_{a}^{b}f(x)\d x$(瑕点为$a$)收敛的充要条件是:任给$\varepsilon>0$,存在$\delta>0$,只要$u_1,u_2\in(a,a+\delta)$,总有
	$$\bigg|\int_{u_1}^{b}f(x)\d x-\int_{u_2}^{b}f(x)\d x\bigg|=\bigg|\int_{u_1}^{u_2}f(x)\d x\bigg|<\varepsilon.$$
\end{theorem}
\begin{proposition}
	设函数$f$的瑕点为$a$,$f$在$\left(a,b\right]$的任一内闭区间$\left[u,b\right]$上可积.则当$\displaystyle\int_{a}^{b}|f(x)|\d x$收敛时,$\displaystyle\int_{a}^{b}f(x)\d x$也必定收敛,并有
	$$\bigg|\int_{a}^{b}f(x)\d x\bigg|\leqslant\int_{a}^{b}|f(x)|\d x.$$
\end{proposition}
\begin{remark}
	同样地,当$\displaystyle\int_{a}^{b}|f(x)|\d x$收敛时,称$\displaystyle\int_{a}^{b}f(x)\d x$为{\heiti 绝对收敛},称收敛而不绝对收敛者为{\heiti 条件收敛}.
\end{remark}
\subsection{非负函数瑕积分敛散判别}
\begin{theorem}[比较原则]
	设定义在$\left(a,b\right]$上的两个非负函数$f$和$g$瑕点都为$a$,在任何区间$\left[u,b\right]\subset\left(a,b\right]$上都可积,且满足
	$$f(x)\leqslant g(x),\ x\in \left(a,b\right],$$
	则有:

		(i)当$\displaystyle\int_{a}^{b}g(x)\d x$收敛时,$\displaystyle\int_{a}^{b}f(x)\d x$必收敛;
		
		(ii)当$\displaystyle\int_{a}^{b}f(x)\d x$发散时,$\displaystyle\int_{a}^{b}g(x)\d x$必发散.

\end{theorem}
\begin{corollary}[比较原则的极限形式]
	$f(x)\geqslant 0,\ g(x)>0$,且$\lim\limits_{x\to +\infty}\dfrac{f(x)}{g(x)}=c$时,有:
	
	(i)当$0<c<+\infty$时,$\displaystyle\int_{a}^{b}f(x)\d x$与$\displaystyle\int_{a}^{b}g(x)\d x$同敛态;
	
	(ii)当$c=0$时,由$\displaystyle\int_{a}^{b}g(x)\d x$收敛可推知$\displaystyle\int_{a}^{b}f(x)\d x$也收敛;
	
	(iii)当$c=+\infty$时,由$\displaystyle\int_{a}^{b}g(x)\d x$发散可推知$\displaystyle\int_{a}^{b}f(x)\d x$也发散.
\end{corollary}
特别地,如果选用$\displaystyle\int_{a}^{+\infty}\frac{\d x}{(x-a)^p}$作为比较对象,则有下面的{\heiti Cauchy判别法}及其极限形式.
\begin{corollary}[Cauchy判别法]
	设$f$定义于$\left(a,b\right]$上,且在任何区间$\left[u,b\right]\subset\left(a,b\right]$上可积,则有:
	
	(i)当$0\leqslant f(x)\leqslant\dfrac{1}{(x-a)^p}$,且$0<p<1$时,$\displaystyle\int_{a}^{b}f(x)\d x$收敛;
	
	(ii)当$f(x)\geqslant\dfrac{1}{(x-a)^p}$,且$p\geqslant 1$时,$\displaystyle\int_{a}^{b}f(x)\d x$发散.
\end{corollary}
\begin{corollary}[Cauchy判别法的极限形式]
	设$f$是定义于$\left(a,b\right]$上的非负函数,$a$是其瑕点,且在任何区间$\left[u,b\right]\subset\left(a,b\right]$上可积,若
	$$\lim\limits_{x\to a^+}(x-a)^pf(x)=\lambda.$$
	则有
	
	(i)当$0<p<1,\ 0\leqslant\lambda<+\infty$时,$\displaystyle\int_{a}^{b}f(x)\d x$收敛;
	
	(ii)当$p\geqslant 1,\ 0<\lambda\leqslant +\infty$时,$\displaystyle\int_{a}^{b}f(x)\d x$发散.
\end{corollary}
\subsection{一般瑕积分的敛散判别法}
\begin{theorem}[Dirichlet判别法]
	设$a$为$f(x)$的瑕点,函数$F(u)=\displaystyle\int_{a}^{b}f(x)\d x$在$\left(a,b\right]$上有界,函数$g(x)$在$\left(a,b\right]$上单调且$\lim\limits_{x\to a^+}g(x)=0$,则瑕积分$\displaystyle\int_{a}^{b}f(x)g(x)\d x$收敛.
\end{theorem}
\begin{theorem}[Abel判别法]
	设$a$是$f(x)$的瑕点,瑕积分$\displaystyle\int_{a}^{b}f(x)\d x$发散,函数$g(x)$在$\left(a,b\right]$上单调且有界,则瑕积分$\displaystyle\int_{a}^{b}f(x)g(x)\d x$收敛.
\end{theorem}
\newpage

\part{级数}

\chapter{数项级数}
\section{基本概念}
\begin{definition}[数项级数]
	给定一个数列$\{u_n\}$,对它的各项依次用“$+$”号连接起来的表达式
	$$u_1+u_2+\cdots+u_n+\cdots$$
	称为{\heiti 常数项无穷级数}或{\heiti 数项级数}(也常简称{\heiti 级数}),其中$\{u_n\}$称为数项级数的{\heiti 通项}或{\heiti 一般项}.
	
	记作$\displaystyle\sum_{n=1}^{\infty}u_n$或简单记作$\sum u_n$.
	
	级数的前$n$项之和记作
	$$S_n=\sum_{k=1}^{n}u_k=u_1+u_2+\cdots+u_n,$$
	称为数项级数的{\heiti 第$n$个部分和},简称{\heiti 部分和}.
\end{definition}
\begin{definition}[收敛与发散]
	若数项级数的部分和数列$\{S_n\}$收敛于$S$,即$\lim\limits_{n\to\infty}S_n=S$,则称数项级数{\heiti 收敛},称$S$为数项级数的{\heiti 和},记作
	$$S=\sum u_n.$$
	若$\{S_n\}$是发散数列,则称数项级数{\heiti 发散}.
\end{definition}
\begin{example}
	讨论{\heiti 几何级数}
	$$\sum_{k=0}^{\infty}aq^{i}\qquad(a\neq 0)$$
	的敛散性.
\end{example}
\begin{solution}
	$q\neq 1$时,级数的部分和
	$$S_n=a+aq+\cdots+aq^{n-1}=a\cdot\frac{1-q^n}{1-q}.$$
	因此,
	\begin{enumerate}[(1)]
		\item 当$|q|<1$时,$\lim\limits_{n\to\infty}S_n=\lim\limits_{n\to\infty}a\cdot\dfrac{1-q^n}{1-q}=\dfrac{a}{1-q}$. 此时级数收敛,其和为$\dfrac{a}{1-q}$.
		\item 当$|q|>1$时,$\lim\limits_{n\to\infty}S_n=\infty$,级数发散.
		\item 当$q=1$时,$S_{n}=na$,级数发散.
		\item 当$q=-1$时,$S_{2k}=0,\ S_{2k+1}=a,\ k=1,2,\cdots$,级数发散.
	\end{enumerate}
	综上所述,$|q|<1$时级数收敛,$|q|\geqslant 1$时级数发散.
\end{solution}
\begin{example}
	设$p$级数
	$$\sum_{n=1}^{\infty}\frac{1}{n^p},$$
	当$p>1$时级数收敛,当$p\leqslant 1$时级数发散,特别地,当$p=1$时称它为{\heiti 调和级数}.
\end{example}
依级数收敛和发散的定义,我们得到了第一个用来判别级数敛散的方法. 即求出级数的部分和数列,判断部分和数列的敛散性进而直接推出级数的敛散性. 然而,很多情况下我们是无法求出级数的部分和的,那么我们就需要其他的判别敛散性的方法.

由于级数的敛散性由它的部分和数列确定,因而可以把级数看作部分和数列的另一种表现形式. 反之,任给数列$\{a_n\}$,我们也可以把它看作某一数项级数的部分和数列,我们就有
$$\sum_{n=1}^{\infty}u_n=a_1+(a_2-a_1)+\cdots+(a_n-a_{n-1})+\cdots.$$
这时$\{a_n\}$与级数$\sum u_n$同敛态. 基于级数和数列的这种关系,我们有下列一些定理.
\begin{theorem}[Cauchy准则]
	$\sum u_n$收敛的充要条件是:任给$\varepsilon>0$,总存在正整数$N$,使得当$m>N$时,对任意的正整数$p$,都有
	$$|S_{m+p}-S_m|<\varepsilon$$
	或
	$$|u_{m+1}+u_{m+2}+\cdots+u_{m+p}|<\varepsilon.$$
\end{theorem}
由此可见,一个级数是否收敛与级数前面有限项的取值无关.

此外,我们立即可得下述推论.
\begin{corollary}
	若$\sum u_n$收敛,则$\lim\limits_{n\to\infty}u_n=0$.
\end{corollary}
\begin{remark}
	此推论只是级数收敛的必要条件而非充分条件,即$u_n\to 0$并不能推出级数收敛.
\end{remark}
\begin{theorem}
	若级数$\sum u_n$和$\sum v_n$都收敛,则对任意常数$c,d$,级数$\sum(cu_n+dv_n)$也收敛,且
	$$\sum(cu_n+dv_n)=c\sum u_n+d\sum v_n$$
\end{theorem}
\begin{theorem}
	去掉、增加或改变级数的有限个项并不改变级数的敛散性.
\end{theorem}
由此定理知道,若级数$\sum u_n$收敛,其和为$S$,则级数
$$u_{n+1}+u_{n+2}+\cdots$$
也收敛,且其和$R_n=S-S_n$. 称为级数$\sum u_n$的{\heiti 第$n$个余项}(或简称{\heiti 余项}),表示以部分和$S_n$代替$S$时产生的误差.
\begin{theorem}
	在收敛级数的项中任意加括号,既不改变级数的收敛性,也不改变它的和.
\end{theorem}
\begin{proof}
	设$\sum u_n$为收敛级数,其和为$S$. 记
	$$v_1=u_1+\cdots+u_{n_1},$$
	$$v_2=u_{n_1+1}+\cdots+u_{n_2},$$
	一般地,
	$$v_k=u_{n_{k-1}+1}+\cdots+u_{n_k},$$
	现在证明$\sum u_n$加括号后的级数
	$$\sum_{k=1}^{\infty}(u_{n_{k-1}+1}+\cdots+u_{n_k})=\sum_{k=1}^{\infty}v_k$$
	也收敛,且其和也是$S$.
	
	事实上,设$\{S_n\}$为收敛级数$\sum u_n$的部分和数列,则级数$\sum v_k$的部分和数列$\{S_{n_k}\}$是$\{S_n\}$的一个子列. 由于$\{S_n\}$收敛,且$\lim\limits_{n\to\infty}S_n=S$. 故由子列性质,$\{S_{n_k}\}$也收敛,且$\lim\limits_{k\to\infty}S_{n_k}=S$,即级数$\sum v_k$收敛,且它的和也等于$S$.$\hfill\blacksquare$
\end{proof}
\begin{remark}
	若级数加括号后收敛,并不能说明原级数也收敛. 若级数存在一种使得该级数发散的加括号的方式,则原级数发散.
\end{remark}
\section{正项级数}
\begin{definition}[正项级数]
	各项都是由非负数组成的级数称为{\heiti 正项级数}.
\end{definition}
\begin{remark}
	实际上,$u_n=0$的项不影响级数的敛散性,在判别正项级数敛散性时可自然排除.
\end{remark}
\subsection{一般判别原则}
\begin{theorem}
	正项级数$\sum u_n$收敛的充要条件是:部分和数列$\{S_n\}$有界. 即存在某正数$M$,对一切正整数$n$有$S_n<M$.
\end{theorem}
\begin{proof}
	正项级数的每一项都是非负的,因此$\{S_n\}$是递增的,由单调有界收敛定理可知$\{S_n\}$收敛,故$\sum u_n$收敛.$\hfill\blacksquare$
\end{proof}
\begin{theorem}[比较原则]
	设$\sum u_n$和$\sum v_n$是两个正项级数,如果存在某正数$N$,对一切$n>N$都有
	$$u_n\leqslant v_n,$$
	则
	\begin{enumerate}[(1)]
		\item 若$\sum v_n$收敛,则$\sum u_n$也收敛;
		\item 若$\sum u_n$发散,则$\sum v_n$也发散.
	\end{enumerate}
\end{theorem}
\begin{proof}
	只需证明(1).因为改变级数的有限项并不会影响原级数的敛散性,不妨设$u_n\leqslant v_n$对一切正整数$n$都成立.
	
	记$S_n'$和$S_n''$分别为$\sum u_n$和$\sum v_n$的部分和,则
	$$S_n'\leqslant S_n'',$$
	若$\sum v_n$收敛,则对一切$n$,有$S_n'\leqslant\lim\limits_{n\to\infty}S_n''$,即正项级数$\sum u_n$的部分和数列$\{S_n'\}$有界,故$\sum u_n$收敛. 由于(2)是(1)的逆否命题,自然成立.$\hfill\blacksquare$
\end{proof}
\begin{corollary}[比较原则的极限形式]
	设$\sum u_n$和$\sum v_n$是两个正项级数,若
	$$\lim\limits_{n\to\infty}\frac{u_n}{v_n}=l,$$
	则
	\begin{enumerate}[(1)]
		\item 当$0<l<+\infty$时,$\sum u_n$和$\sum v_n$同敛态;
		\item 当$l=0$时,由$\sum v_n$收敛可推知$\sum u_n$也收敛;
		\item 当$l=+\infty$时,由$\sum v_n$发散可推知$\sum u_n$也发散.
	\end{enumerate}
\end{corollary}
\begin{proof}
	当$0<l<+\infty$时,对任意正数$\varepsilon<l$,存在某正数$N$,当$n>N$时,有
	$$\left|\frac{u_n}{v_n}-l\right|<\varepsilon,$$
	即
	\begin{equation}\label{unvn}
		(l-\varepsilon)v_n<u_n<(l+\varepsilon).
	\end{equation}
	由比较原则可得$\sum u_n$和$\sum v_n$具有相同的敛散性.
	
	当$l=0$时,由$\ref{unvn}$式右半部分及比较原则得:若$\sum v_n$收敛,则$\sum u_n$也收敛.
	
	当$l=+\infty$时,即对任意正数$M$,存在正数$N$,当$n>N$时,都有
	$$\frac{u_n}{v_n}>M,$$
	即$u_n>Mv_n$. 由比较原则得,若$\sum v_n$发散,则$\sum u_n$也发散.$\hfill\blacksquare$
\end{proof}
有了上述定理,判断级数的敛散性问题就可以转化为判断它的一个同阶无穷小的敛散性.
\subsection{比式判别法和根式判别法}
根据比较原则,可以利用已知收敛或者发散级数作为比较对象来判别其他级数的敛散性. 下面我们介绍的判别法是以等比级数作为比较对象而得到的.
\begin{theorem}[D'Alembert比式判别法]
	设$\sum u_n$为正项级数,且存在某正整数$N_0$及常数$q\ (0<q<1)$,对一切$n>N_0$,
	\begin{enumerate}[(1)]
		\item 若$\dfrac{u_{n+1}}{u_n}\leqslant q$,则级数$\sum u_n$收敛.
		\item 若$\dfrac{u_{n+1}}{u_n}\geqslant 1$,则级数$\sum u_n$发散.
	\end{enumerate}
\end{theorem}
\begin{proof}
	(1)不妨设对一切$n\geqslant 1$都有
	$$\frac{u_{n+1}}{u_n}\leqslant q$$
	成立,于是有
	$$\frac{u_2}{u_1}\leqslant q,\quad \frac{u_3}{u_2}\leqslant q,\cdots,\quad \frac{u_n}{u_{n-1}}\leqslant q,\cdots.$$
	左右分别相乘,得
	$$\frac{u_2}{u_1}\cdot\frac{u_3}{u_2}\cdot\cdots\cdot\frac{u_n}{u_{n-1}}\leqslant q^{n-1},$$
	即
	$$u_n\leqslant u_1 q^{n-1}.$$
	当$0<q<1$时,等比级数$\sum\limits_{n=1}^{\infty}q^{n-1}$收敛,根据比较原则可知$\sum u_n$收敛.
	
	(2)当$n>N_0$时有
	$$\frac{u_{n+1}}{u_n}\geqslant 1$$
	成立,则$u_{n+1}\geqslant u_n\geqslant u_{N_0}$,于是当$n\to\infty$时,$u_n$的极限不可能为零. 由Cauchy收敛准则的推论可知$\sum u_n$发散.$\hfill\blacksquare$
\end{proof}
\begin{corollary}[D'Alembert判别法的极限形式]
	若$\sum u_n$为正项级数,且
	$$\lim\limits_{n\to\infty}\frac{u_{n+1}}{u_n}=q,$$
	则
	\begin{enumerate}[(1)]
		\item 当$q<1$时,级数$\sum u_n$收敛;
		\item 当$q>!$或$q=+\infty$时,级数$\sum u_n$发散;
		\item 当$q=1$时,无法判断级数$\sum u_n$的敛散性.
	\end{enumerate}
\end{corollary}
\begin{proof}
	对取定的正数$\varepsilon=\dfrac{1}{2}|1-q|$,存在正数$N$,当$n>N$时,都有
	$$q-\varepsilon<\frac{u_{n+1}}{u_n}<q+\varepsilon.$$
	
	当$q<1$时,$\dfrac{u_{n+1}}{u_n}<q+\varepsilon=\dfrac{1}{2}(1+q)<1$,故级数$\sum u_n$收敛.
	
	当$q>1$时,$\dfrac{u_{n+1}}{u_n}q-\varepsilon=\dfrac{1}{2}(1+q)>1$,故级数$\sum u_n$发散.
	
	当$q=+\infty$时,存在$N$,当$n>N$时有
	$$\frac{u_{n+1}}{u_n}>1,$$
	从而$\lim\limits_{n\to\infty}u_n\neq 0$,所以级数$\sum u_n$发散.
	
	当$q=1$时,分别取$u_n=\dfrac{1}{n}$和$u_n=\dfrac{1}{n^2}$,前者发散而后者收敛,故并不能以此判别敛散性.$\hfill\blacksquare$
\end{proof}
如果上述$\dfrac{u_{n+1}}{u_n}$的极限不存在,则可应用上、下极限来判别.
\begin{corollary}
	设$\sum u_n$为正项级数.
	\begin{enumerate}[(1)]
		\item 若$\varlimsup\limits_{n\to\infty}\dfrac{u_{n+1}}{u_n}=Q<1$,则级数收敛;
		\item 若$\varliminf\limits_{n\to\infty}\dfrac{u_{n+1}}{u_n}=q>1$,则级数发散.
		\item 若$Q=1$或$q=1$或$q<1<Q$,则无法判断其敛散性.
	\end{enumerate}
\end{corollary}
\begin{theorem}[Cauchy根式判别法]
	设$\sum u_n$为正项级数,且存在某正数$N_0$及正常数$l$,对一切$N_0$,
	\begin{enumerate}[(1)]
		\item 若$\sqrt[n]{u_n}\leqslant l<1$,则级数$\sum u_n$收敛;
		\item 若$\sqrt[n]{u_n}\geqslant 1$,则级数$\sum u_n$发散.
		\item 若$\sqrt[n]{u_n}=1$,则无法判断级数$\sum u_n$的敛散性.
	\end{enumerate}
\end{theorem}
\begin{proof}
	对(1)中的不等式变形,得
	$$u_n\leqslant l^n.$$
	因为等比级数$\sum l^n$在$0<l<1$时收敛,故由比较原则,这时级数$\sum u_n$也收敛.
	
	对(2)中的不等式变形,得
	$$u_n\geqslant 1^n=1.$$
	当$n\to\infty$时,显然$u_n$不可能以零为极限,因此级数$\sum u_n$是发散的.
	
	(3)中,与D'Alembert判别法极限形式的证明一样,考虑$\dfrac{1}{n}$和$\dfrac{1}{n^2}$.
	$\hfill\blacksquare$
\end{proof}
\begin{corollary}[Cauchy判别法的极限形式]
	设$\sum u_n$为正项级数,且
	$$\lim\limits_{n\to\infty}\sqrt[n]{u_n}=l,$$
	则
	\begin{enumerate}[(1)]
		\item 当$l<1$时,级数$\sum u_n$收敛;
		\item 当$l>1$时,级数$\sum u_n$发散.
		\item 当$l=1$时,无法判断其敛散性.
	\end{enumerate}
\end{corollary}
\begin{proof}
	当取$\varepsilon<|1-l|$时,存在某正数$N$,对一切$n>N$,有
	$$l-\varepsilon<\sqrt[n]{u_n}<l+\varepsilon.$$
	于是由Cauchy判别法即得.$\hfill\blacksquare$
\end{proof}
如果上述$\sqrt[n]{u_n}$的极限不存在,则可根据根式$\sqrt[n]{u_n}$的上极限来判断.
\begin{corollary}
	设$\sum u_n$为正项级数,且
	$$\varlimsup\limits_{n\to\infty}\sqrt[n]{u_n}=l,$$
	则
	\begin{enumerate}[(1)]
		\item 当$l<1$时级数收敛;
		\item 当$l>1$时级数发散;
		\item 当$l=1$时无法判断其敛散性.
	\end{enumerate}
\end{corollary}
\subsection{Raabe判别法}
比式判别法和根式判别法只适用于来判断通项收敛速度比某一等比级数快的级数,如果级数的收敛速度较慢,它们就无能为力了. 我们设法找到收敛速度更慢的级数作为标准. 这就有了Raabe判别法.
\begin{theorem}[Raabe判别法]
	设$\sum u_n$为正项级数,且存在某正整数$N_0$及常数$r$,
	\begin{enumerate}[(1)]
		\item 若对一切$n>N_0$,成立不等式
		$$n\left(1-\frac{u_{n+1}}{u_n}\right)\geqslant r>1,$$
		则级数$\sum u_n$收敛;
		\item 若对一切$n>N_0$,成立不等式
		$$n\left(1-\frac{u_{n+1}}{u_n}\right)\leqslant 1,$$
		则级数$\sum u_n$发散.
	\end{enumerate}
\end{theorem}
\begin{proof}
	\begin{enumerate}[(1)]
		\item 由$n\left(1-\frac{u_{n+1}}{u_n}\right)\geqslant r$可得$\dfrac{u_{n+1}}{u_n}<1-\dfrac{r}{n}$. 选$p$使$1<p<r$. 由于
		$$\lim\limits_{n\to\infty}\frac{1-\left(1-\frac{1}{n}\right)^r}{\frac{r}{n}}=\lim\limits_{x\to 0}\frac{1-(1-x)^p}{rx}=\lim\limits_{x\to 0}\frac{p(1-x)^{p-1}}{r}=\frac{p}{r}<1,$$
		因此,存在正数$N$,使对任意$n>N$,
		$$\frac{r}{n}>1-\left(1-\frac{1}{n}\right)^p.$$
		这样
		$$\frac{u_{n+1}}{u_n}<1-\left[1-\left(1-\frac{1}{n}\right)^r\right]=\left(1-\frac{1}{n}\right)^p=\left(\frac{n-1}{n}\right)^p.$$
		于是,当$n>N$时就有
		\begin{align*}
			u_{n+1}&=\frac{u_{n+1}}{u_n}\cdot\frac{u_n}{u_{n-1}}\cdot\cdots\cdot\frac{u_{N+1}}{u_N}\cdot u_N\\
			&\leqslant\left(\frac{n-1}{n}\right)^p\left(\frac{n-2}{n-1}\right)^p\cdot\cdots\cdot\left(\frac{N-1}{N}\right)^p\cdot u_N\\
			&=\frac{(N-1)^p}{n^p}\cdot u_N.
		\end{align*}
		当$p>1$时,$\sum\frac{1}{n^2}$收敛,故级数$\sum u_n$是收敛的.
		\item 由$n\left(1-\dfrac{u_{n+1}}{u_n}\right)\leqslant 1$可得$\dfrac{u_{n+1}}{u_n}\geqslant 1-\dfrac{1}{n}=\frac{n-1}{n}$,于是
		\begin{align*}
			u_{n+1}&=\frac{u_{n+1}}{u_n}\cdot\frac{u_n}{u_{n-1}}\cdot\cdots\cdot\frac{u_3}{u_2}\cdot u_2\\
			&>\frac{n-1}{n}\cdot\frac{n-2}{n-1}\cdot\cdots\cdot\frac{1}{2}\cdot u_2\\
			&=\frac{1}{n}\cdot u_2.
		\end{align*}
		因为$\sum \frac{1}{n}$是发散的,故$\sum u_n$是发散的.$\hfill\blacksquare$
	\end{enumerate}
\end{proof}
\begin{corollary}[Raabe判别法的极限形式]
	设$\sum u_n$为正项级数,且极限
	$$\lim\limits_{n\to\infty}n\left(1-\frac{u_{n+1}}{u_n}\right)=r$$
	存在,则
	\begin{enumerate}
		\item 当$r>1$时,级数$\sum u_n$收敛;
		\item 当$r<1$时,级数$\sum u_n$发散.
	\end{enumerate}
\end{corollary}
Raabe判别法判别的范围比比式判别法和根式判别法更广泛,但也有其无法判别的情况,如$r=1$时. 没有收敛得最慢的收敛级数,因此任何判别法只能判别一部分级数,当然我们可以继续建立比Raabe判别法更精细的判别法,这个过程是无限的.
\subsection{积分判别法}
积分判别法是利用非负函数的单调性和积分性质,并以反常积分为比较对象来判别正项级数的敛散性.
\begin{theorem}
	设$f$为$\left[1,+\infty\right)$上的减函数,则级数$\displaystyle\sum_{n=1}^{\infty}f(n)$收敛的充分必要条件是反常积分$\displaystyle\int_{1}^{+\infty}f(x)\d x$收敛.
\end{theorem}
\begin{proof}
	{\heiti 必要性}\quad 设$\displaystyle\sum_{n=1}^{\infty}f(n)$收敛,其和为$S$,则$\lim\limits_{n\to\infty}f(n)=0$. 又因为$f$为$\left[1,+\infty\right)$上的减函数,所以$f(x)\geqslant 0$,从而$\displaystyle\sum_{n=1}^{\infty}f(n)$为正项级数. 于是对任意正整数$m$,有
	$$\int_{1}^{m}f(x)\d x=\sum_{n=2}^{m}\int_{n-1}^{n}f(x)\d x\leqslant \sum_{n=1}^{m-1}f(n)\leqslant \sum_{n=1}^{\infty}f(n)=S.$$
	由$f$是非负的减函数,故对任何正数$A$,有
	$$0\leqslant \int_{1}^{A}f(x)\d x\leqslant \int_{1}^{m+1}f(x)\d x\leqslant \sum_{n=1}^{m}f(n)\leqslant S,\ m<A\leqslant m+1.$$
	根据比较原则,可知$\displaystyle\int_{1}^{\infty}f(x)$收敛.
	
	{\heiti 充分性}\quad 设$\displaystyle\int_{1}^{\infty}f(x)$收敛,则$\lim\limits_{n\to +\infty}f(x)=0$. 又因为$f$是减函数,因此它是非负的减函数,$\displaystyle\sum_{n=1}^{\infty}f(n)$是正项级数. 因此对任意正整数$m$,有
	$$\sum_{n=1}^{m}f(n)=f(1)+\sum_{n=2}^{m}f(n)\leqslant f(1)+\int_{1}^{m}f(x)\d x\leqslant f(1)+\int_{1}^{+\infty}f(x)\d x.$$
	因此级数$\displaystyle\sum_{n=1}^{\infty}f(n)$收敛.$\hfill\blacksquare$
\end{proof}
\section{一般项级数}
这里只讨论某些特殊类型的级数的收敛性问题.
\subsection{交错级数}
定义{\heiti 交错级数}为各项正负相间的级数. 对于交错级数,我们有Leibnitz判别法.
\begin{theorem}[Leibnitz判别法]
	设交错级数
	$$u_1-u_2+u_3-u_4+\cdots+(-1)^{n+1}u_n+\cdots\quad (u_n>0,\ n=1,2,\cdots),$$
	如果它满足下面条件(Leibnitz条件)
	\begin{enumerate}
		\item 数列$\{u_n\}$单调递减;
		\item $\lim\limits_{n\to\infty}u_n=0$,
	\end{enumerate}
	则交错级数收敛.
\end{theorem}
\begin{proof}
	考察交错级数的部分和数列$\{S_n\}$,它的奇数项和偶数项分别为
	$$S_{2m-1}=u_1-(u_2-u_3)-\cdots-(u_{2m-2}-u_{2m-1}),$$
	$$S_{2m}=(u_1-u_2)+(u_3-u_4)+\cdots+(u_{2m-1}-u_{2m}).$$
	由条件(1),上述两式中各个括号内的数都是非负的,从而数列$\{S_{2m-1}\}$是递减的,而数列$\{S_{2n}\}$是递增的,又由条件(2)知道
	$$0<S_{2m-1}-S_{2m}=u_{2m}\to 0\quad (m\to\infty),$$
	从而$\{\left[S_{2m},S_{2m-1}\right]\}$是一个区间套. 由区间套定理,存在唯一的一个数$S$,使得
	$$\lim\limits_{m\to\infty}S_{2m-1}=\lim\limits_{m\to\infty}S_{2m}=S.$$
	所以数列$\{S_n\}$收敛,即交错级数收敛. $\hfill\blacksquare$
\end{proof}
\begin{corollary}
	若交错级数满足Leibnitz条件,则级数的余项估计式为
	$$|R_n|\leqslant u_{n+1}.$$
\end{corollary}
\subsection{绝对收敛级数及其性质}
\begin{definition}[绝对收敛级数]
	若级数$\sum u_n$各项的绝对值组成的级数$\sum |u_n|$收敛,则称级数$\sum u_n$为{\heiti 绝对收敛级数}.
\end{definition}
\begin{theorem}
	绝对收敛级数一定收敛.
\end{theorem}
\begin{proof}
	由于级数$\sum |u_n|$收敛,由Cauchy收敛准则,对任意$\varepsilon>0$,存在$N>0$,对$m>N$和任意正整数$r$,有
	$$|u_{m+1}|+|u_{m+2}|+\cdots+|u_{m+r}|<\varepsilon.$$
	而
	$$|u_{m+1}+u_{m+2}+\cdots+u_{m+r}|\leqslant|u_{m+1}|+|u_{m+2}|+\cdots+|u_{m+r}|<\varepsilon,$$
	因此由Cauchy收敛准则可知$\sum u_n$收敛.
\end{proof}
因此,判别一个级数是否收敛,我们可以先判别它是否绝对收敛,这时只需判断其各项绝对值组成的级数的敛散性,也就转化为了判断正项级数的敛散性问题. 但是,一个级数收敛并不能推出它是绝对收敛的. 我们有以下定义.
\begin{definition}[条件收敛级数]
	如果级数$\sum u_n$收敛而$\sum |u_n|$不收敛,则称$\sum u_n$为{\heiti 条件收敛级数}.
\end{definition}
至此,全体收敛的级数可以分为绝对收敛级数和条件收敛级数两大类. 

下面讨论绝对收敛级数的两个重要性质.

我们先来介绍级数的重排.
\begin{definition}[重排]
	我们把正整数列$\{1,2,\cdots,n,\cdots\}$到它自身的一一映射$f:n\to k(n)$称为{\heiti 正整数列的重排},相应地对于数列$\{u_n\}$按映射$F:u_n\to u_{k_n}$所得到的数列$\{u_{k(n)}\}$称为{\heiti 原数列的重排}. 相应于此,我们也称级数$\sum u_{k(n)}$是级数$\sum u_n$的{\heiti 重排}.
\end{definition}
\begin{theorem}
	设级数$\sum u_n$绝对收敛,其和为$S$,则任意重排后所得到的级数$\sum u_{k(n)}$也绝对收敛,且有相同的和数.
\end{theorem}
\begin{proof}
	为叙述方便,记$v_n=u_{k(n)}$. 先假设$\sum u_n$是正项级数,用$S_n$表示它的第$n$个部分和. 以$\sigma_m=v_1+\cdots+v_m$表示级数$\sum v_n$的第$m$个部分和. 因为$\sum v_n$是$\sum u_n$的重排,所以每一$v_k\ (1\leqslant k\leqslant m)$都等于某一$u_{i_k}\ (1\leqslant k\leqslant m)$. 记
	$$n=\max\{i_1,i_2,\cdots,i_m\},$$
	则对任何$m$,都存在$n$,使$\sigma_m\leqslant S_n$.
	
	由于$\lim\limits_{n\to\infty}S_n=S$,所以对任何正整数$m$都有$\sigma_m\leqslant S$,即得$\sum v_n$收敛,且其和$\sigma\leqslant S$.
	
	同理,级数$\sum u_n$也可看作$\sum v_n$的重排,所以也有$S\leqslant\sigma$,从而推得$\sigma=S$.
	
	若$\sum u_n$为一般项级数且绝对收敛,则$\sum |u_n|$是收敛的正项级数. 由上述证明即得$\sum |v_n|$也收敛,即$\sum v_n$是绝对收敛的.
	
	下面证明$\sum v_n$的和也等于$S$. 令
	$$p_n=\frac{|u_n|+u_n}{2},\quad q_n=\frac{|u_n|-u_n}{2}.$$
	当$u_n\geqslant 0$时,$p_n=u_n\geqslant 0$,$q_n=0$;当$u_n<0$时,$p_n=0$,$q_n=|u_n|=-u_n>0$. 从而有
	$$0\leqslant p_n\leqslant |u_n|,\quad 0\leqslant q_n\leqslant |u_n|,$$
	$$p_n+q_n=|u_n|,\quad p_n-q_n=u_n.$$
	因为$\sum u_n$绝对收敛,由上式可知$\sum p_n,\ \sum q_n$都是收敛的正项级数. 因此
	$$S=\sum u_n=\sum p_n-\sum q_n.$$
	对于重排后的级数$\sum v_n$,也可同理表示为两个收敛的正项级数之差
	$$\sum v_n=\sum p_n'-\sum q_n',$$
	其中$\sum p_n'$和$\sum q_n'$分别是级数$\sum p_n$和$\sum q_n$的重排,前面已经证明收敛的正项级数重排后,它的和不变,从而有
	$$\sum v_n=\sum p_n'-\sum q_n'=\sum p_n-\sum q_n=S.$$
	$\hfill\blacksquare$
\end{proof}
\begin{remark}
	由条件收敛级数重排后得到的新级数,即使收敛,也不一定收敛于原来的和数. 而且条件收敛级数适当重排后,可得到发散级数,或收敛于任何事先指定的数.
\end{remark}

我们再来介绍级数的乘积.

设有两个收敛级数
$$\sum u_n=A,\quad \sum v_n=B.$$
可以将上述两个级数中的每一项所有可能的乘积列成下表.
\begin{center}
	\begin{tabular}{|cccccc}
		\hline
		$u_1v_1$ & $u_1v_2$ & $u_1v_3$ & $\cdots$ & $u_1v_n$ & $\cdots$ \\
		$u_2v_1$ & $u_2v_2$ & $u_2v_3$ & $\cdots$ & $u_2v_n$ & $\cdots$ \\
		$u_3v_1$ & $u_3v_2$ & $u_3v_3$ & $\cdots$ & $u_3v_n$ & $\cdots$ \\
		$\vdots$ & $\vdots$ & $\vdots$ & $		$ & $\vdots$ & $	  $ \\
		$u_nv_1$ & $u_nv_2$ & $u_nv_3$ & $\cdots$ & $u_nv_n$ & $\cdots$ \\
		$\vdots$ & $\vdots$ & $\vdots$ & $		$ & $\vdots$ & $	  $ \\
	\end{tabular}
\end{center}

这些乘积$u_iv_j$可以按各种方法排成不同的级数.比如按正方形顺序依次相加,得到
$$u_1v_1+u_1v_2+u_2v_2+u_2v_1+u_1v_3+u_2v_3+u_3v_3+u_3v_2+u_3v_1+\cdots,$$
按对角线顺序依次相加,得到
$$u_1v_1+u_1v_2+u_2v_1+u_1v_3+u_2v_2+u_3v_1+\cdots.$$
\begin{theorem}[Cauchy定理]
	若级数$\sum u_n$和级数$\sum v_n$都绝对收敛,且$\sum |u_n|=A$,$\sum |v_n|=B$,则这两个级数的每一项所有可能的乘积按任意顺序排列得到的级数$\sum w_n$也绝对收敛,且其和等于$AB$.
\end{theorem}
\begin{proof}
	以$S_n$表示级数$\sum |w_n|$的部分和,即
	$$S_n=|w_1|+|w_2|+\cdots+|w_n|,$$
	其中$w_k=u_{i_k}+v{j_k}\ (k=1,2,\cdots,n)$,记
	$$m=\max\{i_1,j_1,i_2,j_2,\cdots,i_n,j_n\},$$
	$$A_m=|u_1|+|u_2|+\cdots+|u_m|,$$
	$$B_m=|v_1|+|v_2|+\cdots+|v_m|,$$
	则必有
	$$S_n\leqslant A_mB_m.$$
	级数$\sum u_n$和$\sum v_n$都绝对收敛,因此部分和数列$A_n$和$B_n$都是有界的,所以$\{S_n\}$是有界的,从而级数$\sum w_n$绝对收敛.
	
	由于绝对收敛级数具有可重排的性质,也就是说级数的和与采用哪一种排列的次序无关. 为方便求和,采取按正方形顺序求和并对各被加项加括号,即
	$$u_1v_1+(u_1v_2+u_2v_2+u_2v_1)+(u_1v_3+u_2v_3+u_3v_3+u_3v_2+u_3v_1)+\cdots,$$
	把每一括号作为一项,得到新的级数
	$$\sum p_n=p_1+p_2+p_3+\cdots+p_n+\cdots,$$
	它与级数$\sum w_n$同时收敛且和数相同. 现以$P_n$表示级数$\sum p_n$的部分和,它与级数$\sum u_n$与$\sum v_n$的部分和$A_n$与$B_n$有如下关系式:
	$$P_n=A_nB_n.$$
	从而有
	$$\lim\limits_{n\to\infty}P_n=\lim\limits_{n\to\infty}A_nB_n=\lim\limits_{n\to\infty}A_n\lim\limits_{n\to\infty}B_n=AB.$$
	$\hfill\blacksquare$
\end{proof}
\subsection{Abel判别法和Dirichlet判别法}
我们先介绍一个公式.
\begin{lemma}[分部求和公式,Abel变换]
	设$\varepsilon_i,\ v_i\ (i=1,2,\cdots,n)$为两组实数,若令
	$$\sigma_k=v_1+v_2+\cdots+v_k\qquad(k=1,2,\cdots,n),$$
	则有如下分部求和公式成立:
	$$\sum_{i=1}^{n}\varepsilon_iv_i=(\varepsilon_1-\varepsilon_2)\sigma_1+(\varepsilon_2-\varepsilon_3)\sigma_2+\cdots+(\varepsilon_{n-1}-\varepsilon_n)\sigma_{n-1}+\varepsilon_n\sigma_n.$$
	即
	$$\sum_{i=1}^{n}\varepsilon_iv_i=\sum_{i=1}^{n-1}(\varepsilon_i-\varepsilon_{i+1})\sigma_i+\varepsilon_n\sigma_n.$$
\end{lemma}
\begin{proof}
	将右边展开即得.$\hfill\blacksquare$
\end{proof}
\begin{lemma}[Abel引理]
	若
	\begin{enumerate}[(1)]
		\item $\varepsilon_1,\varepsilon_2,\cdots,\varepsilon_n$是单调数组;
		\item 对任一正整数$k\ (1\leqslant k\leqslant n)$有$|\sigma_k|\leqslant A$ (这里$\sigma_k=v_1+\cdots+v_k$), 
	\end{enumerate}
	则记$\varepsilon=\max\limits_k\{|\varepsilon_k|\}$,有
	$$\left|\sum_{k=1}^{n}\varepsilon_kv_k\right|\leqslant 3\varepsilon A.$$
\end{lemma}
\begin{proof}
	由(1)可知
	$$\varepsilon_i-\varepsilon_{i+1}\quad (i=1,2,\cdots,n-1)$$
	都是同号的. 于是由分部求和公式以及条件(2)推得
	\begin{align*}
		\left|\sum_{i=1}^{n}\varepsilon_kv_k\right|
		&=\left|\sum_{i=1}^{n-1}(\varepsilon_i-\varepsilon_{i+1})\sigma_i+\varepsilon_n\sigma_n\right|\\
		&\leqslant A\left|\sum_{i=1}^{n-1}(\varepsilon_i-\varepsilon_{i+1})\right|+A|\varepsilon_n|\\
		&=A|\varepsilon_1-\varepsilon_n|+A|\varepsilon_n|\\
		&\leqslant A(|\varepsilon_1|+2|\varepsilon_n|)\\
		&\leqslant 3\varepsilon A.
	\end{align*}
	$\hfill\blacksquare$
\end{proof}
下面讨论级数
$$\sum a_nb_n=a_1b_1+a_2b_2+\cdots+a_nb_n$$
敛散性的判别法.
\begin{theorem}[Abel判别法]
	若$\{a_n\}$为单调有界数列,且级数$\sum b_n$收敛,则级数$\sum a_nb_n$收敛.
\end{theorem}
\begin{proof}
	$\sum b_n$收敛,由Cauchy准则,对任意$\varepsilon>0$,存在$N>0$,使当$n>N$时对任一正整数$p$,都有
	$$\left|\sum_{k=n+1}^{n+p}b_k\right|<\varepsilon.$$
	又由于数列$\{a_n\}$有界,所以存在$M>0$,使$|a_n|\leqslant M$,则由Abel引理可得
	$$\left|\sum_{k=n+1}^{n+p}a_kb_k\right|\leqslant 3M\varepsilon.$$
	所以级数$\sum a_nb_n$收敛.$\hfill\blacksquare$
\end{proof}
\begin{theorem}[Dirichlet判别法]
	若数列$\{a_n\}$单调,$\lim\limits_{n\to\infty}a_n=0$,且级数$\sum b_n$的部分和数列有界,则级数$\sum a_nb_n$收敛.
\end{theorem}
\begin{proof}
	由$\{a_n\}$单调收敛,则对任意$n\in\mathbb{Z}^+$,有$a_n\leqslant A$. 又因为$\sum b_n$的部分和数列有界,即存在$M>0$,使
	$$\left|\sum_{k=1}^{n}b_k\right|\leqslant M.$$
	由Abel引理,有
	$$\left|\sum_{k=1}^{n}a_kb_k\right|\leqslant 3AM.$$
	所以级数$\sum a_nb_n$收敛.$\hfill\blacksquare$
\end{proof}
\newpage

\chapter{函数项级数}
前面我们用收敛数列或数项级数来表示或定义一个数. 本章将讨论怎样用函数列或函数项级数来表示或定义一个函数,并研究这个函数的性质.
\section{一致收敛性}
\subsection{函数列及其一致收敛性}
\begin{definition}[函数列]
	设
	$$f_1,f_2,\cdots,f_n,\cdots$$
	是一列定义在同一数集$E$上的函数,称为定义在$E$上的{\heiti 函数列}. 也记作
	$$\{f_n\}\ \text{或}\ f_n,\ n=1,2,\cdots.$$
\end{definition}
\begin{definition}[函数列极限]
	对每一固定的$x\in D$,任给$\varepsilon>0$,恒存在$N(\varepsilon,x)>0$,使得当$n>N$时,总有
	$$|f_n(x)-f(x)|<\varepsilon,$$
	则称函数列$f_n(x)${\heiti 在数集$D$上收敛},记作
	$$\lim\limits_{n\to\infty}f_n(x)=f(x),\qquad x\in D$$
	或
	$$f_n(x)\to f(x),\qquad (n\to\infty)\qquad x\in D.$$
	称$x$是$\{f_n\}$的{\heiti 收敛点},$f(x)$为函数列$\{f_n\}$的{\heiti 极限函数}. 
	
	使函数列$\{f_n\}$收敛的全体收敛点的集合,称为函数列$\{f_n\}$的{\heiti 收敛域}.
\end{definition}
对于函数列,我们不仅要讨论它在哪些点上收敛,而更重要的是要研究极限函数所具有的解析性质. 比如能否由函数列的每项的连续性判断出极限函数的连续性. 又如极限函数的导数或积分,是否分别是函数列每项导数或积分的极限. 我们涉及到换序问题. 只要求函数列在数集$D$上收敛是不够的,必须对它在$D$上的收敛性提出更高的要求才行. 这就是以下所要讨论的一致收敛性问题.
\begin{definition}[一致收敛]
	设函数列$\{f_n\}$与函数$f$定义在同一数集$D$上,若对任给的$\varepsilon>0$,总存在$N(\varepsilon)>0$,使得当$n>N$时,对一切$x\in D$,都有
	$$|f_n(x)-f(x)|<\varepsilon,$$
	则称函数列$\{f_n\}$在$D$上{\heiti 一致收敛}于$f$,记作
	$$f_n(x)\rightrightarrows f(x)\ (n\to\infty),\ x\in D.$$
\end{definition}
\begin{remark}
	一致收敛的定义中,$N$的选取仅与$\varepsilon$有关,与$x$的取值无关. 而在收敛的定义中,$N$的取值既与$\varepsilon$有关,又与$x$有关. 由此,函数列$\{f_n\}$在$D$上一致收敛,则在$D$的每一点都收敛. 反之则不一定成立.
\end{remark}
\begin{theorem}[函数列一致收敛的Cauchy准则]
	函数列$\{f_n\}$在数集$D$上一致收敛的充要条件是:对任意$\varepsilon>0$,总存在$N>0$,使得当$n,m>N$时,对一切$x\in D$,都有
	$$|f_n(x)-f_m(x)|<\varepsilon.$$
\end{theorem}
\begin{proof}
	{\heiti 必要性}\qquad 设$f_n(x)\rightrightarrows f(x)\ (n\to\infty),\ x\in D$,即对任意$\varepsilon>0$,存在$N>0$,使得当$n>N$时,对一切$x\in D$,都有
	$$|f_n(x)-f(x)|<\frac{\varepsilon}{2}.$$
	于是当$n,m>N$时有
	\begin{align*}
		|f_n(x)-f_m(x)|&\leqslant |f_n(x)-f(x)|+|f(x)-f_m(x)|\\
		&<\frac{\varepsilon}{2}+\frac{\varepsilon}{2}=\varepsilon.
	\end{align*}
	
	{\heiti 充分性}\qquad 若
	$$|f_n(x)-f_m(x)|<\varepsilon$$
	成立,由数列收敛的Cauchy准则,$\{f_n\}$在$D$上任一点都收敛,记其极限函数为$f(x),\ x\in D$. 现固定$n$,让$m\to\infty$,于是当$n>N$时,对一切$x\in D$,都有
	$$|f_n(x)-f(x)|\leqslant \varepsilon.$$
	由一致收敛的定义即得
	$$f_n(x)\rightrightarrows f(x)\ (n\to\infty),\ x\in D.$$
	$\hfill\blacksquare$
\end{proof}

根据一致收敛的定义可推出下述定理.
\begin{theorem}\label{chy}
	函数列$\{f_n\}$在区间$D$上一致收敛于$f$的充要条件是:
	$$\lim\limits_{n\to\infty}\sup\limits_{x\in D}|f_n(x)-f(x)|=0.$$
\end{theorem}
\begin{proof}
	{\heiti 必要性}\qquad 若$f_n(x)\rightrightarrows f(x)\ (n\to\infty),\ x\in D$,则对任意$\varepsilon>0$,存在$N(\varepsilon)>0$,当$n>N$时,有
	$$|f_n(x)-f(x)|<\varepsilon,\quad x\in D.$$
	由上确界的定义,有
	$$\sup\limits_{x\in D}|f_n(x)-f(x)|\leqslant\varepsilon.$$
	由极限定义,即得
	$$\lim\limits_{n\to\infty}\sup\limits_{x\in D}|f_n(x)-f(x)|=0.$$
	
	{\heiti 充分性}\qquad 由假设,对任给$\varepsilon>0$,存在$N>0$,使得当$n>N$时,有
	$$\sup\limits_{x\in D}|f_n(x)-f(x)|<\varepsilon.$$
	因为对一切$x\in D$,总有
	$$|f_n(x)-f(x)|\leqslant\sup\limits_{x\in D}|f_n(x)-f(x)|,$$
	所以有
	$$|f_n(x)-f(x)|<\varepsilon.$$
	于是$\{f_n\}$在$D$上一致收敛于$f$.$\hfill\blacksquare$
\end{proof}
\begin{remark}
	在判断函数列是否一致收敛时上述定理更为方便一些,但必须事先知道它的极限函数.
\end{remark}
\begin{corollary}
	函数列$\{f_n\}$在$D$上不一致收敛于$f$的充要条件是:存在$\{x_n\}\in D$,使得$\{f_n(x_n)-f(x_n)\}$不收敛于$0$.
\end{corollary}
\begin{definition}[内闭一致收敛]
	设函数列$\{f_n\}$与$f$定义在区间$I$上,若对任意闭区间$\left[a,b\right]\subset I$,$\{f_n\}$在$\left[a,b\right]$上一致收敛于$f$,则称$\{f_n\}$在$I$上{\heiti 内闭一致收敛}于$f$.
\end{definition}
\begin{remark}
	若$I=\left[\alpha,\beta\right]$是有界闭区间,显然$\{f_n\}$在$I$上内闭一致收敛于$f$与$\{f_n\}$在$I$上一致收敛于$f$是一致的.
\end{remark}
\subsection{函数项级数及其一致收敛性}
\begin{definition}[函数项级数]
	设$\{u_n(x)\}$是定义在数集$E$上的一个函数列,表达式
	$$u_1(x)+u_2(x)+\cdots+u_n(x)+\cdots,\quad x\in E$$
	称为定义在$E$上的{\heiti 函数项级数},简记为$\displaystyle\sum_{n=1}^{\infty}u_n(x)$或$\sum u_n(x)$. 称
	$$S_n(x)=\sum_{k=1}^{n}u_k(x),\quad x\in E,\quad n=1,2,\cdots$$
	为函数项级数$\sum u_n(x)$的{\heiti 部分和函数列}.
\end{definition}
\begin{definition}[函数项级数的收敛]
	若$x_0\in E$,数项级数
	$$u_1(x_0)+u_2(x_0)+\cdots+u_n(x_0)+\cdots$$
	收敛,即部分和$S_n(x_0)$当$n\to\infty$时极限存在,则称函数项级数$\sum u_n(x)$在点$x_0${\heiti 收敛},称$x_0$是级数的{\heiti 收敛点}. 反之,称级数在点$x_0${\heiti 发散}. 若级数$\sum u_n(x)$在$E$的某个子集$D$上每点都收敛,则称级数$\sum u_n(x)$在$D$上收敛. 
	
	使函数项级数$\sum u_n(x)$收敛的全体收敛点的集合,称为函数项级数的{\heiti 收敛域}. 函数项级数$\sum u_n(x)$在收敛域上的每一点$x$与其所对应的数项级数的和$S(x)$构成一个定义在收敛域上的函数,称为函数项级数$\sum u_n(x)$的{\heiti 和函数},并写作
	$$u_1(x)+u_2(x)+\cdots+u_n(x)+\cdots=S(x),\quad x\in D,$$
	即
	$$\lim\limits_{n\to\infty}S_n(x)=S(x),\quad x\in D.$$
\end{definition}
\begin{definition}
	设$\{S_n(x)\}$是函数项级数$\sum u_n(x)$的部分和函数列. 若$\{S_n(x)\}$在数集$D$上一致收敛于$S(x)$,则称$\sum u_n(x)$在$D$上一致收敛于$S(x)$. 若$\sum u_n(x)$在任意闭区间$\left[a,b\right]\subset I$上一致收敛,则称$\sum u_n(x)$在$I$上{\heiti 内闭一致收敛}.
\end{definition}
由于函数项级数的一致收敛性由它的部分和函数列来确定,所以由前段中有关函数列一致收敛的定理,都可推出相应的有关函数项级数的定理.
\begin{theorem}[一致收敛的Cauchy准则]
	函数项级数$\sum u_n(x)$在数集$D$上一致收敛的充要条件为:对任意$\varepsilon>0$,总存在$N>0$,使得当$n>N$时,对一切$x\in D$和一切正整数$p$,都有
	$$|S_{n+p}(x)-S_n(x)|<\varepsilon$$
	或
	$$|u_{n+1}(x)+u_{n+2}(x)+\cdots+u_{n+p}(x)|<\varepsilon.$$
\end{theorem}
当$p=1$时,我们就有下述推论.
\begin{corollary}
	函数项级数$\sum u_n(x)$在数集$D$上一致收敛的必要条件是函数列$\{u_n(x)\}$在$D$上一致收敛于零.
\end{corollary}
与数项级数类似,我们也可以定义函数项级数的余项.
\begin{definition}[余项]
	设函数项级数$\sum u_n(x)$在$D$上的和函数为$S(x)$,称
	$$R_n(x)=S(x)-S_n(x)$$
	为函数项级数$\sum u_n(x)$的{\heiti 余项}.
\end{definition}
\begin{theorem}
	函数项级数$\sum u_n(x)$在数集$D$上一致收敛于$S(x)$的充要条件是
	$$\lim\limits_{n\to\infty}\sup\limits_{x\in D}|R_n(x)|=\lim\limits_{n\to\infty}\sup\limits_{x\in D}|S(x)-S_n(x)|=0.$$
\end{theorem}
\begin{remark}
	这里将$\{S_n(x)\}$作为函数列即转化为定理\ref{chy}的结论.
\end{remark}
\subsection{函数项级数的一致收敛性判别法}
\begin{theorem}[Weierstrass判别法]
	设函数项级数$\sum u_n(x)$定义在数集$D$上,$\sum M_n$为收敛的正项级数,若对一切$x\in D$,有
	$$|u_n(x)|\leqslant M_n,\quad n=1,2,\cdots,$$
	则函数项级数$\sum u_n(x)$在$D$上一致收敛.
\end{theorem}
\begin{proof}
	由假设,$\sum M_n$收敛,由数项级数的Cauchy准则,对任意$\varepsilon>0$,存在$N>0$,使得当$n>N$时,对一切正整数$p$,都有
	$$|M_{n+1}+\cdots+M_{n+p}|=M_{n+1}+\cdots+M_{n+p}<\varepsilon.$$
	对一切$x\in D$,有
	\begin{align*}
		|u_{n+1}(x)+\cdots+u_{n+p}(x)|\leqslant |u_{n+1}(x)|+\cdots+|u_{n+p}(x)|\leqslant M_{n+1}+\cdots+M_{n+p}<\varepsilon.
	\end{align*}
	根据函数项级数一致收敛的Cauchy准则,级数$\sum u_n(x)$在$D$上一致收敛.$\hfill\blacksquare$
\end{proof}
\begin{remark}
	上述判别法也称为$M${\heiti 判别法}或{\heiti 优级数判别法}.当级数$\sum u_n(x)$和级数$\sum M_n$在区间$\left[a,b\right]$上成立
	$$|u_n(x)|\leqslant M_n,\quad n=1,2,\cdots$$
	时,称级数$\sum M_n$在区间$\left[a,,b\right]$上优于级数$\sum u_n(x)$,或称$\sum M_n$为$\sum u_n(x)$的{\heiti 优级数}.
\end{remark}
下面讨论定义在区间$I$上形如
$$\sum u_n(x)v_n(x)=u_1(x)v_1(x)+u_2(x)v_2(x)+\cdots+u_n(x)v_n(x)+\cdots$$
的函数项级数的一致收敛性判别法. 与数项级数一样,我们介绍Abel判别法和Dirichlet判别法.
\begin{theorem}[Abel判别法]
	设
	\begin{enumerate}[(1)]
		\item $\sum u_n(x)$在区间$I$上一致收敛;
		\item 对于每一个$x\in I$,$\{v_n(x)\}$是单调的;
		\item $\{v_n(x)\}$在$I$上一致有界,即存在$M>0$,使得对一切$x\in I$和正整数$n$,有
		$$|v_n(x)|\leqslant M,$$
	\end{enumerate}
	则级数$\sum u_n(x)v_n(x)$在$I$上一致收敛.
\end{theorem}
\begin{proof}
	由(1),任给$\varepsilon>0$,存在$N>0$,使得当$n>N$时,对一切正整数$p$和一切$x\in I$,有
	$$|u_{n+1}(x)+\cdots+u_{n+p}(x)|<\varepsilon.$$
	又由(2)(3)及Abel引理,得
	$$|u_{n+1}(x)v_{n+1}(x)+\cdots+u_{n+p}(x)v_{n+p}(x)|\leqslant (|v_{n+1}(x)|+2|v_{n+p}(x)|)\varepsilon\leqslant 3M\varepsilon.$$
	由函数项级数一致收敛的Cauchy准则即得.$\hfill\blacksquare$
\end{proof}
\begin{theorem}[Dirichlet判别法]
	设
	\begin{enumerate}[(1)]
		\item $\sum u_n(x)$的部分和函数列
		$$S_n(x)=\sum_{k=1}^{n}u_k(x)\quad (n=1,2,\cdots)$$
		在$I$上一致有界;
		\item 对于每一个$x\in I$,$\{v_n(x)\}$是单调的;
		\item 在$I$上$v_n(x)\rightrightarrows 0\ (n\to\infty)$,
	\end{enumerate}
	则级数$\sum u_n(x)v_n(x)$在$I$上一致收敛.
\end{theorem}
\begin{proof}
	由(1),存在$M>0$,对一切$x\in I$,有$S_n(x)\leqslant M$. 因此当$n,p$为任何正整数时,
	$$|u_{n+1}(x)+\cdots+u_{n+p}(x)|=|S_{n+p}(x)-S_n(x)|\leqslant 2M.$$
	对任何一个$x\in I$,再由(2)及Abel引理,得
	$$|u_{n+1}(x)v_{n+1}(x)+\cdots+u_{n+p}(x)v_{n+p}(x)|\leqslant 2M(|v_{n+1}(x)|+2|v_{n+p}(x)|).$$
	再由(3),对任意$\varepsilon>0$,存在$N>0$使得当$n>N$时,对一切$x\in I$,有
	$$|v_n(x)|<\varepsilon,$$
	所以
	$$|u_{n+1}(x)v_{n+1}(x)+\cdots+u_{n+p}(x)v_{n+p}(x)|<2M(\varepsilon+2\varepsilon)=6M\varepsilon.$$
	由函数项级数一致收敛的Cauchy准则即得.$\hfill\blacksquare$
\end{proof}
\section{一致收敛函数列与函数项级数的性质}
本节讨论由函数列和函数项级数所确定的函数的连续性、可积性和可微性.
\begin{theorem}[极限可交换性]\label{limcommutative}
	设函数列$\{f_n\}$在$(a,x_0)\cup(x_0,b)$上一致收敛于$f(x)$,且对每个$n$,$\lim\limits_{x\to x_0}f_n(x)=a_n$,则$\lim\limits_{n\to\infty}a_n$和$\lim\limits_{x\to x_0}f(x)$均存在且相等.
\end{theorem}
\begin{proof}
	先证$\{a_n\}$是收敛数列. 对任意$\varepsilon>0$,由于$\{f_n\}$一致收敛,故有$N>0$,当$n>N$时,对任意正整数$p$和一切$x\in(a,x_0)\cup(x_0,b)$,有
	$$|f_n(x)-f_{n+p}(x)|<\varepsilon.$$
	从而
	$$|a_n-a_{n+p}|=\lim\limits_{x\to x_0}|f_n(x)-f_{n+p}(x)|\leqslant\varepsilon.$$
	这样由Cauchy准则可知$\{a_n\}$是收敛数列.
	
	\hspace*{\fill}
	
	设$\lim\limits_{n\to\infty}a_n=A$. 再证$\lim\limits_{x\to x_0}f(x)=A$.
	
	\hspace*{\fill}
	
	由于$f_n(x)$一致收敛于$f(x)$及$a_n$收敛于$A$,因此对任意$\varepsilon>0$,存在正数$N$,当$n>N$时,对任意$x\in (a,x_0)\cup(x_0,b)$,
	$$|f_n(x)-f(x)|<\frac{\varepsilon}{3}\quad\text{和}\quad |a_n-A|<\frac{\varepsilon}{3}$$
	同时成立. 特别取$n=N+1$,有
	$$|f_{N+1}(x)-f(x)|<\frac{\varepsilon}{3},\quad |a_{N+1}-A|<\frac{\varepsilon}{3}.$$
	又$\lim\limits_{x\to x_0}f_{N+1}(x)=a_{N+1}$,故存在$\delta>0$,当$0<|x-x_0|<\delta$时,
	$$|f_{N+1}(x)-a_{N+1}|<\frac{\varepsilon}{3}.$$
	这样,当$x$满足$0<|x-x_0|<\delta$时,
	$$|f(x)-A|\leqslant |f(x)-f_{N+1}(x)|+|f_{N+1}(x)-a_{N+1}|+|a_{N+1}-A|<\frac{\varepsilon}{3}+\frac{\varepsilon}{3}+\frac{\varepsilon}{3}=\varepsilon.$$
	即$\lim\limits_{x\to x_0}f(x)=A$.$\hfill\blacksquare$
\end{proof}
\begin{remark}
	这个定理指出:在一致收敛的条件下,$\{f_n(x)\}$中两个独立变量$x$与$n$,在分别求极限时其求极限的顺序可以交换,即
	$$\lim\limits_{x\to x_0}\lim\limits_{n\to\infty}f_n(x)=\lim\limits_{n\to\infty}\lim\limits_{x\to x_0}f_n(x).$$
\end{remark}
\begin{remark}
	类似地,对于$f$中$x$的左极限和右极限,我们也可交换其与$n$求极限的顺序,即在一致收敛和相应极限存在的条件下,有
	$$\lim\limits_{x\to a^+}\lim\limits_{n\to\infty}f_n(x)=\lim\limits_{n\to\infty}\lim\limits_{x\to a^+}f_n(x),$$
	$$\lim\limits_{x\to b^-}\lim\limits_{n\to\infty}f_n(x)=\lim\limits_{n\to\infty}\lim\limits_{x\to b^-}f_n(x).$$
\end{remark}
由上述定理可得到极限函数的连续性定理.
\begin{theorem}[连续性]
	若函数列$\{f_n\}$在区间$I$上一致收敛,且每一项都连续,则其极限函数$f$在$I$上也连续.
\end{theorem}
\begin{proof}
	设$x_0$为$I$上任一点. 由于$\lim\limits_{x\to x_0}f_n(x)=f_n(x_0)$,于是由定理$\ref{limcommutative}$知$\lim\limits_{x\to x_0}f(x)$也存在,且
	$$\lim\limits_{x\to x_0}f(x)=\lim\limits_{x\to x_0}\lim\limits_{n\to\infty}f_n(x)=\lim\limits_{n\to\infty}\lim\limits_{x\to x_0}f_n(x)=\lim\limits_{n\to\infty}f_n(x_0)=f(x_0),$$
	因此$f(x)$在$x_0$上连续.$\hfill\blacksquare$
\end{proof}
\begin{remark}
	我们可以得到极限函数连续性定理的逆否命题. 即若各项为连续函数的函数列在区间$I$上其极限函数不连续,则此函数列在区间$I$上不一致收敛.
\end{remark}
\begin{remark}
	一致收敛性是极限函数连续的充分条件,而非必要条件.
\end{remark}
由于函数的连续性是函数的局部性质,因此上述定理中的一致收敛条件可以减弱为内闭一致收敛.
\begin{corollary}
	若连续函数列$\{f_n\}$在区间$I$上内闭一致收敛于$f$,则$f$在$I$上连续.
\end{corollary}
由连续性定理我们可以进一步证明可积性定理.
\begin{theorem}[可积性]
	若函数列$\{f_n\}$在$\left[a,b\right]$上一致收敛,且每一项都连续,则
	$$\int_{a}^{b}\lim\limits_{n\to\infty}f_n(x)\d x=\lim\limits_{n\to\infty}\int_{a}^{b}f_n(x)\d x.$$
\end{theorem}
\begin{proof}
	设$f$为函数列$\{f_n\}$在$\left[a,b\right]$上的极限函数. 由极限函数的连续性定理,$f$在$\left[a,b\right]$上连续,从而$f_n(n=1,2,\cdots)$与$f$在$\left[a,b\right]$上都可积.
	
	因为在$\left[a,b\right]$上$f_n\rightrightarrows f\ (n\to\infty)$,故对任意$\varepsilon>0$,存在$N$,当$n>N$时,对一切$x\in \left[a,b\right]$,都有
	$$|f_n(x)-f(x)|<\varepsilon.$$
	再根据定积分的性质,当$n>N$时有
	$$\left|\int_{a}^{b}f_n(x)\d x-\int_{a}^{b}f(x)\d x\right|\leqslant\int_{a}^{b}|f_n(x)-f(x)|\d x<\varepsilon(b-a).$$
	即
	$$\lim\limits_{n\to\infty}\int_{a}^{b}f_n(x)\d x=\int_{a}^{b}f_n(x)\d x=\int_{a}^{b}\lim\limits_{n\to\infty}f_n(x)\d x.$$
	$\hfill\blacksquare$
\end{proof}
\begin{remark}
	这个定理指出:在一致收敛的条件下,极限运算与积分运算的顺序可以交换.
\end{remark}
\begin{remark}
	一致收敛性是极限运算与积分运算交换的充分条件,而不是必要条件.
\end{remark}
由可积性定理我们可以进一步证明可微性定理.
\begin{theorem}[可微性]
	设$\{f_n\}$为定义在$\left[a,b\right]$上的函数列,若$x_0\in\left[a,b\right]$为$\{f_n\}$的收敛点,$\{f_n\}$的每一项在$\left[a,b\right]$上有连续的导数,且$\{f'_n\}$在$\left[a,b\right]$上一致收敛,则
	$$\frac{\d}{\d x}\left(\lim\limits_{n\to\infty}f_n(x)\right)=\lim\limits_{n\to\infty}\frac{\d}{\d x}f_n(x).$$
\end{theorem}
\begin{proof}
	设$f_n(x_0)\to A\ (n\to\infty)$,$f'_n\rightrightarrows g\ (n\to\infty),\ x\in\left[a,b\right]$. 我们要证明函数列$\{f_n\}$在区间$\left[a,b\right]$上收敛,且其极限函数的导数存在且等于$g$.
	
	由定理条件,对任一$x\in\left[a,b\right]$,总有
	$$f_n(x)=f_n(x_0)+\int_{x_0}^{x}f'_n(t)\d t.$$
	当$n\to\infty$时,右边第一项极限为$A$,由可积性定理,第二项极限为$\displaystyle\int_{x_0}^{x}g(t)\d t$,所以左边极限存在,记为$f$,则
	$$f(x)=\lim\limits_{n\to\infty}f_n(x)=f(x_0)+\int_{x_0}^{x}g(t)\d t,$$
	其中$f(x_0)=A$. 由$g$的连续性及微积分学基本定理有
	$$f'=g.$$
	$\hfill\blacksquare$
\end{proof}
与连续性类似,函数的可微性也是函数的局部性质,因此一致收敛条件可以减弱为内闭一致收敛.
\begin{corollary}
	设函数列$\{f_n\}$定义在区间$I$上,若$x_0\in I$为$\{f_n\}$的收敛点,且$\{f'_n\}$在$I$上内闭一致收敛,则$f$在$I$上可导,且$f'(x)=\lim\limits_{n\to\infty}f'_n(x)$.
\end{corollary}
下面讨论定义在区间$\left[a,b\right]$上的函数项级数$\sum u_n(x)$的连续性、逐项求积和逐项求导的性质,这些性质可由函数列的相应性质推出,故不再证明.
\begin{theorem}[连续性]
	若函数项级数$\sum u_n(x)$在区间$\left[a,b\right]$上一致收敛,且每一项都连续,则其和函数在$\left[a,b\right]$上也连续.
\end{theorem}
\begin{remark}
	这个定理指出,在一致收敛条件下,(无限项)求和运算与求极限运算可以交换顺序,即
	$$\sum\left(\lim\limits_{x\to x_0}u_n(x)\right)=\lim\limits_{x\to x_0}\left(\sum u_n(x)\right).$$
\end{remark}
\begin{theorem}[逐项求积]
	若函数项级数$\sum u_n(x)$在$\left[a,b\right]$上一致收敛,且每一项$u_n(x)$都连续,则
	$$\sum\int_{a}^{b}u_n(x)\d x=\int_{a}^{b}\sum u_n(x)\d x.$$
\end{theorem}
\begin{theorem}[逐项求导]
	若函数项级数$\sum u_n(x)$在$\left[a,b\right]$上每一项都有连续的导函数,$x_0\in\left[a,b\right]$为$\sum u_n(x)$的收敛点,且$\sum u'_n(x)$在$\left[a,b\right]$上内闭一致收敛,则
	$$\sum\left(\frac{\d}{\d x}u_n(x)\right)=\frac{\d}{\d x}(\sum u_n(x)).$$
\end{theorem}
\begin{remark}
	上述两个定理指出,在一致收敛条件下,逐项求积或求导后求和等于求和后再求积或求导.
\end{remark}
\begin{remark}
	与函数列的情况相同,函数项级数的连续性定理与逐项求导定理中的一致收敛条件也可减弱为内闭一致收敛.
\end{remark}
最后,我们指出,本节中六个定理的意义不只是检验函数列或函数项级数是否满足定理中的关系式,更重要的是根据定理的条件,即使没有求出极限函数或和函数,也能由函数列或函数项级数本身获得极限函数或和函数的解析性质.
\section{幂级数}
本章将讨论由幂函数序列$\{a_n(x-x_0)^n\}$所产生的函数项级数
\begin{equation}\label{mi1}
	\sum_{n=0}^{\infty}a_n(x-x_0)^n=a_0+a_1(x-x_0)+a_2(x-x_0)^2+\cdots+a_n(x-x_0)^n+\cdots,
\end{equation}
它称为{\heiti 幂级数},是一类最简单的函数项级数,从某种意义上说,它也可以看作是多项式函数的延伸.

下面将着重讨论$x_0=0$,即
\begin{equation}\label{mi2}
	\sum_{n=0}^{\infty}a_nx^n=a_0+a_1x+a_2x^2+\cdots+a_nx^n+\cdots
\end{equation}
的情形,因为只要把\ref{mi2}中的$x$换成$x-x_0$,就得到\ref{mi1}.
\subsection{幂级数的收敛区间}
首先讨论幂级数$\displaystyle\sum_{n=0}^{\infty}a_nx^n$的收敛性问题. 显然它在$x=0$处总是收敛的. 此外,我们有以下定理.
\begin{theorem}[Abel定理]
	若幂级数$\displaystyle\sum_{n=0}^{\infty}a_nx^n$在$x=\overline{x}$收敛,则它在区间$|x|<|\overline{x}|$中绝对收敛;反之,幂级数在$x=\overline{x}$处发散,则它在$|x|>|\overline{x}|$中均发散.
\end{theorem}
\begin{proof}
	设级数$\displaystyle\sum_{n=0}^{\infty}a_n\overline{x}^n$收敛,则存在$M$,对任意$n>1$,有
	$$|a_n\overline{x}^n|\leqslant M.$$
	于是
	$$\sum_{n=0}^{\infty}|a_nx^n|=\sum_{n=0}^{\infty}\left|a_n\overline{x}^n\cdot\frac{x^n}{\overline{x}^n}\right|=M\cdot\sum_{n=0}^{\infty}\left|\frac{x^n}{\overline{x}^n}\right|.$$
	因此当$|x|<|\overline{x}|$时,$\displaystyle\sum_{n=0}^{\infty}a_nx^n$绝对收敛.$\hfill\blacksquare$
\end{proof}

由上述定理,我们知道,幂级数$\displaystyle\sum_{n=0}^{\infty}a_nx^n$的收敛域是以原点为中心的区间. 若以$2R$表示区间的长度,则称$R$为幂级数的{\heiti 收敛半径},显然$R\geqslant 0$. 实际上,它就是使得幂级数收敛的那些收敛点的绝对值的上确界. 所以有
\begin{enumerate}[(1)]
	\item 当$R=0$时,幂级数仅在$x=0$处收敛;
	\item 当$R=+\infty$时,幂级数在$(-\infty,+\infty)$上收敛;
	\item 当$0<R<+\infty$时,幂级数在$(-R,R)$上收敛,在$(-\infty,-R)$和$(R,+\infty)$上都发散;至于在$x=\pm R$处,幂级数可能收敛也可能发散.
\end{enumerate}
我们称$(-R,R)$为幂级数的{\heiti 收敛区间}.

\begin{remark}
	对于收敛域和收敛区间的区别,收敛域是全体收敛点的集合,它可能是开区间,也可能是闭区间,而收敛区间就是$(-R,R)$,它是一个开区间.
\end{remark}
对于如何求得幂函数的收敛半径,我们有以下定理.
\begin{theorem}[Cauchy-Hadamard定理]
	对于幂级数$\displaystyle\sum_{n=0}^{\infty}a_nx^n$,设
	$$\rho=\varlimsup\limits_{n\to\infty}\sqrt[n]{|a_n|},$$
	则当
	\begin{enumerate}[(1)]
		\item $0<\rho<+\infty$时,收敛半径$R=\frac{1}{\rho}$;
		\item $\rho=0$时,$R=+\infty$;
		\item $\rho=+\infty$时,$R=0$.
	\end{enumerate}
\end{theorem}
\begin{proof}
	对于幂级数$\displaystyle\sum_{n=0}^{\infty}|a_nx^n|$,由于
	$$\varlimsup\limits_{n\to\infty}\sqrt[n]{|a_nx^n|}=\varlimsup\limits_{n\to\infty}\sqrt[n]{|a_n|}|x|=\rho |x|,$$
	根据级数的Cauchy根式判别法,当$\rho |x|<1$时,幂级数收敛;当$\rho |x|>1$时,幂级数发散. 于是,
	\begin{enumerate}[(1)]
		\item 当$0<\rho<+\infty$时,由$\rho |x|<1$得收敛半径$R=\dfrac{1}{\rho}$;
		\item 当$\rho=0$时,对任意$x$都有$\rho |x|<1$,所以$R=+\infty$;
		\item 当$\rho=+\infty$时,则对除$x=0$外的任何$x$都有$\rho |x|>1$,所以$R=0$.
	\end{enumerate}
	$\hfill\blacksquare$
\end{proof}
由D'Alembert比式判别法和Cauchy根式判别法的关系,我们知道,若$\lim\limits_{n\to\infty}\dfrac{|a_{n+1}|}{a_n}=\rho$,则有$\lim\limits_{n\to\infty}\sqrt[n]{|a_n|}=\rho$. 因此,我们也常用级数的D'Alembert判别法来推出幂级数的收敛半径.

下面讨论幂级数的一致收敛性问题.
\begin{theorem}
	若幂级数$\displaystyle\sum_{n=0}^{\infty}a_nx^n$的收敛半径为$R$,则它在它的收敛区间$(-R,R)$上内闭一致收敛.
\end{theorem}
\begin{proof}
	即证幂级数在它的收敛区间内的任一闭区间$\left[a,b\right]$上都一致收敛. 设$\overline{x}=\max\{|a|,|b|\}\in(-R,R)$,那么对于$\left[a,b\right]$上任一点$x$,都有
	$$|a_nx^n|\leqslant|a_n\overline{x}^n|.$$
	由于幂级数在点$\overline{x}$绝对收敛,由Weierstrass判别法可得它在$\left[a,b\right]$上一致收敛. 由$\left[a,b\right]$的任意性,可得幂级数在收敛区间上内闭一致收敛.$\hfill\blacksquare$
\end{proof}
\begin{theorem}
	若幂级数$\displaystyle\sum_{n=0}^{\infty}a_nx^n$的收敛半径为$R$,且在$x=R$(或$x=-R$)时收敛,则幂级数在$\left[0,R\right]$(或$\left[-R,0\right]$)上一致收敛.
\end{theorem}
\begin{proof}
	只需证$x=R$时收敛的情形,$x=-R$时收敛可类似证明.
	
	设幂级数$\displaystyle\sum_{n=0}^{\infty}a_nx^n$在$x=R$时收敛,对于$x\in\left[0,R\right]$有
	$$\sum_{n=0}^{\infty}a_nx^n=\sum_{n=0}^{\infty}a_nR^n\left(\frac{x}{R}\right)^n.$$
	已知级数$\displaystyle\sum_{n=0}^{\infty}a_nR^n$收敛,函数列$\left\{\left(\frac{x}{R}\right)^n\right\}$在$\left[0,R\right]$上递减且一致有界,即
	$$1\geqslant\frac{x}{R}\geqslant\left(\frac{x}{R}\right)^2\geqslant\cdots\geqslant\left(\frac{x}{R}\right)^n\geqslant\cdots\geqslant 0.$$
	故由函数项级数的Abel判别法可知幂级数在$\left[0,R\right]$上一致收敛.$\hfill\blacksquare$
\end{proof}

\subsection{幂级数的性质}
作为函数项级数的具体化,幂级数的性质也包括连续性、逐项求积和逐项求导. 下面我们先来看幂级数的连续性.
\begin{theorem}[连续性]
	幂级数$\displaystyle\sum_{n=0}^{\infty}a_nx^n$的和函数是$(-R,R)$上的连续函数;若幂级数$\displaystyle\sum_{n=0}^{\infty}a_nx^n$在收敛区间的左(右)端点上收敛,则其和函数在这一端点上右(左)连续.
\end{theorem}
\begin{proof}
	由函数项级数的连续性定理立即可得.$\hfill\blacksquare$
\end{proof}
在讨论幂级数的逐项求导和逐项求积之前,我们要确保逐项求导或求积后的幂级数的收敛区间与原幂级数的收敛区间的一致性. 即先来讨论
\begin{equation}\label{dif}
	\sum_{n=0}^{\infty}na_nx^{n-1}=a_1+2a_2x+3a_3x^2+\cdots+na_nx^{n-1}+\cdots
\end{equation}
与
\begin{equation}\label{int}
	\sum_{n=0}^{\infty}\frac{a_n}{n+1}x^{n+1}=a_0x+\frac{a_1}{2}x^2+\frac{a_2}{3}x^3+\cdots+\frac{a_n}{n+1}x^{n+1}+\cdots
\end{equation}
的收敛区间.
\begin{theorem}
	幂级数$\displaystyle\sum_{n=0}^{\infty}a_nx^n$与幂级数\ref{dif}、\ref{int}具有相同的收敛区间.
\end{theorem}
\begin{proof}
	这里只要证明$\displaystyle\sum_{n=0}^{\infty}a_nx^n$与\ref{dif}有相同的收敛区间就可以了,因为对\ref{int}逐项求导就得到$\displaystyle\sum_{n=0}^{\infty}a_nx^n$.
	
	设
	$$\varlimsup\limits_{n\to\infty}\sqrt[n]{|a_n|}=\rho,$$
	则
	$$\varlimsup\limits_{n\to\infty}\sqrt[n]{(n+1)|a_{n+1}|}=\varlimsup\limits_{n\to\infty}\sqrt[n]{|a_{n+1}|}=\rho.$$
	
	因此两个级数的收敛半径相同.$\hfill\blacksquare$	
\end{proof}
\begin{theorem}[任意次可微]
	设幂级数$\displaystyle\sum_{n=0}^{\infty}a_nx^n$收敛半径为$R$,则它的和函数$f(x)=\displaystyle\sum_{n=0}^{\infty}a_nx^n$在$(-R,R)$中任意次可微,且
	$$f^{(k)}=\sum_{n=k}^{\infty}n(n-1)\cdots(n-k+1)a_nx^{n-k}.$$
\end{theorem}
\begin{proof}
	以$k=1$为例. 首先,幂级数$\displaystyle\sum_{n=0}^{\infty}(a_nx^n)'=\displaystyle\sum_{n=1}^{\infty}na_nx^{n-1}$的收敛半径仍为$R$,故它在区间$(-R,R)$上内闭一致收敛. 由函数项级数逐项求导定理,$\displaystyle\sum_{n=0}^{\infty}a_nx^n$在任意闭区间$I\subset (-R,R)$中可微,且
	$$\left(\sum_{n=0}^{\infty}a_nx^n\right)'=\sum_{n=0}^{\infty}(a_nx^n)'=\sum_{n=1}^{\infty}na_nx^{n-1}.$$
	$f(x)$的高阶可微性的证明是完全类似的.$\hfill\blacksquare$
\end{proof}
\begin{corollary}[幂级数系数与各阶导数的关系]\label{relation}
	设$f(x)$是幂级数$\displaystyle\sum_{n=0}^{\infty}a_nx^n$在点$x=0$的某邻域上的和函数,则有
	$$a_n=\frac{f^{(n)}}{n!}\ (n=0,1,2,\cdots).$$
\end{corollary}
\begin{theorem}[逐项积分]
	设幂级数$\displaystyle\sum_{n=0}^{\infty}a_nx^n$的收敛半径$R\neq 0$,则
	$$\int_{0}^{x}\left(\sum_{n=0}^{\infty}a_nt^n\right)\d t=\sum_{n=0}^{\infty}\int_{0}^{x}a_nt^n\d t=\sum_{n=0}^{\infty}\frac{a_n}{n+1}x^{n+1},\quad \forall x\in(-R,R).$$
\end{theorem}
\begin{proof}
	不妨设$x>0$,则根据前面的讨论,$\displaystyle\sum_{n=0}^{\infty}a_nt^n$在$t\in\left[0,x\right]$中一致收敛,因此可以逐项积分.$\hfill\blacksquare$
\end{proof}
下面关于幂级数的求和与求极限运算次序的可交换性是数项级数和函数项级数相应结果的直接应用,这里不再证明.
\begin{theorem}[求和可交换性]
	设$\displaystyle\sum_{n=0}^{\infty}|a_{ij}|=s_j$,$\displaystyle\sum_{j=0}^{\infty}s_jx^j$在$(-R,R)$中收敛,则
	$$\sum_{i=0}^{\infty}\sum_{j=0}^{\infty}a_{ij}x^j=\sum_{j=0}^{\infty}\left(\sum_{i=0}^{\infty}a_{ij}\right)x^j,\quad x\in(-R,R).$$
\end{theorem}
\begin{theorem}[求极限可交换性]
	设$\lim\limits_{m\to\infty}a_{mn}=a_n$,$|a_{mn}|\leqslant A_n$. 如果$\displaystyle\sum_{n=0}^{\infty}A_nx^n$在$(-R,R)$中收敛,则
	$$\lim\limits_{m\to\infty}\sum_{n=0}^{\infty}a_{mn}x^n=\sum_{n=0}^{\infty}a_nx^n,\quad x\in(-R,R).$$
\end{theorem}
\subsection{幂级数的运算}
首先我们给出两个幂级数相等的定义.
\begin{definition}[幂级数相等]
	若两个幂级数在$x=0$的某邻域内有相同的和函数,则称这两个幂级数在该邻域内相等.
\end{definition}
\begin{theorem}
	若幂级数$\displaystyle\sum_{n=0}^{\infty}a_nx^n$与$\displaystyle\sum_{n=0}^{\infty}b_nx^n$在$x=0$的某邻域内相等,则它们同次幂项的系数相等,即
	$$a_n=b_n\ (n=0,1,2,\cdots).$$
\end{theorem}
\begin{proof}
	由推论\ref{relation}可直接得到.$\hfill\blacksquare$
\end{proof}
\begin{remark}
	进一步我们可推得,若幂级数的和函数为奇(偶)函数,则展开式中不出现偶(奇)次幂的项.
\end{remark}
\begin{theorem}[运算法则]
	设幂级数$\displaystyle\sum_{n=0}^{\infty}a_nx^n$与$\displaystyle\sum_{n=0}^{\infty}b_nx^n$的收敛半径分别为$R_a$和$R_b$,设$\lambda$为常数,$R=\min\{R_a,R_b\}$,则
	$$\lambda\sum_{n=0}^{\infty}a_nx^n=\sum_{n=0}^{\infty}\lambda a_nx^n,\ x\in(-R_a,R_a),$$
	$$\sum_{n=0}^{\infty}a_nx^n\pm\sum_{n=0}^{\infty}b_nx^n=\sum_{n=0}^{\infty}(a_n\pm b_n)x^n,\ x\in(-R,R),$$
	$$\left(\sum_{n=0}^{\infty}a_nx^n\right)\left(\sum_{n=0}^{\infty}b_nx^n\right)=\sum_{n=0}^{\infty}\left(\sum_{i+j=n}a_ib_j\right)x^n,\ x\in(-R,R).$$
\end{theorem}
\begin{proof}
	由数项级数的相应性质可直接推出.$\hfill\blacksquare$
\end{proof}
下面我们介绍幂级数的除法运算,并拓展一些应用与示例.
\begin{theorem}[幂级数的除法]
	内容...
\end{theorem}
\subsubsection{Bernoulli数}
\subsubsection{Euler数}
\section{函数的幂级数展开}
\section{母函数方法}
\section{用级数构造函数}
\subsection{处处连续但处处不可导的函数}
\subsection{Peano曲线}
\subsection{光滑函数的Taylor展开的系数可以为任意实数列}

\newpage

\part{多元函数微积分}

\chapter{平面点集与多元函数}
\section{$n$元函数}
所有有序实数组$(x_1,x_2,\cdots,x_n)$的全体称为$n${\heiti 维实向量空间},简称{\heiti $n$维空间},记作$\mathbb{R}^n$. 其中每个有序实数组$(x_1,x_2,\cdots,x_n)$称为$\mathbb{R}^n$中的一个点,$n$个实数$x_1,x_2,\cdots,x_n$是这个点的坐标.
\begin{definition}[$n$元函数]
	设$E$为$\mathbb{R}^n$中的点集,若有某个对应法则$f$,使$E$中每一点$P(x_1,x_2,\cdots,x_n)$都有唯一的一个实数$y$与之对应,则称$f$为定义在$E$上的$n$元函数,记作
	$$f:E\to\mathbb{R},$$
	$$P\mapsto y.$$
\end{definition}
常把$n$元函数简写成
$$y=f(x_1,x_2,\cdots,x_n),\quad (x_1,x_2,\cdots,x_n)\in E$$
或
$$y=f(P),\quad P\in E.$$
其中,后一种写法被称为“点函数”,它使多元函数和一元函数在形式上尽量保持一致,以便仿照一元函数的办法来处理多元函数中的许多问题. 对于多元函数,我们将着重讨论二元函数. 在掌握了二元函数的有关理论和研究方法之后,我们可以把它推广到一般的多元函数中去. 下面我们给出二元函数的定义.
\section{二元函数}
\begin{definition}[二元函数]
	设平面点集$D\subset\mathbb{R}^2$,若按照某对应法则$f$,$D$中每一点$P(x,y)$都有唯一确定的实数$z$与之对应,则称$f$为定义在$D$上的{\heiti 二元函数},记作
	$$f:D\to\mathbb{R},$$
	$$P\mapsto z.$$
	称$D$为$f$的定义域,$P\in D$所对应的$z$为$f$在点$P$的函数值,记作$z=f(P)$或$z=f(x,y)$. 全体函数值的集合为$f$的值域,记作$f(D)\subset\mathbb{R}$. 通常还把$P$的坐标$x$与$y$称为$f$的{\heiti 自变量},把$z$称为{\heiti 因变量}.
\end{definition}
在映射意义下,上述$z=f(P)$称为$P$的{\heiti 象},$P$称为$z$的{\heiti 原象}. 当把$(x,y)\in D$和它所对应的象$z=f(x,y)$一起组成三维数组$(x,y,z)$时,三维Euclid空间$\mathbb{R}^3$中的点集
$$S=\{(x,y,z)|z=f(x,y),(x,y)\in D\}\subset\mathbb{R}^3$$
便是二元函数$f$的{\heiti 图像}. 通常$z=f(x,y)$的图像是一空间曲面,$f$的定义域$D$便是该曲面在$xOy$平面上的投影.

若二元函数的值域是有界数集,则称该函数为{\heiti 有界函数},若值域是无界数集,则称该函数为{\heiti 无界函数}.

\chapter{多元函数的极限与连续}
\section{重极限}
\begin{definition}[重极限]
	设$f$为定义在$D\subset\mathbb{R}^2$上的二元函数,$P_0$为$D$的一个聚点,$A$是一个确定的实数. 若对任意$\varepsilon>0$,存在$\delta>0$,使得当$P\in\mathring{U}(P_0;\delta)\cap D$时,都有
	$$|f(P)-A|<\varepsilon,$$
	则称$f$在$D$上当$P\to P_0$时以$A$为极限,记作
	$$\lim\limits_{P\to P_0\atop P\in D}f(P)=A.$$
	在对于$P\in D$不致产生误解时,也可简单写作
	$$\lim\limits_{P\to P_0}f(P)=A.$$
	当$P,P_0$分别用坐标$(x,y),(x_0,y_0)$表示时,也写作
	$$\lim\limits_{(x,y)\to(x_0,y_0)}f(x,y)=A.$$
\end{definition}
\chapter{多元函数微分学}
\chapter{多元函数积分学}
\chapter{曲线积分与曲面积分}




\end{document}