\documentclass[12pt]{ctexart}
\usepackage{amsfonts,amssymb,amsmath,amsthm,geometry,enumerate}
\usepackage[colorlinks,linkcolor=blue,anchorcolor=blue,citecolor=green]{hyperref}
\usepackage[all]{xy}
%introduce theorem environment
\theoremstyle{definition}
\newtheorem{definition}{定义}
\newtheorem{theorem}{定理}
\newtheorem{lemma}{引理}
\newtheorem{corollary}{推论}
\newtheorem{property}{性质}
\newtheorem{example}{例}
\theoremstyle{plain}
\newtheorem*{solution}{解}
\newtheorem*{remark}{注}
\geometry{a4paper,scale=0.8}

%article info
\title{\vspace{-2em}\textbf{映射与运算}\vspace{-2em}}
\date{ }
\begin{document}
	\maketitle
	\begin{definition}[映射]
		设集合$A$,$B$非空,若$A$中的任一元素都能通过某一对应法则$f$唯一地对应到$B$中的一个元素,则称$f$是从$A$到$B$的\textbf{映射},记作$f:A\to B$. 设$x\in A$,则$y\in B$,称$y$是$x$在$f$下的\textbf{像},记作$f(x)$;称$x$是$y$在$f$下的\textbf{原像},记作$f^{-1}(y)$.
	\end{definition}
	\begin{definition}[单射]
		设映射$f:A\to B$,对任意$x_1,x_2\in A,x_1\neq x_2$,有$f(x_1)\neq f(x_2)$,则称$f$是\textbf{单射}.
	\end{definition}
	不同元素在单射下的像也不同.
	\begin{definition}[满射]
		设映射$f:A\to B$,对任意$y\in B$,都存在原像$f^{-1}(y)$,则称$f$是\textbf{满射}.
	\end{definition}
	\begin{definition}[双射]
		若映射$f$既是单射,又是满射,则称$f$是\textbf{双射}.
	\end{definition}
	双射$f:A\to B$是$A$中元素与$B$中元素的一一对应.
	\begin{definition}[恒等映射]
		设映射$i:A\to A$,$i(x)=x$,则称$i$为\textbf{恒等映射}.
	\end{definition}
	\begin{definition}[嵌入映射]
		设非空集合$A_0\subset A$,映射$f:A_0\to A$,$i(x)=x$,则称$i$是嵌入映射.
	\end{definition}
	嵌入映射有扩大值域的作用.
	\begin{definition}[开拓与限制]
		设非空集合$A_0\subset A$,映射$f:A_0\to B$,$g:A\to B$,对任意$x\in A_0$,有$f(x)=g(x)$,则称$g$是$f$的\textbf{开拓},$f$是$g$的\textbf{限制}.
	\end{definition}
	开拓映射有扩大定义域的作用,限制映射有缩小定义域的作用.
	\begin{definition}[映射的复合]
		设映射$f:A\to B$,$g:B\to C$,则定义复合映射:$g\circ f=g\left(f\left(x\right)\right):A\to C$.
	\end{definition}
	可以用交换图表示这个过程.
	\begin{displaymath}
		\xymatrix{
		A\ar[r]^f \ar[dr]_{g\circ f} & B\ar[d]^g\\
									 & C\\
		}
	\end{displaymath}
	\begin{definition}[直积]
		设非空集合$A$,$B$,定义$A$与$B$的\textbf{直积}$A\times B=\left\{(a,b)\ |\ \forall a\in A,b\in B\right\}$.
	\end{definition}
	\begin{definition}[代数运算]
		设非空集合$A$,$B$,$D$,称映射$A\times B\to D$为$A$与$B$到$D$的一个\textbf{代数运算}. 即对任意$a\in A$,$b\in B$,都有唯一的$d\in D$满足$f(a,b)=d$,记作$a\circ b=d$.
	\end{definition}
	\begin{definition}[二元运算]
		设$A$为非空集合. 代数运算$f:A\times A\to A$称为\textbf{二元运算},简称\textbf{运算}.
	\end{definition}
	在二元运算下,$A$中的元素经过运算仍在$A$中,于是二元运算满足\textbf{封闭性}.
	\begin{definition}[结合性]
		若在非空集合$A$上定义了一种运算“$\circ$”,对任意$a,b,c\in A$,都有
		$$(a\circ b)\circ c=a\circ(b\circ c),$$
		则称运算“$\circ$”是\textbf{结合的}.
	\end{definition}
	\begin{definition}[交换性]
		若在非空集合$A$上定义了一种运算“$\circ$”,对任意$a,b\in A$,都有
		$$a\circ b=b\circ a,$$
		则称运算“$\circ$”是\textbf{交换的}.
	\end{definition}
	习惯上将运算“$\circ$”称为乘法,并略去不写. 要注意这里的乘法是一种抽象的运算,与数的乘法不是一回事. 那么,结合性可以写为$(ab)c=a(bc)$,交换性可以写为$ab=ba$.
	\begin{definition}[分配性]
		若在非空集合$A$上定义了两种运算“$\circ$”和“$+$”,分别称为乘法和加法,对任意$a,b,c\in A$,若有
		$$a\circ(b+c)=(a\circ b)+(a\circ c),$$
		则称乘法对加法是\textbf{左分配的},若有
		$$(a+b)\circ c=(a\circ c)+(b\circ c),$$
		则称乘法对加法是\textbf{右分配的}. 若乘法对加法既是左分配的,又是右分配的,则称乘法对加法是\textbf{分配的}.
	\end{definition}
	与乘法类似,这里的加法也是一种抽象的运算,不能简单看作数的加法.
\end{document}