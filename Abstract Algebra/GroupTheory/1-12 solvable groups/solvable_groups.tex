\documentclass[12pt]{ctexart}
\usepackage{amsfonts,amssymb,amsmath,amsthm,geometry,enumerate}
\usepackage[colorlinks,linkcolor=blue,anchorcolor=blue,citecolor=green]{hyperref}
\usepackage[all]{xy}
%introduce theorem environment
\theoremstyle{definition}
\newtheorem{definition}{定义}
\newtheorem{theorem}{定理}
\newtheorem{lemma}{引理}
\newtheorem{corollary}{推论}
\newtheorem{property}{性质}
\newtheorem{proposition}{命题}
\newtheorem{example}{例}
\theoremstyle{plain}
\newtheorem*{solution}{解}
\newtheorem*{remark}{注}
\geometry{a4paper,scale=0.8}

\newcommand{\id}{\mathrm{id}}
\newcommand{\Aut}{\mathrm{Aut}}
\newcommand{\Inn}{\mathrm{Inn}}
\newcommand{\Orb}{\mathrm{Orb}}
\newcommand{\Stab}{\mathrm{Stab}}

\everymath{\displaystyle}

%article info
\title{\vspace{-2em}\textbf{可解群与幂零群}\vspace{-2em}}
\date{ }
\begin{document}
	\maketitle
	\begin{definition}[换位子]
		设$g_1,g_2\in G$,称
		$$\left[g_1,g_2\right]=g_1^{-1}g_2^{-1}g_1g_2$$
		为$g_1$和$g_2$的\textbf{换位子}.
	\end{definition}
	可见,换位子的作用是换位,即$g_2g_1\left[g_1,g_2\right]=g_1g_2$.且有
	$$\left[g_1,g_2\right]\left[g_2,g_1\right]=e.$$
	即$\left[g_2,g_1\right]=\left[g_1,g_2\right]^{-1}$.
	\begin{definition}[换位子群]
		设$H<G$,$K<G$,称
		$$\left[H,K\right]=\langle \left\{\left[h,k\right]\mid h\in H,k\in K\right\}\rangle$$
		为$H$和$K$的\textbf{换位子群}.
	\end{definition}
	可见,$\left[H,K\right]=\left[K,H\right]$.
	\begin{property}
		设$\alpha:G\to G_1$是同态,则
		\begin{enumerate}
			\item 对任意$g_1,g_2\in G$,$\alpha\left(\left[g_1,g_2\right]\right)=\left[\alpha(g_1),\alpha(g_2)\right]$.
			\item 对任意$H<G$,$K<G$,$\alpha\left(\left[H,K\right]\right)=\left[\alpha(H),\alpha(K)\right]$.
		\end{enumerate}
	\end{property}
	\begin{lemma}
		设$H<G$,$K<G$,则
		\begin{enumerate}
			\item $\left[H,K\right]=\{1\}\iff H\subset C_G(K)$;
			\item $\left[H,K\right]\subset K\iff H\subset N_G(K)$;
			\item 若$H\lhd G$,$G\lhd G$,则$\left[H,K\right]\lhd G$且$\left[H,K\right]\subset H\cap K$;
			\item 若$H_1<H$,$K_1<K$,则$\left[H_1,K_1\right]<\left[H,K\right]$.
		\end{enumerate}
	\end{lemma}
	\begin{corollary}
		设$H<G$,$K<G$,则
		\begin{enumerate}
			\item $G$是交换群当且仅当$\left[G,G\right]=\{1\}$;
			\item $K\lhd G\iff \left[K,K\right]\lhd G$;
			\item $\left[G,G\right]\lhd G$.
		\end{enumerate}
	\end{corollary}
	\begin{definition}[正规列]
		设群$G$的幺元为$1$,它的子群$G_i$有如下排列
		$$G=G_1\supset G_2\supset\cdots\supset G_t\supset G_{t+1}=\{1\},$$
		且$G_i\lhd G_{i-1},2\leqslant i\leqslant t+1$,则称这个序列为\textbf{次正规列}.若还有$G_i\lhd G$,则称这个序列为\textbf{正规列}.上述序列中有$t$个包含号,所以称序列的长度为$t$.称$G_i/G_{i-1}$为次正规序列的\textbf{因子}.
	\end{definition}
	\begin{definition}[因子列]
		次正规序列
		$$G=G_1\supset G_2\supset\cdots\supset G_t\supset G_{t+1}=\{1\}$$
		的因子
		$$G_1/G_2,G_2/G_3,\cdots,G_t/G_{t+1}$$
		称为次正规序列的\textbf{因子列}.
	\end{definition}
	\begin{remark}
		因子列没有包含关系.
	\end{remark}
	\begin{definition}[加细]
		设有两个次正规序列
		$$G=G_1'\supset G_2'\supset\cdots\supset G_r'\supset G_{r+1}'=\{1\},$$
		$$G=G_1\supset G_2\supset\cdots\supset G_t\supset G_{t+1}=\{1\},$$
		若对任意$G_i'$,都有$G_j=G_i'$,则称后者是前者作为次正规序列的\textbf{加细}.
	\end{definition}
	\begin{remark}
		正规序列的加细也有类似的定义.若正规序列加细后仍是正规序列,则称后者是前者作为正规序列的加细.但是,若正规序列加细后不再是正规序列,则把这个正规序列看作次正规序列,后者是前者作为次正规序列的加细.
	\end{remark}
	\begin{definition}[导出列]
		定义$G^{(0)}=G$,$G^{(i)}=\left[G^{(i-1)},G^{(i-1)}\right],i\geqslant1$,称序列
		$$G=G^{(0)}\supset G^{(1)}\supset G^{(2)}\supset\cdots$$
		为$G$的\textbf{导出列}.
	\end{definition}
	\begin{definition}[降中心列]
		定义$\varGamma_1(G)=G$,$\varGamma_i(G)=\left[G,\varGamma_{i-1}(G)\right],i\geqslant 2$,称序列
		$$G=\varGamma_1(G)\supset\varGamma_2(G)\supset\cdots$$
		为$G$的\textbf{降中心列}.
	\end{definition}
	\begin{definition}[升中心列]
		定义$C_0(G)=\{1\}$,$C_i(G)/C_{i-1}(G)=C(G/C_{i-1}(G)),i\geqslant 1$,称序列
		$$\{1\}=C_0(G)\subset C_1(G)\subset C_2(G)\subset\cdots$$
		为$G$的\textbf{升中心列}.
	\end{definition}
	\begin{remark}
		$C_i(G)$是存在的,可以写成显性表达式.
	\end{remark}
	\begin{definition}[可解群,幂零群]
		设$G$是群,若有$k$,使$G^{(k)}=\{1\}$,则称$G$是\textbf{可解群}.若有$k$,使$\varGamma_k(G)=\{1\}$,则称$G$是\textbf{幂零群}.
	\end{definition}
\end{document}