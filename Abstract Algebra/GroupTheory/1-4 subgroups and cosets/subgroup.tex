\documentclass[12pt]{ctexart}
\usepackage{amsfonts,amssymb,amsmath,amsthm,geometry,enumerate}
\usepackage[colorlinks,linkcolor=blue,anchorcolor=blue,citecolor=green]{hyperref}
\usepackage[all]{xy}
%introduce theorem environment
\theoremstyle{definition}
\newtheorem{definition}{定义}
\newtheorem{theorem}{定理}
\newtheorem{lemma}{引理}
\newtheorem{corollary}{推论}
\newtheorem{property}{性质}
\newtheorem{proposition}{命题}
\newtheorem{example}{例}
\theoremstyle{plain}
\newtheorem*{solution}{解}
\newtheorem*{remark}{注}
\geometry{a4paper,scale=0.8}

%article info
\title{\vspace{-2em}\textbf{子群与陪集}\vspace{-2em}}
\date{ }
\begin{document}
	\maketitle
	\begin{definition}[子群]
		设$G$是群,若非空集合$H\subset G$且$H$是群,则称$H$是$G$的\textbf{子群},记作$H<G$.
	\end{definition}
	考虑极端,$G$本身和$\{e\}$都是$G$的子群,称为$G$的\textbf{平凡子群}.
	\begin{theorem}[子群的充要条件]\label{iff}
		设$H$是群$G$的非空子集,则$H<G$当且仅当对任意$a,b\in H$,$ab^{-1}\in H$.
	\end{theorem}
	\begin{proof}
		当$H<G$时,由$H$运算的封闭性以及存在逆元即得$ab^{-1}\in H$.
		
		若对任意$a,b\in H$,有$ab^{-1}\in H$,则
		
		对任意$a,a\in H$,有$e=aa^{-1}\in H$,幺元存在;
		
		对$e\in H$与任意$b\in H$,有$b^{-1}=eb^{-1}\in H$,逆元存在;
		
		对任意$a,b^{-1}\in H$,有$ab=a(b^{-1})^{-1}\in H$,满足封闭律;
		
		$H$中的运算继承$G$中的运算,满足结合律,于是$H<G$.
	\end{proof}
	\begin{theorem}
		设$H$是群$G$的非空有限子集,则$H<G$当且仅当$H$对$G$的运算封闭.
	\end{theorem}
	\begin{proof}
		必要性显然. 充分性:$H$对$G$的运算封闭,故对任意$a\in H$,$a^k\in H$,其中$k$为任意正整数.由于$H$是有限群,于是存在正整数$m>n$满足$a^m=a^n$,故$a^{m-n}=e\in H$. 当$m-n=1$时,$a=e$,于是$a^{-1}=e\in H$;当$m-n>1$时,$a^{m-n-1}a=aa^{m-n-1}=e$,$a^{-1}=a^{m-n-1}\in H$,逆元存在. 封闭性满足,结合律满足,于是$H<G$.
	\end{proof}
	\begin{theorem}
		若$H_1<G$,$H_2<G$,则$H_1\cap H_2<G$.
	\end{theorem}
	\begin{proof}
		设$a,b\in H_1\cap H_2$,则由定理\ref{iff},$ab^{-1}\in H_1$,$ab^{-1}\in H_2$,则$ab^{-1}\in H_1\cap H_2$,再用一次该定理,即得$H_1\cap H_2<G$.
	\end{proof}
	\begin{corollary}
		设$I$为指标集,若对任意$i\in I$,$H_i<G$,则$\bigcap_{i\in I}H_i<G$.
	\end{corollary}
	下面是一些例子.
	\begin{example}
		数域$F$上全体$n$阶可逆方阵关于矩阵乘法构成群,称为\textbf{一般线性群},记作$GL_n(F)$. 其中,行列式为$1$的$n$阶方阵关于矩阵乘法也构成群,称为\textbf{特殊线性群},记作$SL_n(F)$. 显然$SL_n(F)<GL_n(F)$.
	\end{example}
	\begin{example}
		给定$m\in\mathbb{Z}$,定义集合$m\mathbb{Z}=\left\{mn\ |\ \forall n\in\mathbb{Z}\right\}$,定义运算为整数加法,则$m\mathbb{Z}<\mathbb{Z}$.
	\end{example}
	\begin{theorem}
		$(\mathbb{Z},+)$的子群都是形如$m\mathbb{Z}$的.
	\end{theorem}
	\begin{proof}
		设$H<\mathbb{Z}$. 若$H=\{0\}$,则$H=0\mathbb{Z}$.
		
		假设$H\neq {0}$,则存在非零整数. 由于$H$是子群,若$a\in H$,则$-a \in H$,因此$H$中必存在正整数. 定义$m=\min\{a\in H|a>0\}$
		
		任取$n\in H$,设$n=qm+r$,其中$0\leqslant r<m$. 由于$m\in H$且$H$是子群,于是$r=n-qm\in H$. 但$0\leqslant r<m$,而$m$是$H$中最小的正整数,因此$r=0$,即$n=qm\in m\mathbb{Z}$,所以$H\subset\mathbb{Z}$.
		
		由于$m\in H$,且$H$是子群,对任意整数$k$,有$km\in H$,因此$\mathbb{Z}\subset H$.		
	\end{proof}
	\begin{definition}[陪集]
		设$H<G$,给定$a\in G$,集合$aH=\left\{ah|h\in H\right\}$称为以$a$为代表的\textbf{左陪集}. 类似地,集合$Ha=\left\{ha|h\in H\right\}$称为以$a$为代表元的\textbf{右陪集}.
	\end{definition}
	左陪集和右陪集的概念是对偶的,下面考虑左陪集的情形.
	\begin{theorem}
		设$H<G$,则$aRb\iff a^{-1}b\in H$确定了$G$中的等价关系$R$. $a$所在的等价类$\overline{a}$恰为以$a$为代表的左陪集$aH$.
	\end{theorem}
	\begin{proof}
		先证$R$为等价关系.
		
		自反性:$H$为子群,故$a^{-1}a=e\in H$,即$aRa$.
		
		对称性:若$aRb$,则$a^{-1}b\in H$,$b^{-1}a=\left(a^{-1}b\right)^{-1}\in H$,即$bRa$.
		
		传递性:若$aRb$,$bRc$,则$a^{-1}b\in H$,$b^{-1}c\in H$,$a^{-1}c=a^{-1}bb^{-1}c\in H$,即$aRc$.
		
		再证$\overline{a}=aH$. 对任意$b\in\overline{a}$,$a^{-1}b\in H$,$b=aa^{-1}b\in aH$,于是$\overline{a}\subset aH$;对任意$b\in aH$,存在$h\in H$使得$b=ah$. $a^{-1}b=a^{-1}ah=h\in H$,于是$aH\subset\overline{a}$.
	\end{proof}
	由于$R$是一个等价关系,于是全体左陪集$\{aH\}$为$G$的一个分类,称为\textbf{左陪集空间}. 由此可见左陪集空间是群$G$对关系$R$的商集,于是又把左陪集空间称为\textbf{左商集},记作$G/R$.
	\begin{proposition}\label{subgroup-bijection}
		映射$\varphi:H\to aH,h\mapsto ah$是双射.
	\end{proposition}
	\begin{proof}
		$aH$是由$H$得到的,$\varphi$是满射是显然的. 对任意$h_1\neq h_2\in H$,$ah_1\neq ah_2$(否则将违反消去律),故$\varphi$是单射.
	\end{proof}
	\begin{definition}[指数]
		左陪集空间中陪集的个数称为\textbf{指数},记作$\left[G:H\right]$.
	\end{definition}
	\begin{theorem}[Lagrange]
		设$G$为有限群,$H<G$,则
		$$|G|=\left[G:H\right]|H|.$$
	\end{theorem}
	\begin{proof}
		设$\left[G:H\right]=n$,$a_i$为每个左陪集的代表元. $G=\bigsqcup_{i=1}^{n}a_iH$,而由命题\ref{subgroup-bijection},$|a_iH|=|H|$,故$|G|=n|H|=\left[G:H\right]|H|$.
	\end{proof}
	\begin{corollary}
		设$G$是有限群,$K<H<G$,则$\left[G:K\right]=\left[G:H\right]\left[H:K\right]$.
	\end{corollary}
\end{document}