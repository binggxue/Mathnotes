\documentclass[12pt]{ctexart}
\usepackage{amsfonts,amssymb,amsmath,amsthm,geometry,enumerate}
\usepackage[colorlinks,linkcolor=blue,anchorcolor=blue,citecolor=green]{hyperref}
\usepackage[all]{xy}
%introduce theorem environment
\theoremstyle{definition}
\newtheorem{definition}{定义}
\newtheorem{theorem}{定理}
\newtheorem{lemma}{引理}
\newtheorem{corollary}{推论}
\newtheorem{property}{性质}
\newtheorem{proposition}{命题}
\newtheorem{example}{例}
\theoremstyle{plain}
\newtheorem*{solution}{解}
\newtheorem*{remark}{注}
\geometry{a4paper,scale=0.8}

\newcommand{\id}{\mathrm{id}}
\newcommand{\Aut}{\mathrm{Aut}}
\newcommand{\Inn}{\mathrm{Inn}}
%article info
\title{\vspace{-2em}\textbf{变换群与置换群}\vspace{-2em}}
\date{ }
\begin{document}
	\maketitle
	\begin{definition}[变换]
		设$A$是一个集合,映射$f:A\to A$称为\textbf{变换},即集合到自身的映射.
	\end{definition}
	\begin{definition}[变换群]
		集合$A$上所有的可逆变换组成的集合,关于映射的复合构成群,称为集合$A$的\textbf{全变换群},记作$S_A$. 全变换群的一个子群称为$A$的一个\textbf{变换群}.
	\end{definition}
	可以依定义验证$S_A$构成群. 可逆变换即双射,要求集合中的元素在变换前后是一一对应的.
	\begin{definition}[对称群]
		若集合$A$是含$n$个元素的有限集,$S_A$也称为$n$元\textbf{对称群},也记作$S_n$. $S_n$中的变换称为\textbf{置换}.
	\end{definition}
	\begin{definition}[置换群]
		对称群$S_n$中若干置换可以构成一个$S_n$的子群,称为\textbf{置换群}.
	\end{definition}
	由定义,对称群是最大的置换群.
	\begin{theorem}[Cayley]
		任何群都与一个变换群同构.
	\end{theorem}
	\begin{proof}
		设$G$是群,任意$a\in G$,定义$\varphi_a:G\to G,\ g\mapsto ag$. $g$在$\varphi_a$下的像$ag$是唯一的,所以$\varphi$是映射.
		
		由于$a^{-1}g\in G$,而$\varphi_a(a^{-1}g)=g$,也就是$G$中任何元素$g$都有原像$a^{-1}g$,所以$\varphi_a$是满射.
		
		对任意$g_1,g_2\in G$,若$\varphi_a(g_1)=\varphi_a(g_2)$,则$ag_1=ag_2$.由消去律有$g_1=g_2$,于是$\varphi_a$是单射. $\varphi_a$又是满射,所以是双射,即可逆映射.故$\varphi_a\in S_G$.
		
		设$T=\left\{\varphi_a\ |\ a\in G\right\}$,则$T\subset S_G$. 又因为$(\varphi_b)^{-1}=\varphi_{b^{-1}}$,则$\varphi_a(\varphi_b)^{-1}=\varphi_a\varphi_{b^{-1}}=\varphi_{ab^{-1}}\in T$,于是由子群的充要条件,有$T<S_G$,则$T$是$G$的一个变换群,下面证明$T\cong G$.
		
		设$f:G\to T,\ a\mapsto\varphi_a$,显然$f$是满映射.对任意$g_1,g_2\in G$,若$f(g_1)=f(g_2)$,则$\varphi_{g_1}=\varphi_{g_2}$,$\varphi_{g_1}(e)=\varphi_{g_2}(e)$,即$g_1=g_2$,所以$f$是单射.于是$f$是双射.
		
		对任意$a,b\in G$,$f(ab)=\varphi_{ab}=\varphi_a\varphi_b=f(a)(b)$,所以$f$是同构映射,$G\cong T$.
	\end{proof}
	\begin{corollary}
		任何有限群都与一个置换群同构.
	\end{corollary}
	下面介绍置换群相关内容. 设$\sigma\in S_n$,设$A=\left\{a_1,a_2,\cdots,a_n\right\}$,则置换$\sigma$可以表示为
	$$\sigma(A)=
	\begin{pmatrix}
		a_1 & a_2 & \cdots & a_n \\
		\sigma(a_1) & \sigma(a_2) & \cdots & \sigma(a_n)
	\end{pmatrix}
	$$
	其中,$\sigma(a_1),\sigma(a_2),\cdots,\sigma(a_n)$是$a_1,a_2,\cdots,a_n$的一个排列. 注意到一共有$n!$种不同的排列方式,于是$|S_n|=n!$. 特别地,若$\id(a_i)=a_i,i=1,2,\cdots,n$,则称$\id$为\textbf{恒等置换}.
	\begin{definition}[轮换]
		设$I_r=\left\{i_1,i_2,\cdots,i_r\right\}\subset\left\{a_1,a_2,\cdots,a_n\right\}=A$,置换$\sigma$满足
		$$\sigma(I_r)=
		\begin{pmatrix}
			i_1 & i_2 & \cdots & i_r \\
			i_2 & i_3 & \cdots & i_1
		\end{pmatrix}
		$$
		$$\sigma(A\backslash I_r)=\id(A\backslash I_r),$$
		则称$\sigma$为$\boldsymbol{r}$\textbf{-轮换},记作$\sigma=(i_1i_2\cdots i_r)$. $i_1,i_2,\cdots,i_r$称为轮换中的\textbf{文字},$r$称为轮换的\textbf{长}.
	\end{definition}
	特别地,当$r=2$时称为\textbf{对换},$r=1$时为恒等置换.
	
	\begin{proposition}
		$r$-轮换的阶为$r$.
	\end{proposition}
	\begin{proposition}
		$(i_1i_2\cdots i_r)=(i_2i_3\cdots i_1)=\cdots (i_ri_1\cdots i_{r-1})$.
	\end{proposition}
	上述两个命题都是显然的.
	\begin{definition}
		在$S_n$中,如果若干个轮换间无共同文字,则称它们是\textbf{不相交的轮换}.
	\end{definition}
	\begin{proposition}
		在$S_n$中不相交轮换的乘积可换.
	\end{proposition}
	\begin{proof}
		对于两个不相交的轮换$\sigma_1$和$\sigma_2$,$\sigma_1$作用在$\sigma_2$作用的文字上时是恒等置换,同理$\sigma_2$作用在$\sigma_1$作用的文字上时也是恒等置换,而恒等置换与置换的乘积是可换的,于是不相交轮换的乘积可换. 对于多个不相交的轮换,以此类推即可.
	\end{proof}
	\begin{theorem}\label{permutationtocycle}
		$S_n$中任一置换都可表为若干不相交轮换的乘积.
	\end{theorem}
	\begin{proof}
		设$a\in\left\{1,2,\cdots,n\right\}$,置换$\sigma$作用到$a$上得到一些不同的文字.
		$$a=\sigma^{0}(a),\sigma(a),\sigma^2(a),\cdots,$$
		假设$\sigma^m(a)$与前面某一文字$\sigma^k(a)$重复,那么$k=0$,否则$\sigma^{k-1}(a)=\sigma^{m-1}(a)$从而矛盾. 于是置换$\sigma$在$a$上的作用等同于轮换
		$$\sigma_1=(a\sigma(a)\sigma^2(a)\cdots\sigma^{m}(a)),$$
		下面考虑$b\in\left\{1,2,\cdots,n\right\}\backslash\left\{a,\sigma(a),\cdots\sigma^m(a)\right\}$,得到轮换
		$$\sigma_2=(b\sigma(b)\cdots\sigma^l(b)),$$
		这里$\sigma_1$和$\sigma_2$是不相交的轮换. 以此类推,可以通过有限次操作取遍$\left\{1,2,\cdots,n\right\}$中的元素. 于是任一置换可以表为若干不相交轮换的乘积.
	\end{proof}
	\begin{proposition}
		任一个$r$-轮换都可以写成$r-1$个对换的乘积.
	\end{proposition}
	\begin{proof}
		$(i_1i_2\cdots i_r)=(i_1i_r)(i_1i_{r-1})\cdots(i_1i_3)(i_1i_2)$.
	\end{proof}
	\begin{proposition}
		任一置换都可以表为一些对换的乘积,这些对换的表示不一定唯一,但对换个数的奇偶性不变.
	\end{proposition}
	\begin{proof}
		由定理\ref{permutationtocycle},任一置换可以表示为若干不相交轮换的乘积,而任一轮换可以写成对换的乘积,因此任一置换都可以表为一些对换的乘积. 对换的表示不唯一,因为对任一对换的乘积,乘以$(i_ji_k)(i_ki_j)$之后仍然不变. 对换的表示改变了,但对换个数的奇偶性没有变.
	\end{proof}
	对换的表示并不是置换的本质,对换个数的奇偶性才是,于是有奇置换与偶置换的概念.
	\begin{definition}[奇置换与偶置换]
		可以表为奇数个对换的乘积的置换称为\textbf{奇置换},可以表为偶数个对换的乘积的置换称为\textbf{偶置换}.
	\end{definition}
	下面是奇置换与偶置换的一些简单性质,这与整数的奇偶性可以类比.
	\begin{property}\label{oddeven}
		两个奇置换之积是偶置换,两个偶置换之积是奇置换. 奇置换与偶置换之积是奇置换,偶置换与奇置换之积是奇置换. 置换的逆不改变置换的奇偶性.
	\end{property}
	\begin{definition}[交错群]
		按照群的定义可以验证,$n$元偶置换全体对置换的乘法构成群,称为$n$元\textbf{交错群},记作$A_n$.
	\end{definition}
	\begin{proposition}
		$A_n\lhd S_n$,$|A_n|=n!/2$.
	\end{proposition}
	\begin{proof}
		对任意$\sigma\in A_n$,$\varphi\in S_n$,$\varphi\sigma\varphi^{-1}\in A_n$,因此$A_n\lhd S_n$. 而$A_n$中不是奇置换就是偶置换. 对任意$\sigma\in S_n$,映射$\sigma\to (1,2)\sigma$建立了一个奇置换与偶置换之间的双射,于是
		$|A_n|=n!/2$.
	\end{proof}
	\begin{proposition}
		设置换$\sigma=\sigma_1\sigma_2\cdots\sigma_n$表示为$n$个不相交的轮换的乘积,其中$\sigma_i$是$r_i$-轮换,则$\sigma$的阶为$\left[r_1,r_2,\cdots,r_n\right]$.
	\end{proposition}
	\begin{proof}
		设$|\sigma|=d$,$m=\left[r_1,r_2,\cdots,r_n\right]$,则通过展开即可得$\sigma^m=\id$,于是$d\mid m$. 
		
		已知$\sigma^d = \text{id}$,所以对每个$i$,有$\sigma_i^d = \text{id}$.而 $\sigma_i$是一个 $r_i$-轮换,其阶为 $r_i$,因此$r_i \mid d$. 所以$d$是$r_i$的公倍数,所以$m\mid d$.
	\end{proof}
	\begin{definition}[自同构群]
		群$G$到自身的同构映射称为它的一个\textbf{自同构},全体自同构组成的集合对映射的复合作成群,称为$G$的\textbf{自同构群},记作$\Aut G$.
	\end{definition}
	同构映射是双射,因此$\Aut G<S_G$.
	\begin{definition}[内自同构群]
		设$G$是群,给定$a\in G$,定义映射$\sigma_a:G\to G,\ g\mapsto aga^{-1}$,则映射$\sigma_a\in\Aut G$,称为由$a$决定的\textbf{内自同构}.记
		$$\Inn G=\left\{\sigma_a\ |\ a\in G\right\},$$
		则$\Inn G\lhd\Aut G$,称为$G$的\textbf{内自同构群}.
	\end{definition}
	\begin{proof}
		对任意$g\in G$,$(\sigma_{a^{-1}})\sigma_a(g)=a^{-1}aga^{-1}a=g$. 因此$\sigma_{a^{-1}}$是$\sigma_a$的逆映射,$\sigma_a$是双射.又对任意$g_1,g_2\in G$,
		$$\sigma_a(g_1g_2)=ag_1g_2a^{-1}=ag_1a^{-1}ag_2a^{-1}=\sigma_a(g_1)\sigma_a(g_2),$$
		于是$\sigma_a$是同构,$\sigma_a\in\Aut G$.
		
		于是$\Inn G\subset\Aut G$. 对任意$a,b\in G$,任意$g\in G$,有
		$$\sigma_a\sigma_b(g)=\sigma_a(bgb^{-1})=abgb^{-1}a^{-1}=(ab)g(ab)^{-1}=\sigma_{ab}(g)\in\Inn G,$$
		于是$\Inn G<\Aut G$. 对任意$\sigma_a\in\Inn G$,$\varphi\in\Aut G$,有
		$$\varphi\sigma_a\varphi^{-1}(g)=\varphi(a\varphi^{-1}(g)a^{-1})=\varphi(a)\varphi\varphi^{-1}(g)\varphi(a^{-1})=\sigma_{\varphi(a)}(g)\in\Inn G,$$
		故$\Inn G\lhd\Aut G$.
	\end{proof}
\end{document}