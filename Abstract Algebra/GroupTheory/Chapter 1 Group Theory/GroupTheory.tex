\documentclass[12pt]{ctexart}
\usepackage{amsfonts,amssymb,amsmath,amsthm,geometry,enumerate}
\usepackage[colorlinks,linkcolor=blue,anchorcolor=blue,citecolor=green]{hyperref}
\usepackage[all]{xy}
%introduce theorem environment
\theoremstyle{definition}
\newtheorem{definition}{定义}[section]
\newtheorem{theorem}{定理}[section]
\newtheorem{lemma}{引理}[section]
\newtheorem{corollary}{推论}[section]
\newtheorem{property}{性质}[section]
\newtheorem{proposition}{命题}[section]
\newtheorem{example}{例}[section]
\theoremstyle{plain}
\newtheorem*{solution}{解}
\newtheorem*{remark}{注}
\geometry{a4paper,scale=0.8}

\newcommand{\id}{\mathrm{id}}
\newcommand{\Aut}{\mathrm{Aut}}
\newcommand{\Inn}{\mathrm{Inn}}
\newcommand{\Orb}{\mathrm{Orb}}
\newcommand{\Stab}{\mathrm{Stab}}

\everymath{\displaystyle}

%article info
\title{\textbf{群论}}
\date{ }
\begin{document}
	\maketitle
	\tableofcontents
	\newpage
\section{映射与运算}
\begin{definition}[映射]
	设集合$A$,$B$非空,若$A$中的任一元素都能通过某一对应法则$f$唯一地对应到$B$中的一个元素,则称$f$是从$A$到$B$的\textbf{映射},记作$f:A\to B$. 设$x\in A$,则$y\in B$,称$y$是$x$在$f$下的\textbf{像},记作$f(x)$;称$x$是$y$在$f$下的\textbf{原像},记作$f^{-1}(y)$.
\end{definition}
\begin{definition}[单射]
	设映射$f:A\to B$,对任意$x_1,x_2\in A,x_1\neq x_2$,有$f(x_1)\neq f(x_2)$,则称$f$是\textbf{单射}.
\end{definition}
不同元素在单射下的像也不同.
\begin{definition}[满射]
	设映射$f:A\to B$,对任意$y\in B$,都存在原像$f^{-1}(y)$,则称$f$是\textbf{满射}.
\end{definition}
\begin{definition}[双射]
	若映射$f$既是单射,又是满射,则称$f$是\textbf{双射}.
\end{definition}
双射$f:A\to B$是$A$中元素与$B$中元素的一一对应.
\begin{definition}[恒等映射]
	设映射$i:A\to A$,$i(x)=x$,则称$i$为\textbf{恒等映射}.
\end{definition}
\begin{definition}[嵌入映射]
	设非空集合$A_0\subset A$,映射$f:A_0\to A$,$i(x)=x$,则称$i$是嵌入映射.
\end{definition}
嵌入映射有扩大值域的作用.
\begin{definition}[开拓与限制]
	设非空集合$A_0\subset A$,映射$f:A_0\to B$,$g:A\to B$,对任意$x\in A_0$,有$f(x)=g(x)$,则称$g$是$f$的\textbf{开拓},$f$是$g$的\textbf{限制}.
\end{definition}
开拓映射有扩大定义域的作用,限制映射有缩小定义域的作用.
\begin{definition}[映射的复合]
	设映射$f:A\to B$,$g:B\to C$,则定义复合映射:$g\circ f=g\left(f\left(x\right)\right):A\to C$.
\end{definition}
可以用交换图表示这个过程.
\begin{displaymath}
	\xymatrix{
		A\ar[r]^f \ar[dr]_{g\circ f} & B\ar[d]^g\\
		& C\\
	}
\end{displaymath}
\begin{definition}[直积]
	设非空集合$A$,$B$,定义$A$与$B$的\textbf{直积}$A\times B=\left\{(a,b)\ |\ \forall a\in A,b\in B\right\}$.
\end{definition}
\begin{definition}[代数运算]
	设非空集合$A$,$B$,$D$,称映射$A\times B\to D$为$A$与$B$到$D$的一个\textbf{代数运算}. 即对任意$a\in A$,$b\in B$,都有唯一的$d\in D$满足$f(a,b)=d$,记作$a\circ b=d$.
\end{definition}
\begin{definition}[二元运算]
	设$A$为非空集合. 代数运算$f:A\times A\to A$称为\textbf{二元运算},简称\textbf{运算}.
\end{definition}
在二元运算下,$A$中的元素经过运算仍在$A$中,于是二元运算满足\textbf{封闭性}.
\begin{definition}[结合性]
	若在非空集合$A$上定义了一种运算“$\circ$”,对任意$a,b,c\in A$,都有
	$$(a\circ b)\circ c=a\circ(b\circ c),$$
	则称运算“$\circ$”是\textbf{结合的}.
\end{definition}
\begin{definition}[交换性]
	若在非空集合$A$上定义了一种运算“$\circ$”,对任意$a,b\in A$,都有
	$$a\circ b=b\circ a,$$
	则称运算“$\circ$”是\textbf{交换的}.
\end{definition}
习惯上将运算“$\circ$”称为乘法,并略去不写. 要注意这里的乘法是一种抽象的运算,与数的乘法不是一回事. 那么,结合性可以写为$(ab)c=a(bc)$,交换性可以写为$ab=ba$.
\begin{definition}[分配性]
	若在非空集合$A$上定义了两种运算“$\circ$”和“$+$”,分别称为乘法和加法,对任意$a,b,c\in A$,若有
	$$a\circ(b+c)=(a\circ b)+(a\circ c),$$
	则称乘法对加法是\textbf{左分配的},若有
	$$(a+b)\circ c=(a\circ c)+(b\circ c),$$
	则称乘法对加法是\textbf{右分配的}. 若乘法对加法既是左分配的,又是右分配的,则称乘法对加法是\textbf{分配的}.
\end{definition}
与乘法类似,这里的加法也是一种抽象的运算,不能简单看作数的加法.
\section{等价关系与集合分类}
\begin{definition}[关系]
	设集合$R\subset A\times A$,$a,b\in A$,若$(a,b)\in R$,则称$a$和$b$有\textbf{关系}$R$,记作$aRb$;若$(a,b)\notin R$,则称$a$与$b$没有关系.
\end{definition}
\begin{definition}[等价关系]
	若关系$R$满足
	\begin{enumerate}
		\item 反身性:$aRa,\ \forall a\in A$;
		\item 对称性:$aRb$,则$bRa$,$\forall a,b\in A$;
		\item 传递性:$aRb,bRc$则$aRc$,$\forall a,b,c\in A$.
	\end{enumerate}
	则称$R$为\textbf{等价关系}.
\end{definition}
\begin{definition}[集合的分类]
	非空集合$A$可以分成若干不交非空子集,即$A=\bigcup_{i\in I}M_i$,$M_i\cap M_j=\varnothing,\ i\neq j$,则$\{M_i|i\in I\}$称为$A$的一个\textbf{分类}或\textbf{分划}.
\end{definition}
\begin{theorem}
	集合$A$的一个分类决定$A$中的一个等价关系.
\end{theorem}
\begin{proof}
	设关系$R$满足
	$$aRb\iff a\ \text{和}\ b\ \text{在同一类},$$
	则根据定义易得$R$是等价关系.
\end{proof}
\begin{definition}[等价类]
	设在集合$A$上定义了一个等价关系$R$,$a\in A$,则所有与$a$有关系的元素构成一个集合$\{b\in A\ |\ bRa\}$,称为$a$所在的\textbf{等价类},记作$\overline{a}$,$a$称为这个等价类的\textbf{代表元}.
\end{definition}
\begin{definition}[商集]
	设集合$A$中有等价关系$R$,则以$R$为前提的所有等价类的集合$\{\overline{a}\}$称为$A$对$R$的\textbf{商集},记作$A/R$.
\end{definition}
\begin{definition}[自然映射]
	称从非空集合$A$到它的商集合$A/R$的映射$\pi:A\to A/R,\ \pi(a)=\overline{a}$为\textbf{自然映射}.
\end{definition}
容易验证$\pi$是映射,且是满射,但未必是单射,因为以$a$为代表元的等价类不一定只有$a$这一个元素,如果$b\in\overline{a}$,那么$\pi(a)=\pi(b)=\overline{a}$.
\begin{theorem}
	集合$A$中的一个等价关系决定$A$的一个分类.
\end{theorem}
\begin{proof}
	对任意$a\in A$,$\pi(a)$是$a$所在的等价类,于是$A$中的任何元素都有所在的等价类,这些等价类互不相交,于是构成了$A$的一个分类.
\end{proof}
\begin{definition}[同余关系]
	设集合$A$中有等价关系$R$,并带有二元运算“$\circ$”,若满足
	$$aRb,\ cRd\Rightarrow (a\circ b)R(c\circ d),\quad\forall a,b,c,d\in A,$$
	则称$R$是\textbf{同余关系},相应地,$a$的等价类也称为$a$的\textbf{同余类}.
\end{definition}
\begin{theorem}
	设“$\circ$”是$A$中的二元运算,并定义“$\overline{\circ}$”:$\overline{a}\overline{\circ}\overline{c}=\overline{a\circ c}$,则“$\overline{\circ}$”是$A$中的二元运算当且仅当$R$是同余关系.
\end{theorem}
\begin{proof}
	若“$\overline{\circ}$”是二元运算,则对任意$\overline{a},\overline{c}\in A/R$,有$\overline{a}\ \overline{\circ}\ \overline{c}\in A/R$,于是$\overline{a\circ c}\in A/R$,设$aRb$,$cRd$,则
	$$\overline{a}\ \overline{\circ}\ \overline{c}=\overline{b}\ \overline{\circ}\ \overline{d}=\overline{b\circ d}=\overline{a\circ c},$$
	故$(a\circ c)R(b\circ d)$.
	
	若$R$是同余关系,则对任意$aRb$,$cRd$,有$(a\circ c)R(b\circ d)$,进而$\overline{a\circ c}=\overline{b\circ d}$,故$\overline{a}\ \overline{\circ}\ \overline{c}=\overline{b}\ \overline{\circ}\ \overline{d}\in A/R$,所以“$\circ$”是二元运算.
\end{proof}
\section{群的概念}
\begin{definition}[半群]
	设集合$S$带有二元运算“$\circ$”,若对任意$a,b,c\in S$,有$(a\circ b)\circ c=a\circ(b\circ c)$,则称$(S,\circ)$是\textbf{半群}.
\end{definition}
\begin{definition}[幺半群]
	设半群$(M,\circ)$,若存在$e\in M$,对任意$a\in M$,有$e\circ a=a\circ e=a$,则称$(M,\circ)$为\textbf{幺半群},$e$称为$(M,\circ)$的\textbf{幺元}.
\end{definition}
\begin{definition}[群]
	设集合$G$带有二元运算“$\circ$”,满足以下条件:
	\begin{enumerate}
		\item 对任意$a,b,c\in G$,有$(a\circ b)\circ c=a\circ (b\circ c)$;
		\item 存在$e\in G$,对任意$a\in G$,有$e\circ a=a\circ e=a$;
		\item 对任意$a\in G$,存在$a^{-1}\in G$,使得$a\circ a^{-1}=a^{-1}\circ a=e$,
	\end{enumerate}
	则称$(G,\circ)$是一个\textbf{群}.
\end{definition}
\begin{remark}
	在不引起歧义的情况下,可以说“设$G$是一个群”,而将运算“$\circ$”看作广义上的“乘法”而省略,即“$a\circ b$”记作“$ab$”.
\end{remark}
\begin{remark}
	称“$a^{-1}$”为$a$的\textbf{逆元}.
\end{remark}
\begin{definition}[交换群]
	若$G$中任意元素$a,b$满足$ab=ba$,则称$G$是\textbf{交换群}或\textbf{Abel群}.
\end{definition}
\begin{theorem}
	群$G$的幺元是唯一的.
\end{theorem}
\begin{proof}
	设$e,e'\in G$都是幺元,则$e=ee'=e'$.
\end{proof}
\begin{theorem}
	群$G$的逆元是唯一的.
\end{theorem}
\begin{proof}
	对任意$a\in G$,设$b,c\in G$是$a$的逆元,则$b=b(ac)=(ba)c=c$.
\end{proof}
由于结合律成立,则$a^n=\underbrace{a\circ a\circ\cdots\circ a}_{n\text{个}}$是良定义的.同时有$a^ma^n=a^{m+n}$,$(a^s)^t=a^{st}$.
\begin{theorem}[消去律]
	设$a,b,c\in G$,若$ab=ac$,则$b=c$;若$ba=ca$,则$b=c$.
\end{theorem}
\begin{proof}
	$ab=ac\Rightarrow a^{-1}ab=a^{-1}ac\Rightarrow b=c$,类似地$ba=ca\Rightarrow baa^{-1}=caa^{-1}\Rightarrow b=c$.
\end{proof}
\begin{definition}[群的阶]
	群$G$中元素的个数称为群的\textbf{阶},记作$|G|$.当$|G|$有限时,称$G$为\textbf{有限群},否则称$G$为\textbf{无限群}.
\end{definition}
\begin{definition}[群中元素的阶]
	设$G$为一个群,$a\in G$,若存在正整数$k$使得$a^k=e$,则最小的正整数$k$称为$a$的\textbf{阶},记作$|a|$.即$|a|=\min\left\{k\in \mathbb{N}^+|a^k=e\right\}$. 若不存在这样的$k$,则称$a$的阶为无穷.
\end{definition}
\begin{theorem}
	$|a|=\infty\iff \forall m, n\in \mathbb{N}^+,m\neq n,a^m\neq a^n$.
\end{theorem}
\begin{proof}
	设$m.n.k\in\mathbb{N}^+$且$m=n+k$.则$|a|=\infty\iff \forall k\in\mathbb{N}^+$,$a^k\neq e\iff a^{m-n}\neq e\iff a^m\neq a^n$.
\end{proof}
\begin{theorem}
	设$|a|=d$,则$\forall h\in\mathbb{Z}$,$a^h=e\iff d|h$.
\end{theorem}
\begin{proof}
	充分性:$d|h$,则存在$k\in\mathbb{Z}$使得$h=kd$,则$a^{h}=a^{kd}=(a^{d})^k=e$.
	
	必要性:设$h=qd+r$,$0 \leqslant r<d$,则$a^{qd+r}=a^{qd}a^{r}=a^r=e$,则$r=0$.
\end{proof}
\begin{corollary}
	对任意$m,n\in\mathbb{Z}$,$a^m=a^n\iff d|m-n\iff m\equiv n\pmod d$.
\end{corollary}
\begin{theorem}
	设$|a|=d$,$k\in\mathbb{N}^+$,则$|a^k|=\dfrac{d}{(d,k)}$.
\end{theorem}
\begin{proof}
	设$|a^k|=h$,则$a^{kh}=e$.而$|a|=d$,则$d|kh$.
	
	设$d=d_1(d,k)$,$k=k_1(d,k)$,则$(d_1,k_1)=1$,$d_1|k_1h$,则$d_1|h$.
	
	$(a^k)^{d_1}=a^{kd_1}=a^{dk_1}=e$,故$h|d_1$.于是$|a^k|=h=d_1=\dfrac{d}{(d,k)}$.
\end{proof}
\begin{corollary}
	$|a^k|=d\iff (d,k)=1$.
\end{corollary}
\begin{theorem}
	设$a,b\in G$,$|a|=m$,$|b|=n$,$ab=ba$,$(m,n)=1$,则$|ab|=mn$.
\end{theorem}
\begin{proof}
	设$|ab|=d$,而$(ab)^{mn}=(a^m)^n(b^n)^m=e$,于是$d|mn$;
	
	又$(ab)^{md}=a^{md}b^{md}=b^{md}=e$,故$n|md$,而$(m,n)=1$,于是$n|d$.同理$m|d$,于是$mn|d$,故$mn=d$.
\end{proof}
\section{子群与陪集}
\begin{definition}[子群]
	设$G$是群,若非空集合$H\subset G$且$H$是群,则称$H$是$G$的\textbf{子群},记作$H<G$.
\end{definition}
考虑极端,$G$本身和$\{e\}$都是$G$的子群,称为$G$的\textbf{平凡子群}.
\begin{theorem}[子群的充要条件]\label{iff}
	设$H$是群$G$的非空子集,则$H<G$当且仅当对任意$a,b\in H$,$ab^{-1}\in H$.
\end{theorem}
\begin{proof}
	当$H<G$时,由$H$运算的封闭性以及存在逆元即得$ab^{-1}\in H$.
	
	若对任意$a,b\in H$,有$ab^{-1}\in H$,则
	
	对任意$a,a\in H$,有$e=aa^{-1}\in H$,幺元存在;
	
	对$e\in H$与任意$b\in H$,有$b^{-1}=eb^{-1}\in H$,逆元存在;
	
	对任意$a,b^{-1}\in H$,有$ab=a(b^{-1})^{-1}\in H$,满足封闭律;
	
	$H$中的运算继承$G$中的运算,满足结合律,于是$H<G$.
\end{proof}
\begin{theorem}
	设$H$是群$G$的非空有限子集,则$H<G$当且仅当$H$对$G$的运算封闭.
\end{theorem}
\begin{proof}
	必要性显然. 充分性:$H$对$G$的运算封闭,故对任意$a\in H$,$a^k\in H$,其中$k$为任意正整数.由于$H$是有限群,于是存在正整数$m>n$满足$a^m=a^n$,故$a^{m-n}=e\in H$. 当$m-n=1$时,$a=e$,于是$a^{-1}=e\in H$;当$m-n>1$时,$a^{m-n-1}a=aa^{m-n-1}=e$,$a^{-1}=a^{m-n-1}\in H$,逆元存在. 封闭性满足,结合律满足,于是$H<G$.
\end{proof}
\begin{theorem}
	若$H_1<G$,$H_2<G$,则$H_1\cap H_2<G$.
\end{theorem}
\begin{proof}
	设$a,b\in H_1\cap H_2$,则由定理\ref{iff},$ab^{-1}\in H_1$,$ab^{-1}\in H_2$,则$ab^{-1}\in H_1\cap H_2$,再用一次该定理,即得$H_1\cap H_2<G$.
\end{proof}
\begin{corollary}
	设$I$为指标集,若对任意$i\in I$,$H_i<G$,则$\bigcap_{i\in I}H_i<G$.
\end{corollary}
下面是一些例子.
\begin{example}
	数域$F$上全体$n$阶可逆方阵关于矩阵乘法构成群,称为\textbf{一般线性群},记作$GL_n(F)$. 其中,行列式为$1$的$n$阶方阵关于矩阵乘法也构成群,称为\textbf{特殊线性群},记作$SL_n(F)$. 显然$SL_n(F)<GL_n(F)$.
\end{example}
\begin{example}
	给定$m\in\mathbb{Z}$,定义集合$m\mathbb{Z}=\left\{mn\ |\ \forall n\in\mathbb{Z}\right\}$,定义运算为整数加法,则$m\mathbb{Z}<\mathbb{Z}$.
\end{example}
\begin{theorem}
	$(\mathbb{Z},+)$的子群都是形如$m\mathbb{Z}$的.
\end{theorem}
\begin{proof}
	设$H<\mathbb{Z}$. 若$H=\{0\}$,则$H=0\mathbb{Z}$.
	
	假设$H\neq {0}$,则存在非零整数. 由于$H$是子群,若$a\in H$,则$-a \in H$,因此$H$中必存在正整数. 定义$m=\min\{a\in H|a>0\}$
	
	任取$n\in H$,设$n=qm+r$,其中$0\leqslant r<m$. 由于$m\in H$且$H$是子群,于是$r=n-qm\in H$. 但$0\leqslant r<m$,而$m$是$H$中最小的正整数,因此$r=0$,即$n=qm\in m\mathbb{Z}$,所以$H\subset\mathbb{Z}$.
	
	由于$m\in H$,且$H$是子群,对任意整数$k$,有$km\in H$,因此$\mathbb{Z}\subset H$.		
\end{proof}
\begin{definition}[陪集]
	设$H<G$,给定$a\in G$,集合$aH=\left\{ah|h\in H\right\}$称为以$a$为代表的\textbf{左陪集}. 类似地,集合$Ha=\left\{ha|h\in H\right\}$称为以$a$为代表元的\textbf{右陪集}.
\end{definition}
左陪集和右陪集的概念是对偶的,下面考虑左陪集的情形.
\begin{theorem}
	设$H<G$,则$aRb\iff a^{-1}b\in H$确定了$G$中的等价关系$R$. $a$所在的等价类$\overline{a}$恰为以$a$为代表的左陪集$aH$.
\end{theorem}
\begin{proof}
	先证$R$为等价关系.
	
	自反性:$H$为子群,故$a^{-1}a=e\in H$,即$aRa$.
	
	对称性:若$aRb$,则$a^{-1}b\in H$,$b^{-1}a=\left(a^{-1}b\right)^{-1}\in H$,即$bRa$.
	
	传递性:若$aRb$,$bRc$,则$a^{-1}b\in H$,$b^{-1}c\in H$,$a^{-1}c=a^{-1}bb^{-1}c\in H$,即$aRc$.
	
	再证$\overline{a}=aH$. 对任意$b\in\overline{a}$,$a^{-1}b\in H$,$b=aa^{-1}b\in aH$,于是$\overline{a}\subset aH$;对任意$b\in aH$,存在$h\in H$使得$b=ah$. $a^{-1}b=a^{-1}ah=h\in H$,于是$aH\subset\overline{a}$.
\end{proof}
由于$R$是一个等价关系,于是全体左陪集$\{aH\}$为$G$的一个分类,称为\textbf{左陪集空间}. 由此可见左陪集空间是群$G$对关系$R$的商集,于是又把左陪集空间称为\textbf{左商集},记作$G/R$.
\begin{proposition}\label{subgroup-bijection}
	映射$\varphi:H\to aH,h\mapsto ah$是双射.
\end{proposition}
\begin{proof}
	$aH$是由$H$得到的,$\varphi$是满射是显然的. 对任意$h_1\neq h_2\in H$,$ah_1\neq ah_2$(否则将违反消去律),故$\varphi$是单射.
\end{proof}
\begin{definition}[指数]
	左陪集空间中陪集的个数称为\textbf{指数},记作$\left[G:H\right]$.
\end{definition}
\begin{theorem}[Lagrange]
	设$G$为有限群,$H<G$,则
	$$|G|=\left[G:H\right]|H|.$$
\end{theorem}
\begin{proof}
	设$\left[G:H\right]=n$,$a_i$为每个左陪集的代表元. $G=\bigsqcup_{i=1}^{n}a_iH$,而由命题\ref{subgroup-bijection},$|a_iH|=|H|$,故$|G|=n|H|=\left[G:H\right]|H|$.
\end{proof}
\begin{corollary}
	设$G$是有限群,$K<H<G$,则$\left[G:K\right]=\left[G:H\right]\left[H:K\right]$.
\end{corollary}
\section{正规子群与商群}
\begin{definition}[共轭]
	设$f,h\in G$,若存在$g\in G$,使得
	$$f=ghg^{-1},$$
	则称$f$和$g$\textbf{共轭}.
\end{definition}
容易验证,共轭是一个等价关系,于是决定了一个等价类,称为共轭类.
\begin{definition}[共轭类]
	设$G$是群,对任意$f\in G$,与$f$共轭的元素组成的集合称为以$f$为代表元的\textbf{共轭类}.
\end{definition}
\begin{definition}[自共轭元素]
	如果以$f$为代表的共轭类中的元素只有$f$一个,则称$f$是群$G$的\textbf{自共轭元素}.
\end{definition}
\begin{property}
	一个共轭类中所有元素的阶相同.
\end{property}
\begin{property}
	共轭类的元素数目是群的阶的因子.
\end{property}
\begin{definition}[共轭子群]
	若$H<G$,$K<G$,存在$g\in G$使得
	$$K=gHg^{-1},$$
	则称子群$H$和$K$\textbf{共轭}.
\end{definition}
\begin{definition}[元素的中心化子]
	设$a\in G$,集合$C_G(a)=\left\{g\in G\ |\ ga=ag\right\}$是$G$的子群,称为$a$的\textbf{中心化子}.
\end{definition}
\begin{definition}[子集的中心化子]
	设$S\subset G$,集合$C_G(S)=\left\{
	g\in G\ |\ gs=sg,\forall s\in S\right\}$是$G$的子群,称为$S$的\textbf{中心化子}.
\end{definition}
\begin{definition}[中心]
	$C_G(G)$称为$G$的\textbf{中心}.
\end{definition}
\begin{definition}[正规化子]
	设$M\subset G$,集合$N_G(M)=\left\{g\in G\ |\ gM=Mg\right\}$是$G$的子群,称为$M$的\textbf{正规化子}.
\end{definition}
如果对任意$g\in G$,子群$H$都是自共轭的,则称$H$为正规子群,即如下定义.
\begin{definition}[正规子群]
	设$G$是群,$H<G$,若有
	$$ghg^{-1}\in H,\ \forall\ g\in G,\ \forall\ h\in H,$$
	则称$H$是$G$的\textbf{正规子群},记作$H\vartriangleleft G$.
\end{definition}
\begin{example}
	定义运算为矩阵乘法,$SL_n(F)\vartriangleleft GL_n(F)$.
\end{example}
\begin{example}
	定义运算为数的加法,$m\mathbb{Z}\vartriangleleft \mathbb{Z}$.
\end{example}
\begin{theorem}
	设$H<G$,则下列条件等价.
	\begin{enumerate}
		\item $H\vartriangleleft G$;
		\item $gH=Hg,\ \forall\ g\in G$;
		\item $g_1Hg_2H=g_1g_2H,\ \forall\ g_1,g_2\in G$.
	\end{enumerate}
\end{theorem}
\begin{proof}
	1 $\to$ 2 : 若$H\vartriangleleft G$,则$gHg^{-1}=H$,故$gH=Hg$.
	
	2 $\to$ 3 : 若$gH=Hg$,则$g_1Hg_2H=g_1(Hg_2)H=g_1g_2H$.
	
	3 $\to$ 1 : 若$g_1Hg_2H=g_1g_2H$,则$gHg^{-1}H=gg^{-1}H=H$,则$gHg^{-1}=H$.
\end{proof}
\begin{definition}[商群]
	设$H<G$,关系$R$定义为$aRb\iff a^{-1}b\in H$,则
	$$R\text{为同余关系}\iff H\vartriangleleft G,$$
	商集合$G/R$对同余关系$R$导出的运算也构成一个群,称为$G$对$H$的\textbf{商群},记为$G/H$.
\end{definition}
\begin{proof}
	必要性:因为$g^{-1}(gh)=h\in H$,故$gR(gh)$.又$g^{-1}Rg^{-1}$,于是由同余关系,$$(gg^{-1})R(ghg^{-1}),$$即$eR(ghg^{-1})$,$ghg^{-1}=e^{-1}ghg^{-1}\in H$,$H\vartriangleleft G$.
	
	充分性:设任意$a_1,a_2,b_1,b_2\in H$,$a_1Rb_1$,$a_2Rb_2$,则
	$$(a_1a_2)^{-1}(b_1b_2)=a_2^{-1}a_1^{-1}b_1b_2=a_2{-1}a_1^{-1}b_1a_2a_2^{-1}b_2,$$
	由于$H\vartriangleleft G$,$a_1^{-1}b_1\in H$,则$a_2{-1}a_1^{-1}b_1a_2\in H$,又$a_2^{-1}b_2\in H$,则$(a_1a_2)^{-1}(b_1b_2)\in H$,故$R$为同余关系.
\end{proof}
\begin{example}
	由于$m\mathbb{Z}\vartriangleleft\mathbb{Z}$,于是有商群
	$$\mathbb{Z}/m\mathbb{Z}=\left\{\overline{0},\overline{1},\cdots,\overline{m-1}\right\},$$
	记为$\mathbb{Z}_m$,称为$\mathbb{Z}$的\textbf{模$\boldsymbol{m}$的剩余类加群}.
\end{example}
剩余类加群的每一个元素叫做一个剩余类. 同一剩余类中的两个元素同余,例如设$a,b\in\overline{k}$,$k=0,1,\cdots,m-1$,则$a\equiv b\pmod m$. 
\section{群的同态与同构}
\begin{definition}[同态映射]
	设群$(G_1,\circ)$,$(G_2,\ast)$之间存在映射$f:G_1\to G_2$,若对任意$g_1,g_2\in G_1$,有$f(g_1\circ g_2)=f(g_1)\ast f(g_2)$,则称$f$为\textbf{同态映射}. 符号不至于混淆时,常记作$f(g_1g_2)=f(g_1)f(g_2)$.
\end{definition}
如果$f$是单射,则称为\textbf{单同态};如果$f$是满射,则称为\textbf{满同态}. 若$f:G_1\to G_2$是满同态,则称$G_1$和$G_2$是\textbf{同态的}.
\begin{definition}[同构]
	若同态映射$f:G_1\to G_2$是双射,则称$f$是\textbf{同构映射},$G_1$和$G_2$是\textbf{同构的},记作$G_1\cong G_2$.
\end{definition}
\begin{remark}
	容易验证,同构是等价关系.
\end{remark}
\begin{definition}[自然同态]
	设$H\lhd G$,映射$\pi:G\to G/H,g\mapsto gH$是同态映射,称为\textbf{自然同态}.
\end{definition}
\begin{property}
	设同态映射$f:G_1\to G_2$,$g:G_2\to G_3$,则$gf:G_1\to G_3$也是同态映射.
\end{property}
\begin{property}
	幺元同态到幺元,逆元同态到逆元,子群同态到子群.
\end{property}
\begin{definition}[核]
	设同态映射$f:G_1\to G_2$,$e_1\in G_1$,$e_2\in G_2$是幺元,$G_2$的幺元$e_2$的完全原像$\{a\in G_1\ |\ f(a)=e_2\}$称为同态映射$f$的\textbf{核},记作$\ker f$.
\end{definition}
\begin{example}
	同态映射$f$是单同态当且仅当$\ker f=\{e_1\}$.
\end{example}
\begin{proof}
	必要性显然.充分性:若$\ker f=\{e_1\}$,则$f(e_1)=e_2$,对任意$x_1,x_2\in G_1$,若$f(x_1)=f(x_2)$,则$f(x_1)\left[f(x_2)\right]^{-1}=f(x_1x_2^{-1})=e_2$,于是$x_1x_2^{-1}=e_1$,则$x_1=x_2$.
\end{proof}
\begin{proposition}
	若$H\lhd G$,$\pi:G\to G/H$,则$\ker\pi=H$.
\end{proposition}
\begin{proposition}
	设同态映射$f:G_1\to G_2$,则$\ker f\lhd G_1$.
\end{proposition}
\begin{proof}
	对任意$g\in G_1$,$a\in\ker f$,
	$$f(gag^{-1})=f(g)f(a)f(g^{-1})=f(g)f(g^{-1})=f(gg^{-1})=f(e_1)=e_2,$$
	于是$gag^{-1}\in\ker f$. 由正规子群定义得$\ker f\lhd G_1$.
\end{proof}
\begin{theorem}[群同态基本定理]
	设满同态$f:G_1\to G_2$,则$G_1/\ker f\cong G_2$.
\end{theorem}
\begin{proof}
	记$N=\ker f\lhd G_1$,设$\varphi:G_1/N\to G_2,\ gN\mapsto f(g)$.若$g_1N=g_2N$,则$g_1^{-1}g_2\in N$,$f(g_1^{-1}g_2)=f(g_1)^{-1}f(g_2)=e_2$,于是$f(g_1)=f(g_2)$. 这表明$g_N$在$\varphi$下的像是唯一的,所以$\varphi$是映射.
	
	若$f(g_1)=f(g_2)$,则$e_2=f(g_1)^{-1}f(g_2)=f(g_1^{-1}g_2)$,于是$g_1^{-1}g_2\in N$,$g_1N=g_2N$,因此$\varphi$是单射.
	
	由于$f$是满射,因此$\varphi$是满射,故$\varphi$是双射.
	
	对任意$aN,bN\in G/N$,由于$f$是同态,有
	$$\varphi(aNbN)=\varphi(abN)=f(ab)=f(a)f(b)=\varphi(aN)\varphi(bN).$$
	因此$\varphi$是同构映射,故$G_1/\ker f\cong G_2$.
\end{proof}
\begin{corollary}[第一同构定理]
	设$f$是群$G$的同态,则$G/\ker f\cong f(G)$.
\end{corollary}
\begin{theorem}[第二同构定理]
	若$H<G$,$N\lhd G$,则$H\cap N\lhd H$且
	$$H/(H\cap N)\cong HN/N.$$
\end{theorem}
\begin{proof}
	令$\varphi:H\to HN/N,\ h\mapsto hN$,显然$\varphi$是映射. 对任意$hnN\in HN/N$,由于$hnN=hN$,有$$\varphi(h)=hN=hnN,$$故$\varphi$是满射. 对任意$h_1,h_2\in H$,
	$$\varphi(h_1h_2)=h_1h_2N=h_1Nh_2N=\varphi(h_1)\varphi(h_2),$$
	故$\varphi$是同态. 而
	$$\ker\varphi=\left\{h\in H\ |\ \varphi(h)=hN=e_2=N\right\}=\left\{h\in H\ |\ h\in N\right\}=H\cap N,$$
	由同态基本定理,有
	$$H/(H\cap N)\cong HN/N.$$
\end{proof}
\begin{theorem}[第三同构定理]
	若$H\lhd G$,$N\lhd G$,$N\subset H$,则
	$$G/H\cong (G/N)/(H/N).$$
\end{theorem}
\begin{proof}
	由$H\lhd G$,$N\lhd G$以及$N\subset H$,有$N\lhd H$,且$H/N\lhd G/N$.
	
	设$\pi:G\to G/N,\ g\mapsto gN$以及$\psi:G/N\to(G/N)/(H/N),\ gN\mapsto (gN)(H/N)$,则$\varphi=\psi\circ\pi:G\to(G/N)/(H/N)$是群同态.
	
	由于$\pi$,$\psi$是满射,故$\varphi$是满射. 又
	$$\ker\varphi=\left\{g\in G\ |\ \varphi(g)=H/N\right\},$$
	$$\varphi(g)=\psi(\pi(g))=(gN)(H/N),$$
	$$(gN)(H/N)=H/N\iff gN=H/N\iff g\in H,$$
	故$\ker\varphi=G\cap H=H$,由群同态基本定理,
	$$G/H\cong (G/N)/(H/N).$$
\end{proof}
\section{循环群与生成组}
\begin{definition}[循环群]
	由一个元素$a$反复运算得到的群称为\textbf{循环群},记作$\langle a\rangle$.这个元素称为群的\textbf{生成元}.
\end{definition}
\begin{theorem}
	循环群都是交换群.
\end{theorem}
\begin{proof}
	对任意$a^m,a^n\in\langle a\rangle$,$a^ma^n=a^{m+n}=a^{n+m}=a^na^m$.
\end{proof}
\begin{theorem}
	循环群的子群仍是循环群.
\end{theorem}
\begin{proof}
	设$G_1<\langle a\rangle$,设$k=\min\left\{m\in\mathbb{N}^{+}\ |\ a^m\in G_1\right\}$,则$\langle a^k\rangle\subset G_1$.
	
	对任意$a^n\in G_1$,设$n=qk+r$,则$a^n=a^{qk}a^r\in G_1$,于是$a^r\in G_1$,$0\leqslant r<k$,则$r=0$. 于是$a_n\in\langle a^k\rangle$,则$G_1\subset\langle a^k\rangle$.
\end{proof}
\begin{theorem}
	设循环群$G=\langle a\rangle$.若$|G|=m$,则$G\cong(\mathbb{Z}_m,+)$;若$|G|=\infty$,则$G\cong(\mathbb{Z},+)$.
\end{theorem}
\begin{proof}
	设$f:\mathbb{Z}\to G,\ n\to a^n$. 显然$f$是映射.任意$a^n$都有$n$对应,故$f$是满射.对任意$m,n\in\mathbb{Z}$,
	$$f(m+n)=a^{m+n}=a^ma^n=f(m)f(n),$$
	故$f$是满同态. 由群同态基本定理,
	$$\mathbb{Z}/\ker f\cong G.$$
	而$\ker f\lhd\mathbb{Z}=m\mathbb{Z}$,这里存在$m\in\mathbb{N}$. 当$m=0$时,$\ker f=\{0\}$,则$\mathbb{Z}\cong G$. 当$m>0$时,$\mathbb{Z}/\ker f=\mathbb{Z}/m\mathbb{Z}=\mathbb{Z}_m\cong G$.
\end{proof}
\begin{theorem}
	设$|G|=m$,则$G$是循环群的充要条件是对每一个正整数因子$m_1|m$,都存在唯一的$m_1$阶子群.
\end{theorem}
\begin{proposition}\label{ord}
	有限群$G$中元素的阶是$|G|$的因子.
\end{proposition}
\begin{proof}
	显然有限群中元素的阶有限,设$a\in G,|a|=d$,则
	$$\langle a\rangle=\left\{e,a,a^2,\cdots,a^{d-1}\right\},$$
	而$\langle a\rangle<G,|\langle a\rangle|=d$,由Lagrange定理得证.
\end{proof}
\begin{proposition}
	素数阶群必为循环群.
\end{proposition}
\begin{proof}
	设有限群$|G|=p$,$p$是素数,则由命题\ref{ord},对任意$g\in G$,$|g|=1$或$p$.当$|g|=1$时,$g=e$.当$|g|=p$时,$|\langle g\rangle|=p=|G|$,而$\langle g\rangle<G$,于是$\langle g\rangle=G$,即$G$是由$g$生成的循环群.
\end{proof}
\begin{definition}[生成的子群]
	设$S$是群$G$的非空子集,包含$S$的最小子群称为$S$\textbf{生成的子群},记作$\langle S\rangle$. 等价定义为包含$S$的所有子群的交.
\end{definition}
\begin{theorem}
	设$S$是群$G$的非空子集,$S^{-1}=\left\{a^{-1}\ |\ a\in S\right\}$,则
	$$\langle S\rangle=\left\{x_1x_2\cdots x_m\ |\ x_i\in S\cup S^{-1}\right\}$$
\end{theorem}
\begin{proof}
	设$T=\left\{x_1x_2\cdots x_m\ |\ x_i\in S\cup S^{-1}\right\}$. 由于$S\subset\langle S\rangle$,$S^{-1}\subset\langle S\rangle$,于是$S\cup S^{-1}\subset\langle S\rangle$,则$T\subset\langle S\rangle$.下面证明$T$是子群.
	
	设$x_1x_2\cdots x_n,y_1y_2\cdots y_m\in T$,则$y_i^{-1}\in S\cup S^{-1}$,于是
	$$x_1x_2\cdots x_n(y_1y_2\cdots y_m)^{-1}=x_1x_2\cdots x_ny_m^{-1}y_{m-1}^{-1}\cdots y_1^{-1}\in T.$$
	故$T<\langle S\rangle$,而$\langle S\rangle$是包含$S$的最小子群,故$T=S$.
\end{proof}
\begin{definition}[生成组]
	若$G=\langle S\rangle$,则称$S$为$G$的\textbf{生成组}.
\end{definition}
\begin{definition}[有限生成群]
	若存在群$G$的有限个元素的生成组,则称$G$是\textbf{有限生成群}. 若$G$还是交换群,则称为\textbf{有限生成Abel群}.
\end{definition}
注意到有限群是有限生成群,但有限生成群不一定是有限群,例如$(\mathbb{Z},+)=\langle 1\rangle$.
\section{变换群与置换群}
	\begin{definition}[变换]
	设$A$是一个集合,映射$f:A\to A$称为\textbf{变换},即集合到自身的映射.
\end{definition}
\begin{definition}[变换群]
	集合$A$上所有的可逆变换组成的集合,关于映射的复合构成群,称为集合$A$的\textbf{全变换群},记作$S_A$. 全变换群的一个子群称为$A$的一个\textbf{变换群}.
\end{definition}
可以依定义验证$S_A$构成群. 可逆变换即双射,要求集合中的元素在变换前后是一一对应的.
\begin{definition}[对称群]
	若集合$A$是含$n$个元素的有限集,$S_A$也称为$n$元\textbf{对称群},也记作$S_n$. $S_n$中的变换称为\textbf{置换}.
\end{definition}
\begin{definition}[置换群]
	对称群$S_n$中若干置换可以构成一个$S_n$的子群,称为\textbf{置换群}.
\end{definition}
由定义,对称群是最大的置换群.
\begin{theorem}[Cayley]
	任何群都与一个变换群同构.
\end{theorem}
\begin{proof}
	设$G$是群,任意$a\in G$,定义$\varphi_a:G\to G,\ g\mapsto ag$. $g$在$\varphi_a$下的像$ag$是唯一的,所以$\varphi$是映射.
	
	由于$a^{-1}g\in G$,而$\varphi_a(a^{-1}g)=g$,也就是$G$中任何元素$g$都有原像$a^{-1}g$,所以$\varphi_a$是满射.
	
	对任意$g_1,g_2\in G$,若$\varphi_a(g_1)=\varphi_a(g_2)$,则$ag_1=ag_2$.由消去律有$g_1=g_2$,于是$\varphi_a$是单射. $\varphi_a$又是满射,所以是双射,即可逆映射.故$\varphi_a\in S_G$.
	
	设$T=\left\{\varphi_a\ |\ a\in G\right\}$,则$T\subset S_G$. 又因为$(\varphi_b)^{-1}=\varphi_{b^{-1}}$,则$\varphi_a(\varphi_b)^{-1}=\varphi_a\varphi_{b^{-1}}=\varphi_{ab^{-1}}\in T$,于是由子群的充要条件,有$T<S_G$,则$T$是$G$的一个变换群,下面证明$T\cong G$.
	
	设$f:G\to T,\ a\mapsto\varphi_a$,显然$f$是满映射.对任意$g_1,g_2\in G$,若$f(g_1)=f(g_2)$,则$\varphi_{g_1}=\varphi_{g_2}$,$\varphi_{g_1}(e)=\varphi_{g_2}(e)$,即$g_1=g_2$,所以$f$是单射.于是$f$是双射.
	
	对任意$a,b\in G$,$f(ab)=\varphi_{ab}=\varphi_a\varphi_b=f(a)(b)$,所以$f$是同构映射,$G\cong T$.
\end{proof}
\begin{corollary}
	任何有限群都与一个置换群同构.
\end{corollary}
下面介绍置换群相关内容. 设$\sigma\in S_n$,设$A=\left\{a_1,a_2,\cdots,a_n\right\}$,则置换$\sigma$可以表示为
$$\sigma(A)=
\begin{pmatrix}
	a_1 & a_2 & \cdots & a_n \\
	\sigma(a_1) & \sigma(a_2) & \cdots & \sigma(a_n)
\end{pmatrix}
$$
其中,$\sigma(a_1),\sigma(a_2),\cdots,\sigma(a_n)$是$a_1,a_2,\cdots,a_n$的一个排列. 注意到一共有$n!$种不同的排列方式,于是$|S_n|=n!$. 特别地,若$\id(a_i)=a_i,i=1,2,\cdots,n$,则称$\id$为\textbf{恒等置换}.
\begin{definition}[轮换]
	设$I_r=\left\{i_1,i_2,\cdots,i_r\right\}\subset\left\{a_1,a_2,\cdots,a_n\right\}=A$,置换$\sigma$满足
	$$\sigma(I_r)=
	\begin{pmatrix}
		i_1 & i_2 & \cdots & i_r \\
		i_2 & i_3 & \cdots & i_1
	\end{pmatrix}
	$$
	$$\sigma(A\backslash I_r)=\id(A\backslash I_r),$$
	则称$\sigma$为$\boldsymbol{r}$\textbf{-轮换},记作$\sigma=(i_1i_2\cdots i_r)$. $i_1,i_2,\cdots,i_r$称为轮换中的\textbf{文字},$r$称为轮换的\textbf{长}.
\end{definition}
特别地,当$r=2$时称为\textbf{对换},$r=1$时为恒等置换.

\begin{proposition}
	$r$-轮换的阶为$r$.
\end{proposition}
\begin{proposition}
	$(i_1i_2\cdots i_r)=(i_2i_3\cdots i_1)=\cdots (i_ri_1\cdots i_{r-1})$.
\end{proposition}
上述两个命题都是显然的.
\begin{definition}
	在$S_n$中,如果若干个轮换间无共同文字,则称它们是\textbf{不相交的轮换}.
\end{definition}
\begin{proposition}
	在$S_n$中不相交轮换的乘积可换.
\end{proposition}
\begin{proof}
	对于两个不相交的轮换$\sigma_1$和$\sigma_2$,$\sigma_1$作用在$\sigma_2$作用的文字上时是恒等置换,同理$\sigma_2$作用在$\sigma_1$作用的文字上时也是恒等置换,而恒等置换与置换的乘积是可换的,于是不相交轮换的乘积可换. 对于多个不相交的轮换,以此类推即可.
\end{proof}
\begin{theorem}\label{permutationtocycle}
	$S_n$中任一置换都可表为若干不相交轮换的乘积.
\end{theorem}
\begin{proof}
	设$a\in\left\{1,2,\cdots,n\right\}$,置换$\sigma$作用到$a$上得到一些不同的文字.
	$$a=\sigma^{0}(a),\sigma(a),\sigma^2(a),\cdots,$$
	假设$\sigma^m(a)$与前面某一文字$\sigma^k(a)$重复,那么$k=0$,否则$\sigma^{k-1}(a)=\sigma^{m-1}(a)$从而矛盾. 于是置换$\sigma$在$a$上的作用等同于轮换
	$$\sigma_1=(a\sigma(a)\sigma^2(a)\cdots\sigma^{m}(a)),$$
	下面考虑$b\in\left\{1,2,\cdots,n\right\}\backslash\left\{a,\sigma(a),\cdots\sigma^m(a)\right\}$,得到轮换
	$$\sigma_2=(b\sigma(b)\cdots\sigma^l(b)),$$
	这里$\sigma_1$和$\sigma_2$是不相交的轮换. 以此类推,可以通过有限次操作取遍$\left\{1,2,\cdots,n\right\}$中的元素. 于是任一置换可以表为若干不相交轮换的乘积.
\end{proof}
\begin{proposition}
	任一个$r$-轮换都可以写成$r-1$个对换的乘积.
\end{proposition}
\begin{proof}
	$(i_1i_2\cdots i_r)=(i_1i_r)(i_1i_{r-1})\cdots(i_1i_3)(i_1i_2)$.
\end{proof}
\begin{proposition}
	任一置换都可以表为一些对换的乘积,这些对换的表示不一定唯一,但对换个数的奇偶性不变.
\end{proposition}
\begin{proof}
	由定理\ref{permutationtocycle},任一置换可以表示为若干不相交轮换的乘积,而任一轮换可以写成对换的乘积,因此任一置换都可以表为一些对换的乘积. 对换的表示不唯一,因为对任一对换的乘积,乘以$(i_ji_k)(i_ki_j)$之后仍然不变. 对换的表示改变了,但对换个数的奇偶性没有变.
\end{proof}
对换的表示并不是置换的本质,对换个数的奇偶性才是,于是有奇置换与偶置换的概念.
\begin{definition}[奇置换与偶置换]
	可以表为奇数个对换的乘积的置换称为\textbf{奇置换},可以表为偶数个对换的乘积的置换称为\textbf{偶置换}.
\end{definition}
下面是奇置换与偶置换的一些简单性质,这与整数的奇偶性可以类比.
\begin{property}\label{oddeven}
	两个奇置换之积是偶置换,两个偶置换之积是奇置换. 奇置换与偶置换之积是奇置换,偶置换与奇置换之积是奇置换. 置换的逆不改变置换的奇偶性.
\end{property}
\begin{definition}[交错群]
	按照群的定义可以验证,$n$元偶置换全体对置换的乘法构成群,称为$n$元\textbf{交错群},记作$A_n$.
\end{definition}
\begin{proposition}
	$A_n\lhd S_n$,$|A_n|=n!/2$.
\end{proposition}
\begin{proof}
	对任意$\sigma\in A_n$,$\varphi\in S_n$,$\varphi\sigma\varphi^{-1}\in A_n$,因此$A_n\lhd S_n$. 而$A_n$中不是奇置换就是偶置换. 对任意$\sigma\in S_n$,映射$\sigma\to (1,2)\sigma$建立了一个奇置换与偶置换之间的双射,于是
	$|A_n|=n!/2$.
\end{proof}
\begin{proposition}
	设置换$\sigma=\sigma_1\sigma_2\cdots\sigma_n$表示为$n$个不相交的轮换的乘积,其中$\sigma_i$是$r_i$-轮换,则$\sigma$的阶为$\left[r_1,r_2,\cdots,r_n\right]$.
\end{proposition}
\begin{proof}
	设$|\sigma|=d$,$m=\left[r_1,r_2,\cdots,r_n\right]$,则通过展开即可得$\sigma^m=\id$,于是$d\mid m$. 
	
	已知$\sigma^d = \text{id}$,所以对每个$i$,有$\sigma_i^d = \text{id}$.而 $\sigma_i$是一个 $r_i$-轮换,其阶为 $r_i$,因此$r_i \mid d$. 所以$d$是$r_i$的公倍数,所以$m\mid d$.
\end{proof}
\begin{definition}[自同构群]
	群$G$到自身的同构映射称为它的一个\textbf{自同构},全体自同构组成的集合对映射的复合作成群,称为$G$的\textbf{自同构群},记作$\Aut G$.
\end{definition}
同构映射是双射,因此$\Aut G<S_G$.
\begin{definition}[内自同构群]
	设$G$是群,给定$a\in G$,定义映射$\sigma_a:G\to G,\ g\mapsto aga^{-1}$,则映射$\sigma_a\in\Aut G$,称为由$a$决定的\textbf{内自同构}.记
	$$\Inn G=\left\{\sigma_a\ |\ a\in G\right\},$$
	则$\Inn G\lhd\Aut G$,称为$G$的\textbf{内自同构群}.
\end{definition}
\begin{proof}
	对任意$g\in G$,$(\sigma_{a^{-1}})\sigma_a(g)=a^{-1}aga^{-1}a=g$. 因此$\sigma_{a^{-1}}$是$\sigma_a$的逆映射,$\sigma_a$是双射.又对任意$g_1,g_2\in G$,
	$$\sigma_a(g_1g_2)=ag_1g_2a^{-1}=ag_1a^{-1}ag_2a^{-1}=\sigma_a(g_1)\sigma_a(g_2),$$
	于是$\sigma_a$是同构,$\sigma_a\in\Aut G$.
	
	于是$\Inn G\subset\Aut G$. 对任意$a,b\in G$,任意$g\in G$,有
	$$\sigma_a\sigma_b(g)=\sigma_a(bgb^{-1})=abgb^{-1}a^{-1}=(ab)g(ab)^{-1}=\sigma_{ab}(g)\in\Inn G,$$
	于是$\Inn G<\Aut G$. 对任意$\sigma_a\in\Inn G$,$\varphi\in\Aut G$,有
	$$\varphi\sigma_a\varphi^{-1}(g)=\varphi(a\varphi^{-1}(g)a^{-1})=\varphi(a)\varphi\varphi^{-1}(g)\varphi(a^{-1})=\sigma_{\varphi(a)}(g)\in\Inn G,$$
	故$\Inn G\lhd\Aut G$.
\end{proof}
\section{群作用}
\begin{definition}[群作用]
	设$G$是群,$X$是非空集合. 定义映射$G\times X\to X,\ (g,x)\mapsto g\cdot x$,若满足
	\begin{enumerate}
		\item 幺元:$e\cdot x=x$;
		\item 兼容性:对任意$g_1,g_2\in G$,$(g_1g_2)\cdot x=g_1\cdot(g_2\cdot x)$,
	\end{enumerate}
	则称映射“$\cdot$”为$G$在$X$上的一个\textbf{作用}.
\end{definition}

群作用由公理化的定义有些抽象,下面建立群作用和置换之间的联系,也可看作是群作用的另一种定义.
\begin{theorem}
	设$G$是群,$X$是非空集合,映射$\varphi:G\to S_X,\ g\mapsto\varphi_{g}$.则$\varphi$是同态当且仅当$\varphi$给出了一个群$G$在集合$X$上的群作用.
\end{theorem}
\begin{proof}
	必要性:定义映射$\cdot:G\times X\to X,\ (g,x)\mapsto\varphi_g(x)$,这里$\varphi_g\in S_X$是同态. 下面证明“$\cdot$”是群作用.
	
	对任意$g\in G$,$e\in G$是幺元,则
	$$\varphi_{ge}=\varphi_g\varphi_e=\varphi_g,$$
	于是$\varphi_e=\id$.则对任意$x\in X$,
	$$e\cdot x=\varphi_e(x)=\id(x)=x.$$
	
	对任意$g_1,g_2\in G$,有
	$$(g_1g_2)\cdot x=\varphi_{g_1}\varphi_{g_2}(x)=\varphi_{g_1}(\varphi_{g_2}(x))=g_1\cdot(g_2\cdot x).$$
	于是“$\cdot$”是群作用.
	
	充分性:定义映射$\varphi_g:X\to X,\ \varphi_g(x)\mapsto g\cdot x$,先证明$\varphi_g$是置换.
	
	对任意$x_1,x_2\in X$,若$\varphi_g(x_1)=\varphi_g(x_2)$,则$g\cdot x_1=g\cdot x_2$,用$g^{-1}$作用,得
	$$g^{-1}\cdot(g\cdot x_1)=g^{-1}\cdot(g\cdot x_2),$$由兼容性公理得$x_1=x_2$,故$\varphi_g$是单射.
	
	而$\varphi_{g}(g^{-1}\cdot x)=g\cdot(g^{-1}\cdot x)=e\cdot x=x$,故对所有$x\in X$,都存在原像$g^{-1}\cdot x$,于是$\varphi_g$是满射.
	
	因此$\varphi_g$是置换. 定义映射$\varphi:G\to S_X,\ g\mapsto\varphi_g$,下面证明$\varphi$是同态.
	
	对任意$g_1,g_2\in G$,任意$x\in X$,有
	$$\varphi_{g_1g_2}(x)=(g_1g_2)\cdot x=g_1\cdot(g_2\cdot x)=\varphi_{g_1}\cdot(\varphi_{g_2}(x))=\varphi_{g_1}\varphi_{g_2}(x),$$
	于是$\varphi$是同态.
\end{proof}
\begin{definition}[轨道]
	设$G$是群,$X$是非空集合. 对任意给定的$x\in X$,称集合$\left\{g\cdot x\ |\ \forall g\in G\right\}$为$x$的\textbf{轨道},记作$\Orb(x)$.
\end{definition}
轨道就是在群$G$的作用下,$x$所能到达的所有取值的集合.
\begin{theorem}
	设群$G$作用在集合$X$上. 定义关系$xRy\iff \exists g\in G,y=g\cdot x$,则$R$是等价关系,且$x$所在的等价类是$\Orb(x)$.
\end{theorem}
\begin{proof}
	$R$是等价关系由反身性、对称性、传递性验证即可. 由$R$的定义可知$x$所在的等价类为$\Orb(x)$.
\end{proof}
\begin{remark}
	由此,群$G$作用在集合$X$上,可将$X$分为若干轨道的无交并.
\end{remark}
\begin{definition}[稳定化子]
	设$G$是群,$X$是非空集合. 对任意给定的$x\in X$,集合$\left\{g\ |\ g\cdot x=x\right\}$关于群$G$的运算构成群,称为$x$的\textbf{稳定化子}或\textbf{迷向子群},记作$\Stab(x)$.
\end{definition}
也就是说,群$G$中有一些元素,作用在$x$上,得到的还是$x$自身. 例如$e\in G$,$e\cdot x=x$. 下面证明$\Stab(x)<G$.
\begin{proof}
	已知$\Stab(x)\subset G$. 对任意$g_1,g_2\in \Stab(x)$,
	$$e\cdot x=(g_2^{-1}g_2)\cdot x=g_2^{-1}\cdot(g_2\cdot x)=g_2^{-1}\cdot x=x.$$
	于是
	$$(g_1g_2^{-1})\cdot x=g_1\cdot (g_2^{-1}\cdot x)=g_1\cdot x=x.$$
	故$g_1g_2^{-1}\in \Stab(x)$,由子群的充要条件,$\Stab(x)<G$.
\end{proof}
\begin{definition}[不动点]
	设群$G$作用在集合$X$上,集合$\left\{x\mid g\cdot x=x,\forall g\in G\right\}$称为$X$在群$G$作用下的\textbf{不动点}.
\end{definition}
\begin{definition}[齐性空间]
	若群$G$作用在集合$X$上,对任意$x,y\in X$,都有$g\in G$满足$g\cdot x=y$,则称这个作用是\textbf{可传递}的或\textbf{可迁}的,集合$X$称为\textbf{齐性空间}.
\end{definition}
\begin{remark}
	群$G$在每个轨道$\Orb(x)$上的作用是可传递的.
\end{remark}
\begin{definition}
	若对任意$x\in X$,$\Stab(x)=e$,则称$G$的作用是\textbf{自由的}.
\end{definition}
\begin{definition}
	若对任意$g\neq e$,存在$x\in X$使得$g\cdot x\neq x$,则称$G$的作用是\textbf{忠实的}或\textbf{有效的}.
\end{definition}
\begin{definition}
	若对任意$g\in G$,$x\in X$,都有$g\cdot x=x$,则称$G$的作用是\textbf{平凡的}.
\end{definition}
\begin{theorem}[轨道-稳定化子定理]
	设$G/\Stab(x)$是$G$关于$\Stab(x)$的左陪集空间,则存在双射$\varphi:\Orb(x)\to G/\Stab(x)$.特别地,当$G$是有限群时,有$|G|=|\Orb(x)||\Stab(x)|$.
\end{theorem}
\begin{proof}
	设$\varphi:\Orb(x)\to G/\Stab(x),\ g\cdot x\mapsto g\Stab(x)$. 设$g\cdot x=h\cdot x$,则$(h^{-1}g)\cdot x=x$,于是$h^{-1}g\in\Stab(x)$,$h\Stab(x)=g\Stab(x)$,所以$\varphi$是映射.
	
	对于任一$g\Stab(x)$,都有原像$g\cdot x$,故$\varphi$是满射.
	
	对任意$g_1\cdot x,g_2\cdot x\in\Orb(x)$,若$\varphi(g_1\cdot x)=\varphi(g_2\cdot x)$,则$g_1\Stab(x)=g_2\Stab(x)$,$g_2^{-1}g_1\in\Stab(x)$,于是$g_2^{-1}g_1\cdot x=x$,于是$g_1\cdot x=g_2\cdot x$,$\varphi$是单射.
	
	于是$\varphi$是双射,有$|\Orb(x)|=|G\Stab(x)|=\left[G:\Stab(x)\right]$.特别地,当$G$有限时,由Lagrange定理,有$|G|=|\Orb(x)||\Stab(x)|$.
\end{proof}
可以把群作用推广到集类上.
\begin{definition}
	设$G$是群,$\mathcal{X}$是非空集类,定义映射$G\times \mathcal{X}\to \mathcal{X},\ (g,H)\mapsto g\cdot H$,若满足
	\begin{enumerate}
		\item 幺元:$e\cdot H=H$;
		\item 兼容性:对任意$g_1,g_2\in G$,$(g_1g_2)\cdot H=g_1\cdot(g_2\cdot H)$,
	\end{enumerate}
	则称映射“$\cdot$”为$G$在$\mathcal{X}$上的一个\textbf{作用}.
\end{definition}
\begin{definition}[共轭作用]
	若群$G$作用在自身,对任意$g,x\in G$,有
	$$g\cdot x=gxg^{-1},$$
	可以验证这是一个作用,称为\textbf{共轭作用}.
\end{definition}
现在可以用群作用的语言来描述共轭类、共轭子群、中心化子以及正规化子的概念.
\begin{definition}[共轭类]
	设群$G$到自身有共轭作用,则对任意$x\in G$,称$\Orb(x)$为$x$所在的\textbf{共轭类}.
\end{definition}
\begin{definition}[共轭子群]
	设$H<G$,则称$g\cdot H=gHg^{-1}$为$H$在$G$作用下的\textbf{共轭子群}.
\end{definition}
\begin{definition}[中心化子]
	$x\in G$在共轭作用下的稳定化子$\Stab(x)$称为$x$在$G$中的\textbf{中心化子},记作$C_G(x)$.
\end{definition}
\begin{definition}[中心]
	群$G$中所有元素的中心化子的交称为群$G$的\textbf{中心},即与$G$中所有元素都共轭的元素组成的集合,记作$C_G(G)$.
\end{definition}
\begin{definition}[正规化子]
	$H<G$在共轭作用下的稳定化子$\Stab(H)$称为$H$在$G$中的\textbf{正规化子},记作$N_G(H)$.
\end{definition}
\begin{definition}[类方程]
	由于群$G$可以划分为若干共轭类,即
	$$G=\bigsqcup_{i\in I}C(x_i),$$
	其中$C(x_i)$为以$x_i$为代表元的共轭类.即$C(x_i)=\Orb(x_i)$.对任意$z\in C_G(G)$,任意$g\in G$,都有$z=gzg^{-1}$,即$|C(z)|=1$.则
	\begin{equation}\label{classeq}
		|G|=\sum_{i\in I}|C(x_i)|=|C_G(G)|+\sum_{i\in I'}|C(x_i)|=|C_G(G)|+\sum_{i\in I'}\frac{|G|}{|C_G(x_i)|}.
	\end{equation}
	其中$C_G(G)=\left\{x_i\mid i\in I\backslash I'\right\}$.称方程(\ref{classeq})为\textbf{类方程}.
\end{definition}
\begin{example}
	考虑对称群$S_3$,$|S_3|=6$.它的共轭类有$\{e\}$,$\left\{(12),(13),(23)\right\}$以及$\left\{(123),(132)\right\}$.于是
	$$6=1+3+2.$$
\end{example}
\section{Sylow子群}
\begin{definition}[$p$-群]
	设$G$是有限群,$p$是素数,若$|G|=p^k,k\in\mathbb{N}^+$,则称$G$是一个$\boldsymbol{p}$\textbf{-群}.
\end{definition}
\begin{lemma}\label{fixedpoint}
	设$p$-群$G$作用在集合$X$上,若$|X|=n$,$X$中的不动点个数为$t\ (t\in\mathbb{N})$,则
	\begin{enumerate}[(1)]
		\item $t\equiv n\pmod p$;
		\item 若$(n,p)=1$,则不动点存在.
	\end{enumerate}
\end{lemma}
\begin{proof}
	(1)设$X=\displaystyle\bigsqcup_{i\in I}\Orb(x_i)$. $x_i$为不动点当且仅当$|\Orb(x_i)|=1$,于是
	$$n=t+\sum_{|\Orb(x_i)\neq 1|}|\Orb(x_i)|.$$
	而由轨道-稳定化子定理,$|\Orb(x_i)|$能整除$|G|$,而$|G|=p^k\ (k\in\mathbb{N}^{+})$,于是$p$能整除$|\Orb(x_i)|$.故$t\equiv n\pmod p$.
	
	(2)若$(n,p)=1$,则$n\nmid p$,由(1),得$t\nmid p$,则$t\neq 0$,即存在不动点.
\end{proof}
\begin{lemma}\label{divide}
	在正整数中,设$p$是素数,$n=p^lm$,若$k\leqslant l$,则$p^{l-k}$恰能整除$\mathrm{C}_{n}^{p^k}$.
\end{lemma}
\begin{proof}
	由组合数公式,
	$$\mathrm{C}_{n}^{p^k}=\frac{n}{p^k}\prod_{i=1}^{p^k-1}\frac{n-i}{p^k-i},$$
	而
	$$\frac{n}{p^k}=p^{l-k}m\Rightarrow p^{l-k}\mid\mathrm{C}_{n}^{p^k},$$
	
	设$1\leqslant i\leqslant p^k-1$表示为$i=p^tj$,其中$(p,j)=1$,$t<k\leqslant l$. 则
	$$n-i=p^t\left(p^{l-t}m-j\right),$$
	$$p^{k}-i=p^t\left(p^{k-t}-j\right),$$
	于是$p\nmid\prod_{i=1}^{p^k-1}\frac{n-i}{p^k-i}$,故$p^{l-k}$恰能整除$\mathrm{C}_{n}^{p^k}$.
\end{proof}
下面若无特殊说明,默认$G$的阶为$p^lm$,其中$p$为素数,$(p,m)=1$,$l\geqslant 1$.
\begin{theorem}[Sylow第一定理,存在性]
	若$1\leqslant k\leqslant l$,则$G$存在$p^k$阶子群.
\end{theorem}
\begin{proof}
	设$G$中所有$p^k$阶子集组成的集合为$\mathcal{X}$. 则$|\mathcal{X}|=\mathrm{C}_{n}^{p^k}$,这里$n=p^lm$.
	设$G$作用在$\mathcal{X}$上,则有轨道分解
	$$\mathcal{X}=\bigsqcup_{i\in I}\Orb(A_i),\quad A_i\in\mathcal{X}.$$
	于是
	$$|\mathcal{X}|=\sum_{i\in I}|\Orb(A_i)|.$$
	由引理\ref{divide},存在$A\in\mathcal{X}$,$p^{l-k+1}\nmid|\Orb(A)|$.由轨道-稳定化子定理,
	$$p^{l-k+1}\nmid \frac{|G|}{|\Stab(A)|}\Rightarrow p^{l-k+1}\nmid\frac{p^lm}{|\Stab(A)|}.$$
	设$|\Stab(A)|=p^ab$,其中$(a,b)=1$.则$p^{l-a}<p^{l-k+1}$,即$a>k-1$,$a\geqslant k$.于是$p^k\mid|\Stab(A)|$.
	
	由于$\Stab(A)<G$,对任意$g\in\Stab(A),a\in A$,定义群$\Stab(A)$对集合$A$的作用$g\cdot a=ga$. 由于$\Stab(A)=\left\{g\in G\mid g\cdot a=a,\ \forall a\in A\right\}$,于是$ga\in A$. 则$\Stab(A)\cdot a\subset A$. 而$\Stab(A)$到$\Stab(A)\cdot a$之间是双射,于是$|\Stab(A)|=|\Stab(A)\cdot a|\leqslant|A|=p^k$.即$\Stab(A)$是一个$p^k$阶子群.
\end{proof}
\begin{definition}[Sylow $p$-子群]
	设$G$的阶是$p^lm$,其中$p$是素数,则$G$的$p^l$阶子群称为$G$的\textbf{Sylow\ $\boldsymbol{p}$-子群}.
\end{definition}
\begin{theorem}[Sylow第二定理,共轭性]
	设$P$是$G$的一个Sylow $p$-子群,$H$是$P$的一个$p^k$阶子群,则$H$包含于$P$的共轭子群中. 特别地,Sylow $p$-子群之间互相共轭.
\end{theorem}
\begin{proof}
	设$G$作用在$G/P$上,$g\cdot gP=ggP$,称为左平移作用.将这个作用限制在$H$上,则$h\cdot gP=hgP$. 由于$|G/P|=m$,$(m,p)=1$,由引理\ref{fixedpoint}(2),存在$gP\in G/P$满足$hgP=gP$.于是$hg\in gP$,即$h\in gPg^{-1}$,$H$包含于$P$的共轭子群中.特别地,当$|H|=p^l$,则$P$也包含在$H$的共轭子群中,于是$H=gPg^{-1}$.
\end{proof}
\begin{theorem}[Sylow第三定理,计数定理]
	设$G$的Sylow $p$-子群的个数为$k$,则
	\begin{enumerate}[(1)]
		\item 当且仅当$k=1$时,这个Sylow $p$-子群$P\lhd G$;
		\item $k\equiv 1\pmod p$且$k\mid m$.
	\end{enumerate}
\end{theorem}
\begin{proof}
	(1)设$P$是群$G$的一个Sylow $p$-子群. 若$P'$是另外一个Sylow $p$-子群,则由Sylow第二定理,有$P'\subset\left\{gPg^{-1}\mid g\in G\right\}$,同时有$P\subset\left\{gP'g^{-1}\mid g\in G\right\}$. 若$k=1$,则$P=gPg^{-1}$,对任意$g\in G$成立,于是$P\lhd G$.反之,若$P\lhd G$,则$P=gPg^{-1}$,得$k=1$.
	
	(2)设$\mathcal{X}$是群$G$的所有Sylow $p$-子群的集合,群$P\in\mathcal{X}$作用在集合$\mathcal{X}$上的作用为共轭作用.对任意$g\in P$,
	$$g\cdot P=gPg^{-1}=P,$$
	因此$P$是该作用下的一个不动点.假设$P_1$也是一个不动点,则对任意$g\in P$,
	$$gP_1g^{-1}=P_1,$$
	因此$g\in N_G(P_1)$,$P\subset N_G(P_1)$.而$|P|=p^l$,于是设$|N_G(P_1)|=p^lm_1$,其中$m_1\mid m$.于是$P,P_1$都是$N_G(P_1)$的Sylow $p$-子群,而$P_1\lhd N_G(P_1)$,由(1),得$k=1$,即$P=P_1$,该作用下只有一个不动点. 由引理\ref{fixedpoint}(1),有$k\equiv 1\pmod p$.
	
	设群$G$在集合$\mathcal{X}$上的作用为共轭作用.则由Sylow第二定理,对任意$P_1,P_2\in\mathcal{X}$,存在$g\in G$,使得
	$$P_1=g\cdot P_2=gP_2g^{-1},$$
	于是$\mathcal{X}$是可传递的.对任意$P\in\mathcal{X}$,
	$$k=|\mathcal{X}|=|\Orb(P)|=\frac{|G|}{|\Stab(P)|},$$
	于是$k\mid|G|$,即$k\mid p^lm$.而由于$k\equiv 1\pmod p$,于是$(k,p)=1$,则$k\mid m$.
\end{proof}
下面介绍Sylow定理的若干应用.
\begin{definition}[单群]
	没有非平凡正规子群的群称为\textbf{单群}。
\end{definition}
\begin{example}
	$72$阶群不是单群.
\end{example}
\begin{proof}
	首先,$72=2^3\times3^2$,设有限群$G$的阶$|G|=72$,设$G$的Sylow $2$-子群的个数为$k_1$,Sylow $3$-子群的个数为$k_2$. 由Sylow第三定理,$k_1$可能为$1,3,9$,$k_2$可能为$1,4$.
	
	当$k_1=1$时,由Sylow第三定理(1),这个$8$阶的Sylow $2$-子群是$G$的正规子群.当$k_2=1$时,这个$9$阶的Sylow $3$子群也是$G$的正规子群. 它们都不是平凡的.
	
	当$k_2=4$时,设$X=\left\{P_1,P_2,P_3,P_4\right\}$,其中$P_i$是互不相同的Sylow $3$-子群.设$G$作用在$X$上的作用为共轭作用.即
	$$g\cdot P_i=gP_ig^{-1},\ \forall g\in G,$$
	则这个作用决定了一个同态$\varphi:G\to S_X$.
	
	而$\ker\varphi\lhd G$,假设$\ker\varphi=G$,则对任意$g\in G$,
	$$g\cdot P_i=\id(P_i)=P_i,$$
	则Sylow子群之间不能互相共轭,这与Sylow第二定理矛盾.
	
	假设$\ker\varphi=\left\{e\right\}$,则由同态基本定理,
	$$G/\ker\varphi\cong\varphi(G),$$
	于是
	$$|G/\ker\varphi|=|G/e|=|G|=|\varphi(G)|<|S_4|=24,$$
	而$|G|=72$,矛盾.
	
	于是$\ker\varphi$是$G$的非平凡正规子群,故$72$阶群不是单群.
\end{proof}
\begin{example}
	56阶群不是单群.
\end{example}
\begin{proof}
	首先,$56=2^3\times 7$,设有限群$G$的阶$|G|=56$,设$G$的Sylow $2$-子群的个数为$k_1$,Sylow $7$-子群的个数为$k_2$. 由Sylow第三定理,$k_1$可能为$1,7$,$k_2$可能为$1,8$.
	
	当$k_1=1$时,由Sylow第三定理(1),这个$8$阶的Sylow $2$-子群是$G$的正规子群.当$k_2=1$时,这个$7$阶的Sylow $7$-子群也是$G$的正规子群. 它们都不是平凡的.
	
	当$k_1=7$且$k_2=8$时,由于素数阶群必为循环群,于是这$8$个Sylow $7$-子群中,除幺元外的$6$个元素都是$7$阶的,且各不相同.于是一共含有$|G|$中的$1+6\times 8=49$个元素.对任意的一个Sylow $2$-子群,除幺元外含有$7$个元素,且与Sylow $7$-子群中的元素不同. 这就有$49+7=56$个元素. 而这$7$个Sylow $2$-子群元素不是完全一致的,于是Sylow $7$子群和Sylow $8$子群中不重复的元素个数就超过了$56$,这与$|G|=56$矛盾!于是$k_1=1$或$k_2=1$,则由上述可知$56$阶群不是单群.
\end{proof}
\begin{example}
	设$|G|=p^lm$,$(p,m)=1$,$p>m\neq 1$,则$G$是单群.
\end{example}
\begin{proof}
	设$G$的Sylow $p$-子群的个数为$k$,由Sylow第三定理,$k$的取值只能为$1$.而$m>1$,于是$G$的Sylow $p$-子群是$G$的$p^l$阶真正规子群.
\end{proof}
\begin{remark}
	$k$的取值只能为$1$,因为当$k$取$1+p$时,$1+p>m$于是不能整除. 那么其他取值更不能取到了.
\end{remark}
\section{群的直积}
	\begin{definition}[群的扩张]
	设$G,A,B$是群,若有$N\lhd G$,使得$A\cong N$,$B\cong G/N$,则称群$G$是$B$过$A$的\textbf{扩张}.称$N$为\textbf{扩张核}.
\end{definition}
\begin{remark}
	群的扩张与域的扩张完全不同,域从子域扩成扩域,而群不一定是子群,甚至不一定和子群同构.
\end{remark}
\begin{definition}[正合序列]
	设$G_1,G_2,\cdots,G_n$是群,有同态映射如下,
	\begin{displaymath}
		\xymatrix{
			G_1\ar[r]^{f_1} & G_2\ar[r]^{f_2} & \cdots\ar[r]^{f_{n-1}} & G_n
		}
	\end{displaymath}
	且满足$f_i(G_i)=\ker f_{i+1}$,则称这个序列为\textbf{正合序列}.
\end{definition}
\begin{remark}
	这里群的个数可以是有限的,也可以是无限的.
\end{remark}
\begin{definition}[短正合序列]
	设$1$是$A$的幺元,$1'$是$B$的幺元,则正合序列
	\begin{displaymath}
		\xymatrix{
			\{1\}\ar[r]^{i} & A\ar[r]^{\lambda} & G\ar[r]^{\mu} & B\ar[r]^{\varphi} & \{1'\}
		}
	\end{displaymath}
	称为\textbf{短正合序列}.
\end{definition}
\begin{remark}
	不难看出,$\lambda$是单射,$\mu$是满射.这是短正合序列的本质体现.因此在书写上,可简写为
	\begin{displaymath}
		\xymatrix{
			1\ar[r] & A\ar[r]^{\lambda} & G\ar[r]^{\mu} & B\ar[r] & 1
		}
	\end{displaymath}
\end{remark}
\begin{theorem}\label{iffshort}
	设$G,A,B$是群,则$G$是$B$过$A$的扩张当且仅当存在短正合序列
	\begin{displaymath}
		\xymatrix{
			1\ar[r] & A\ar[r]^{\lambda} & G\ar[r]^{\mu} & B\ar[r] & 1
		}
	\end{displaymath}
\end{theorem}
\begin{proof}
	必要性:设存在$N\lhd G$,使得$A\cong N$,$B\cong G/N$.设同构映射$f:A\to N$,$h:G/N\to B$,把$f$开拓到$\lambda$,则$\lambda$是单同态. 设$\mu=h\circ \pi$,则$\mu$是满同态. 于是存在短正合序列.
	
	充分性:设存在短正合序列
	\begin{displaymath}
		\xymatrix{
			1\ar[r] & A\ar[r]^{\lambda} & G\ar[r]^{\mu} & B\ar[r] & 1
		}
	\end{displaymath}
	则$\lambda$是单同态,$\mu$是满同态.且
	$$\lambda(A)=\ker\mu\lhd G.$$
	设$N=\ker\mu$,而$\lambda:A\to\lambda(A)$是单同态,又是满射,于是$\lambda$是同构,$A\cong\lambda(A)=N$.
	
	对于满同态$\mu:G\to B$,由同态基本定理,有$G/\ker\mu\cong B$,于是$G/N\cong B$. 因此$G$是$B$过$A$的扩张.
\end{proof}
\begin{theorem}
	设$G,G',A,B$是群.
	\begin{enumerate}
		\item 若$G$是$B$过$A$的扩张,$G\cong G'$,则$G'$也是$B$过$A$的扩张.
		\item 若$G$和$G'$都是$B$过$A$的扩张,且存在同态$f:G\to G'$,使下图交换,则$f$是同构映射.称$G$和$G'$是$B$过$A$的\textbf{等价扩张}.
		\begin{displaymath}
			\xymatrix{
				1\ar[r] & A\ar[r]^{\lambda}\ar[d]^{\id_A} & G\ar[r]^{\mu}\ar[d]^{f} & B\ar[r]\ar[d]^{\id_B} & 1\\
				1\ar[r] & A\ar[r]^{\lambda'} & G'\ar[r]^{\mu'} & B\ar[r] & 1
			}
		\end{displaymath}
	\end{enumerate}
\end{theorem}
\begin{proof}
	1.由于$G$是$B$过$A$的扩张,于是有短正合序列
	\begin{displaymath}
		\xymatrix{
			1\ar[r] & A\ar[r]^{\lambda} & G\ar[r]^{\mu} & B\ar[r] & 1
		}
	\end{displaymath}
	设$f:G\to G'$是同构,则$f\circ\lambda:A\to G'$是单同态,$\mu\circ f^{-1}$是满同态,且$f\circ\lambda(A)=f(\ker\mu)=\ker\mu f^{-1}$,于是$G'$是$B$过$A$的扩张.
	
	2.先证$f$是单射. 只需证明$\ker f=\{e\}$,这里$e$是$G$的幺元,并设$e'$是$G'$的幺元.设$f(x)=e'$,下证$x=e$.
	
	由交换图,
	$$\mu(x)=\mu'f(x)=\mu'(e')=1,$$
	于是$x\in\ker\mu=\lambda(A)$,存在$a\in A$使$x=\lambda(a)$,即
	$$e'=f(x)=f\lambda(a)=\lambda'(a),$$
	又$\lambda$是单射,于是$a=1$,$x=\lambda(1)=e$.
	
	再证$f$是满射. 对任意$x'\in G'$,由$\mu$是满射,$\mu(G)=B$,则存在$x\in G$,使得$\mu(x)=\mu'(x')$.即
	$$\mu'f(x)=\mu'(x'),$$
	于是
	$$\mu'\left(x'\left[f(x)\right]^{-1}\right)=\mu'(e')=1,$$
	即
	$$x'f(x)^{-1}\in\ker\mu'=\lambda'(A)=f\lambda(A)\subset f(G),$$
	于是$x'\in f(G)f(x)\subset f(G)$.
\end{proof}
\begin{remark}
	1的证明中,用了两次扩张的充要条件,即定理\ref{iffshort}.对于$f(\ker\mu)=\ker\mu f^{-1}$,可以由核的定义以及集合的包含关系证得.
\end{remark}
\begin{definition}[内直积]
	设$G$是$B$过$A$的扩张,$N$为扩张核,若存在$H<G$,使得$H\cap N=\{e\}$且$G=HN$,则称此扩张为\textbf{非本质扩张},$G$称为$N$与$H$的\textbf{半直积},记作$G=H\ltimes N$.进一步,若$H\lhd G$,则称这种扩张为\textbf{平凡扩张},$G$是$N$与$H$的\textbf{内直积},记作$G=H\otimes N$.
\end{definition}
\begin{remark}
	对非本质扩张,有$B\cong H$.因为
	$$B\cong G/N=HN/N\cong H/(H\cap N)=H/\{e\}\cong H.$$
\end{remark}
\begin{example}
	设$G=(\mathbb{Z},+)$,$A=N=2\mathbb{Z}\lhd\mathbb{Z}$,$B=G/N=\mathbb{Z}_2$,则$G$是$B$过$A$的扩张. 由于不存在子群$H\cong B=\mathbb{Z}_2$,于是这个扩张不是非本质扩张.
\end{example}
\begin{theorem}
	设$A<G$,$B<G$,则
	\begin{enumerate}
		\item $G=AB$且$A\cap B=\{e\}$当且仅当对任意$g\in G$,存在唯一$a\in A,b\in B$使得$g=ab$.
		\item 若$G=AB$且$A\cap B=\{e\}$,则$A,B$都是$G$的正规子群的充要条件为对任意$a\in A,b\in B,ab=ba$.此时$G=A\otimes B$.
	\end{enumerate}
\end{theorem}
\begin{proof}
	1.必要性:由$G=AB$,对任意$g\in G$,存在$a\in A,b\in B$使得$g=ab$,假设另有$a'\in A,b'\in B$使得$g=a'b'$,则$ab=a'b'$,$bb'^{-1}=a^{-1}a'=e$,于是$a=a',b=b'$.
	
	充分性:若对任意$g\in G$,存在唯一$a\in A,b\in B$使得$g=ab$,则$G=AB$. 若$c\in A\cap B$,则$c=ec=ce$,于是$c=e$.
	
	2.必要性:若$A\lhd G$,则$bab^{-1}\in A$,于是$a^{-1}bab^{-1}\in A$. 又$B\lhd G$,则$a^{-1}ba\in B$,于是$a^{-1}bab^{-1}\in B$,故$a^{-1}bab^{-1}\in A\cap B$. 于是$a^{-1}bab^{-1}=e$,$ba=ab$.
	
	充分性:若对任意$a\in A$,$b\in B$有$ab=ba$,由于$G=AB$,对任意$g\in G$,存在$a\in A,b\in B$使得$g=ab$,于是对任意$a_0\in A$,
	$$ga_0g^{-1}=aba_0b^{-1}a^{-1}=aa_0bb^{-1}a^{-1}=aa_0a^{-1}\in A,$$
	于是$A\lhd G$,同理$B\lhd G$.
\end{proof}
可以将内直积的概念推广到多个正规子群的情况.
\begin{definition}
	设$N_1,N_2,\cdots,N_k$是$G$的正规子群.若$G$中任意元素分解为$N_i$中元素的乘积是唯一的,则称$G$是$N_1,N_2,\cdots,N_k$的内直积,记作
	$$G=N_1\otimes N_2\otimes\cdots\otimes N_k=\bigotimes_{i=1}^{k}N_i.$$
\end{definition}
以上讨论了一个群的内直积分解,下面说明两个群的内直积总是存在且唯一.
\begin{definition}[外直积]
	设$A,B$是两个群,定义集合$G=\left\{(a,b)\mid a\in A,b\in B\right\}$,定义$G$中元素的运算$(a_1,b_1)(a_2,b_2)=(a_1a_2,b_1b_2)$.则可验证$G$关于上述运算构成群,称为$A$和$B$的\textbf{外直积},记作$G=A\times B$.
\end{definition}
\begin{theorem}
	设$A$和$B$是两个群,则一定存在$B$过$A$的平凡扩张$G$,且$G$在同构意义下唯一.
\end{theorem}
\begin{proof}
	设$G=A\times B$,则$G$是群.记$A'=\left\{(a,1')\mid a\in A\right\}$,$B'=\left\{(1,b)\mid b\in B\right\}$,则可证$A'\lhd G$,$B'\lhd G$,且$G=A'B'$,$A'\cap B'=\left\{(1,1')\right\}$.
	
	由内直积定义,$G=A'\otimes B'=B'\otimes A'$.则$G$是$B'$过$A'$的平凡扩张.容易在$A$和$A'$,$B$和$B'$建立同构,即$A\to A',a\mapsto(a,1')$,$B\to B',b\mapsto(1,b)$,故$G$是$B$过$A$的平凡扩张.
	
	设$G_1$也是$B$过$A$的平凡扩张.则有$A_1\lhd G_1$,$B_1\lhd G_1$,$G_1=A_1B_1$,$A_1\cap B_1=\{e'\}$,且$A\cong A_1$,$B\cong B_1$.下证$G\cong G_1$.
	
	设$f_1:A\to A_1,a\mapsto a_1$,$f_2:B\to B_1,b\mapsto b_1$是两个同构映射.令$f:G\to G_1,(a,b)\mapsto f_1(a)f_2(b)$.下证$f$是同构.
	
	因为$f_1$和$f_2$都是满射,于是对任意$f_1(a)$和$f_2(b)$都有原像$a$和$b$.于是$f$是满射.
	
	假设$f_1(a')f_2(b')=f_1(a)f_2(b)$,由于$G_1$是平凡扩张,因此分解是唯一的. $f_1(a')=f_1(a)$,$f_2(b')=f_2(b)$.又因为$f_1$和$f_2$是单射,于是$a'=a,b'=b$,$(a,b)=(a',b')$.
	
	而
	\begin{equation*}
		\begin{aligned}
			&f\left((a,b)(a',b')\right)=f\left((aa',bb')\right)=f_1(aa')f_2(bb')=f_1(a)f_1(a')f_2(b)f_2(b')\\
			&=f_1(a)f_2(b)f_1(a')f_2(b')=f\left((a,b)\right)f\left((a',b')\right),
		\end{aligned}
	\end{equation*}
	于是$f$是同构映射.
\end{proof}
\begin{remark}
	外直积$G=G_1\times G_2$中,$G_1$和$G_2$一般不是$G$的子群,但是存在某个同构关系,使得$G_1$,$G_2$分别和$G$的两个子群同构. 而在内直积$G=H\otimes N$中,$H$和$N$都是$G$的正规子群.内直积和外直积在本质上是一致的.
\end{remark}
\begin{remark}
	上述定理实际上说明了对于$G=A\times B$,存在$A'\lhd G$,$B'\lhd G$且$A'\cong A$,$B'\cong B$,使得$G=A'\otimes B'$.
\end{remark}
反之,对于$G=A\otimes B$,它和外直积的关系如下.
\begin{theorem}
	若$G=A\otimes B$,则$A\times B\cong G$.
\end{theorem}
\begin{proof}
	令$f:A\times B\to G,(a,b)\mapsto ab$,容易判断这是良定义的.对任意$g\in G$,都有唯一$a\in A$,$b\in B$使得$g=ab$,于是$f$是满射.对任意$g_1=g_2$,有$(a_1,b_1)=(a_2,b_2)$,于是$a_1=a_2$,$b_1=b_2$,于是$f$是单射.又
	$$f\left((a_1,b_1)(a_2,b_2)\right)=f\left((a_1a_2,b_1b_2)\right)=a_1a_2b_1b_2=a_1b_1a_2b_2=f\left((a_1,b_1)\right)f\left((a_2,b_2)\right),$$
	于是$f$是同构映射.
\end{proof}
下面介绍外直积的若干性质.
\begin{theorem}
	设$G=A\times B$,则
	\begin{enumerate}
		\item $G$是有限群当且仅当$A$和$B$都是有限群,且当$G$为有限群时,有$|G|=|A||B|$.
		\item $G$是交换群当且仅当$A$和$B$都是交换群.
		\item $A\times B\cong B\times A$.
	\end{enumerate}
\end{theorem}
\begin{proof}
	1.由Cartesian积的性质立即可得.
	
	2.若$G$是交换群,对任意$a_1,a_2\in A$,$b_1,b_2\in B$有$(a_1,b_1)(a_2,b_2)=(a_2,b_2)(a_1,b_1)$,即$(a_1a_2,b_1b_2)=(a_2a_1,b_2b_1)$,于是$a_1a_2=a_2a_1$,$b_1b_2=b_2b_1$.反之,若$A$和$B$都是交换群,对任意$(a_1,b_1),(a_2,b_2)\in G$,有
	$$(a_1,b_1)(a_2,b_2)=(a_1a_2,b_1b_2)=(a_2a_1,b_2b_1)=(a_2,b_2)(a_1,b_1),$$
	于是$G$是交换群.
	
	3.设映射$f:A\times B\to B\times A,(a,b)\mapsto(b,a)$,则这是良定义的,且是双射.而
	$$f\left((a_1,b_1)(a_2,b_2)\right)=f\left((a_1a_2,b_1b_2)\right)=(b_1b_2,a_1a_2)=(b_1,a_1)(b_2,a_2)=f\left((a_1,b_1)\right)f\left((a_2,b_2)\right),$$
	于是$f$是同构映射.
\end{proof}
\begin{theorem}
	设$A,B$是群,$a\in A,b\in B$是两个有限阶元,则对$(a,b)\in A\times B$,有
	$$|(a,b)|=\left[|a|,|b|\right].$$
\end{theorem}
\begin{proof}
	设$|a|=m$,$|b|=n$,$|(a,b)|=t$,$\left[|a|,|b|\right]=s$.则$(a,b)^s=(a^s,b^s)=(e_1,e_2)$,于是$t\mid s$.又$(e_1,e_2)=(a,b)^t=(a^t,b^t)$,于是$a^t=e_1$,$b^t=e_2$,于是$m\mid t$,$n\mid t$,而$s=\left[m,n\right]$,于是$s\mid t$,所以$t=s$,即$|(a,b)|=\left[|a|,|b|\right]$.
\end{proof}
\section{可解群与幂零群}
\begin{definition}[换位子]
	设$g_1,g_2\in G$,称
	$$\left[g_1,g_2\right]=g_1^{-1}g_2^{-1}g_1g_2$$
	为$g_1$和$g_2$的\textbf{换位子}.
\end{definition}
可见,换位子的作用是换位,即$g_2g_1\left[g_1,g_2\right]=g_1g_2$.且有
$$\left[g_1,g_2\right]\left[g_2,g_1\right]=e.$$
即$\left[g_2,g_1\right]=\left[g_1,g_2\right]^{-1}$.
\begin{definition}[换位子群]
	设$H<G$,$K<G$,称
	$$\left[H,K\right]=\langle \left\{\left[h,k\right]\mid h\in H,k\in K\right\}\rangle$$
	为$H$和$K$的\textbf{换位子群}.
\end{definition}
可见,$\left[H,K\right]=\left[K,H\right]$.
\begin{property}
	设$\alpha:G\to G_1$是同态,则
	\begin{enumerate}
		\item 对任意$g_1,g_2\in G$,$\alpha\left(\left[g_1,g_2\right]\right)=\left[\alpha(g_1),\alpha(g_2)\right]$.
		\item 对任意$H<G$,$K<G$,$\alpha\left(\left[H,K\right]\right)=\left[\alpha(H),\alpha(K)\right]$.
	\end{enumerate}
\end{property}
\begin{lemma}
	设$H<G$,$K<G$,则
	\begin{enumerate}
		\item $\left[H,K\right]=\{1\}\iff H\subset C_G(K)$;
		\item $\left[H,K\right]\subset K\iff H\subset N_G(K)$;
		\item 若$H\lhd G$,$G\lhd G$,则$\left[H,K\right]\lhd G$且$\left[H,K\right]\subset H\cap K$;
		\item 若$H_1<H$,$K_1<K$,则$\left[H_1,K_1\right]<\left[H,K\right]$.
	\end{enumerate}
\end{lemma}
\begin{corollary}
	设$H<G$,$K<G$,则
	\begin{enumerate}
		\item $G$是交换群当且仅当$\left[G,G\right]=\{1\}$;
		\item $K\lhd G\iff \left[K,K\right]\lhd G$;
		\item $\left[G,G\right]\lhd G$.
	\end{enumerate}
\end{corollary}
\begin{definition}[正规列]
	设群$G$的幺元为$1$,它的子群$G_i$有如下排列
	$$G=G_1\supset G_2\supset\cdots\supset G_t\supset G_{t+1}=\{1\},$$
	且$G_i\lhd G_{i-1},2\leqslant i\leqslant t+1$,则称这个序列为\textbf{次正规列}.若还有$G_i\lhd G$,则称这个序列为\textbf{正规列}.上述序列中有$t$个包含号,所以称序列的长度为$t$.称$G_i/G_{i-1}$为次正规序列的\textbf{因子}.
\end{definition}
\begin{definition}[因子列]
	次正规序列
	$$G=G_1\supset G_2\supset\cdots\supset G_t\supset G_{t+1}=\{1\}$$
	的因子
	$$G_1/G_2,G_2/G_3,\cdots,G_t/G_{t+1}$$
	称为次正规序列的\textbf{因子列}.
\end{definition}
\begin{remark}
	因子列没有包含关系.
\end{remark}
\begin{definition}[加细]
	设有两个次正规序列
	$$G=G_1'\supset G_2'\supset\cdots\supset G_r'\supset G_{r+1}'=\{1\},$$
	$$G=G_1\supset G_2\supset\cdots\supset G_t\supset G_{t+1}=\{1\},$$
	若对任意$G_i'$,都有$G_j=G_i'$,则称后者是前者作为次正规序列的\textbf{加细}.
\end{definition}
\begin{remark}
	正规序列的加细也有类似的定义.若正规序列加细后仍是正规序列,则称后者是前者作为正规序列的加细.但是,若正规序列加细后不再是正规序列,则把这个正规序列看作次正规序列,后者是前者作为次正规序列的加细.
\end{remark}
\begin{definition}[导出列]
	定义$G^{(0)}=G$,$G^{(i)}=\left[G^{(i-1)},G^{(i-1)}\right],i\geqslant1$,称序列
	$$G=G^{(0)}\supset G^{(1)}\supset G^{(2)}\supset\cdots$$
	为$G$的\textbf{导出列}.
\end{definition}
\begin{definition}[降中心列]
	定义$\varGamma_1(G)=G$,$\varGamma_i(G)=\left[G,\varGamma_{i-1}(G)\right],i\geqslant 2$,称序列
	$$G=\varGamma_1(G)\supset\varGamma_2(G)\supset\cdots$$
	为$G$的\textbf{降中心列}.
\end{definition}
\begin{definition}[升中心列]
	定义$C_0(G)=\{1\}$,$C_i(G)/C_{i-1}(G)=C(G/C_{i-1}(G)),i\geqslant 1$,称序列
	$$\{1\}=C_0(G)\subset C_1(G)\subset C_2(G)\subset\cdots$$
	为$G$的\textbf{升中心列}.
\end{definition}
\begin{remark}
	$C_i(G)$是存在的,可以写成显性表达式.
\end{remark}
\begin{definition}[可解群,幂零群]
	设$G$是群,若有$k$,使$G^{(k)}=\{1\}$,则称$G$是\textbf{可解群}.若有$k$,使$\varGamma_k(G)=\{1\}$,则称$G$是\textbf{幂零群}.
\end{definition}
\end{document}
