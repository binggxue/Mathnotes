\documentclass[12pt]{ctexart}
\usepackage{amsfonts,amssymb,amsmath,amsthm,geometry,enumerate}
\usepackage[colorlinks,linkcolor=blue,anchorcolor=blue,citecolor=green]{hyperref}
\usepackage[all]{xy}
%introduce theorem environment
\theoremstyle{definition}
\newtheorem{definition}{定义}
\newtheorem{theorem}{定理}
\newtheorem{lemma}{引理}
\newtheorem{corollary}{推论}
\newtheorem{property}{性质}
\newtheorem{example}{例}
\theoremstyle{plain}
\newtheorem*{solution}{解}
\newtheorem*{remark}{注}
\geometry{a4paper,scale=0.8}

%article info
\title{\vspace{-2em}\textbf{群的概念}\vspace{-2em}}
\date{ }
\begin{document}
	\maketitle
	\begin{definition}[半群]
		设集合$S$带有二元运算“$\circ$”,若对任意$a,b,c\in S$,有$(a\circ b)\circ c=a\circ(b\circ c)$,则称$(S,\circ)$是\textbf{半群}.
	\end{definition}
	\begin{definition}[幺半群]
		设半群$(M,\circ)$,若存在$e\in M$,对任意$a\in M$,有$e\circ a=a\circ e=a$,则称$(M,\circ)$为\textbf{幺半群},$e$称为$(M,\circ)$的\textbf{幺元}.
	\end{definition}
	\begin{definition}[群]
		设集合$G$带有二元运算“$\circ$”,满足以下条件:
		\begin{enumerate}
			\item 对任意$a,b,c\in G$,有$(a\circ b)\circ c=a\circ (b\circ c)$;
			\item 存在$e\in G$,对任意$a\in G$,有$e\circ a=a\circ e=a$;
			\item 对任意$a\in G$,存在$a^{-1}\in G$,使得$a\circ a^{-1}=a^{-1}\circ a=e$,
		\end{enumerate}
		则称$(G,\circ)$是一个\textbf{群}.
	\end{definition}
	\begin{remark}
		在不引起歧义的情况下,可以说“设$G$是一个群”,而将运算“$\circ$”看作广义上的“乘法”而省略,即“$a\circ b$”记作“$ab$”.
	\end{remark}
	\begin{remark}
		称“$a^{-1}$”为$a$的\textbf{逆元}.
	\end{remark}
	\begin{definition}[交换群]
		若$G$中任意元素$a,b$满足$ab=ba$,则称$G$是\textbf{交换群}或\textbf{Abel群}.
	\end{definition}
	\begin{theorem}
		群$G$的幺元是唯一的.
	\end{theorem}
	\begin{proof}
		设$e,e'\in G$都是幺元,则$e=ee'=e'$.
	\end{proof}
	\begin{theorem}
		群$G$的逆元是唯一的.
	\end{theorem}
	\begin{proof}
		对任意$a\in G$,设$b,c\in G$是$a$的逆元,则$b=b(ac)=(ba)c=c$.
	\end{proof}
	由于结合律成立,则$a^n=\underbrace{a\circ a\circ\cdots\circ a}_{n\text{个}}$是良定义的.同时有$a^ma^n=a^{m+n}$,$(a^s)^t=a^{st}$.
	\begin{theorem}[消去律]
		设$a,b,c\in G$,若$ab=ac$,则$b=c$;若$ba=ca$,则$b=c$.
	\end{theorem}
	\begin{proof}
		$ab=ac\Rightarrow a^{-1}ab=a^{-1}ac\Rightarrow b=c$,类似地$ba=ca\Rightarrow baa^{-1}=caa^{-1}\Rightarrow b=c$.
	\end{proof}
	\begin{definition}[群的阶]
		群$G$中元素的个数称为群的\textbf{阶},记作$|G|$.当$|G|$有限时,称$G$为\textbf{有限群},否则称$G$为\textbf{无限群}.
	\end{definition}
	\begin{definition}[群中元素的阶]
		设$G$为一个群,$a\in G$,若存在正整数$k$使得$a^k=e$,则最小的正整数$k$称为$a$的\textbf{阶},记作$|a|$.即$|a|=\min\left\{k\in \mathbb{N}^+|a^k=e\right\}$. 若不存在这样的$k$,则称$a$的阶为无穷.
	\end{definition}
	\begin{theorem}
		$|a|=\infty\iff \forall m, n\in \mathbb{N}^+,m\neq n,a^m\neq a^n$.
	\end{theorem}
	\begin{proof}
		设$m.n.k\in\mathbb{N}^+$且$m=n+k$.则$|a|=\infty\iff \forall k\in\mathbb{N}^+$,$a^k\neq e\iff a^{m-n}\neq e\iff a^m\neq a^n$.
	\end{proof}
	\begin{theorem}
		设$|a|=d$,则$\forall h\in\mathbb{Z}$,$a^h=e\iff d|h$.
	\end{theorem}
	\begin{proof}
		充分性:$d|h$,则存在$k\in\mathbb{Z}$使得$h=kd$,则$a^{h}=a^{kd}=(a^{d})^k=e$.
		
		必要性:设$h=qd+r$,$0 \leqslant r<d$,则$a^{qd+r}=a^{qd}a^{r}=a^r=e$,则$r=0$.
	\end{proof}
	\begin{corollary}
		对任意$m,n\in\mathbb{Z}$,$a^m=a^n\iff d|m-n\iff m\equiv n\pmod d$.
	\end{corollary}
	\begin{theorem}
		设$|a|=d$,$k\in\mathbb{N}^+$,则$|a^k|=\dfrac{d}{(d,k)}$.
	\end{theorem}
	\begin{proof}
		设$|a^k|=h$,则$a^{kh}=e$.而$|a|=d$,则$d|kh$.
		
		设$d=d_1(d,k)$,$k=k_1(d,k)$,则$(d_1,k_1)=1$,$d_1|k_1h$,则$d_1|h$.
		
		$(a^k)^{d_1}=a^{kd_1}=a^{dk_1}=e$,故$h|d_1$.于是$|a^k|=h=d_1=\dfrac{d}{(d,k)}$.
	\end{proof}
	\begin{corollary}
		$|a^k|=d\iff (d,k)=1$.
	\end{corollary}
	\begin{theorem}
		设$a,b\in G$,$|a|=m$,$|b|=n$,$ab=ba$,$(m,n)=1$,则$|ab|=mn$.
	\end{theorem}
	\begin{proof}
		设$|ab|=d$,而$(ab)^{mn}=(a^m)^n(b^n)^m=e$,于是$d|mn$;
		
		又$(ab)^{md}=a^{md}b^{md}=b^{md}=e$,故$n|md$,而$(m,n)=1$,于是$n|d$.同理$m|d$,于是$mn|d$,故$mn=d$.
	\end{proof}
\end{document}