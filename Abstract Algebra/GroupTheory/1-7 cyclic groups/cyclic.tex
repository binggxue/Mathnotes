\documentclass[12pt]{ctexart}
\usepackage{amsfonts,amssymb,amsmath,amsthm,geometry,enumerate}
\usepackage[colorlinks,linkcolor=blue,anchorcolor=blue,citecolor=green]{hyperref}
\usepackage[all]{xy}
%introduce theorem environment
\theoremstyle{definition}
\newtheorem{definition}{定义}
\newtheorem{theorem}{定理}
\newtheorem{lemma}{引理}
\newtheorem{corollary}{推论}
\newtheorem{property}{性质}
\newtheorem{proposition}{命题}
\newtheorem{example}{例}
\theoremstyle{plain}
\newtheorem*{solution}{解}
\newtheorem*{remark}{注}
\geometry{a4paper,scale=0.8}

%article info
\title{\vspace{-2em}\textbf{循环群与生成组}\vspace{-2em}}
\date{ }
\begin{document}
	\maketitle
	\begin{definition}[循环群]
		由一个元素$a$反复运算得到的群称为\textbf{循环群},记作$\langle a\rangle$.这个元素称为群的\textbf{生成元}.
	\end{definition}
	\begin{theorem}
		循环群都是交换群.
	\end{theorem}
	\begin{proof}
		对任意$a^m,a^n\in\langle a\rangle$,$a^ma^n=a^{m+n}=a^{n+m}=a^na^m$.
	\end{proof}
	\begin{theorem}
		循环群的子群仍是循环群.
	\end{theorem}
	\begin{proof}
		设$G_1<\langle a\rangle$,设$k=\min\left\{m\in\mathbb{N}^{+}\ |\ a^m\in G_1\right\}$,则$\langle a^k\rangle\subset G_1$.
		
		对任意$a^n\in G_1$,设$n=qk+r$,则$a^n=a^{qk}a^r\in G_1$,于是$a^r\in G_1$,$0\leqslant r<k$,则$r=0$. 于是$a_n\in\langle a^k\rangle$,则$G_1\subset\langle a^k\rangle$.
	\end{proof}
	\begin{theorem}
		设循环群$G=\langle a\rangle$.若$|G|=m$,则$G\cong(\mathbb{Z}_m,+)$;若$|G|=\infty$,则$G\cong(\mathbb{Z},+)$.
	\end{theorem}
	\begin{proof}
		设$f:\mathbb{Z}\to G,\ n\to a^n$. 显然$f$是映射.任意$a^n$都有$n$对应,故$f$是满射.对任意$m,n\in\mathbb{Z}$,
		$$f(m+n)=a^{m+n}=a^ma^n=f(m)f(n),$$
		故$f$是满同态. 由群同态基本定理,
		$$\mathbb{Z}/\ker f\cong G.$$
		而$\ker f\lhd\mathbb{Z}=m\mathbb{Z}$,这里存在$m\in\mathbb{N}$. 当$m=0$时,$\ker f=\{0\}$,则$\mathbb{Z}\cong G$. 当$m>0$时,$\mathbb{Z}/\ker f=\mathbb{Z}/m\mathbb{Z}=\mathbb{Z}_m\cong G$.
	\end{proof}
	\begin{theorem}
		设$|G|=m$,则$G$是循环群的充要条件是对每一个正整数因子$m_1|m$,都存在唯一的$m_1$阶子群.
	\end{theorem}
	\begin{proposition}\label{ord}
		有限群$G$中元素的阶是$|G|$的因子.
	\end{proposition}
	\begin{proof}
		显然有限群中元素的阶有限,设$a\in G,|a|=d$,则
		$$\langle a\rangle=\left\{e,a,a^2,\cdots,a^{d-1}\right\},$$
		而$\langle a\rangle<G,|\langle a\rangle|=d$,由Lagrange定理得证.
	\end{proof}
	\begin{proposition}
		素数阶群必为循环群.
	\end{proposition}
	\begin{proof}
		设有限群$|G|=p$,$p$是素数,则由命题\ref{ord},对任意$g\in G$,$|g|=1$或$p$.当$|g|=1$时,$g=e$.当$|g|=p$时,$|\langle g\rangle|=p=|G|$,而$\langle g\rangle<G$,于是$\langle g\rangle=G$,即$G$是由$g$生成的循环群.
	\end{proof}
	\begin{definition}[生成的子群]
		设$S$是群$G$的非空子集,包含$S$的最小子群称为$S$\textbf{生成的子群},记作$\langle S\rangle$. 等价定义为包含$S$的所有子群的交.
	\end{definition}
	\begin{theorem}
		设$S$是群$G$的非空子集,$S^{-1}=\left\{a^{-1}\ |\ a\in S\right\}$,则
		$$\langle S\rangle=\left\{x_1x_2\cdots x_m\ |\ x_i\in S\cup S^{-1}\right\}$$
	\end{theorem}
	\begin{proof}
		设$T=\left\{x_1x_2\cdots x_m\ |\ x_i\in S\cup S^{-1}\right\}$. 由于$S\subset\langle S\rangle$,$S^{-1}\subset\langle S\rangle$,于是$S\cup S^{-1}\subset\langle S\rangle$,则$T\subset\langle S\rangle$.下面证明$T$是子群.
		
		设$x_1x_2\cdots x_n,y_1y_2\cdots y_m\in T$,则$y_i^{-1}\in S\cup S^{-1}$,于是
		$$x_1x_2\cdots x_n(y_1y_2\cdots y_m)^{-1}=x_1x_2\cdots x_ny_m^{-1}y_{m-1}^{-1}\cdots y_1^{-1}\in T.$$
		故$T<\langle S\rangle$,而$\langle S\rangle$是包含$S$的最小子群,故$T=S$.
	\end{proof}
	\begin{definition}[生成组]
		若$G=\langle S\rangle$,则称$S$为$G$的\textbf{生成组}.
	\end{definition}
	\begin{definition}[有限生成群]
		若存在群$G$的有限个元素的生成组,则称$G$是\textbf{有限生成群}. 若$G$还是交换群,则称为\textbf{有限生成Abel群}.
	\end{definition}
	注意到有限群是有限生成群,但有限生成群不一定是有限群,例如$(\mathbb{Z},+)=\langle 1\rangle$.
\end{document}