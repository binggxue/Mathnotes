\documentclass[12pt]{ctexart}
\usepackage{amsfonts,amssymb,amsmath,amsthm,geometry,enumerate}
\usepackage[colorlinks,linkcolor=blue,anchorcolor=blue,citecolor=green]{hyperref}
\usepackage[all]{xy}
%introduce theorem environment
\theoremstyle{definition}
\newtheorem{definition}{定义}
\newtheorem{theorem}{定理}
\newtheorem{lemma}{引理}
\newtheorem{corollary}{推论}
\newtheorem{property}{性质}
\newtheorem{proposition}{命题}
\newtheorem{example}{例}
\theoremstyle{plain}
\newtheorem*{solution}{解}
\newtheorem*{remark}{注}
\geometry{a4paper,scale=0.8}

\newcommand{\id}{\mathrm{id}}
\newcommand{\Aut}{\mathrm{Aut}}
\newcommand{\Inn}{\mathrm{Inn}}
\newcommand{\Orb}{\mathrm{Orb}}
\newcommand{\Stab}{\mathrm{Stab}}

\everymath{\displaystyle}

%article info
\title{\vspace{-2em}\textbf{Sylow子群}\vspace{-2em}}
\date{ }
\begin{document}
	\maketitle
	\begin{definition}[$p$-群]
		设$G$是有限群,$p$是素数,若$|G|=p^k,k\in\mathbb{N}^+$,则称$G$是一个$\boldsymbol{p}$\textbf{-群}.
	\end{definition}
	\begin{lemma}\label{fixedpoint}
		设$p$-群$G$作用在集合$X$上,若$|X|=n$,$X$中的不动点个数为$t\ (t\in\mathbb{N})$,则
		\begin{enumerate}[(1)]
			\item $t\equiv n\pmod p$;
			\item 若$(n,p)=1$,则不动点存在.
		\end{enumerate}
	\end{lemma}
	\begin{proof}
		(1)设$X=\displaystyle\bigsqcup_{i\in I}\Orb(x_i)$. $x_i$为不动点当且仅当$|\Orb(x_i)|=1$,于是
		$$n=t+\sum_{|\Orb(x_i)\neq 1|}|\Orb(x_i)|.$$
		而由轨道-稳定化子定理,$|\Orb(x_i)|$能整除$|G|$,而$|G|=p^k\ (k\in\mathbb{N}^{+})$,于是$p$能整除$|\Orb(x_i)|$.故$t\equiv n\pmod p$.
		
		(2)若$(n,p)=1$,则$n\nmid p$,由(1),得$t\nmid p$,则$t\neq 0$,即存在不动点.
	\end{proof}
	\begin{lemma}\label{divide}
		在正整数中,设$p$是素数,$n=p^lm$,若$k\leqslant l$,则$p^{l-k}$恰能整除$\mathrm{C}_{n}^{p^k}$.
	\end{lemma}
	\begin{proof}
		由组合数公式,
		$$\mathrm{C}_{n}^{p^k}=\frac{n}{p^k}\prod_{i=1}^{p^k-1}\frac{n-i}{p^k-i},$$
		而
		$$\frac{n}{p^k}=p^{l-k}m\Rightarrow p^{l-k}\mid\mathrm{C}_{n}^{p^k},$$
		
		设$1\leqslant i\leqslant p^k-1$表示为$i=p^tj$,其中$(p,j)=1$,$t<k\leqslant l$. 则
		$$n-i=p^t\left(p^{l-t}m-j\right),$$
		$$p^{k}-i=p^t\left(p^{k-t}-j\right),$$
		于是$p\nmid\prod_{i=1}^{p^k-1}\frac{n-i}{p^k-i}$,故$p^{l-k}$恰能整除$\mathrm{C}_{n}^{p^k}$.
	\end{proof}
	下面若无特殊说明,默认$G$的阶为$p^lm$,其中$p$为素数,$(p,m)=1$,$l\geqslant 1$.
	\begin{theorem}[Sylow第一定理,存在性]
		若$1\leqslant k\leqslant l$,则$G$存在$p^k$阶子群.
	\end{theorem}
	\begin{proof}
		设$G$中所有$p^k$阶子集组成的集合为$\mathcal{X}$. 则$|\mathcal{X}|=\mathrm{C}_{n}^{p^k}$,这里$n=p^lm$.
		设$G$作用在$\mathcal{X}$上,则有轨道分解
		$$\mathcal{X}=\bigsqcup_{i\in I}\Orb(A_i),\quad A_i\in\mathcal{X}.$$
		于是
		$$|\mathcal{X}|=\sum_{i\in I}|\Orb(A_i)|.$$
		由引理\ref{divide},存在$A\in\mathcal{X}$,$p^{l-k+1}\nmid|\Orb(A)|$.由轨道-稳定化子定理,
		$$p^{l-k+1}\nmid \frac{|G|}{|\Stab(A)|}\Rightarrow p^{l-k+1}\nmid\frac{p^lm}{|\Stab(A)|}.$$
		设$|\Stab(A)|=p^ab$,其中$(a,b)=1$.则$p^{l-a}<p^{l-k+1}$,即$a>k-1$,$a\geqslant k$.于是$p^k\mid|\Stab(A)|$.
		
		由于$\Stab(A)<G$,对任意$g\in\Stab(A),a\in A$,定义群$\Stab(A)$对集合$A$的作用$g\cdot a=ga$. 由于$\Stab(A)=\left\{g\in G\mid g\cdot a=a,\ \forall a\in A\right\}$,于是$ga\in A$. 则$\Stab(A)\cdot a\subset A$. 而$\Stab(A)$到$\Stab(A)\cdot a$之间是双射,于是$|\Stab(A)|=|\Stab(A)\cdot a|\leqslant|A|=p^k$.即$\Stab(A)$是一个$p^k$阶子群.
	\end{proof}
	\begin{definition}[Sylow $p$-子群]
	设$G$的阶是$p^lm$,其中$p$是素数,则$G$的$p^l$阶子群称为$G$的\textbf{Sylow\ $\boldsymbol{p}$-子群}.
	\end{definition}
	\begin{theorem}[Sylow第二定理,共轭性]
		设$P$是$G$的一个Sylow $p$-子群,$H$是$P$的一个$p^k$阶子群,则$H$包含于$P$的共轭子群中. 特别地,Sylow $p$-子群之间互相共轭.
	\end{theorem}
	\begin{proof}
		设$G$作用在$G/P$上,$g\cdot gP=ggP$,称为左平移作用.将这个作用限制在$H$上,则$h\cdot gP=hgP$. 由于$|G/P|=m$,$(m,p)=1$,由引理\ref{fixedpoint}(2),存在$gP\in G/P$满足$hgP=gP$.于是$hg\in gP$,即$h\in gPg^{-1}$,$H$包含于$P$的共轭子群中.特别地,当$|H|=p^l$,则$P$也包含在$H$的共轭子群中,于是$H=gPg^{-1}$.
	\end{proof}
	\begin{theorem}[Sylow第三定理,计数定理]
		设$G$的Sylow $p$-子群的个数为$k$,则
		\begin{enumerate}[(1)]
			\item 当且仅当$k=1$时,这个Sylow $p$-子群$P\lhd G$;
			\item $k\equiv 1\pmod p$且$k\mid m$.
		\end{enumerate}
	\end{theorem}
	\begin{proof}
		(1)设$P$是群$G$的一个Sylow $p$-子群. 若$P'$是另外一个Sylow $p$-子群,则由Sylow第二定理,有$P'\subset\left\{gPg^{-1}\mid g\in G\right\}$,同时有$P\subset\left\{gP'g^{-1}\mid g\in G\right\}$. 若$k=1$,则$P=gPg^{-1}$,对任意$g\in G$成立,于是$P\lhd G$.反之,若$P\lhd G$,则$P=gPg^{-1}$,得$k=1$.
		
		(2)设$\mathcal{X}$是群$G$的所有Sylow $p$-子群的集合,群$P\in\mathcal{X}$作用在集合$\mathcal{X}$上的作用为共轭作用.对任意$g\in P$,
		$$g\cdot P=gPg^{-1}=P,$$
		因此$P$是该作用下的一个不动点.假设$P_1$也是一个不动点,则对任意$g\in P$,
		$$gP_1g^{-1}=P_1,$$
		因此$g\in N_G(P_1)$,$P\subset N_G(P_1)$.而$|P|=p^l$,于是设$|N_G(P_1)|=p^lm_1$,其中$m_1\mid m$.于是$P,P_1$都是$N_G(P_1)$的Sylow $p$-子群,而$P_1\lhd N_G(P_1)$,由(1),得$k=1$,即$P=P_1$,该作用下只有一个不动点. 由引理\ref{fixedpoint}(1),有$k\equiv 1\pmod p$.
		
		设群$G$在集合$\mathcal{X}$上的作用为共轭作用.则由Sylow第二定理,对任意$P_1,P_2\in\mathcal{X}$,存在$g\in G$,使得
		$$P_1=g\cdot P_2=gP_2g^{-1},$$
		于是$\mathcal{X}$是可传递的.对任意$P\in\mathcal{X}$,
		$$k=|\mathcal{X}|=|\Orb(P)|=\frac{|G|}{|\Stab(P)|},$$
		于是$k\mid|G|$,即$k\mid p^lm$.而由于$k\equiv 1\pmod p$,于是$(k,p)=1$,则$k\mid m$.
	\end{proof}
	下面介绍Sylow定理的若干应用.
	\begin{definition}[单群]
		没有非平凡正规子群的群称为\textbf{单群}。
	\end{definition}
	\begin{example}
		$72$阶群不是单群.
	\end{example}
	\begin{proof}
		首先,$72=2^3\times3^2$,设有限群$G$的阶$|G|=72$,设$G$的Sylow $2$-子群的个数为$k_1$,Sylow $3$-子群的个数为$k_2$. 由Sylow第三定理,$k_1$可能为$1,3,9$,$k_2$可能为$1,4$.
		
		当$k_1=1$时,由Sylow第三定理(1),这个$8$阶的Sylow $2$-子群是$G$的正规子群.当$k_2=1$时,这个$9$阶的Sylow $3$子群也是$G$的正规子群. 它们都不是平凡的.
		
		当$k_2=4$时,设$X=\left\{P_1,P_2,P_3,P_4\right\}$,其中$P_i$是互不相同的Sylow $3$-子群.设$G$作用在$X$上的作用为共轭作用.即
		$$g\cdot P_i=gP_ig^{-1},\ \forall g\in G,$$
		则这个作用决定了一个同态$\varphi:G\to S_X$.
		
		而$\ker\varphi\lhd G$,假设$\ker\varphi=G$,则对任意$g\in G$,
		$$g\cdot P_i=\id(P_i)=P_i,$$
		则Sylow子群之间不能互相共轭,这与Sylow第二定理矛盾.
		
		假设$\ker\varphi=\left\{e\right\}$,则由同态基本定理,
		$$G/\ker\varphi\cong\varphi(G),$$
		于是
		$$|G/\ker\varphi|=|G/e|=|G|=|\varphi(G)|<|S_4|=24,$$
		而$|G|=72$,矛盾.
		
		于是$\ker\varphi$是$G$的非平凡正规子群,故$72$阶群不是单群.
	\end{proof}
	\begin{example}
		56阶群不是单群.
	\end{example}
	\begin{proof}
		首先,$56=2^3\times 7$,设有限群$G$的阶$|G|=56$,设$G$的Sylow $2$-子群的个数为$k_1$,Sylow $7$-子群的个数为$k_2$. 由Sylow第三定理,$k_1$可能为$1,7$,$k_2$可能为$1,8$.
		
		当$k_1=1$时,由Sylow第三定理(1),这个$8$阶的Sylow $2$-子群是$G$的正规子群.当$k_2=1$时,这个$7$阶的Sylow $7$-子群也是$G$的正规子群. 它们都不是平凡的.
		
		当$k_1=7$且$k_2=8$时,由于素数阶群必为循环群,于是这$8$个Sylow $7$-子群中,除幺元外的$6$个元素都是$7$阶的,且各不相同.于是一共含有$|G|$中的$1+6\times 8=49$个元素.对任意的一个Sylow $2$-子群,除幺元外含有$7$个元素,且与Sylow $7$-子群中的元素不同. 这就有$49+7=56$个元素. 而这$7$个Sylow $2$-子群元素不是完全一致的,于是Sylow $7$子群和Sylow $8$子群中不重复的元素个数就超过了$56$,这与$|G|=56$矛盾!于是$k_1=1$或$k_2=1$,则由上述可知$56$阶群不是单群.
	\end{proof}
	\begin{example}
		设$|G|=p^lm$,$(p,m)=1$,$p>m\neq 1$,则$G$是单群.
	\end{example}
	\begin{proof}
		设$G$的Sylow $p$-子群的个数为$k$,由Sylow第三定理,$k$的取值只能为$1$.而$m>1$,于是$G$的Sylow $p$-子群是$G$的$p^l$阶真正规子群.
	\end{proof}
	\begin{remark}
		$k$的取值只能为$1$,因为当$k$取$1+p$时,$1+p>m$于是不能整除. 那么其他取值更不能取到了.
	\end{remark}
	
\end{document}