\documentclass[12pt]{ctexart}
\usepackage{amsfonts,amssymb,amsmath,amsthm,geometry,enumerate}
\usepackage[colorlinks,linkcolor=blue,anchorcolor=blue,citecolor=green]{hyperref}
\usepackage[all]{xy}
%introduce theorem environment
\theoremstyle{definition}
\newtheorem{definition}{定义}
\newtheorem{theorem}{定理}
\newtheorem{lemma}{引理}
\newtheorem{corollary}{推论}
\newtheorem{property}{性质}
\newtheorem{proposition}{命题}
\newtheorem{example}{例}
\theoremstyle{plain}
\newtheorem*{solution}{解}
\newtheorem*{remark}{注}
\geometry{a4paper,scale=0.8}

\newcommand{\id}{\mathrm{id}}
\newcommand{\Aut}{\mathrm{Aut}}
\newcommand{\Inn}{\mathrm{Inn}}
\newcommand{\Orb}{\mathrm{Orb}}
\newcommand{\Stab}{\mathrm{Stab}}

\everymath{\displaystyle}

%article info
\title{\vspace{-2em}\textbf{群的直积}\vspace{-2em}}
\date{ }
\begin{document}
	\maketitle
	\begin{definition}[群的扩张]
		设$G,A,B$是群,若有$N\lhd G$,使得$A\cong N$,$B\cong G/N$,则称群$G$是$B$过$A$的\textbf{扩张}.称$N$为\textbf{扩张核}.
	\end{definition}
	\begin{remark}
		群的扩张与域的扩张完全不同,域从子域扩成扩域,而群不一定是子群,甚至不一定和子群同构.
	\end{remark}
	\begin{definition}[正合序列]
		设$G_1,G_2,\cdots,G_n$是群,有同态映射如下,
		\begin{displaymath}
			\xymatrix{
				G_1\ar[r]^{f_1} & G_2\ar[r]^{f_2} & \cdots\ar[r]^{f_{n-1}} & G_n
			}
		\end{displaymath}
		且满足$f_i(G_i)=\ker f_{i+1}$,则称这个序列为\textbf{正合序列}.
	\end{definition}
	\begin{remark}
		这里群的个数可以是有限的,也可以是无限的.
	\end{remark}
	\begin{definition}[短正合序列]
		设$1$是$A$的幺元,$1'$是$B$的幺元,则正合序列
		\begin{displaymath}
			\xymatrix{
				\{1\}\ar[r]^{i} & A\ar[r]^{\lambda} & G\ar[r]^{\mu} & B\ar[r]^{\varphi} & \{1'\}
			}
		\end{displaymath}
		称为\textbf{短正合序列}.
	\end{definition}
	\begin{remark}
		不难看出,$\lambda$是单射,$\mu$是满射.这是短正合序列的本质体现.因此在书写上,可简写为
		\begin{displaymath}
			\xymatrix{
				1\ar[r] & A\ar[r]^{\lambda} & G\ar[r]^{\mu} & B\ar[r] & 1
			}
		\end{displaymath}
	\end{remark}
	\begin{theorem}\label{iffshort}
		设$G,A,B$是群,则$G$是$B$过$A$的扩张当且仅当存在短正合序列
		\begin{displaymath}
			\xymatrix{
				1\ar[r] & A\ar[r]^{\lambda} & G\ar[r]^{\mu} & B\ar[r] & 1
			}
		\end{displaymath}
	\end{theorem}
	\begin{proof}
		必要性:设存在$N\lhd G$,使得$A\cong N$,$B\cong G/N$.设同构映射$f:A\to N$,$h:G/N\to B$,把$f$开拓到$\lambda$,则$\lambda$是单同态. 设$\mu=h\circ \pi$,则$\mu$是满同态. 于是存在短正合序列.
		
		充分性:设存在短正合序列
		\begin{displaymath}
			\xymatrix{
				1\ar[r] & A\ar[r]^{\lambda} & G\ar[r]^{\mu} & B\ar[r] & 1
			}
		\end{displaymath}
		则$\lambda$是单同态,$\mu$是满同态.且
		$$\lambda(A)=\ker\mu\lhd G.$$
		设$N=\ker\mu$,而$\lambda:A\to\lambda(A)$是单同态,又是满射,于是$\lambda$是同构,$A\cong\lambda(A)=N$.
		
		对于满同态$\mu:G\to B$,由同态基本定理,有$G/\ker\mu\cong B$,于是$G/N\cong B$. 因此$G$是$B$过$A$的扩张.
	\end{proof}
	\begin{theorem}
		设$G,G',A,B$是群.
		\begin{enumerate}
			\item 若$G$是$B$过$A$的扩张,$G\cong G'$,则$G'$也是$B$过$A$的扩张.
			\item 若$G$和$G'$都是$B$过$A$的扩张,且存在同态$f:G\to G'$,使下图交换,则$f$是同构映射.称$G$和$G'$是$B$过$A$的\textbf{等价扩张}.
			\begin{displaymath}
				\xymatrix{
					1\ar[r] & A\ar[r]^{\lambda}\ar[d]^{\id_A} & G\ar[r]^{\mu}\ar[d]^{f} & B\ar[r]\ar[d]^{\id_B} & 1\\
					1\ar[r] & A\ar[r]^{\lambda'} & G'\ar[r]^{\mu'} & B\ar[r] & 1
				}
			\end{displaymath}
		\end{enumerate}
	\end{theorem}
	\begin{proof}
		1.由于$G$是$B$过$A$的扩张,于是有短正合序列
		\begin{displaymath}
			\xymatrix{
				1\ar[r] & A\ar[r]^{\lambda} & G\ar[r]^{\mu} & B\ar[r] & 1
			}
		\end{displaymath}
		设$f:G\to G'$是同构,则$f\circ\lambda:A\to G'$是单同态,$\mu\circ f^{-1}$是满同态,且$f\circ\lambda(A)=f(\ker\mu)=\ker\mu f^{-1}$,于是$G'$是$B$过$A$的扩张.
		
		2.先证$f$是单射. 只需证明$\ker f=\{e\}$,这里$e$是$G$的幺元,并设$e'$是$G'$的幺元.设$f(x)=e'$,下证$x=e$.
		
		由交换图,
		$$\mu(x)=\mu'f(x)=\mu'(e')=1,$$
		于是$x\in\ker\mu=\lambda(A)$,存在$a\in A$使$x=\lambda(a)$,即
		$$e'=f(x)=f\lambda(a)=\lambda'(a),$$
		又$\lambda$是单射,于是$a=1$,$x=\lambda(1)=e$.
		
		再证$f$是满射. 对任意$x'\in G'$,由$\mu$是满射,$\mu(G)=B$,则存在$x\in G$,使得$\mu(x)=\mu'(x')$.即
		$$\mu'f(x)=\mu'(x'),$$
		于是
		$$\mu'\left(x'\left[f(x)\right]^{-1}\right)=\mu'(e')=1,$$
		即
		$$x'f(x)^{-1}\in\ker\mu'=\lambda'(A)=f\lambda(A)\subset f(G),$$
		于是$x'\in f(G)f(x)\subset f(G)$.
	\end{proof}
	\begin{remark}
		1的证明中,用了两次扩张的充要条件,即定理\ref{iffshort}.对于$f(\ker\mu)=\ker\mu f^{-1}$,可以由核的定义以及集合的包含关系证得.
	\end{remark}
	\begin{definition}[内直积]
		设$G$是$B$过$A$的扩张,$N$为扩张核,若存在$H<G$,使得$H\cap N=\{e\}$且$G=HN$,则称此扩张为\textbf{非本质扩张},$G$称为$N$与$H$的\textbf{半直积},记作$G=H\ltimes N$.进一步,若$H\lhd G$,则称这种扩张为\textbf{平凡扩张},$G$是$N$与$H$的\textbf{内直积},记作$G=H\otimes N$.
	\end{definition}
	\begin{remark}
		对非本质扩张,有$B\cong H$.因为
		$$B\cong G/N=HN/N\cong H/(H\cap N)=H/\{e\}\cong H.$$
	\end{remark}
	\begin{example}
		设$G=(\mathbb{Z},+)$,$A=N=2\mathbb{Z}\lhd\mathbb{Z}$,$B=G/N=\mathbb{Z}_2$,则$G$是$B$过$A$的扩张. 由于不存在子群$H\cong B=\mathbb{Z}_2$,于是这个扩张不是非本质扩张.
	\end{example}
	\begin{theorem}
		设$A<G$,$B<G$,则
		\begin{enumerate}
			\item $G=AB$且$A\cap B=\{e\}$当且仅当对任意$g\in G$,存在唯一$a\in A,b\in B$使得$g=ab$.
			\item 若$G=AB$且$A\cap B=\{e\}$,则$A,B$都是$G$的正规子群的充要条件为对任意$a\in A,b\in B,ab=ba$.此时$G=A\otimes B$.
		\end{enumerate}
	\end{theorem}
	\begin{proof}
		1.必要性:由$G=AB$,对任意$g\in G$,存在$a\in A,b\in B$使得$g=ab$,假设另有$a'\in A,b'\in B$使得$g=a'b'$,则$ab=a'b'$,$bb'^{-1}=a^{-1}a'=e$,于是$a=a',b=b'$.
		
		充分性:若对任意$g\in G$,存在唯一$a\in A,b\in B$使得$g=ab$,则$G=AB$. 若$c\in A\cap B$,则$c=ec=ce$,于是$c=e$.
		
		2.必要性:若$A\lhd G$,则$bab^{-1}\in A$,于是$a^{-1}bab^{-1}\in A$. 又$B\lhd G$,则$a^{-1}ba\in B$,于是$a^{-1}bab^{-1}\in B$,故$a^{-1}bab^{-1}\in A\cap B$. 于是$a^{-1}bab^{-1}=e$,$ba=ab$.
		
		充分性:若对任意$a\in A$,$b\in B$有$ab=ba$,由于$G=AB$,对任意$g\in G$,存在$a\in A,b\in B$使得$g=ab$,于是对任意$a_0\in A$,
		$$ga_0g^{-1}=aba_0b^{-1}a^{-1}=aa_0bb^{-1}a^{-1}=aa_0a^{-1}\in A,$$
		于是$A\lhd G$,同理$B\lhd G$.
	\end{proof}
	可以将内直积的概念推广到多个正规子群的情况.
	\begin{definition}
		设$N_1,N_2,\cdots,N_k$是$G$的正规子群.若$G$中任意元素分解为$N_i$中元素的乘积是唯一的,则称$G$是$N_1,N_2,\cdots,N_k$的内直积,记作
		$$G=N_1\otimes N_2\otimes\cdots\otimes N_k=\bigotimes_{i=1}^{k}N_i.$$
	\end{definition}
	以上讨论了一个群的内直积分解,下面说明两个群的内直积总是存在且唯一.
	\begin{definition}[外直积]
		设$A,B$是两个群,定义集合$G=\left\{(a,b)\mid a\in A,b\in B\right\}$,定义$G$中元素的运算$(a_1,b_1)(a_2,b_2)=(a_1a_2,b_1b_2)$.则可验证$G$关于上述运算构成群,称为$A$和$B$的\textbf{外直积},记作$G=A\times B$.
	\end{definition}
	\begin{theorem}
		设$A$和$B$是两个群,则一定存在$B$过$A$的平凡扩张$G$,且$G$在同构意义下唯一.
	\end{theorem}
	\begin{proof}
		设$G=A\times B$,则$G$是群.记$A'=\left\{(a,1')\mid a\in A\right\}$,$B'=\left\{(1,b)\mid b\in B\right\}$,则可证$A'\lhd G$,$B'\lhd G$,且$G=A'B'$,$A'\cap B'=\left\{(1,1')\right\}$.
		
		由内直积定义,$G=A'\otimes B'=B'\otimes A'$.则$G$是$B'$过$A'$的平凡扩张.容易在$A$和$A'$,$B$和$B'$建立同构,即$A\to A',a\mapsto(a,1')$,$B\to B',b\mapsto(1,b)$,故$G$是$B$过$A$的平凡扩张.
		
		设$G_1$也是$B$过$A$的平凡扩张.则有$A_1\lhd G_1$,$B_1\lhd G_1$,$G_1=A_1B_1$,$A_1\cap B_1=\{e'\}$,且$A\cong A_1$,$B\cong B_1$.下证$G\cong G_1$.
		
		设$f_1:A\to A_1,a\mapsto a_1$,$f_2:B\to B_1,b\mapsto b_1$是两个同构映射.令$f:G\to G_1,(a,b)\mapsto f_1(a)f_2(b)$.下证$f$是同构.
		
		因为$f_1$和$f_2$都是满射,于是对任意$f_1(a)$和$f_2(b)$都有原像$a$和$b$.于是$f$是满射.
		
		假设$f_1(a')f_2(b')=f_1(a)f_2(b)$,由于$G_1$是平凡扩张,因此分解是唯一的. $f_1(a')=f_1(a)$,$f_2(b')=f_2(b)$.又因为$f_1$和$f_2$是单射,于是$a'=a,b'=b$,$(a,b)=(a',b')$.
		
		而
		\begin{equation*}
			\begin{aligned}
				&f\left((a,b)(a',b')\right)=f\left((aa',bb')\right)=f_1(aa')f_2(bb')=f_1(a)f_1(a')f_2(b)f_2(b')\\
				&=f_1(a)f_2(b)f_1(a')f_2(b')=f\left((a,b)\right)f\left((a',b')\right),
			\end{aligned}
		\end{equation*}
		于是$f$是同构映射.
	\end{proof}
	\begin{remark}
		外直积$G=G_1\times G_2$中,$G_1$和$G_2$一般不是$G$的子群,但是存在某个同构关系,使得$G_1$,$G_2$分别和$G$的两个子群同构. 而在内直积$G=H\otimes N$中,$H$和$N$都是$G$的正规子群.内直积和外直积在本质上是一致的.
	\end{remark}
\end{document}