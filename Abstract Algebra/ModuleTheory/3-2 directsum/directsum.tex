\documentclass[12pt]{ctexart}
\usepackage{amsfonts,amssymb,amsmath,amsthm,geometry,enumerate,graphicx}
\usepackage[colorlinks,linkcolor=blue,anchorcolor=blue,citecolor=green]{hyperref}
\usepackage[all]{xy}
%introduce theorem environment
\theoremstyle{definition}
\newtheorem{definition}{定义}
\newtheorem{theorem}{定理}
\newtheorem{lemma}{引理}
\newtheorem{corollary}{推论}
\newtheorem{property}{性质}
\newtheorem{proposition}{命题}
\newtheorem{example}{例}
\theoremstyle{plain}
\newtheorem*{solution}{解}
\newtheorem*{remark}{注}
\geometry{a4paper,scale=0.8}

\newcommand{\id}{\mathrm{id}}
\newcommand{\Aut}{\mathrm{Aut}}
\newcommand{\Inn}{\mathrm{Inn}}
\newcommand{\Orb}{\mathrm{Orb}}
\newcommand{\Stab}{\mathrm{Stab}}

\everymath{\displaystyle}

%article info
\title{\vspace{-2em}\textbf{模的直和}\vspace{-2em}}
\date{ }
\begin{document}
	\maketitle
	\begin{definition}[外直和]
		设$M_1,M_2,\cdots,M_n$是$R$-模,$M=\left\{(x_1,x_2,\cdots,x_n)\mid x_i\in M_i,1\leqslant i\leqslant n\right\}$.若满足
		\begin{enumerate}
			\item $(x_1,x_2,\cdots,x_n)+(y_1,y_2,\cdots,y_n)=(x_1+y_1,x_2+y_2,\cdots,x_n+y_n)$;
			\item $a(x_1,x_2,\cdots,x_n)=(ax_1,ax_2,\cdots,ax_n)$,
		\end{enumerate}
		则可验证$M$也是$R$-模,称为$M_1,M_2,\cdots,M_n$的\textbf{外直和},记作
		$$M=M_1\oplus M_2\oplus\cdots\oplus M_n=\bigoplus_{i=1}^{n}M_i.$$
	\end{definition}
	\begin{remark}
		$M_i$大多数情况下不是$M$的子模,甚至不是子集,因为它们的元素都不一样.
	\end{remark}
	令$M_i'=\left\{x_i'=(0,0,\cdots,\mathop{x_i}\limits_{\text{第}i\text{个}},\cdots,0)\mid x_i\in M_i,1\leqslant i\leqslant n\right\}$,这样$M_i'$是$M$的子模,且在$M_i$与$M_i'$中存在同构映射.
	\begin{theorem}
		设$M_1,M_2,\cdots,M_n$与$N$都是$R$-模,$M=\bigoplus_{i=1}^{n}M_i$,$\varphi_i$是$M_i$到$N$的模同态,则存在唯一的模同态$\varphi:M\to N$使得
		$$\varphi(x_i')=\varphi_i(x_i).$$
	\end{theorem}
	\begin{proof}
		存在性:定义
		$$\varphi(x_1,x_2,\cdots,x_n)=\sum_{i=1}^{n}\varphi_i(x_i),$$
		根据模同态的定义验证即可.
		
		唯一性:设$\psi:M\to N$也是模同态满足$\psi(x_i')=\varphi_i(x_i)$,则
		$$\psi(x_1,x_2,\cdots,x_n)=\psi\left(\sum_{i=1}^{n}x_i'\right)=\sum_{i=1}^{n}\varphi_i(x_i)=\varphi(x_1,x_2,\cdots,x_n),$$
		于是$\psi=\varphi$.
	\end{proof}
	\begin{definition}[内直和]
		若$R$-模$N$的子模$M_1,M_2,\cdots,M_n$满足
		\begin{enumerate}
			\item $N=M_1+M_2+\cdots+M_n$;
			\item $M_i\cap\left(\sum_{j\neq i}M_j\right)=\{0\},\quad 1\leqslant i\leqslant n$,
		\end{enumerate}
		则称$N$是$M_1,M_2,\cdots,M_n$的\textbf{内直和},也记作
		$$N=M_1\oplus M_2\oplus\cdots\oplus M_n=\bigoplus_{i=1}^{n}M_i.$$
	\end{definition}
	\begin{remark}
		条件2比“$M_i\cap M_j=\{0\},\forall i\neq j$”的条件更强,也等价于$N$中元素表示为$M_1,M_2,\cdots,M_n$中元素之和的表法唯一.
	\end{remark}
	\begin{theorem}
		设$M$是$R$-模$M_1,M_2,\cdots,M_n$的外直和,$N$是$M_1,M_2,\cdots,M_n$的内直和,则
		$$\varphi(x_1,x_2,\cdots,x_n)=\sum_{i=1}^{n}x_i$$
		是$M$到$N$的模同构映射.
	\end{theorem}
	\begin{proof}
		根据定义可以验证$\varphi$是模同态.下证$\varphi$是双射.
		
		由于$N$是$M_1,M_2,\cdots,M_n$的内直和,于是$N=M_1+M_2+\cdots+M_n$,对任意$y\in N$,都存在$x_i\in M_i$使得$y=\sum_{i=1}^{n}x_i$,于是$y$有原像$(x_1,x_2,\cdots,x_n)$,故$\varphi$是满射.
		
		若$\sum_{i=1}^{n}x_i=0$,则
		$$x_i=-\sum_{j\neq i}x_j\in M_i\cap\left(\sum_{j\neq i}M_j\right)=\{0\},$$
		于是$x_i=0\ (1\leqslant i\leqslant n)$,于是$\ker\varphi =(0,0,\cdots,0)$,故$\varphi$是单射.
	\end{proof}
	\begin{property}
		直和的表法是唯一的.
	\end{property}
	\begin{proof}
		对任意$a\in N$,假设有两种表法
		$$a=x_1+x_2+\cdots+x_n=y_1+y_2+\cdots+y_n,$$
		则
		$$a-a=(x_1-y_1)+(x_2-y_2)+\cdots+(x_n-y_n)=0,$$
		于是
		$$(x_i-y_i)=-\sum_{j\neq i}(a_j-b_j)\in M_i\cap\left(\sum_{j\neq i}M_j\right)=\{0\},$$
		于是$x_i-y_i=0\ (1\leqslant i\leqslant n)$,故$x_i=y_i$.
	\end{proof}
	\begin{property}
		直和的直和仍是直和.
	\end{property}
	\begin{property}
		直和可以任意结合(任意加括号).
	\end{property}
	\begin{definition}[无关]
		若$R$-模$N$的子模$M_1,M_2,\cdots,M_n$满足
		$$M_i\cap\left(\sum_{j\neq i}M_j\right)=\{0\},\quad 1\leqslant i\leqslant n,$$
		则称$M_1.M_2,\cdots,M_n$是\textbf{无关的}.
	\end{definition}
\end{document}