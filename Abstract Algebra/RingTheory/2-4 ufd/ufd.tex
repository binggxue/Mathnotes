\documentclass[12pt]{ctexart}
\usepackage{amsfonts,amssymb,amsmath,amsthm,geometry,enumerate,graphicx}
\usepackage[colorlinks,linkcolor=blue,anchorcolor=blue,citecolor=green]{hyperref}
\usepackage[all]{xy}
%introduce theorem environment
\theoremstyle{definition}
\newtheorem{definition}{定义}
\newtheorem{theorem}{定理}
\newtheorem{lemma}{引理}
\newtheorem{corollary}{推论}
\newtheorem{property}{性质}
\newtheorem{proposition}{命题}
\newtheorem{example}{例}
\theoremstyle{plain}
\newtheorem*{solution}{解}
\newtheorem*{remark}{注}
\geometry{a4paper,scale=0.8}

\newcommand{\id}{\mathrm{id}}
\newcommand{\Aut}{\mathrm{Aut}}
\newcommand{\Inn}{\mathrm{Inn}}
\newcommand{\Orb}{\mathrm{Orb}}
\newcommand{\Stab}{\mathrm{Stab}}

\everymath{\displaystyle}

%article info
\title{\vspace{-2em}\textbf{唯一析因环}\vspace{-2em}}
\date{ }
\begin{document}
	\maketitle
	\begin{definition}[有限析因条件]
		在整环$R$中,对任意$a\in R^\ast\backslash U$,都可以分解为有限个不可约元素的乘积$p_1p_2\cdots p_n$,则称整环$R$满足\textbf{有限析因条件}.
	\end{definition}
	\begin{definition}[唯一析因环]
		若整环$R$满足有限析因条件且分解在相伴意义下唯一,则称$R$为\textbf{唯一析因环}或\textbf{唯一分解整环},记为UFD.
	\end{definition}
	\begin{remark}
		UFD是使因式分解唯一性成立的整环,类比于数论中的算术基本定理.
	\end{remark}
	\begin{definition}[因子链]
		若$R^\ast$中的一个序列$a_1,a_2,\cdots,a_n,a_{n+1},\cdots$满足$a_{n+1}\mid a_{n}$,则称这个序列是$R$的一个\textbf{因子链}.
	\end{definition}
	\begin{definition}[因子链条件]
		若$R$中不存在无限真因子链,则称$R$满足\textbf{因子链条件}.
	\end{definition}
	\begin{remark}
		也就是说,存在$m$,对任意$n\geqslant m$,有$a_m\sim a_{n}$.
	\end{remark}
	\begin{lemma}\label{factofinite}
		若整环$R$满足因子链条件,则一定满足有限析因条件.
	\end{lemma}
	\begin{proof}
		对任意$a\in R^\ast\backslash U$,先证$a$有不可约因子.若$a$是不可约元素,则自然成立.下设$a$是可约的.设$a=a_1b_1$,这里$a_1,b_1$都是非平凡的真因子.若$a_1$是不可约元素,则自然成立.下设$a_1$是不可约的.以此类推,得到因子链
		$$a,a_1,a_2,\cdots,$$
		由因子链条件,存在$m$使得$a_m\sim a_{m+1}$,于是$a_m$是不可约的,即$a_m$是$a$的不可约因子.
		
		再证$a$可以分解成有限个不可约因子的乘积.设$p_1$是$a$的一个不可约因子,则有$a=p_1a'$.若$a'\in U$,则$a=(p_1a')$,其中$(p_1a')$是不可约元素.若$a'\in R^\ast\backslash U$,则$a'$有不可约因子,设$a'=p_2a''$,若$a''\in U$,则$a=p_1(p_2a'')$.否则,$a''$有不可约因子.以此类推,得到因子链
		$$a,a',a'',\cdots,a^{(n)},a^{(n+1)},\cdots,$$
		由因子链条件,存在$s$使得$a^{(s)}\sim a^{(s+1)}$,于是$a^{(s)}$是不可约的,得到
		$$a=p_1p_2\cdots (p_{s}a^{(s)}).$$
	\end{proof}
	\begin{theorem}
		若$R$是整环,则以下命题等价.
		\begin{enumerate}
			\item $R$是唯一析因环.
			\item $R$满足因子链条件和最大公因式条件.
			\item $R$满足因子链条件和素性条件.
		\end{enumerate}
	\end{theorem}
	\begin{proof}
		1\ $\to$\ 2:
		设$a\in R^\ast\backslash U$可以分解为
		$$a=p_1p_2\cdots p_r,$$
		其中$p_i$为不可约元素.称$r$为$a$的长度,记作$|a|=r$.
		
		设$a_1,a_2,\cdots,a_n,\cdots$是$R$中的一条因子链,于是
		$$|a_1|\geqslant |a_2|\geqslant \cdots\geqslant |a_n|\geqslant\cdots,$$
		由于$|a_i|$是非负整数,于是存在$m$使得$|a_m|=|a_{m+1}|$,于是$a_m\sim a_{m+1}$,$R$满足因子链条件.
		
		设$a,b\in R$,若其中之一为单位,则$(a,b)=1$.现设$a,b\in R^\ast\backslash U$,在相伴意义下对分解归类,有
		$$a=u_1p_1^{n_1}p_2^{n_2}\cdots p_r^{n_r},$$
		$$b=u_2p_1^{m_1}p_2^{m_2}\cdots p_r^{r_s},$$
		其中$u_1,u_2\in U$,$p_i$是互不相同的不可约元素.令$k_i=\min\left\{n_i,m_i\right\}$,则
		$$d=p_1^{k_1}p_2^{k_2}\cdots p_r^{k_r}$$
		是$a$和$b$的最大公因子.
		
		2\ $\to$\ 3:
		由于$R$满足最大公因式条件,于是$(a,b)\sim 1$,$(a,c)\sim 1$可以推出$(a,bc)\sim 1$.对任意不可约元素$p\in R$,若$p\nmid a$,$p\nmid b$,则$(p,a)\sim 1$,$(p,b)\sim 1$,于是$(p,ab)\sim 1$,即$p\nmid ab$,于是$p$是素元素,$R$满足素性条件.
		
		3\ $\to$\ 1:
		由引理\ref{factofinite},$R$满足有限析因条件,下证相伴意义下的唯一性.设$a\in R^\ast\backslash U$有两种分解
		$$a=p_1p_2\cdots p_s=q_1q_2\cdots q_t,$$
		
		当$s=1$时,$a=p_1$,由于$p_1$是不可约元素,于是不妨设$a=p_1\mid q_1$,于是$a\sim q_1$,$t=1$.
		
		假设当$s-1$时成立.因为$p_1\mid a$,于是$p_1\mid q_1q_2\cdots q_t$,存在$q_i$使得$p_1\mid q_i$.不妨设$p_1\mid q_1$.因为$q_1$是不可约元素,只有平凡因子,于是存在$u\in U$使得$q_1=up_1$,有
		$$p_1p_2\cdots p_s=up_1q_2\cdots q_t,$$
		$$p_2\cdots p_s=(uq_2)\cdots q_t,$$
		由归纳假设,$s-1=t-1$,于是$s=t$且$p_i\sim q_{\pi(i)}$,这里$\pi\in S_t$.
	\end{proof}
\end{document}