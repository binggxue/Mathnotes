\documentclass[12pt]{ctexart}
\usepackage{amsfonts,amssymb,amsmath,amsthm,geometry,enumerate,graphicx}
\usepackage[colorlinks,linkcolor=blue,anchorcolor=blue,citecolor=green]{hyperref}
\usepackage[all]{xy}
%introduce theorem environment
\theoremstyle{definition}
\newtheorem{definition}{定义}
\newtheorem{theorem}{定理}
\newtheorem{lemma}{引理}
\newtheorem{corollary}{推论}
\newtheorem{property}{性质}
\newtheorem{proposition}{命题}
\newtheorem{example}{例}
\theoremstyle{plain}
\newtheorem*{solution}{解}
\newtheorem*{remark}{注}
\geometry{a4paper,scale=0.8}

\newcommand{\id}{\mathrm{id}}
\newcommand{\Aut}{\mathrm{Aut}}
\newcommand{\Inn}{\mathrm{Inn}}
\newcommand{\Orb}{\mathrm{Orb}}
\newcommand{\Stab}{\mathrm{Stab}}

\everymath{\displaystyle}

%article info
\title{\vspace{-2em}\textbf{主理想整环与Euclid环}\vspace{-2em}}
\date{ }
\begin{document}
	\maketitle
	\begin{definition}[生成的理想]
		设$R$是环,集合$A\subset R$,称包含$A$的$R$的最小理想为集合$A$\textbf{生成的理想},记作$\langle A\rangle$.
	\end{definition}
	由于理想的交仍为理想,所以$A$生成的理想也可定义为$R$中所有包含$A$的理想的交.且$\langle A\rangle$总是存在.
	\begin{definition}[主理想]
		若集合$A$只包含一个元素$a$,则由$A$生成的理想称为\textbf{主理想},记作$\langle a\rangle$.
	\end{definition}
	\begin{definition}[主理想环]
		若交换幺环$R$的理想都是主理想,则称$R$为\textbf{主理想环}.
	\end{definition}
	\begin{remark}
		主理想环定义在交换幺环上,避免左理想、右理想以及双边理想的讨论.
	\end{remark}
	\begin{definition}[主理想整环]
		定义在整环上的主理想环,即无零因子的主理想环称为\textbf{主理想整环},记为PID.
	\end{definition}
	交换幺环$R$中,由$a$生成的主理想可以写成
	$$\langle a\rangle=\left\{ra\mid r\in R\right\}.$$
	下面是主理想整环中的一些性质.
	\begin{property}
		设$R$是PID,$a,b\in R$,则$a\mid b\iff\langle a\rangle\supset\langle b\rangle$.
	\end{property}
	\begin{proof}
		若$a\mid b$,则存在$r\in R$使得$b=ra$,于是
		$$\langle b\rangle=\{bx\mid x\in R\}=\{rax\mid x\in R\}=r\cdot\langle a\rangle\subset\langle a\rangle.$$
		
		反之,若$\langle a\rangle\supset\langle b\rangle$,则$b\in\langle b\rangle\subset\langle a\rangle$,存在$c\in R$使得$b=ca$,于是$a\mid b$.
	\end{proof}
	\begin{corollary}
		$a\sim b\iff\langle a\rangle=\langle b\rangle$.
	\end{corollary}
	\begin{corollary}
		$a\sim 1\iff\langle a\rangle=\langle 1\rangle=R$.
	\end{corollary}
	\begin{definition}[主理想升链]
		对整环$R$中的一个主理想序列,若有
		$$\langle a_1\rangle\subset\langle a_2\rangle\subset\cdots\subset\langle a_n\rangle\subset\langle a_{n+1}\rangle\subset\cdots,$$
		则称这个序列为\textbf{主理想升链}.
	\end{definition}
	\begin{definition}[主理想升链条件]
		若在$R$的任一主理想升链中,存在$m$,当$n>m$时,有$\langle a_m\rangle=\langle a_n\rangle$,则称$R$满足\textbf{主理想升链条件}.
	\end{definition}
	由主理想整环中的性质立即可得,主理想升链条件和因子链条件是互相等价的.
	\begin{theorem}
		主理想整环是唯一析因环.
	\end{theorem}
	\begin{proof}
		只需证明主理想整环$R$满足主理想升链条件和最大公因子条件即可.
		
		设主理想整环$R$中有一主理想升链
		$$\langle a_1\rangle\subset\langle a_2\rangle\subset\cdots\subset\langle a_n\rangle\subset\langle a_{n+1}\rangle\subset\cdots,$$
		并设$I=\bigcup\langle a_k\rangle$,对任意$a\in\langle a_i\rangle,b\in\langle a_j\rangle$,不妨设$i\leqslant j$,则$a\in\langle a_i\rangle\subset\langle a_j\rangle$,于是$a-b\in\langle a_j\rangle\subset I$.且对任意$r\in R$,有$ra\in\langle a_j\rangle\subset I$,于是$I$是$R$的理想.又因为$R$是主理想整环,于是存在$d\in R$使得$I=\langle d\rangle$.于是$d\in I$,存在$m\in\mathbb{N}^+$使得$d\in\langle a_m\rangle=\{ra_m\mid r\in R\}$,存在$r_0\in R$使得$d=r_0a_m$,于是$a_m\mid d$,有$\langle d\rangle\subset\langle a_m\rangle$,于是有
		$$I=\langle d\rangle\subset\langle a_m\rangle\subset\cdots\subset\langle a_n\rangle\subset I,$$
		对任意$n>m$,有$\langle a_m\rangle=\langle a_n\rangle$,于是$R$满足主理想升链条件.
		
		对任意$\langle a_i\rangle,\langle a_j\rangle$,按理想的等价条件可以证明$\langle a_i\rangle+\langle a_j\rangle$仍是理想.于是存在$d\in R$使得$\langle d\rangle=\langle a_i\rangle+\langle a_j\rangle$,于是$\langle a_i\rangle\subset\langle d\rangle$,$\langle a_j\rangle\subset\langle d\rangle$,则$d\mid a$,$d\mid b$.若另有$c\mid a$,$c\mid b$,则$\langle a_i\rangle\subset\langle c\rangle$,$\langle a_j\rangle\subset\langle c\rangle$,则$\langle a_i\rangle+\langle a_j\rangle\subset\langle c\rangle$,即$\langle d\rangle\subset\langle c\rangle$,于是$c\mid d$.
	\end{proof}
	\begin{corollary}
		设$R$是PID,$a,b,d\in R$,若$d$是$a,b$的最大公因子,则存在$u,v\in R$使得
		$$d=ua+vb.$$
	\end{corollary}
	\begin{corollary}
		$a,b$互素的充要条件是存在$u,v\in R$使得
		$$ua+vb=1.$$
	\end{corollary}
	由于主理想整环是唯一析因环,于是满足最大公因子条件,即任意两个非零元素的最大公因子都存在.在整数环中,求任意两个整数的最大公因子,用到了\textbf{辗转相除法},但并不是所有环都可以用这个方法.把那些可以进行辗转相除法的整环称为Euclid环.具体地,有
	\begin{definition}[Euclid环]
		设$R$是整环,若存在映射$\delta:R\to\mathbb{N}$使得任意$a,b\in R$,且$b\neq 0$,都存在$q,r\in R$使得
		$$a=qb+r,\qquad \delta(b)>\delta(r),$$
		则称$R$为\textbf{Euclid环},记为ED.
	\end{definition}
	\begin{remark}
		可以进行辗转相除等价于满足带余除法条件.定义中的映射$\delta$未必唯一.
	\end{remark}
	\begin{theorem}
		Euclid环是主理想整环.
	\end{theorem}
	\begin{proof}
		证明Euclid环$R$中的理想都是主理想即可.设$I\lhd R$,若$I=\{0\}$,则$I=\langle 0\rangle$是主理想.下设$I\neq 0$,则$I$中存在$b\neq 0$,取为
		$$\delta(b)=\min\left\{\delta(c)\mid c\in I,c\neq 0\right\},$$
		对任意$a\in I$,由Euclid环定义,存在$q,r\in R$使得$a=qb+r$且$\delta(b)>\delta(r)$,于是$r=0$,则$a=qb$,于是$I=\langle b\rangle$.
	\end{proof}
	\begin{corollary}
		Euclid环是主理想整环,因而也是唯一析因环.
	\end{corollary}
\end{document}