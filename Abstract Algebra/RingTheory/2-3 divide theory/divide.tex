\documentclass[12pt]{ctexart}
\usepackage{amsfonts,amssymb,amsmath,amsthm,geometry,enumerate,graphicx}
\usepackage[colorlinks,linkcolor=blue,anchorcolor=blue,citecolor=green]{hyperref}
\usepackage[all]{xy}
%introduce theorem environment
\theoremstyle{definition}
\newtheorem{definition}{定义}
\newtheorem{theorem}{定理}
\newtheorem{lemma}{引理}
\newtheorem{corollary}{推论}
\newtheorem{property}{性质}
\newtheorem{proposition}{命题}
\newtheorem{example}{例}
\theoremstyle{plain}
\newtheorem*{solution}{解}
\newtheorem*{remark}{注}
\geometry{a4paper,scale=0.8}

\newcommand{\id}{\mathrm{id}}
\newcommand{\Aut}{\mathrm{Aut}}
\newcommand{\Inn}{\mathrm{Inn}}
\newcommand{\Orb}{\mathrm{Orb}}
\newcommand{\Stab}{\mathrm{Stab}}

\everymath{\displaystyle}

%article info
\title{\vspace{-2em}\textbf{整环中的整除理论}\vspace{-2em}}
\date{ }
\begin{document}
	\maketitle
	本节讨论的是整环$R$去掉$\{0\}$后的关于乘法作成的交换幺半群$R^\ast$.
	\begin{definition}[单位]
		$R^\ast$中的可逆元全体$U$关于乘法作成Abel群,称为$R$的\textbf{单位群},$U$中的元素称为\textbf{单位}.
	\end{definition}
	\begin{definition}[整除]
		对$a,b\in R^\ast$,若存在$p\in R^\ast$使得$b=pa$,则称$a$整除$b$,记作$a\mid b$.反之,称$a$不整除$b$,记作$a\nmid b$.称$a$是$b$的\textbf{因子},$b$是$a$的\textbf{倍式}.
	\end{definition}
	\begin{definition}[平凡因子]
		任意单位$u\in U\subset R^\ast$,对任意$a\in R^\ast$,都有$u\mid a$,称为$a$的\textbf{平凡因子}.
	\end{definition}
	\begin{remark}
		因为$u(u^{-1}a)=a$,而$u^{-1}a\in R^\ast$,于是$u\mid a$.
	\end{remark}
	\begin{definition}[相伴]
		对$a,b\in R^\ast$,若$a\mid b$,$b\mid a$,则称$a$与$b$\textbf{相伴},记作$a\sim b$.
	\end{definition}
	\begin{property}
		$a\sim b$当且仅当存在$u\in U$使得$b=ua$.
	\end{property}
	\begin{property}
		相伴关系是同余关系.
	\end{property}
	\begin{proof}
		易证相伴关系是等价关系.对任意$a,b,c,d\in R^\ast$,若$a\sim b$,$c\sim d$,则存在$u_1,u_2\in U$使得$b=u_1a$,$d=u_2c$,于是
		$$bd=u_1au_2c=u_1u_2ac,$$
		而$u_1u_2\in U$,于是$bd\sim ac$.
	\end{proof}
	\begin{property}
		$u\in U\iff u\sim 1$.
	\end{property}
	\begin{definition}[真因子]
		对$a,b\in R^\ast$,若$a\mid b$,$b\nmid a$,则称$a$是$b$的\textbf{真因子}.
	\end{definition}
	\begin{definition}
		真因子即不与之相伴的因子.
	\end{definition}
	\begin{definition}[不可约元素,可约元素]
		若$a\in R^\ast\backslash U$没有非平凡的真因子,即只有平凡的真因子,则称$a$为\textbf{不可约元素}.反之,则称$a$为\textbf{可约元素}.
	\end{definition}
	\begin{remark}
		不可约和可约的概念只对$R^\ast$中非单位的元素有定义,单位不存在这个概念.
	\end{remark}
	\begin{definition}[素元素]
		设$p\in R^\ast\backslash U$,$a,b\in R^\ast$,若$p\mid ab$能推出$p\mid a$或$p\mid b$,则称$p$为\textbf{素元素}.
	\end{definition}
	不可约元素和素元素存在密切的关系.
	\begin{lemma}
		素元素是不可约元素.
	\end{lemma}
	\begin{proof}
		设$a\mid p$,则存在$b\in R^\ast$使得$p=ab$,于是$p\mid a$或$p\mid b$.若$p\mid a$,则$a\sim p$,$a$不是$p$的真因子.若$p\mid b$,则存在$c\in R^\ast$使得$b=pc$.于是$p=ab=apc=pac$,$ac=1$,于是$a\in U$是$p$的平凡因子.
	\end{proof}
	不可约元素不一定是素元素,下面是一个反例.
	\begin{example}\label{opeg}
		设$R=\mathbb{Z}\left[\sqrt{-5}\right]=\left\{a+b\sqrt{-5}\mid a,b\in\mathbb{Z}\right\}$,则$3$是$R$的不可约元素,但不是素元素.
	\end{example}
	为找出单位群,先定义$Z\left[\sqrt{-5}\right]$中范数的概念.
	\begin{definition}
		设$\alpha=a+b\sqrt{-5}\in\mathbb{Z}\left[\sqrt{-5}\right]$,$\overline{\alpha}=a-b\sqrt{-5}$,$N(\alpha)=\alpha\overline{\alpha}=a^2+5b^2$为$\alpha$的\textbf{范数}.
	\end{definition}
	显然$N(\alpha)$是非负整数,$N(\alpha)=0$当且仅当$\alpha=0$.且对任意$\alpha,\beta\in R$,有
	$$N(\alpha\beta)=\alpha\beta\overline{\alpha\beta}=\alpha\beta\overline{\alpha}\overline{\beta}=\alpha\overline{\alpha}\beta\overline{\beta}=N(\alpha)N(\beta).$$
	
	下面是例\ref{opeg}的证明.
	\begin{proof}
		对任意$\alpha\in U$,有$\alpha\alpha^{-1}=1$.于是$N(\alpha)N(\alpha^{-1})=N(\alpha\alpha^{-1})=N(1)=1$.由于$N(\alpha)$是非负整数,于是$N(\alpha)=N(\alpha^{-1})=1$.于是$\alpha=\pm 1$.当$\alpha=\pm 1$时,显然$\alpha\in U$,于是$U=\left\{1,-1\right\}$.
		
		设$\alpha=a+b\sqrt{-5}$是$3$的一个因子,则存在$\beta\in R^\ast$使得$3=\alpha\beta$.于是$N(\alpha)N(\beta)=N(\alpha\beta)=N(3)=9$.$N(\alpha)$的取值有$1,3,9$.
		
		当$N(\alpha)=1$时,$\alpha=1$是$3$的平凡真因子.
		
		当$N(\alpha)=3$时,$a^2+5b^2=3$无整数解,于是此情况不存在.
		
		当$N(\alpha)=9$时,$N(\beta)=1$,$\beta=\pm 1$,于是$\alpha\sim 3$.则$\alpha$不是$3$的真因子.
		
		上述说明了$3$是不可约元素.另一方面,由于$3\mid 9=(2+\sqrt{-5})(2-\sqrt{-5})$,而$3\nmid 2\pm\sqrt{-5}$,于是$3$不是素元素.
	\end{proof}
	\begin{definition}[素性条件]
		若整环$R$中的不可约元素都是素元素,则称$R$满足\textbf{素性条件}.
	\end{definition}
	\begin{definition}[公因子]
		设$a,b\in R^\ast$,若有$d\in R^\ast$满足$d\mid a$且$d\mid b$,则称$d$为$a$和$b$的\textbf{公因子}.若有公因子$d$,对任意公因子$d_1$都有$d_1\mid d$,则称$d$是$a$和$b$的\textbf{最大公因子}.类似地可以定义有限多个元素的最大公因子.
	\end{definition}
	最大公因子不一定存在.若$R^\ast$中的任意两个元素的最大公因子都存在,则称$R$满足\textbf{最大公因子条件}.
	\begin{lemma}
		设整环$R$满足最大公因子条件,则有
		\begin{enumerate}
			\item $R$中任意两个元素$a,b$的最大公因子在相伴意义下唯一,记为$(a,b)$.
			\item $R$中任意$r$个元素$a_1,a_2,\cdots,a_r$的最大公因子存在.
			\item $((a,b),c)\sim(a,(b,c))$.
			\item $c(a,b)=(ca,cb)$.
			\item 若$(a,b)\sim 1$,$(a,c)\sim 1$,则$(a,bc)\sim 1$.(若$(a,b)\sim 1$,则称$a$和$b$\textbf{互素})
		\end{enumerate}
	\end{lemma}
	\begin{proof}
		1.设$d$,$d_1$是$a,b$的两个最大公因子,则$d\mid d_1$,$d_1\mid d$,于是$d\sim d_1$.
		
		2.设$d_1=(a_1,a_2),d_2=(d_1,a_3),\cdots,d=d_{r-1}=(d_{r-2},a_r)$,则$d\mid d_{r-2}$,$d_{r-2}\mid d_{r-3}$,以此类推,有$d\mid d_1=(a_1,a_2)$,又$d\mid a_r$,于是$d$是$a_1,a_2,\cdots,a_r$的公因子.对任意公因子$a$,$a\mid a_1$,$a\mid a_2$,于是$a\mid d_1$,又$a\mid a_3$,于是$a\mid d_2$,以此类推,$a\mid d$.于是$d$是$a_1,a_2,\cdots,a_r$的最大公因子.
		
		3.由结论2,$((a,b),c)$和$(a,(b,c))$都是$a,b,c$的最大公因子,由结论1,它们相伴.
		
		4.设$d=(a,b)$,$e=(ca,cb)$,则$d\mid a$,于是$cd\mid ca$,同理$cd\mid cb$,于是$cd\mid e$,存在$u\in R$使得$e=cdu$.而$e\mid ca$,存在$x\in R$使得$ca=ex=cdux$,于是$a=dux$,$a\mid du$,同理$b\mid du$,于是$du\mid d$,即$u\in U$,于是$e\sim cd$.
		
		5.由$(a,b)\sim 1$,有$(ac,bc)\sim c$,于是
		$$1\sim (a,c)\sim(a,(ac,bc))\sim((a,ac),bc)\sim (a,bc).$$
	\end{proof}
\end{document}