\documentclass[12pt]{ctexart}
\usepackage{amsfonts,amssymb,amsmath,amsthm,geometry,enumerate,graphicx}
\usepackage[colorlinks,linkcolor=blue,anchorcolor=blue,citecolor=green]{hyperref}
\usepackage[all]{xy}
%introduce theorem environment
\theoremstyle{definition}
\newtheorem{definition}{定义}
\newtheorem{theorem}{定理}
\newtheorem{lemma}{引理}
\newtheorem{corollary}{推论}
\newtheorem{property}{性质}
\newtheorem{proposition}{命题}
\newtheorem{example}{例}
\theoremstyle{plain}
\newtheorem*{solution}{解}
\newtheorem*{remark}{注}
\geometry{a4paper,scale=0.8}

\newcommand{\id}{\mathrm{id}}
\newcommand{\Aut}{\mathrm{Aut}}
\newcommand{\Inn}{\mathrm{Inn}}
\newcommand{\Orb}{\mathrm{Orb}}
\newcommand{\Stab}{\mathrm{Stab}}
\newcommand{\End}{\mathrm{End}}

\everymath{\displaystyle}

%article info
\title{\vspace{-2em}\textbf{环的同态与同构}\vspace{-2em}}
\date{ }
\begin{document}
	\maketitle
	\begin{definition}[同态与同构]
		设$R_1$和$R_2$是两个环,映射$f:R_1\to R_2$满足对任意$a,b\in R_1$,
		$$f(a+b)=f(a)+f(b),\qquad f(ab)=f(a)f(b),$$
		则称$f$是环$R_1$到$R_2$上的\textbf{同态映射}.若$f$是单射,则称\textbf{单同态},若$f$是满射,则称\textbf{满同态},这时称$R_1$和$R_2$是\textbf{同态的},若$f$是双射,则称\textbf{同构映射},$R_1$和$R_2$是\textbf{同构的}.
	\end{definition}
	\begin{definition}[零同态]
		设$f$是环$R_1$到$R_2$上的同态,若对任意$a\in R_1$,$f(a)=0$,则称$f$是$R_1$到$R_2$的\textbf{零同态}.
	\end{definition}
	\begin{definition}[自然同态]
		设$R$是环,$I\lhd R$,映射$\pi:R\to R/I$是同态,称为$R$关于$I$的\textbf{自然同态}.
	\end{definition}
	环的同态与同构与群的类似,很多性质可以类推.
	\begin{definition}[核]
		设$f$是环$R_1$到$R_2$的同态,称$\ker f=\left\{a\in R_1\mid f(a)=0\right\}$为$f$的\textbf{核}.
	\end{definition}
	立即可得,零同态$f$的核$\ker f=R_1$.
	\begin{theorem}[环同态基本定理]
		设$f$是环$R_1$到$R_2$的满同态,则$R_1/\ker f\cong R_2$.
	\end{theorem}
	\begin{proof}
		由环的定义,$R_1$和$R_2$对加法构成Abel群,于是由群同态基本定理,存在同构$\varphi:R_1/\ker f\to R_2$,对任意$a+I,b+I\in R_1/\ker f$,有
		$$\varphi((a+I)(b+I))=\varphi(ab+I)=f(ab)=f(a)f(b)=\varphi(a+I)\varphi(b+I),$$
		于是$\varphi$对乘法也是同构.
	\end{proof}
	\begin{theorem}[挖补定理]
		设环$S,R'$,$R'\cap S=\varnothing$,设$R<S$使得$R\cong R'$,则存在$S'\cong S$,且$R'<S'$.
	\end{theorem}
	\begin{proof}
		设$\varphi:R\to R'$是同构映射.令$S'=R'\cup(S\backslash R)$,设映射$\phi:S\to S'$,满足
		$$\phi(x)=\left\{
			\begin{aligned}
				& x, &x\in S\backslash R,\\
				& \varphi(x), & x\in R,
			\end{aligned}
		\right.$$
		容易验证$\phi$是双射.定义$S'$中的加法与乘法,对任意$x',y'\in S'$,
		$$x'+y'=\phi(x+y),$$
		$$x'y'=\phi(xy),$$
		其中
		$$x=\left\{
			\begin{aligned}
				& x', & x'\in S\backslash R,\\
				& \varphi^{-1}(x'), & x'\in R',
			\end{aligned}
		\right.$$
		$$y=\left\{
		\begin{aligned}
			& y', & y'\in S\backslash R,\\
			& \varphi^{-1}(y'), & y'\in R',
		\end{aligned}
		\right.$$
		如此,对任意$x,y\in S$,设$x'=\phi(x)$,$y'=\phi(y)$,则
		$$x'+y'=\phi(a+b),\qquad x'y'=\phi(ab)$$
		若$x \in R$,则 $x' = \varphi(x) \in R'$,故 $a = \varphi^{-1}(x') = x$;若$x \in S \setminus R$,则$x' = x$,故$a = x$.同理$b=y$,于是
		$$\phi(x)+\phi(y)=x'+y'=\phi(x+y),$$
		$$\phi(x)\phi(y)=x'y'=\phi(xy),$$
		于是$\phi$是同构.
		
		$S'$的加法和乘法在$R'$上的限制就是$R'$的加法和乘法,于是$R'<S'$.
	\end{proof}
	\begin{definition}[群的自同态环]
		设$A$是Abel群,$A$到自身的同态映射称为\textbf{自同态},记$A$中自同态组成的集合为$\End M$,定义加法
		$$(f+g)(a)=f(a)+g(a),\ \forall f,g\in\End M,$$
		则$\End M$关于加法和映射的复合作成幺环,称为群$A$的\textbf{自同态环}.
	\end{definition}
	\begin{proof}
		显然$\id\in\End A$,于是$\End A$非空.先证$\End A$关于加法作成Abel群.
		\begin{enumerate}
			\item 封闭律:对任意$f,g\in\End A$,$a\in A$,$(f+g)(a)=f(a)+g(a)\in A$;
			\item 结合律:对任意$f,g,h\in\End A$,$a\in A$,$((f+g)+h)(a)=(f+g)(a)+h(a)=f(a)+g(a)+h(a)=f(a)+(g+h)(a)=(f+(g+h))(a)$;
			\item 零元律:对任意$a\in A$,设$\varphi(a)=0$,则对任意$f\in\End A$,有$(f+\varphi)(a)=f(a)+\varphi(a)=f(a)$,于是$\varphi$是零元.
			\item 逆元律:对任意$f\in\End A$,定义$f^{-1}=-f$,则$(f+f^{-1})(a)=f(a)+f^{-1}(a)=f(a)+(-f(a))=0$.
		\end{enumerate}
		再证$\End A$关于乘法(映射复合)作成半群.
		\begin{enumerate}
			\item 封闭律:对任意$f,g\in\End A$,$a\in A$,$(fg)(a)=f(g(a))\in A$;
			\item 结合律:对任意$f,g,h\in\End A$,$a\in A$,$((fg)h)(a)=fg(h(a))=f(g(h(a)))=f(gh(a))=(f(gh))(a)$.
		\end{enumerate}
		最后证$\End A$满足两条分配律.
		\begin{enumerate}
			\item $f(g+h)(a)=f((g+h)(a))=f(g(a)+h(a))=f(g(a))+f(h(a))=fg(a)+fh(a)$;
			\item $(f+g)h(a)=f(h(a))+g(h(a))=fh(a)+gh(a)$.
		\end{enumerate}
	\end{proof}
\end{document}