\chapter{反常积分}
在讨论Riemann积分时有两个最基本的限制:积分区间的有穷性和被积函数的有界性.但在很多实际问题中要突破这些限制,考虑无穷区间上的积分或无界函数的积分,这便是本章的主题.
\section{反常积分概念}
第一类,我们考虑无穷区间上的积分.
\begin{definition}[无穷积分]
	设函数$f$定义在无穷区间$\left[a,+\infty\right)$上,且在任何有限区间$\left[a,u\right]$上可积.如果存在极限
	$$\lim\limits_{u\to +\infty}\int_{a}^{u}f(x)\d x=J,$$
	则称此极限$J$为函数$f$在$\left[a,+\infty\right)$上的{\heiti 无穷限反常积分}(简称{\heiti 无穷积分}),记作
	$$J=\int_{a}^{+\infty}f(x)\d x,$$
	并称$\displaystyle\int_{a}^{+\infty}f(x)\d x${\heiti 收敛}.如果极限不存在,则称$\displaystyle\int_{a}^{+\infty}f(x)\d x${\heiti 发散}.
\end{definition}
类似地,可定义$f$在$\left(-\infty,b\right]$上的无穷积分:
$$\int_{-\infty}^{b}f(x)\d x=\lim\limits_{u\to -\infty}\int_{u}^{b}f(x)\d x.$$

对于$f$在$(-\infty,+\infty)$上的无穷积分,用前面两种积分来定义:
$$\int_{-\infty}^{+\infty}f(x)\d x=\int_{-\infty}^{a}f(x)\d x+\int_{a}^{+\infty}f(x)\d x,\quad a\in\mathbb{R}.$$
当且仅当等号右边两个无穷积分都收敛时它才是收敛的.

可以根据函数极限的性质与定积分的性质,导出无穷积分的一些相应性质.
\begin{proposition}
	对于任意$i=1,2,\cdots,n$,若$\displaystyle\int_{a}^{+\infty}f_i(x)\d x$都收敛,$k_i$是任意常数,则$\displaystyle\int_{a}^{+\infty}\displaystyle\sum_{i=1}^{n}k_if_i(x)\d x$也收敛,且
	$$\int_{a}^{+\infty}\sum_{i=1}^{n}k_if_i(x)\d x=\sum_{i=1}^{n}k_i\int_{a}^{+\infty}f_i(x)\d x.$$
\end{proposition}
\begin{proposition}
	若$f$在任何有限区间$\left[a,u\right]$上可积,$a<b$,则$\displaystyle\int_{a}^{+\infty}f(x)\d x$与$\displaystyle\int_{b}^{+\infty}f(x)\d x$同敛态,且有
	$$\int_{a}^{+\infty}f(x)\d x=\int_{a}^{b}f(x)\d x+\int_{b}^{+\infty}f(x)\d x.$$
	其中右边第一项为Riemann积分.
\end{proposition}
第二类,我们考虑无界函数的积分.
\begin{definition}[瑕积分]
	设函数$f$定义在区间$\left(a,b\right]$上,在点$a$的任一右邻域上无界,但在任何内闭区间$\left[u,b\right]\subset\left(a,b\right]$上有界且可积.如果存在极限
	$$\lim\limits_{u\to a^+}\int_{u}^{b}f(x)\d x=J,$$
	则称此极限为无界函数$f$在$\left(a,b\right]$上的反常积分,记作
	$$J=\int_{a}^{b}f(x)\d x,$$
	其中,$f$在$a$近旁是无界的,这时我们称$a$为$f$的{\heiti 瑕点},无界函数的反常积分又称为{\heiti 瑕积分}.
\end{definition}
类似地,可定义瑕点为$b$时的瑕积分:
$$\int_{a}^{b}f(x)\d x=\lim\limits_{u\to b^-}\int_{a}^{u}f(x)\d x.$$

若$f$的瑕点$c\in(a,b)$,则定义瑕积分
$$\int_{a}^{b}f(x)\d x=\int_{a}^{c}f(x)\d x+\int_{c}^{b}f(x)\d x=\lim\limits_{u\to c^-}f(x)\d x+\lim\limits_{u\to c^+}f(x)\d x$$
当且仅当等号右边两个瑕积分都收敛时它才是收敛的.

若$a,b$两点都是$f$的瑕点,而$f$在任何$\left[u,v\right]\in\left(a,b\right)$上可积,这时定义瑕积分
$$\int_{a}^{b}f(x)\d x=\int_{a}^{c}f(x)\d x+\int_{c}^{b}f(x)\d x=\lim\limits_{u\to a^+}\int_{u}^{c}f(x)\d x+\lim\limits_{v\to b^-}\int_{c}^{v}f(x)\d x.$$
其中$c$为$(a,b)$上任一实数.当且仅当等号右边两个瑕积分都收敛时它才是收敛的.

同样可以根据函数极限的性质与定积分的性质,导出瑕积分的一些相应性质.
\begin{proposition}
	对于任意$i=1,2,\cdots,n$,设函数$f_i$的瑕点都是$x=a$,$k_i$为常数,则当瑕积分$\displaystyle\int_{a}^{b}f_i(x)\d x$都收敛时,$\displaystyle\int_{a}^{b}\displaystyle\sum_{i=1}^{n}k_if_i(x)\d x$必定收敛,且
	$$\int_{a}^{b}\sum_{i=1}^{n}k_if_i(x)\d x=\sum_{i=1}^{n}k_i\int_{a}^{b}f_i(x)\d x.$$
\end{proposition}
\begin{proposition}
	设函数$f$的瑕点为$x=a$,$c\in(a,b)$为任一常数.则瑕积分$\displaystyle\int_{a}^{b}f(x)\d x$与$\int_{a}^{c}f(x)\d x$同敛态,并有
	$$\int_{a}^{b}f(x)\d x=\int_{a}^{c}f(x)\d x+\int_{c}^{b}f(x)\d x.$$
	其中等号右边第二项为Riemann积分.
\end{proposition}
\section{无穷积分的敛散判别}
\subsection{无穷积分收敛的Cauchy准则}
由定义可知,无穷积分$\displaystyle\int_{a}^{+\infty}f(x)\d x$收敛与否,取决于函数$F(u)=\displaystyle\int_{a}^{u}f(x)\d x$在$u\to +\infty$时是否存在极限.因此可由函数极限的Cauchy准则导出无穷积分收敛的Cauchy准则.
\begin{theorem}[无穷积分收敛的Cauchy准则]
	无穷积分收敛的充要条件是:任给$\varepsilon>0$,存在$G\geqslant a$,只要$u_1,u_2>G$,便有
	$$\bigg|\int_{a}^{u_2}f(x)\d x-\int_{a}^{u_1}f(x)\d x\bigg|=\bigg|\int_{u_1}^{u_2}f(x)\d x\bigg|<\varepsilon.$$
\end{theorem}
\begin{proposition}
	若$f$在任何有限区间$\left[a,u\right]$上可积,且有$\displaystyle\int_{a}^{+\infty}|f(x)|\d x$收敛,则$\displaystyle\int_{a}^{+\infty}f(x)\d x$亦必收敛,并有
	$$\bigg|\int_{a}^{+\infty}f(x)\d x\bigg|\leqslant\int_{a}^{+\infty}|f(x)|\d x.$$
\end{proposition}
\begin{proof}
	由$\displaystyle\int_{a}^{+\infty}|f(x)|\d x$收敛,根据Cauchy准则(必要性),任给$\varepsilon>0$,存在$G\geqslant a$,当$u_2>u_1>G$时,总有
	$$\bigg|\int_{u_1}^{u_2}|f(x)|\d x\bigg|=\int_{u_1}^{u_2}|f(x)|\d x<\varepsilon.$$
	利用定积分的绝对值不等式,又有
	$$\bigg|\int_{u_1}^{u_2}f(x)\d x\bigg|\leqslant\int_{u_1}^{u_2}|f(x)|\d x<\varepsilon.$$
	再由Cauchy准则(充分性),证得$\displaystyle\int_{a}^{+\infty}f(x)\d x$收敛.
	
	又因为$\bigg|\displaystyle\int_{a}^{u}f(x)\d x\bigg|\leqslant\int_{a}^{u}|f(x)|\d x\ (u>a)$,令$u\to +\infty$,得
	$$\bigg|\int_{a}^{+\infty}f(x)\d x\bigg|\leqslant\int_{a}^{+\infty}|f(x)|\d x.$$
	$\hfill\blacksquare$
\end{proof}
\begin{remark}
	当$\displaystyle\int_{a}^{+\infty}|f(x)|\d x$收敛时,称$\displaystyle\int_{a}^{+\infty}f(x)\d x$为{\heiti 绝对收敛}.由上述命题我们知道,绝对收敛的无穷积分,其自身也一定收敛.但是,收敛的无穷积分不一定绝对收敛.我们称收敛而不绝对收敛者为{\heiti 条件收敛}.
\end{remark}
\subsection{非负函数无穷积分敛散判别}
\begin{theorem}[比较原则]
	设定义在$\left[a,+\infty\right)$上的两个非负函数$f$和$g$都在任何有限区间$\left[a,u\right]$上可积,且满足
	$$f(x)\leqslant g(x),\ x\in \left[a,+\infty\right),$$
	则

		(i)当$\displaystyle\int_{a}^{+\infty}g(x)\d x$收敛时,$\displaystyle\int_{a}^{+\infty}f(x)\d x$必收敛;
		
		(ii)当$\displaystyle\int_{a}^{+\infty}f(x)\d x$发散时,$\displaystyle\int_{a}^{+\infty}g(x)\d x$必发散.

	
\end{theorem}
\begin{proof}
	只需证明(i).由于$\displaystyle\int_{a}^{u}g(x)\d x$关于上界$u$是单调递增的,因此$\displaystyle\int_{a}^{+\infty}g(x)\d x$收敛的充要条件是$\displaystyle\int_{a}^{u}g(x)\d x$在$\left[a,+\infty\right)$上存在上界$M$.
	
	由$f(x)\leqslant g(x)$且$f,g$非负,有
	$$\int_{a}^{u}f(x)\d x\leqslant\int_{a}^{u}g(x)\d x\leqslant M.$$
	因此$\displaystyle\int_{a}^{+\infty}f(x)\d x$收敛.$\hfill\blacksquare$
\end{proof}
\begin{corollary}[比较原则的极限形式]
	若$f$和$g$都在任何有限区间$\left[a,u\right]$上可积,当$x\in\left[a,+\infty\right)$时,$f(x)\geqslant 0,\ g(x)>0$,且$\lim\limits_{x\to +\infty}\dfrac{f(x)}{g(x)}=c$,则有:
	
	(i)当$0<c<+\infty$时,$\displaystyle\int_{a}^{+\infty}f(x)\d x$与$\displaystyle\int_{a}^{+\infty}g(x)\d x$同敛态;
	
	(ii)当$c=0$时,由$\displaystyle\int_{a}^{+\infty}g(x)\d x$收敛可推知$\displaystyle\int_{a}^{+\infty}f(x)\d x$也收敛;
	
	(iii)当$c=+\infty$时,由$\displaystyle\int_{a}^{+\infty}g(x)\d x$发散可推知$\displaystyle\int_{a}^{+\infty}f(x)\d x$也发散.
\end{corollary}
特别地,如果选用$\displaystyle\int_{a}^{+\infty}\frac{\d x}{x^p}$作为比较对象,则有下面的{\heiti Cauchy判别法}及其极限形式.
\begin{corollary}[Cauchy判别法]
	设$f$定义于$\left[a,+\infty\right)\ (a>0)$,且在任何有限区间$\left[a,u\right]$上可积,则有:
	
	(i)当$0\leqslant f(x)\leqslant\dfrac{1}{x^p},\ x\in\left[a,+\infty\right)$,且$p>1$时,$\displaystyle\int_{a}^{+\infty}f(x)\d x$收敛;
	
	(ii)当$f(x)\geqslant\dfrac{1}{x^p},\ x\in\left[a,+\infty\right)$,且$p\leqslant 1$时,$\displaystyle\int_{a}^{+\infty}f(x)\d x$发散.
\end{corollary}
\begin{corollary}[Cauchy判别法的极限形式]
	设$f$是定义于$\left[a,+\infty\right)$上的非负函数,在任何有限区间$\left[a,u\right]$上可积,且
	$$\lim\limits_{x\to +\infty}x^pf(x)=\lambda.$$
	则有
	
	(i)当$p>1,\ 0\leqslant\lambda<+\infty$时,$\displaystyle\int_{a}^{+\infty}f(x)\d x$收敛;
	
	(ii)当$p\leqslant 1,\ 0<\lambda\leqslant +\infty$时,$\displaystyle\int_{a}^{+\infty}f(x)\d x$发散.
\end{corollary}
\subsection{一般无穷积分的敛散判别法}
\begin{theorem}[Dirichlet判别法]
	若$F(u)=\displaystyle\int_{a}^{u}f(x)\d x$在$\left[a,+\infty\right)$上有界,$g(x)$在$\left[a,+\infty\right)$上当$x\to +\infty$时单调趋于$0$,则$\displaystyle\int_{a}^{+\infty}f(x)g(x)\d x$收敛.
\end{theorem}
\begin{proof}
	设$\bigg|\displaystyle\int_{a}^{u}f(x)\d x\bigg|\leqslant M,\ u\in\left[a,+\infty\right)$.任给$\varepsilon>0$,由于$\lim\limits_{x\to +\infty}g(x)=0$,因此存在$G\geqslant a$,当$x>G$时,有
	$$|g(x)|<\frac{\varepsilon}{4M}.$$
	又因为$g$是单调函数,由积分第二中值定理的推论,对于任何$u_2>u_1>G$,存在$\xi\in\left[u_1,u_2\right]$,使得
	$$\int_{u_1}^{u_2}f(x)g(x)\d x=g(u_1)\int_{u_1}^{\xi}f(x)\d x+g(u_2)\int_{\xi}^{u_2}f(x)\d x.$$
	于是有
	\begin{align*}
		\bigg|\int_{u_1}^{u_2}f(x)g(x)\d x\bigg|
		&\leqslant|g(u_1)|\cdot\bigg|\int_{u_1}^{\xi}f(x)\d x\bigg|+|g(u_2)|\cdot\bigg|\int_{\xi}^{u_2}f(x)\d x\bigg|\\
		&=|g(u_1)|\cdot\bigg|\int_{a}^{\xi}f(x)\d x-\int_{u_1}^{a}f(x)\d x\bigg|+|g(u_2)|\cdot\bigg|\int_{a}^{u_2}f(x)\d x-\int_{a}^{\xi}f(x)\d x\bigg|\\
		&<\frac{\varepsilon}{4M}\cdot 2M+\frac{\varepsilon}{4M}\cdot 2M=\varepsilon.
	\end{align*}
	根据Cauchy准则,证得$\displaystyle\int_{a}^{+\infty}f(x)g(x)\d x$收敛.
	$\hfill\blacksquare$
\end{proof}
\begin{theorem}[Abel判别法]
	若$\displaystyle\int_{a}^{+\infty}f(x)\d x$收敛,$g(x)$在$\left[a,+\infty\right)$上单调有界,则$\displaystyle\int_{a}^{+\infty}f(x)g(x)\d x$收敛.
\end{theorem}
\begin{proof}
	由于$\displaystyle\int_{a}^{+\infty}f(x)\d x$收敛,即$\lim\limits_{u\to +\infty}\displaystyle\int_{a}^{u}f(x)\d x$存在,则$\displaystyle\int_{a}^{u}f(x)\d x$在$\left[0,+\infty\right)$上有界,又$g(x)$在$\left[a,+\infty\right)$上单调有界,则必有极限.设$\lim\limits_{x\to +\infty}g(x)=a$,即有
	$$\lim\limits_{x\to +\infty}\left[g(x)-a\right]=0.$$
	由Dirichlet判别法知,$\displaystyle\int_{a}^{+\infty}f(x)\left[g(x)-a\right]\d x$收敛,即
	$$\displaystyle\int_{a}^{+\infty}\left[f(x)g(x)-af(x)\right]\d x=\displaystyle\int_{a}^{+\infty}f(x)g(x)\d x-a\displaystyle\int_{a}^{+\infty}f(x)\d x.$$
	由于$\displaystyle\int_{a}^{+\infty}f(x)\d x$收敛,故$\displaystyle\int_{a}^{+\infty}f(x)g(x)\d x$收敛.$\hfill\blacksquare$
\end{proof}
\hspace*{\fill}

\begin{remark}
	上述证明利用了Dirichlet判别法,当然,这里也可利用积分第二中值定理证明,在此不再赘述.
\end{remark}
\section{瑕积分的敛散判别}
瑕积分与无穷积分的敛散判别法相类似,我们只是给出陈述,不再加以证明.
\subsection{瑕积分收敛的Cauchy准则}
\begin{theorem}[瑕积分收敛的Cauchy准则]
	瑕积分$\displaystyle\int_{a}^{b}f(x)\d x$(瑕点为$a$)收敛的充要条件是:任给$\varepsilon>0$,存在$\delta>0$,只要$u_1,u_2\in(a,a+\delta)$,总有
	$$\bigg|\int_{u_1}^{b}f(x)\d x-\int_{u_2}^{b}f(x)\d x\bigg|=\bigg|\int_{u_1}^{u_2}f(x)\d x\bigg|<\varepsilon.$$
\end{theorem}
\begin{proposition}
	设函数$f$的瑕点为$a$,$f$在$\left(a,b\right]$的任一内闭区间$\left[u,b\right]$上可积.则当$\displaystyle\int_{a}^{b}|f(x)|\d x$收敛时,$\displaystyle\int_{a}^{b}f(x)\d x$也必定收敛,并有
	$$\bigg|\int_{a}^{b}f(x)\d x\bigg|\leqslant\int_{a}^{b}|f(x)|\d x.$$
\end{proposition}
\begin{remark}
	同样地,当$\displaystyle\int_{a}^{b}|f(x)|\d x$收敛时,称$\displaystyle\int_{a}^{b}f(x)\d x$为{\heiti 绝对收敛},称收敛而不绝对收敛者为{\heiti 条件收敛}.
\end{remark}
\subsection{非负函数瑕积分敛散判别}
\begin{theorem}[比较原则]
	设定义在$\left(a,b\right]$上的两个非负函数$f$和$g$瑕点都为$a$,在任何区间$\left[u,b\right]\subset\left(a,b\right]$上都可积,且满足
	$$f(x)\leqslant g(x),\ x\in \left(a,b\right],$$
	则有:

		(i)当$\displaystyle\int_{a}^{b}g(x)\d x$收敛时,$\displaystyle\int_{a}^{b}f(x)\d x$必收敛;
		
		(ii)当$\displaystyle\int_{a}^{b}f(x)\d x$发散时,$\displaystyle\int_{a}^{b}g(x)\d x$必发散.

\end{theorem}
\begin{corollary}[比较原则的极限形式]
	$f(x)\geqslant 0,\ g(x)>0$,且$\lim\limits_{x\to +\infty}\dfrac{f(x)}{g(x)}=c$时,有:
	
	(i)当$0<c<+\infty$时,$\displaystyle\int_{a}^{b}f(x)\d x$与$\displaystyle\int_{a}^{b}g(x)\d x$同敛态;
	
	(ii)当$c=0$时,由$\displaystyle\int_{a}^{b}g(x)\d x$收敛可推知$\displaystyle\int_{a}^{b}f(x)\d x$也收敛;
	
	(iii)当$c=+\infty$时,由$\displaystyle\int_{a}^{b}g(x)\d x$发散可推知$\displaystyle\int_{a}^{b}f(x)\d x$也发散.
\end{corollary}
特别地,如果选用$\displaystyle\int_{a}^{+\infty}\frac{\d x}{(x-a)^p}$作为比较对象,则有下面的{\heiti Cauchy判别法}及其极限形式.
\begin{corollary}[Cauchy判别法]
	设$f$定义于$\left(a,b\right]$上,且在任何区间$\left[u,b\right]\subset\left(a,b\right]$上可积,则有:
	
	(i)当$0\leqslant f(x)\leqslant\dfrac{1}{(x-a)^p}$,且$0<p<1$时,$\displaystyle\int_{a}^{b}f(x)\d x$收敛;
	
	(ii)当$f(x)\geqslant\dfrac{1}{(x-a)^p}$,且$p\geqslant 1$时,$\displaystyle\int_{a}^{b}f(x)\d x$发散.
\end{corollary}
\begin{corollary}[Cauchy判别法的极限形式]
	设$f$是定义于$\left(a,b\right]$上的非负函数,$a$是其瑕点,且在任何区间$\left[u,b\right]\subset\left(a,b\right]$上可积,若
	$$\lim\limits_{x\to a^+}(x-a)^pf(x)=\lambda.$$
	则有
	
	(i)当$0<p<1,\ 0\leqslant\lambda<+\infty$时,$\displaystyle\int_{a}^{b}f(x)\d x$收敛;
	
	(ii)当$p\geqslant 1,\ 0<\lambda\leqslant +\infty$时,$\displaystyle\int_{a}^{b}f(x)\d x$发散.
\end{corollary}
\subsection{一般瑕积分的敛散判别法}
\begin{theorem}[Dirichlet判别法]
	设$a$为$f(x)$的瑕点,函数$F(u)=\displaystyle\int_{a}^{b}f(x)\d x$在$\left(a,b\right]$上有界,函数$g(x)$在$\left(a,b\right]$上单调且$\lim\limits_{x\to a^+}g(x)=0$,则瑕积分$\displaystyle\int_{a}^{b}f(x)g(x)\d x$收敛.
\end{theorem}
\begin{theorem}[Abel判别法]
	设$a$是$f(x)$的瑕点,瑕积分$\displaystyle\int_{a}^{b}f(x)\d x$发散,函数$g(x)$在$\left(a,b\right]$上单调且有界,则瑕积分$\displaystyle\int_{a}^{b}f(x)g(x)\d x$收敛.
\end{theorem}