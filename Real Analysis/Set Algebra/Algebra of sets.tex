\documentclass[12pt]{ctexart}
\usepackage{amsfonts,amssymb,amsmath,amsthm,geometry,graphicx,caption,color,xpatch}
\usepackage[colorlinks,linkcolor=blue,anchorcolor=blue,citecolor=green]{hyperref}
%define color 'orange'
\definecolor{orange}{RGB}{255,127,0}

%introduce theorem environment
\theoremstyle{definition}
\newtheorem{definition}{定义}
\newtheorem{theorem}{定理}
\newtheorem{property}{性质}
\newtheorem{example}{例}
\theoremstyle{plain}
\newtheorem*{solution}{\textcolor{green}{Soln}}
\newtheorem*{remark}{\textcolor{orange}{Rmk}}
\geometry{a4paper,scale=0.8}

%remove the dot after the theoremstyle
%\makeatletter
%\xpatchcmd{\@thm}{\thm@headpunct{.}}{\thm@headpunct{}}{}{}
%\makeatother

%article info
\title{\vspace{-2em}\textbf{集代数}\vspace{-2em}}
\date{ }

\begin{document}
	\maketitle
	\begin{definition}[幂集]
		集合$S$的所有子集组成的集合称为集合$S$的\textbf{幂集},记作$2^S$.
	\end{definition}
	\begin{definition}[环与代数]
		设非空集类$\mathcal{R}\subset 2^S$满足以下条件:
		\begin{enumerate}
			\item 对差封闭:$\forall A,B\in\mathcal{R},A-B\in\mathcal{R}$;
			\item 对有限并封闭:$\forall E_i\in\mathcal{R},i=1,2,\cdots,n,\ \bigcup_{i=1}^{n}E_i\in\mathcal{R}$.
		\end{enumerate}
		则称$\mathcal{R}$为\textbf{环}.特别地,若全集在其中,即$S\in\mathcal{R}$,则称$\mathcal{R}$为\textbf{代数}.当有限并可以改为可列并时,分别称为$\boldsymbol{\sigma}$\textbf{-环}和$\boldsymbol{\sigma}$\textbf{-代数}.
	\end{definition}
	\begin{definition}[$\pi$-系统]
		设非空集类$\mathcal{F}\subset 2^S$满足以下条件:
		\begin{enumerate}
			\item 对有限交封闭:$\forall E_i\in\mathcal{F},i=1,2,\cdots,n,\ \bigcap_{i=1}^{n}E_i\in\mathcal{F}$.
		\end{enumerate}
		则称$\mathcal{F}$为$\boldsymbol{\pi}$\textbf{-系统}.
	\end{definition}
	\begin{definition}[$\lambda$-系统]
		设非空集类$\mathcal{F}\subset 2^S$满足以下条件:
		\begin{enumerate}
			\item 全集在其中:$S\in\mathcal{F}$;
			\item 对差封闭:$\forall A,B\in\mathcal{F},A-B\in\mathcal{F}$;
			\item 对不交可列并封闭:$\forall E_i\in\mathcal{R},i=1,2,\cdots,\ \bigsqcup_{i=1}^{\infty}E_i\in\mathcal{F}$.
		\end{enumerate}
		则称$\mathcal{F}$为$\boldsymbol{\lambda}$\textbf{-系统}.
	\end{definition}
	\begin{definition}[单调类]
		若非空集类$\mathcal{F}\subset 2^S$对单调集列的极限封闭,则称$\mathcal{F}$为\textbf{单调类}.
	\end{definition}
	\begin{definition}[生成的$\sigma$-环]
		设非空集类$\mathcal{F}\subset 2^S$,称包含$\mathcal{F}$的最小的$\sigma$-环为$\mathcal{F}$生成的$\sigma$-环,记作$R(\mathcal{F})$.
	\end{definition}
	\begin{definition}[生成的$\sigma$-代数]
		设非空集类$\mathcal{F}\subset 2^S$,称包含$\mathcal{F}$的最小的$\sigma$-代数为$\mathcal{F}$生成的$\sigma$-代数,记作$\sigma(\mathcal{F})$.
	\end{definition}
	\begin{definition}[生成的$\lambda$-系]
		设非空集类$\mathcal{F}\subset 2^S$,称包含$\mathcal{F}$的最小的$\lambda$-系为$\mathcal{F}$生成的$\lambda$-系,记作$\delta(\mathcal{F})$.
	\end{definition}
	\begin{theorem}[$\lambda$-$\pi$系定理]
		设$\mathcal{F}$为$\pi$-系,则
		$$\delta(\mathcal{F})=\sigma(\mathcal{F}).$$
	\end{theorem}
	\begin{proof}
		$\sigma(\mathcal{F})$为包含$\mathcal{F}$的$\lambda$-系,而$\delta(\mathcal{F})$是包含$\mathcal{F}$的最小的$\lambda$-系,于是$\delta(\mathcal{F})\subset\sigma(\mathcal{F})$.
		
		为证$\sigma(\mathcal{F})\subset\delta(\mathcal{F})$,只需证$\delta(\mathcal{F})$为$\sigma$-代数.
		
		只需证$\delta(\mathcal{F})$关于有限交封闭.
		
		对任意$A\in\delta(\mathcal{F})$,令
		$$\kappa(A)=\left\{B\in\delta(\mathcal{F}):B\cap A\in\delta(\mathcal{F})\right\}.$$
		以下证$\kappa(A)=\delta(\mathcal{F})$.
		
		首先,若$A\in\mathcal{F}$,因$\mathcal{F}$为$\pi$-系,故$F\subset \kappa(A)$.
		
		以下说明$\kappa(A)$为$\lambda$-系.如此$\kappa(A)\supset\delta(\mathcal{F})$,而由$\kappa(A)$定义,显然$\kappa(A)\subset\delta(\mathcal{F})$,如此$\kappa(A)=\delta(\mathcal{F})$.
		
		\begin{enumerate}
			\item 全集在其中:$S\cap A\in\kappa(A)$;
			\item 不交可列并封闭:设$B_n$两两不交,$B_n\subset\kappa(A)$,即$B_n\cap A\in\delta(\mathcal{F})$.有
			$$\left(\bigcup_nB_n\right)\cap A=\bigcup_n\left(B_n\cap A\right)=\bigsqcup_n(B_n\cap A)\in\delta(\mathcal{F}).$$
			\item 补集在其中:若$B\in\kappa(A)$,即$B\cap A\in\delta(\mathcal{F})$,下证$B^c\in\kappa(A)$,即证$B^c\cap A\in\delta(\mathcal{F})$.
			$$B^c\cap A=(B\cap A)^c\cap A=\left((B\cap A)\cup A^c\right)^c=\left((B\cap A)\sqcup A^c\right)^c.$$
		\end{enumerate}
		
		其次,若$A\in\delta(\mathcal{F})$,$\mathcal{F}\subset\kappa(A)$,同理证明$\kappa(A)$是$\lambda$-系即可.
	\end{proof}
	\begin{theorem}[单调类方法]
		若$\mathcal{F}$是环,则
		$$M(\mathcal{F})=R(\mathcal{F}).$$
	\end{theorem}
\end{document}