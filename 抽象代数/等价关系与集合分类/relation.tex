\documentclass[12pt]{ctexart}
\usepackage{amsfonts,amssymb,amsmath,amsthm,geometry,enumerate}
\usepackage[colorlinks,linkcolor=blue,anchorcolor=blue,citecolor=green]{hyperref}
\usepackage[all]{xy}
%introduce theorem environment
\theoremstyle{definition}
\newtheorem{definition}{定义}
\newtheorem{theorem}{定理}
\newtheorem{lemma}{引理}
\newtheorem{corollary}{推论}
\newtheorem{property}{性质}
\newtheorem{example}{例}
\theoremstyle{plain}
\newtheorem*{solution}{解}
\newtheorem*{remark}{注}
\geometry{a4paper,scale=0.8}

%article info
\title{\vspace{-2em}\textbf{等价关系与集合分类}\vspace{-2em}}
\date{ }
\begin{document}
	\maketitle
	\begin{definition}[关系]
		设集合$R\subset A\times A$,$a,b\in A$,若$(a,b)\in R$,则称$a$和$b$有\textbf{关系}$R$,记作$aRb$;若$(a,b)\notin R$,则称$a$与$b$没有关系.
	\end{definition}
	\begin{definition}[等价关系]
		若关系$R$满足
		\begin{enumerate}
			\item 反身性:$aRa,\ \forall a\in A$;
			\item 对称性:$aRb$,则$bRa$,$\forall a,b\in A$;
			\item 传递性:$aRb,bRc$则$aRc$,$\forall a,b,c\in A$.
		\end{enumerate}
		则称$R$为\textbf{等价关系}.
	\end{definition}
	\begin{definition}[集合的分类]
		非空集合$A$可以分成若干不交非空子集,即$A=\bigcup_{i\in I}M_i$,$M_i\cap M_j=\varnothing,\ i\neq j$,则$\{M_i|i\in I\}$称为$A$的一个\textbf{分类}或\textbf{分划}.
	\end{definition}
	\begin{theorem}
		集合$A$的一个分类决定$A$中的一个等价关系.
	\end{theorem}
	\begin{proof}
		设关系$R$满足
		$$aRb\iff a\ \text{和}\ b\ \text{在同一类},$$
		则根据定义易得$R$是等价关系.
	\end{proof}
	\begin{definition}[等价类]
		设在集合$A$上定义了一个等价关系$R$,$a\in A$,则所有与$a$有关系的元素构成一个集合$\{b\in A\ |\ bRa\}$,称为$a$所在的\textbf{等价类},记作$\overline{a}$,$a$称为这个等价类的\textbf{代表元}.
	\end{definition}
	\begin{definition}[商集]
		设集合$A$中有等价关系$R$,则以$R$为前提的所有等价类的集合$\{\overline{a}\}$称为$A$对$R$的\textbf{商集},记作$A/R$.
	\end{definition}
	\begin{definition}[自然映射]
		称从非空集合$A$到它的商集合$A/R$的映射$\pi:A\to A/R,\ \pi(a)=\overline{a}$为\textbf{自然映射}.
	\end{definition}
	容易验证$\pi$是映射,且是满射,但未必是单射,因为以$a$为代表元的等价类不一定只有$a$这一个元素,如果$b\in\overline{a}$,那么$\pi(a)=\pi(b)=\overline{a}$.
	\begin{theorem}
		集合$A$中的一个等价关系决定$A$的一个分类.
	\end{theorem}
	\begin{proof}
		对任意$a\in A$,$\pi(a)$是$a$所在的等价类,于是$A$中的任何元素都有所在的等价类,这些等价类互不相交,于是构成了$A$的一个分类.
	\end{proof}
	\begin{definition}[同余关系]
		设集合$A$中有等价关系$R$,并带有二元运算“$\circ$”,若满足
		$$aRb,\ cRd\Rightarrow (a\circ b)R(c\circ d),\quad\forall a,b,c,d\in A,$$
		则称$R$是\textbf{同余关系},相应地,$a$的等价类也称为$a$的\textbf{同余类}.
	\end{definition}
	\begin{theorem}
		设“$\circ$”是$A$中的二元运算,并定义“$\overline{\circ}$”:$\overline{a}\overline{\circ}\overline{c}=\overline{a\circ c}$,则“$\overline{\circ}$”是$A$中的二元运算当且仅当$R$是同余关系.
	\end{theorem}
	\begin{proof}
		若“$\overline{\circ}$”是二元运算,则对任意$\overline{a},\overline{c}\in A/R$,有$\overline{a}\ \overline{\circ}\ \overline{c}\in A/R$,于是$\overline{a\circ c}\in A/R$,设$aRb$,$cRd$,则
		$$\overline{a}\ \overline{\circ}\ \overline{c}=\overline{b}\ \overline{\circ}\ \overline{d}=\overline{b\circ d}=\overline{a\circ c},$$
		故$(a\circ c)R(b\circ d)$.
		
		若$R$是同余关系,则对任意$aRb$,$cRd$,有$(a\circ c)R(b\circ d)$,进而$\overline{a\circ c}=\overline{b\circ d}$,故$\overline{a}\ \overline{\circ}\ \overline{c}=\overline{b}\ \overline{\circ}\ \overline{d}\in A/R$,所以“$\circ$”是二元运算.
	\end{proof}
\end{document}