\documentclass[12pt]{ctexart}
\usepackage{amsfonts,amssymb,amsmath,amsthm,geometry,enumerate}
\usepackage[colorlinks,linkcolor=blue,anchorcolor=blue,citecolor=green]{hyperref}
\usepackage[all]{xy}
%introduce theorem environment
\theoremstyle{definition}
\newtheorem{definition}{定义}
\newtheorem{theorem}{定理}
\newtheorem{lemma}{引理}
\newtheorem{corollary}{推论}
\newtheorem{property}{性质}
\newtheorem{proposition}{命题}
\newtheorem{example}{例}
\theoremstyle{plain}
\newtheorem*{solution}{解}
\newtheorem*{remark}{注}
\geometry{a4paper,scale=0.8}

%article info
\title{\vspace{-2em}\textbf{群的同态与同构}\vspace{-2em}}
\date{ }
\begin{document}
	\maketitle
	\begin{definition}[同态映射]
		设群$(G_1,\circ)$,$(G_2,\ast)$之间存在映射$f:G_1\to G_2$,若对任意$g_1,g_2\in G_1$,有$f(g_1\circ g_2)=f(g_1)\ast f(g_2)$,则称$f$为\textbf{同态映射}. 符号不至于混淆时,常记作$f(g_1g_2)=f(g_1)f(g_2)$.
	\end{definition}
	如果$f$是单射,则称为\textbf{单同态};如果$f$是满射,则称为\textbf{满同态}. 若$f:G_1\to G_2$是满同态,则称$G_1$和$G_2$是\textbf{同态的}.
	\begin{definition}[同构]
		若同态映射$f:G_1\to G_2$是双射,则称$f$是\textbf{同构映射},$G_1$和$G_2$是\textbf{同构的},记作$G_1\cong G_2$.
	\end{definition}
	\begin{remark}
		容易验证,同构是等价关系.
	\end{remark}
	\begin{definition}[自然同态]
		设$H\lhd G$,映射$\pi:G\to G/H,g\mapsto gH$是同态映射,称为\textbf{自然同态}.
	\end{definition}
	\begin{property}
		设同态映射$f:G_1\to G_2$,$g:G_2\to G_3$,则$gf:G_1\to G_3$也是同态映射.
	\end{property}
	\begin{property}
		幺元同态到幺元,逆元同态到逆元,子群同态到子群.
	\end{property}
	\begin{definition}[核]
		设同态映射$f:G_1\to G_2$,$e_1\in G_1$,$e_2\in G_2$是幺元,$G_2$的幺元$e_2$的完全原像$\{a\in G_1\ |\ f(a)=e_2\}$称为同态映射$f$的\textbf{核},记作$\ker f$.
	\end{definition}
	\begin{example}
		若$f$是单同态,则$\ker f=\{e_1\}$.
	\end{example}
	\begin{proposition}
		若$H\lhd G$,$\pi:G\to G/H$,则$\ker\pi=H$.
	\end{proposition}
	\begin{proposition}
		设同态映射$f:G_1\to G_2$,则$\ker f\lhd G_1$.
	\end{proposition}
	\begin{proof}
		对任意$g\in G_1$,$a\in\ker f$,
		$$f(gag^{-1})=f(g)f(a)f(g^{-1})=f(g)f(g^{-1})=f(gg^{-1})=f(e_1)=e_2,$$
		于是$gag^{-1}\in\ker f$. 由正规子群定义得$\ker f\lhd G_1$.
	\end{proof}
	\begin{theorem}[群同态基本定理]
		设满同态$f:G_1\to G_2$,则$G_1/\ker f\cong G_2$.
	\end{theorem}
	\begin{proof}
		记$N=\ker f\lhd G_1$,设$\varphi:G_1/N\to G_2,\ gN\mapsto f(g)$.若$g_1N=g_2N$,则$g_1^{-1}g_2\in N$,$f(g_1^{-1}g_2)=f(g_1)^{-1}f(g_2)=e_2$,于是$f(g_1)=f(g_2)$. 这表明$g_N$在$\varphi$下的像是唯一的,所以$\varphi$是映射.
		
		若$f(g_1)=f(g_2)$,则$e_2=f(g_1)^{-1}f(g_2)=f(g_1^{-1}g_2)$,于是$g_1^{-1}g_2\in N$,$g_1N=g_2N$,因此$\varphi$是单射.
		
		由于$f$是满射,因此$\varphi$是满射,故$\varphi$是双射.
		
		对任意$aN,bN\in G/N$,由于$f$是同态,有
		$$\varphi(aNbN)=\varphi(abN)=f(ab)=f(a)f(b)=\varphi(aN)\varphi(bN).$$
		因此$\varphi$是同构映射,故$G_1/\ker f\cong G_2$.
	\end{proof}
	\begin{corollary}[第一同构定理]
		设$f$是群$G$的同态,则$G/\ker f\cong f(G)$.
	\end{corollary}
	\begin{theorem}[第二同构定理]
		若$H<G$,$N\lhd G$,则$H\cap N\lhd H$且
		$$H/(H\cap N)\cong HN/N.$$
	\end{theorem}
	\begin{proof}
		令$\varphi:H\to HN/N,\ h\mapsto hN$,显然$\varphi$是映射. 对任意$hnN\in HN/N$,由于$hnN=hN$,有$$\varphi(h)=hN=hnN,$$故$\varphi$是满射. 对任意$h_1,h_2\in H$,
		$$\varphi(h_1h_2)=h_1h_2N=h_1Nh_2N=\varphi(h_1)\varphi(h_2),$$
		故$\varphi$是同态. 而
		$$\ker\varphi=\left\{h\in H\ |\ \varphi(h)=hN=e_2=N\right\}=\left\{h\in H\ |\ h\in N\right\}=H\cap N,$$
		由同态基本定理,有
		$$H/(H\cap N)\cong HN/N.$$
	\end{proof}
	\begin{theorem}[第三同构定理]
		若$H\lhd G$,$N\lhd G$,$N\subset H$,则
		$$G/H\cong (G/N)/(H/N).$$
	\end{theorem}
	\begin{proof}
		由$H\lhd G$,$N\lhd G$以及$N\subset H$,有$N\lhd H$,且$H/N\lhd G/N$.
		
		设$\pi:G\to G/N,\ g\mapsto gN$以及$\psi:G/N\to(G/N)/(H/N),\ gN\mapsto (gN)(H/N)$,则$\varphi=\psi\circ\pi:G\to(G/N)/(H/N)$是群同态.
		
		由于$\pi$,$\psi$是满射,故$\varphi$是满射. 又
		$$\ker\varphi=\left\{g\in G\ |\ \varphi(g)=H/N\right\},$$
		$$\varphi(g)=\psi(\pi(g))=(gN)(H/N),$$
		$$(gN)(H/N)=H/N\iff gN=H/N\iff g\in H,$$
		故$\ker\varphi=G\cap H=H$,由群同态基本定理,
		$$G/H\cong (G/N)/(H/N).$$
	\end{proof}
\end{document}