\documentclass[12pt]{ctexart}
\usepackage{amsfonts,amssymb,amsmath,amsthm,geometry,enumerate}
\usepackage[colorlinks,linkcolor=blue,anchorcolor=blue,citecolor=green]{hyperref}
\usepackage[all]{xy}
%introduce theorem environment
\theoremstyle{definition}
\newtheorem{definition}{定义}
\newtheorem{theorem}{定理}
\newtheorem{lemma}{引理}
\newtheorem{corollary}{推论}
\newtheorem{property}{性质}
\newtheorem{proposition}{命题}
\newtheorem{example}{例}
\theoremstyle{plain}
\newtheorem*{solution}{解}
\newtheorem*{remark}{注}
\geometry{a4paper,scale=0.8}

%article info
\title{\vspace{-2em}\textbf{正规子群与商群}\vspace{-2em}}
\date{ }
\begin{document}
	\maketitle
	\begin{definition}[共轭]
		设$f,h\in G$,若存在$g\in G$,使得
		$$f=ghg^{-1},$$
		则称$f$和$g$\textbf{共轭}.
	\end{definition}
	容易验证,共轭是一个等价关系,于是决定了一个等价类,称为共轭类.
	\begin{definition}[共轭类]
		设$G$是群,对任意$f\in G$,与$f$共轭的元素组成的集合称为以$f$为代表元的\textbf{共轭类}.
	\end{definition}
	\begin{definition}[自共轭元素]
		如果以$f$为代表的共轭类中的元素只有$f$一个,则称$f$是群$G$的\textbf{自共轭元素}.
	\end{definition}
	\begin{property}
		一个共轭类中所有元素的阶相同.
	\end{property}
	\begin{property}
		共轭类的元素数目是群的阶的因子.
	\end{property}
	\begin{definition}[共轭子群]
		若$H<G$,$K<G$,存在$g\in G$使得
		$$K=gHg^{-1},$$
		则称子群$H$和$K$\textbf{共轭}.
	\end{definition}
	如果对任意$g\in G$,子群$H$都是自共轭的,则称$H$为正规子群,即如下定义.
	\begin{definition}[正规子群]
		设$G$是群,$H<G$,若有
		$$ghg^{-1}\in H,\ \forall\ g\in G,\ \forall\ h\in H,$$
		则称$H$是$G$的\textbf{正规子群},记作$H\vartriangleleft G$.
	\end{definition}
	\begin{example}
		定义运算为矩阵乘法,$SL_n(F)\vartriangleleft GL_n(F)$.
	\end{example}
	\begin{example}
		定义运算为数的加法,$m\mathbb{Z}\vartriangleleft \mathbb{Z}$.
	\end{example}
	\begin{theorem}
		设$H<G$,则下列条件等价.
		\begin{enumerate}
			\item $H\vartriangleleft G$;
			\item $gH=Hg,\ \forall\ g\in G$;
			\item $g_1Hg_2H=g_1g_2H,\ \forall\ g_1,g_2\in G$.
		\end{enumerate}
	\end{theorem}
	\begin{proof}
		1 $\to$ 2 : 若$H\vartriangleleft G$,则$gHg^{-1}=H$,故$gH=Hg$.
		
		2 $\to$ 3 : 若$gH=Hg$,则$g_1Hg_2H=g_1(Hg_2)H=g_1g_2H$.
		
		3 $\to$ 1 : 若$g_1Hg_2H=g_1g_2H$,则$gHg^{-1}H=gg^{-1}H=H$,则$gHg^{-1}=H$.
	\end{proof}
	\begin{definition}[商群]
		设$H<G$,关系$R$定义为$aRb\iff a^{-1}b\in H$,则
		$$R\text{为同余关系}\iff H\vartriangleleft G,$$
		商集合$G/R$对同余关系$R$导出的运算也构成一个群,称为$G$对$H$的\textbf{商群},记为$G/H$.
	\end{definition}
	\begin{proof}
		必要性:因为$g^{-1}(gh)=h\in H$,故$gR(gh)$.又$g^{-1}Rg^{-1}$,于是由同余关系,$$(gg^{-1})R(ghg^{-1}),$$即$eR(ghg^{-1})$,$ghg^{-1}=e^{-1}ghg^{-1}\in H$,$H\vartriangleleft G$.
		
		充分性:设任意$a_1,a_2,b_1,b_2\in H$,$a_1Rb_1$,$a_2Rb_2$,则
		$$(a_1a_2)^{-1}(b_1b_2)=a_2^{-1}a_1^{-1}b_1b_2=a_2{-1}a_1^{-1}b_1a_2a_2^{-1}b_2,$$
		由于$H\vartriangleleft G$,$a_1^{-1}b_1\in H$,则$a_2{-1}a_1^{-1}b_1a_2\in H$,又$a_2^{-1}b_2\in H$,则$(a_1a_2)^{-1}(b_1b_2)\in H$,故$R$为同余关系.
	\end{proof}
	\begin{example}
		由于$m\mathbb{Z}\vartriangleleft\mathbb{Z}$,于是有商群
		$$\mathbb{Z}/m\mathbb{Z}=\left\{\overline{0},\overline{1},\cdots,\overline{m-1}\right\},$$
		记为$\mathbb{Z}_m$,称为$\mathbb{Z}$的\textbf{模$m$的剩余类加群}.
	\end{example}
	剩余类加群的每一个元素叫做一个剩余类. 同一剩余类中的两个元素同余,例如设$a,b\in\overline{k}$,$k=0,1,\cdots,m-1$,则$a\equiv b\pmod m$. 
\end{document}