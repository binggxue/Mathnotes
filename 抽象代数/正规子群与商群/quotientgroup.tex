\documentclass[12pt]{ctexart}
\usepackage{amsfonts,amssymb,amsmath,amsthm,geometry,enumerate}
\usepackage[colorlinks,linkcolor=blue,anchorcolor=blue,citecolor=green]{hyperref}
\usepackage[all]{xy}
%introduce theorem environment
\theoremstyle{definition}
\newtheorem{definition}{定义}
\newtheorem{theorem}{定理}
\newtheorem{lemma}{引理}
\newtheorem{corollary}{推论}
\newtheorem{property}{性质}
\newtheorem{proposition}{命题}
\newtheorem{example}{例}
\theoremstyle{plain}
\newtheorem*{solution}{解}
\newtheorem*{remark}{注}
\geometry{a4paper,scale=0.8}

%article info
\title{\vspace{-2em}\textbf{正规子群与商群}\vspace{-2em}}
\date{ }
\begin{document}
	\maketitle
	\begin{definition}[正规子群]
		设$G$是群,$H<G$,若有
		$$ghg^{-1}\in H,\ \forall\ g\in G,\ \forall\ h\in H,$$
		则称$H$是$G$的\textbf{正规子群},记作$H\vartriangleleft G$.
	\end{definition}
	\begin{theorem}
		设$H<G$,则下列条件等价.
		\begin{enumerate}
			\item $H\vartriangleleft G$;
			\item $gH=Hg,\ \forall\ g\in G$;
			\item $g_1Hg_2H=g_1g_2H,\ \forall\ g_1,g_2\in G$.
		\end{enumerate}
	\end{theorem}
	\begin{proof}
		
	\end{proof}
	\begin{definition}[商群]
		设$H<G$,关系$R$定义为$aRb\iff a^{-1}b\in H$,则
		$$R\text{为同余关系}\iff H\vartriangleleft G,$$
		商集合$G/R$对同余关系$R$导出的运算也构成一个群,称为$G$对$H$的\textbf{商群},记为$G/H$.
	\end{definition}
	\begin{proof}
		
	\end{proof}
	\begin{example}
		
	\end{example}
\end{document}