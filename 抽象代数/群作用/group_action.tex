\documentclass[12pt]{ctexart}
\usepackage{amsfonts,amssymb,amsmath,amsthm,geometry,enumerate}
\usepackage[colorlinks,linkcolor=blue,anchorcolor=blue,citecolor=green]{hyperref}
\usepackage[all]{xy}
%introduce theorem environment
\theoremstyle{definition}
\newtheorem{definition}{定义}
\newtheorem{theorem}{定理}
\newtheorem{lemma}{引理}
\newtheorem{corollary}{推论}
\newtheorem{property}{性质}
\newtheorem{proposition}{命题}
\newtheorem{example}{例}
\theoremstyle{plain}
\newtheorem*{solution}{解}
\newtheorem*{remark}{注}
\geometry{a4paper,scale=0.8}

\newcommand{\id}{\mathrm{id}}
\newcommand{\Aut}{\mathrm{Aut}}
\newcommand{\Inn}{\mathrm{Inn}}
\newcommand{\Orb}{\mathrm{Orb}}
\newcommand{\Stab}{\mathrm{Stab}}
%article info
\title{\vspace{-2em}\textbf{群作用}\vspace{-2em}}
\date{ }
\begin{document}
	\maketitle
	\begin{definition}[群作用]
		设$G$是群,$X$是非空集合. 定义映射$G\times X\to X,\ (g,x)\mapsto g\cdot x$,若满足
		\begin{enumerate}
			\item 幺元:$e\cdot x=x$;
			\item 兼容性:对任意$g_1,g_2\in G$,$(g_1g_2)\cdot x=g_1\cdot(g_2\cdot x)$,
		\end{enumerate}
		则称映射“$\cdot$”为$G$在$X$上的一个\textbf{作用}.
	\end{definition}
	
	群作用由公理化的定义有些抽象,下面建立群作用和置换之间的联系,也可看作是群作用的另一种定义.
	\begin{theorem}
		设$G$是群,$X$是非空集合,映射$\varphi:G\to S_X,\ g\mapsto\varphi_{g}$.则$\varphi$是同态当且仅当$\varphi$给出了一个群$G$在集合$X$上的群作用.
	\end{theorem}
	\begin{proof}
		必要性:定义映射$\cdot:G\times X\to X,\ (g,x)\mapsto\varphi_g(x)$,这里$\varphi_g\in S_X$是同态. 下面证明“$\cdot$”是群作用.
		
		对任意$g\in G$,$e\in G$是幺元,则
		$$\varphi_{ge}=\varphi_g\varphi_e=\varphi_g,$$
		于是$\varphi_e=\id$.则对任意$x\in X$,
		$$e\cdot x=\varphi_e(x)=\id(x)=x.$$
		
		对任意$g_1,g_2\in G$,有
		$$(g_1g_2)\cdot x=\varphi_{g_1}\varphi_{g_2}(x)=\varphi_{g_1}(\varphi_{g_2}(x))=g_1\cdot(g_2\cdot x).$$
		于是“$\cdot$”是群作用.
		
		充分性:定义映射$\varphi_g:X\to X,\ \varphi_g(x)\mapsto g\cdot x$,先证明$\varphi_g$是置换.
		
		对任意$x_1,x_2\in X$,若$\varphi_g(x_1)=\varphi_g(x_2)$,则$g\cdot x_1=g\cdot x_2$,用$g^{-1}$作用,得
		$$g^{-1}\cdot(g\cdot x_1)=g^{-1}\cdot(g\cdot x_2),$$由兼容性公理得$x_1=x_2$,故$\varphi_g$是单射.
		
		而$\varphi_{g}(g^{-1}\cdot x)=g\cdot(g^{-1}\cdot x)=e\cdot x=x$,故对所有$x\in X$,都存在原像$g^{-1}\cdot x$,于是$\varphi_g$是满射.
		
		因此$\varphi_g$是置换. 定义映射$\varphi:G\to S_X,\ g\mapsto\varphi_g$,下面证明$\varphi$是同态.
		
		对任意$g_1,g_2\in G$,任意$x\in X$,有
		$$\varphi_{g_1g_2}(x)=(g_1g_2)\cdot x=g_1\cdot(g_2\cdot x)=\varphi_{g_1}\cdot(\varphi_{g_2}(x))=\varphi_{g_1}\varphi_{g_2}(x),$$
		于是$\varphi$是同态.
	\end{proof}
	\begin{definition}[轨道]
		设$G$是群,$X$是非空集合. 对任意给定的$x\in X$,称集合$\left\{g\cdot x\ |\ \forall g\in G\right\}$为$x$的\textbf{轨道},记作$\Orb(x)$.
	\end{definition}
	轨道就是在群$G$的作用下,$x$所能到达的所有取值的集合.
	\begin{definition}[稳定化子]
		设$G$是群,$X$是非空集合. 对任意给定的$x\in X$,集合$\left\{g\ |\ g\cdot x=x\right\}$关于群$G$的运算构成群,称为$x$的\textbf{稳定化子}或\textbf{迷向子群},记作$\Stab(x)$.
	\end{definition}
	也就是说,群$G$中有一些元素,作用在$x$上,得到的还是$x$自身. 例如$e\in G$,$e\cdot x=x$. 下面证明$\Stab(x)<G$.
	\begin{proof}
		已知$\Stab(x)\subset G$. 对任意$g_1,g_2\in \Stab(x)$,
		$$e\cdot x=(g_2^{-1}g_2)\cdot x=g_2^{-1}\cdot(g_2\cdot x)=g_2^{-1}\cdot x=x.$$
		于是
		$$(g_1g_2^{-1})\cdot x=g_1\cdot (g_2^{-1}\cdot x)=g_1\cdot x=x.$$
		故$g_1g_2^{-1}\in \Stab(x)$,由子群的充要条件,$\Stab(x)<G$.
	\end{proof}
	\begin{definition}[齐性空间]
		若群$G$作用在集合$X$上,对任意$x,y\in X$,都有$g\in G$满足$g\cdot x=y$,则称这个作用是\textbf{可传递}的或\textbf{可迁}的,集合$X$称为\textbf{齐性空间}.
	\end{definition}
	\begin{definition}
		若对任意$x\in X$,$\Stab(x)=e$,则称$G$的作用是\textbf{自由的}.
	\end{definition}
	\begin{definition}
		若对任意$g\neq e$,存在$x\in X$使得$g\cdot x\neq x$,则称$G$的作用是\textbf{忠实的}或\textbf{有效的}.
	\end{definition}
	\begin{definition}
		若对任意$g\in G$,$x\in X$,都有$g\cdot x=x$,则称$G$的作用是\textbf{平凡的}.
	\end{definition}
	\begin{theorem}[轨道-稳定化子定理]
		设$G/\Stab(x)$是$G$关于$\Stab(x)$的左陪集空间,则存在双射$\varphi:\Orb(x)\to G/\Stab(x)$.特别地,当$G$是有限群时,有$|G|=|\Orb(x)||\Stab(x)|$.
	\end{theorem}
	\begin{proof}
		设$\varphi:\Orb(x)\to G/\Stab(x),\ g\cdot x\mapsto g\Stab(x)$. 设$g\cdot x=h\cdot x$,则$(h^{-1}g)\cdot x=x$,于是$h^{-1}g\in\Stab(x)$,$h\Stab(x)=g\Stab(x)$,所以$\varphi$是映射.
		
		对于任一$g\Stab(x)$,都有原像$g\cdot x$,故$\varphi$是满射.
		
		对任意$g_1\cdot x,g_2\cdot x\in\Orb(x)$,若$\varphi(g_1\cdot x)=\varphi(g_2\cdot x)$,则$g_1\Stab(x)=g_2\Stab(x)$,$g_2^{-1}g_1\in\Stab(x)$,于是$g_2^{-1}g_1\cdot x=x$,于是$g_1\cdot x=g_2\cdot x$,$\varphi$是单射.
		
		于是$\varphi$是双射,有$|\Orb(x)|=|G\Stab(x)|=\left[G:\Stab(x)\right]$.特别地,当$G$有限时,由Lagrange定理,有$|G|=|\Orb(x)||\Stab(x)|$.
	\end{proof}
\end{document}