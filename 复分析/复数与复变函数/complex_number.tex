\documentclass[12pt]{ctexart}
\usepackage{amsfonts,amssymb,amsmath,amsthm,geometry,enumerate}
\usepackage[colorlinks,linkcolor=blue,anchorcolor=blue,citecolor=green]{hyperref}
\usepackage[all]{xy}

%introduce theorem environment
\theoremstyle{definition}
\newtheorem{definition}{定义}
\newtheorem{theorem}{定理}
\newtheorem{lemma}{引理}
\newtheorem{corollary}{推论}
\newtheorem{property}{性质}
\newtheorem{proposition}{命题}
\newtheorem{example}{例}
\theoremstyle{plain}
\newtheorem*{solution}{解}
\newtheorem*{remark}{注}
\geometry{a4paper,scale=0.8}

\newcommand{\iu}{\mathrm{i}}
\newcommand{\eu}{\mathrm{e}}
\newcommand{\Arg}{\operatorname{Arg}}
\renewcommand{\Re}{\operatorname{Re}}
\renewcommand{\Im}{\operatorname{Im}}


%article info
\title{\vspace{-2em}\textbf{复数与复变函数}\vspace{-2em}}
\date{ }
\begin{document}
	\maketitle
	复数是实数域的扩充.
	\begin{definition}
		设$a,b\in\mathbb{R}$,定义虚数单位$\iu$满足$\iu^2=-1$.称$z=a+\iu b$为\textbf{复数},$a$,$b$分别称为复数$z$的\textbf{实部}和\textbf{虚部},记作$\Re z=a$,$\Im z=b$.
	\end{definition}
	复数的四则运算如下.
	$$(a+\iu b)\pm(c+\iu d)=(a\pm c)+\iu(b\pm d),$$
	$$(a+\iu b)(c+\iu d)=(ac-bd)+\iu(ad+bc),$$
	若$c+\iu d\neq 0$,则
	$$\frac{a+\iu b}{c+\iu d}=\frac{(a+\iu b)(c-\iu d)}{(c+\iu d)(c-\iu d)}=\frac{(ac+bd)+\iu (bc-ad)}{c^2+d^2}.$$
	\begin{definition}
		设$z=a+\iu b$,称$a-\iu b$为$z$的\textbf{共轭},记作$\overline{z}$.
	\end{definition}
	\begin{definition}
		设$z=a+\iu b$,称$\sqrt{a^2+b^2}$为$z$的\textbf{模},记作$|z|$.
	\end{definition}
	\begin{definition}
		对于平面上给定的直角坐标系,复数$z=a+\iu b$可以用坐标为$(a,b)$的点表示. 第一个坐标称为\textbf{实轴},第二个坐标称为\textbf{虚轴},所在的平面称为\textbf{复平面}.
	\end{definition}
	复数可以表示为一点,或者从原点指向该点的向量,可以用极坐标表示.
	\begin{definition}
		设复数$z=a+\iu b=r\left(\cos\varphi+\iu\sin\varphi\right)$,则$r=|z|$是复数$z$的\textbf{模}. 当$z\neq 0$时,称$\varphi$为$z$的\textbf{辐角}.
	\end{definition}
	\begin{remark}
		$z=0$时,辐角无意义.
	\end{remark}
	设$z_1=r_1\left(\cos\varphi_1+\iu\sin\varphi_1\right)$,$z_2=r_2\left(\cos\varphi_2+\iu\sin\varphi_2\right)$,则
	$$z_1z_2=r_1r_2\left(\cos(\varphi_1+\varphi_2)+\iu\sin(\varphi_1+\varphi_2)\right).$$
	非零复数$z$的辐角有无数个,将$z$的全体辐角记作$\Arg z$. 将辐角范围取为$0\leqslant\varphi<2\pi$,称为$z$的\textbf{主辐角},记作$\arg z$.
	
	下面介绍复球面. 取一个在原点$O$与复平面相切的球面,通过点$O$作一条与复平面垂直的直线交球面于点$N$,称$N$点为北极点. 取复平面上任一点$z$,用直线与$N$点相连,交球面于一点$P(z)$,于是建立了复平面到球面上的双射. 
	
	在复平面上,以原点$O$为圆心的圆$C$上的点,通过双射,在球面上也是一个圆$\varGamma$(纬线). 当$C$的半径增大时,$\varGamma$就越接近北极点$N$. 我们有一个假想点$\infty$与北极点$N$对应,称这个假想点为\textbf{无穷远点}. 添加无穷远点后的复平面称为\textbf{扩充复平面}. 扩充复平面与整个球面是一一对应的.
	
	先研究单值的复变函数. 
	\begin{definition}
		定义在区间$[a,b]$上的连续复值函数$\gamma(t)=x(t)+\iu y(t)(a\leqslant t\leqslant b)$称为\textbf{曲线},其中$x(t)$和$y(t)$均为$t$的连续实函数. $\gamma(a)$和$\gamma(b)$称为曲线$\gamma(t)$的\textbf{端点}. 若$\gamma(a)=\gamma(b)$,则称$\gamma(t)$为\textbf{闭曲线}. 曲线的\textbf{方向}定义为$t$增加的方向. 若$\gamma'(t)$存在且连续,则称$\gamma(t)$为\textbf{光滑曲线}.
	\end{definition}
	\begin{definition}
		若仅当$t_1=t_2$时$\gamma(t_1)=\gamma(t_2)$,则称$\gamma(t)$为\textbf{简单曲线}或\textbf{Jordan曲线}.
	\end{definition}
	\begin{definition}
		复平面上的一个非空连通开集$D$称为一个\textbf{域}.
	\end{definition}
	\begin{theorem}[Jordan]
		一条简单闭曲线$\gamma$将复平面分为两个域,其中一个是有界的,称为$\gamma$的内部,另一个是无界的,称为$\gamma$的外部,$\gamma$是这两个域的共同边界.
	\end{theorem}
	\begin{remark}
		定理是容易接受的,证明需要拓扑等知识,在此述而不证.
	\end{remark}
	\begin{definition}
		若$D\subset\mathbb{C}$是域,$f(z):D\to\mathbb{C}$称为\textbf{复变函数}.
	\end{definition}
\end{document}