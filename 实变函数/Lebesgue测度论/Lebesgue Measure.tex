\documentclass[lang=cn,12pt]{ctexart}

\usepackage{amsthm,amsfonts,amssymb,mathtools,geometry}
\usepackage[colorlinks,linkcolor=blue,anchorcolor=blue,citecolor=green]{hyperref}
\theoremstyle{definition}
\newtheorem{definition}{定义}
\newtheorem{theorem}{定理}
\newtheorem{lemma}{引理}
\newtheorem{corollary}{推论}
\newtheorem{proposition}{性质}
\newtheorem{example}{例}
\theoremstyle{plain}
\newtheorem*{solution}{解}
\newtheorem*{remark}{注}

\geometry{a4paper,scale=0.8}

\title{\Huge\textbf{Lebesgue测度论}}

\date{}

%table of content depth,目录显示的深度
\setcounter{tocdepth}{3} 


\begin{document}
	\maketitle
	\tableofcontents
	
	\newpage
\part{$\mathbb{R}^n$的拓扑}
\section{度量空间,$n$维Euclid空间}
把多个元素放在一起就构成了集合,但是集合间的元素是松散的.我们还需要定义
集合的元素之间的“关系”或“结构”,有了这层“关系”或“结构”,就构成了一个{\heiti 空间}.
\begin{definition}[度量空间]
	设$X$是一个集合,若对于$X$中任意两个元素$x,y$,都有唯一确定的实数$d(x,y)$与之对应,而且这一对应关系满足下列条件:
	\begin{enumerate}
		\item $d(x,y)\geqslant 0$,当且仅当$x=y$时等号成立;
		\item $d(x,y)\leqslant d(x,z)+d(y,z)$,对任意$z$都成立,
	\end{enumerate}
	则称$d(x,y)$是$x,y$之间的{\heiti 距离},称$(X,d)$为{\heiti 度量空间}或{\heiti 距离空间}.$X$中的元素称为{\heiti 点},条件(2)称为{\heiti 三点不等式}.
\end{definition}
\begin{remark}
	距离$d$有对称性,即$d(x,y)=d(y,x)$.事实上,在三点不等式中取$z=x$,则
	$$d(x,y)\leqslant d(x,x)+d(y,x)=d(y,x).$$
	由于$x,y$的次序是任意的,同理可证$d(y,x)\leqslant d(x,y)$,这就得到$d(x,y)=d(y,x)$.
\end{remark}
\begin{remark}
	如果$(X,d)$是度量空间,$Y$是$X$的一个非空子集,则$(Y,d)$也是一个度量空间,称为$X$的{\heiti 子空间}.
\end{remark}
\begin{definition}[$n$维Euclid空间]
	设$n$是一个正整数,将由$n$个实数$x_1,x_2,\cdots,x_n$按确定的次序排成的数组$(x_1,x_2,\cdots,x_n)$的全体组成的集合记为$\mathbb{R}^n$,对$\mathbb{R}^n$中任意两点
	$$x=(\xi_1,\xi_2,\cdots,\xi_n),\qquad y=(\eta_1,\eta_2,\cdots,\eta_n),$$
	规定距离
	$$d(x,y)=\sqrt{\sum_{i=1}^{n}(\xi_i-\eta_i)^2}.$$
	容易验证$d(x,y)$满足距离的条件.将$(\mathbb{R}^n,d)$称为{\heiti $n$维Euclid空间},其中$d$称为{\heiti Euclid距离}.
\end{definition}
\begin{remark}
	对$d(x,y)$满足距离的条件的验证:首先,条件(1)显然成立,对于条件(2),由Cauchy-Schwarz不等式
	$$\left(\sum_{i=1}^{n}a_ib_i\right)^2\leqslant\left(\sum_{i=1}^{n}a_i^2\right)\left(\sum_{i=1}^{n}b_i^2\right)$$
	得到
	\begin{align*}
		\sum_{i=1}^{n}(a_i+b_i)^2
		&=\sum_{i=1}^{n}a_i^2+2\sum_{i=1}^{n}a_ib_i+\sum_{i=1}^{n}b_i^2\\
		&\leqslant\sum_{i=1}^{n}a_i^2+2\sqrt{\sum_{i=1}^{n}a_i^2\cdot\sum_{i=1}^{n}b_i^2}+\sum_{i=1}^{n}b_i^2\\
		&=\left(\sqrt{\sum_{i=1}^{n}a_i^2}+\sqrt{\sum_{i=1}^{n}b_i^2}\right)^2.
	\end{align*}
	令$z=(\zeta_1,\zeta_2,\cdots,\zeta_n)$,$a_i=\zeta_i-\xi_i$,$b_i=\eta_i-\zeta_i$,则
	$$\eta_i-\xi_i=a_i+b_i.$$
	代入上面不等式即为三点不等式.
\end{remark}

此外,在$\mathbb{R}^n$中还可以用下面的方法定义其他的距离:
$$\rho(x,y)=\sum_{i=1}^{n}|\xi_i-\eta_i|.$$
容易验证$\rho$也满足条件(1)和条件(2).(称$\rho$为Manhattan距离)由此可知,在一个集合中引入距离的方法可以不限于一种.之后我们仅讨论$n$维Euclid空间和Euclid距离$d(x,y)$.

下面我们将考察$\mathbb{R}^n$中的极限、开集、闭集、紧集等一系列概念,它们的基础都是邻域,而邻域仅依靠距离即可作出.本章的结论对于一般的度量空间也是成立的,之后在泛函分析的学习中还会涉及.

我们从定义邻域的概念开始.
\begin{definition}[邻域]
	$\mathbb{R}^n$中所有和定点$P_0$的距离小于定数$\delta(>0)$的点的全体,即集合
	$$\{P|d(P,P_0)<\delta\}$$
	称为点$P_0$的$\delta${\heiti 邻域},记作$U(P_0,\delta)$.$P_0$称为邻域的{\heiti 中心},$\delta$称为邻域的{\heiti 半径}.在不需要特别指出是怎样的一个半径时,也干脆说是$P_0$的一个邻域,记作$U(P_0)$.显然,在$\mathbb{R}$,$\mathbb{R}^2$,$\mathbb{R}^3$中的$U(P_0,\delta)$就是以$P_0$为中心,$\delta$为半径的{\heiti 开区间},{\heiti 开圆}和{\heiti 开球}.
\end{definition}
容易证明邻域具有下面的基本性质:
\begin{proposition}
	\begin{enumerate}
		\item $P\in U(P)$;
		\item 对于$U_1(P)$和$U_2(P)$,存在$U_3(P)\subset U_1(P)\cap U_2(P)$;
		\item 对于$Q\in U(P)$,存在$U(Q)\subset U(P)$;
		\item 对于$P\neq Q$,存在$U(P)$和$U(Q)$,使$U(P)\cap U(Q)=\varnothing$.
	\end{enumerate}
\end{proposition}
\begin{definition}[极限]
	设$\{P_n\}$为$\mathbb{R}^m$中一点列,$P_0\in\mathbb{R}^m$,如果当$n\to\infty$时有$d(P_n,P_0)\to 0$,则称点列$\{P_n\}${\heiti 收敛于}$P_0$.记为$\lim\limits_{n\to\infty}P_n=P_0$或$P_n\to P_0(n\to\infty)$.
\end{definition}
\begin{remark}
	用邻域的术语来定义$\{P_n\}$收敛于$P_0$:对于$P_0$的任一邻域$U(P_0)$,存在某个自然数$N$,对任意$n>N$,都有$P_n\in U(P_0)$.
\end{remark}
\begin{definition}[点集的距离]
	两个非空点集$A,B$的距离定义为
	$$d(A,B)=\inf\{d(x,y)|x\in A,y\in B\}.$$
\end{definition}
\begin{remark}
	特别地,当其中一个点集为单点集时,我们就定义了点与点集的距离.
\end{remark}
\begin{definition}[点集的直径]
	一个非空点集$E$的{\heiti 直径}定义为
	$$\delta(E)=\sup\limits_{P,Q\in E}d(P,Q).$$
\end{definition}
\begin{definition}[有界点集]
	设$E$是$\mathbb{R}^n$中一点集,若$\delta(E)<\infty$,则称$E$为{\heiti 有界点集}.
\end{definition}
\begin{remark}
	空集也作为有界点集.
\end{remark}
\begin{remark}
	显然,$E$为有界点集的充要条件是存在常数$K>0$,使对于所有的$x=(x_1,x_2,\cdots,x_n)\in E$,都有$|x_i|\leqslant K(i=1,2,\cdots,n)$.这等价于:存在$K>0$,对所有$x\in E$,都有$d(x,\mathbf{0})\leqslant K$,这里$\mathbf{0}=(0,0,\cdots,0)$,称为$n$维Euclid空间的原点.
\end{remark}
\begin{definition}
	点集$\{(x_1,x_2,\cdots,x_n)|a_i<x_i<b_i,\ i=1,2,\cdots,n\}$称为一个{\heiti 开区间}($n$维),若将其中不等式一律换成$a_i\leqslant x_i\leqslant b_i,\ i=1,2,\cdots,n$,则称之为一个{\heiti 闭区间}.类似地,我们还可以定义左开右闭区间、左闭右开区间.当上述各种区间无区别的必要时,统称为{\heiti 区间},记作$I$.把$b_i-a_i(i=1,2,\cdots,n)$称为$I$的第$i$个“边长”,$\prod\limits_{i=1}^{n}(b_i-a_i)$称为$I$的“体积”,记为$|I|$.
\end{definition}
\section{内点,界点,聚点}
\begin{definition}[内点,外点,界点]
	如果存在$P_0$的某一邻域$U(P_0)$,使$U(P_0)\subset E$,则称$P_0$为$E$的{\heiti 内点}.如果$P_0$是$E^c$的内点,则称$P_0$是$E$的{\heiti 外点}.如果$P_0$既非$E$的内点又非$E$的外点,也就是说$P_0$的\underline{任一}邻域既有属于$E$的点,又有不属于$E$的点,则称$P_0$为$E$的{\heiti 界点}或{\heiti 边界点}.
\end{definition}
\begin{remark}
	上述三个概念中当然以内点最为重要,因为其他两个概念都是由此派生出来的.
\end{remark}
\begin{definition}[聚点]
	设$E$是$\mathbb{R}^n$中一点集,$P_0$为$\mathbb{R}^n$中一定点,如果$P_0$的任一邻域内都含有无穷多个属于$E$的点,则称$P_0$为$E$的一个{\heiti 聚点}.
\end{definition}
\begin{remark}
	由聚点定义可知有限集没有聚点.
\end{remark}
\begin{theorem}\label{judian}
	下面三个陈述是等价的:
	\begin{enumerate}
		\item $P_0$是$E$的聚点;
		\item 在$P_0$的任一邻域内,至少含有一个属于$E$而异于$P_0$的点;
		\item 存在$E$中互异的点所成点列$\{P_n\}$,使$P_n\to P_0(n\to \infty)$.
	\end{enumerate}
\end{theorem}

显然$E$的内点一定是$E$的聚点,但$E$的聚点不一定是$E$的内点,还可能是$E$的界点.其次,$E$的内点一定属于$E$,但$E$的聚点可以属于$E$也可以不属于$E$.
\begin{definition}[孤立点]
	设$E$是$\mathbb{R}^n$中一点集,$P_0$为$\mathbb{R}^n$中一定点,如果$P_0$属于$E$但不是$E$的聚点,则$P_0$称为$E$的{\heiti 孤立点}.
\end{definition}
\begin{remark}
	由定理\ref{judian}可知,$P_0$是$E$的孤立点的充要条件是:存在$P_0$的某邻域$U(P_0)$,使得$E\cap U(P_0)=\{P_0\}$.由此又知,{\heiti $E$的界点不是聚点就是孤立点}.
\end{remark}
综上所述,所有$\mathbb{R}^n$中的点,对$E$来说可以分为内点、界点、外点或分为聚点、孤立点、外点.但是,对一个具体的点集$E$来说,以上两种分类的三种点不一定都出现.界点和聚点可以属于$E$,也可以不属于$E$.

根据上面引入的概念,对于一个给定的点集$E$,我们可以考虑上述各种点的集合,其中最重要的是下面四种.
\begin{definition}
	设$E$是$\mathbb{R}^n$中的一个点集,有
	\begin{enumerate}
		\item $E$的全体内点所成的集合,称为$E$的{\heiti 开核},记作$\mathring{E}$.
		\item $E$的全体聚点所成的集合,称为$E$的{\heiti 导集},记作$E'$.
		\item $E$的全体界点所成的集合,称为$E$的{\heiti 边界},记作$\partial E$.
		\item $E\cup E'$称为$E$的{\heiti 闭包},记作$\overline{E}$.
	\end{enumerate}
\end{definition}
它们都可以用集合的语言描述如下.
\begin{enumerate}
	\item $\mathring{E}=\{x|\exists\ U(x)\subset E\}$;
	\item $E'=\{x|\forall\ U(x),\ U(x)\cap E\backslash\{x\}\neq\varnothing\}$;
	\item $\partial E=\{x|U(x)\cap E\neq\varnothing\text{且}U(x)\cap E^c\neq\varnothing\}$;
	\item $\overline{E}=\{x|\forall\ U(x),\ U(x)\cap E\neq\varnothing\}$.
\end{enumerate}
\begin{remark}
	由(4)可以看出,闭包就是包含$E$的内点、界点、聚点、孤立点(可能会有重合)而只不含$E$的外点的集合.
\end{remark}
\begin{remark}
	由(4)还可得到
	$$\overline{E}=E\cup\partial E=\mathring{E}\cup\partial E=E'\cup\{E\text{的孤立点}\}$$
	以及闭包与内核的对偶关系
	$$(\mathring{E})^c=\overline{E^c},\qquad (\overline{E})^c=\mathring{E^c}.$$
\end{remark}
\begin{theorem}\label{baohanyu}
	设$A\subset B$,则$A'\subset B',\ \mathring{A}\subset\mathring{B},\ \overline{A}\subset\overline{B}$.
\end{theorem}
\begin{theorem}\label{bing}
	$(A\cup B)'=A'\cup B'$.
\end{theorem}
\begin{proof}
	因为$A\subset A\cup B,\ B\subset A\cup B$,故从定理\ref{baohanyu}可知,$A'\subset (A\cup B)',\ B'\subset (A\cup B)'$,从而
	$$A'\cup B'\subset(A\cup B)'.$$
	
	另一方面,假设$P\in(A\cup B)'$,则必有$P\in A'\cup B'$.即$(A\cup B)'\subset A'\cup B'$.否则,若$P\notin A'\cup B'$,那么将有$P\notin A'$且$P\notin B'$.因而有$P$的某一邻域$U_1(P)$,在$U_1(P)$内除$P$外不含$A$的任何点,同时有$P$的某一邻域$U_2(P)$,在$U_2(P)$内除$P$外不含$B$的任何点,则由邻域的基本性质(2)知,存在$U_3(P)\subset U_1(P)\cap U_2(P)$,在$U_3(P)$中除点$P$外不含$A\cup B$中的任何点,这与$P\in(A\cup B)'$的假设矛盾.
\end{proof}
\begin{theorem}[Bolzano-Weierstrass定理]
	有界无限点集至少有一个聚点.
\end{theorem}
证明方法同数学分析中$\mathbb{R}$和$\mathbb{R}^2$时的证明,在此不再赘述.
\begin{theorem}
	设$E\neq\varnothing,\ E\neq\mathbb{R}^n$,则$E$至少有一界点(即$\partial E\neq\varnothing$).
\end{theorem}
\begin{proof}
	设$P_0\in E,\ P_1\in E^c$,定义$P_t=(1-t)P_0+tP_1,\ t\in\left[0,1\right]$.设$t_0=\sup\{t|P_t\in E\}$.下证$P_{t_0}\in\partial E$.
	
	若$P_{t_0}\in E$,则$t_0\neq 1$.对任意$t\in\left(t_0,1\right]$,$P_t\notin E$.对任意$\delta>0$,存在$t\in\left(t_0,1\right]$,使得$P_t\in E^c\cap U(P_{t_0},\delta)$,于是$P_{t_0}\in\partial E$.
	
	若$P_{t_0}\in E^c$,即$P_{t_0}\notin E$,则$t_0\neq 0$.存在$t_n\in\left[0,t_0\right),\ t_n\to t_0$,且$P_{t_n}\in E$.对任意$\delta>0$,存在$P_{t_n}\in E\cap U(P_{t_0},\delta)$.因$P_{t_0}\in U(P_{t_0},\delta)\cap E^c$,于是也有$P_{t_0}\in\partial E$.
\end{proof}
\section{开集、闭集、紧集、完备集}
\begin{definition}[开集]
	设$E\subset\mathbb{R}^n$,如果$E$的每一点都是$E$的内点,则称$E$为{\heiti 开集}.
\end{definition}
例如整个空间$\mathbb{R}^n$是开集,空集是开集,在$\mathbb{R}$中任意开区间$(a,b)$是开集,在$\mathbb{R}^2$中$E=\{(x,y)|x^2+y^2<1\}$是开集(但它在$\mathbb{R}^3$中就不是开集了,想想看,这是为什么?).
\begin{definition}[闭集]
	设$E\subset\mathbb{R}^n$,如果$E$的每一个聚点都属于$E$,则称$E$为{\heiti 闭集}.
\end{definition}
例如整个空间$\mathbb{R}^n$是闭集,空集是闭集,在$\mathbb{R}$中任意闭区间$\left[a,b\right]$是闭集,任意的有限集合都是闭集.

开集、闭集利用开核、闭包等术语来说,就是

$E$为开集$\iff E\subset\mathring{E}$,即$E=\mathring{E}$.

$E$为闭集$\iff E'\subset E$(或$\partial E\subset E$).\\
今后开集常用字母$G$表示,闭集常用字母$F$表示.
\begin{theorem}
	对任何$E\subset\mathbb{R}^n$,$\mathring{E}$是开集,$E'$和$\overline{E}$都是闭集.(这也是$\mathring{E}$称为开核,$\overline{E}$称为闭包的缘由)
\end{theorem}
\begin{proof}
	首先证明$\mathring{E}$是开集.设$P\in\mathring{E}$,由$E$的定义知,存在邻域$U(P)\subset E$,对于任意的$Q\in U(P)$,由邻域的基本性质(3)知,存在$U(Q)$使得$U(Q)\subset U(P)\subset E$,即$Q$是$E$的内点,故$U(P)\subset\mathring{E}$,所以$P$是$\mathring{E}$的内点,故$\mathring{E}$是开集.
	
	其次证明$E'$是闭集.设$P_0\in(E')'$,则由定理\ref{judian}(2)可知,在$P_0$的任一邻域$U(P_0)$内,至少含有一个属于$E'$而异于$P_0$的点$P_1$.因为$P_1\in E'$,于是又有属于$E$的$P_2\in U(P_0)$,而且还可以要求$P_2\neq P_0$,再次利用该定理,即得$P_0\in E'$.所以$E'$是闭集.
	
	最后证明$\overline{E}$是闭集.由闭包的定义及定理\ref{bing},有
	$$(\overline{E})'=E'\cup(E')'\subset E'\cup E'=E'\subset\overline{E}.$$
	从而$\overline{E}$是闭集.
\end{proof}
\begin{theorem}[开集与闭集的对偶性]
	设$E$是开集,则$E^c$是闭集;设$E$是闭集,则$E^c$是开集.
\end{theorem}
\begin{proof}
	只需证明第一部分.
	
	证法一:设$E$是开集,而$P_0$是$E^c$的任一聚点,那么,$P_0$的任一邻域都有不属于$E$的点.这样$P_0$就不可能是$E$的内点,从而不属于$E$(因为$E$是开集),也就是$P_0\in E^c$.由闭集的定义得$E^c$为闭集.
	
	证法二:设$E$是开集,则$E=\mathring{E}$,由闭包、开核对偶关系,得$\overline{E^c}=(\mathring{E})^c=E^c$,可见$E^c$是闭集.
\end{proof}
由于开集和并集的这种对偶关系,在许多情形下,我们将闭集看作是开集派生出来的概念.也就是说,如果定义了开集,闭集也就随之确定.
\begin{theorem}\label{kai}
	任意多个开集的并仍是开集,有限多个开集的交仍是开集.
\end{theorem}
\begin{proof}
	第一部分显然.对第二部分,有限多个开集的交仍是开集总能递归为两个开集的交仍是开集.故只需证明两个开集的交的情况.
	
	设$G_1,G_2$为开集,任取$P_0\in G_1\cap G_2$.因$P_0\in G_i(i=1,2)$,故存在$U_i(P_0)\subset G_i(i=1,2)$.由邻域的基本性质(2),存在$U_3(P_0)\subset U_1(P_0)\cap U_2(P_0)$,从而$U_3(P_0)\subset G_1\cap G_2$,可见$P_0$是$G_1\cap G_2$的内点.
\end{proof}
\begin{remark}
	任意多个开集的交不一定是开集.例如
	$$G_n=\left(-1-\frac{1}{n},1+\frac{1}{n}\right),\ n=1,2,\cdots,$$
	每个$G_n$是开集,但$\bigcap\limits_{n=1}^{\infty}G_n=\left[-1,1\right]$不是开集.
\end{remark}
\begin{theorem}
	任意多个闭集的交仍为闭集,有限多个闭集的并仍为闭集.
\end{theorem}
\begin{proof}
	利用De\ Morgan公式.
	
	设$\Lambda=\{1,2,\cdots\}$,$F_i, i\in\Lambda$(或$i=1,2,\cdots,m$)是闭集,则由开集和并集的对偶关系知$F_i^c$是开集,从而由定理\ref{kai}知,$\bigcup\limits_{i\in\Lambda}F_i^c$(或$\bigcap\limits_{i=1}^{m}F_i^c$)也是开集,由De\ Morgan公式有
	$$\bigcap\limits_{i\in\Lambda}F_i=(\bigcup\limits_{i\in\Lambda}F_i^c)^c\quad\text{或}\bigcup\limits_{i=1}^{m}F_i=(\bigcap\limits_{i=1}^{m}F_i^c)^c,$$
	故再由开集和并集的对偶关系可知$\bigcap\limits_{i\in\Lambda}F_i$或$\bigcup\limits_{i=1}^{m}F_i$是闭集.
\end{proof}
\begin{remark}
	任意多个闭集的并不一定是闭集.例如
	$$F_n=\left[\frac{1}{n},1-\frac{1}{n}\right],\ n=3,4,\cdots,$$
	每个$F_n$是闭集,但$\bigcup\limits_{n=3}^{\infty}F_n=(0,1)$不是闭集.
\end{remark}
\begin{proposition}
	设$F_1,F_2$是$\mathbb{R}$中两个互不相交的闭集,则存在两个互不相交的开集$G_1,G_2$,使$G_1\supset F_1,\ G_2\supset F_2$.
\end{proposition}
在数学分析中我们已经学习了以下形式的Heine-Borel有限覆盖定理:设$I$是$\mathbb{R}^n$中的闭区间,$\mathcal{M}$是一族开区间,它覆盖了$I$,则在$\mathcal{M}$中一定存在有限多个开区间,它们同样覆盖了$I$.

我们下面要把上述定理推广成更一般的形式.
\begin{theorem}[Heine-Borel定理]
	设$F$是一个有界闭集,$\mathcal{M}$是一族开集,它覆盖了$F$,则在$\mathcal{M}$中一定存在有限多个开集,它们同样覆盖了$F$.
\end{theorem}
\begin{proof}
	因$F$是有界闭集,所以在$\mathbb{R}^n$中存在闭区间$I$包含$F$.记$\mathcal{D}$为由$\mathcal{M}$中的全体开集与开集$F^c$一起组成的新开集族,则$\mathcal{D}$覆盖了$\mathbb{R}^n$,因此也覆盖了$I$.对于$I$中任一点$P$,存在$\mathcal{D}$中开集$U_P$,使得$P\in U_P$,因而存在开区间$I_P\subset U_P$,并且$P\in I_P$,所以开区间族$\{I_P|P\in I\}$覆盖了$I$.由数学分析中的有限覆盖定理,在这族开区间中存在有限个开区间,设为$I_{P_1},I_{P_2},\cdots,I_{P_m}$仍然覆盖了$I$,则由$F\subset I$,及$I_{P_i}\subset U_{P_i}(i=1,2,\cdots,m)$,得$F\subset\bigcup\limits_{i=1}^{m}U_{P_i}$.如果开集$F^c$不在这$m$个开集中,则$U_{P_1},U_{P_2},\cdots,U_{P_m}$覆盖了$F$,定理得证;否则从这$m$个开集中去掉$F^c$,因为$F^c$与$F$不相交,所以剩下的$m-1$个开集仍然覆盖了$F$.
\end{proof}
\begin{definition}[紧集]
	设$M$是度量空间$X$中一集合,$\mathcal{M}$是$X$中任一族覆盖了$M$的开集,如果必可从$\mathcal{M}$中选出有限个开集仍然覆盖$M$,则称$M$为$X$中的{\heiti 紧集}.
\end{definition}
由Heine-Borel定理知$\mathbb{R}^n$中的有界闭集必为紧集,是否$\mathbb{R}^n$中的紧集都是有界闭集呢?答案是肯定的,我们有以下定理.
\begin{theorem}
	设$M$是$\mathbb{R}^n$中的紧集,则$M$是$\mathbb{R}^n$中的有界闭集.
\end{theorem}
\begin{proof}
	设点$Q\in M^c$,对于$M$中的任意一点$P$,由于$P\neq Q$,由邻域性质,存在$\delta_P>0$,使得
	$$U(P,\delta_P)\cap U(Q,\delta_P)=\varnothing.$$
	
	显然开集族$\{U(P,\delta_P)|P\in M\}$覆盖了$M$,由于$M$是紧集,因此存在有限个邻域$U(P_i,\delta_{P_i})(i=1,2,\cdots,m)$,使得
	\begin{equation}\label{fugai}
		M\subset\bigcup\limits_{i=1}^{m}U(P_i,\delta_{P_i})
	\end{equation}
	由此立即可知$M$是有界集.又令
	$$\delta=\min\{\delta_{P_1},\delta_{P_2},\cdots,\delta_{P_m}\},$$
	则$\delta>0$,并且$U(Q,\delta)\cap U(P_i,\delta_i)=\varnothing(i=1,2,\cdots,m)$,由\ref{fugai}式得$U(Q,\delta)\cap M=\varnothing$,因此$Q$不是$M$的聚点,所以$M'\cap M^c=\varnothing$,这说明$M'\subset M$,即$M$是闭集.
\end{proof}
\begin{remark}
	上述定理说明了$\mathbb{R}^n$中紧集和有界闭集是一致的.但是在一般的度量空间中,紧集一定是有界闭集(与上述定理证明相类似),但有界闭集不一定是紧集.
\end{remark}
\begin{definition}[自密集]
	设$E\subset\mathbb{R}^n$,如果$E\subset E'$,就称$E$是{\heiti 自密集}.
\end{definition}
\begin{remark}
	换句话说,当集合中每点都是这个集的聚点时,这个集是自密集.另一个说法是没有孤立点的集是自密集.
\end{remark}
例如,空集是自密集,$\mathbb{R}$中有理数全体组成的集是自密集.
\begin{definition}[完备集]
	设$E\subset\mathbb{R}^n$,如果$E=E'$,就称$E$是{\heiti 完备集}或{\heiti 完全集}.
\end{definition}
可以看出,完备集就是自密闭集,即没有孤立点的闭集.例如,空集是完备集,$\mathbb{R}$中任一闭区间$\left[a,b\right]$及全直线都是完备集.

下面我们简单介绍直线上(即$\mathbb{R}$中)开集与闭集的构造.

在直线上,开区间是开集,但是开集不一定是开区间,它往往是一系列开区间的并集.为研究直线上开集的结构,我们先引入构成区间的概念.
\begin{definition}[构成区间]
	设$G$是直线上的开集,如果开区间$(\alpha,\beta)\subset G$,而且端点$\alpha,\beta\notin G$,那么称$(\alpha,\beta)$为$G$的{\heiti 构成区间}.
\end{definition}
\begin{theorem}[开集构造定理]
	直线上任一个{\heiti 非空开集}可以表示成{\heiti 至多可数个}互{\heiti 不相交}的构成区间的并.
\end{theorem}
\begin{proof}
	
\end{proof}
既然闭集的余集是开集,那么从开集的构造可以引入余区间的概念.
\begin{definition}[余区间]
	设$A$是直线上的闭集,称$A$的余集$A^c$的构成区间为$A$的{\heiti 余区间}或{\heiti 邻接区间}.
\end{definition}
我们得到闭集的构造如下:
\begin{theorem}\label{wa}
	直线上的闭集$F$或者是全直线,或者是从直线上挖掉至多可数个互不相交的开区间所得到的集.
\end{theorem}
由孤立点的定义很容易知道,直线上点集$A$的孤立点必是包含在$A$的余集中的某两个开区间的公共端点.因此,闭集的孤立点一定是它的两个余区间的公共端点.完备集是没有孤立点的闭集,所以,{\heiti 完备集就是没有相邻接的余区间的闭集}.
\section{Cantor三分集}
下面我们将讨论Cantor三分疏朗集,这是实分析中的一个重要概念,也常作为反例出现.为此我们先给出疏朗集和稠密集的定义.
\begin{definition}[稠密和疏朗]
	设$E\subset\mathbb{R}^n$,
	\begin{enumerate}
		\item 设$F\subset\mathbb{R}^n$,若对任意$x\in F$和任意邻域$U(x)$,$U(x)\cap E\neq\varnothing$,则称$E$在$F$中{\heiti 稠密}.
		\item 若对任意$x\in\mathbb{R}^n$和任意邻域$U(x)$,存在$U(y)\subset U(x)\cap E^c$,则称$E$是{\heiti 疏朗集}或{\heiti 无处稠密集}.
	\end{enumerate}
\end{definition}
例如有限点集或收敛可数列都是疏朗集,有理点集$\mathbb{Q}^n$在$\mathbb{R}^n$中稠密.
\begin{definition}[Cantor三分集]
	将$E_0=\left[0,1\right]$三等分,去掉中间的开区间,剩下两个闭区间,记这两个闭区间的并为$E_1$,再把剩下的两个闭区间分别三等分,分别去掉中间的开区间,剩下$2^2$个闭区间,记这些闭区间的并为$E_2$.以此类推,当进行到第$n$次时,一共去掉$2^{n-1}$个开区间,剩下$2^n$个长度为$3^{-n}$的相互隔离的闭区间,记这些闭区间的并为$E_n$.如此继续下去,就从$\left[0,1\right]$中去掉了可数多个互不相交且没有公共端点的开区间.由定理\ref{wa},剩下的必是一个闭集,称它为{\heiti Cantor三分集},记为$P$.
\end{definition}
下面列举了Cantor集$P$的一些性质.
\begin{proposition}
	\begin{enumerate}
		\item $P$是完备集.
		\item $P$没有内点.
		\item $P$是零测集.
		\item $P$的基数为$\aleph$.
	\end{enumerate}
\end{proposition}

综上所述,我们将Cantor三分集的特点归纳为:它是一个测度为零且基数为$\aleph$的疏朗完备集.
\newpage

\part{Lebesgue测度论}
虽然我们在小学时期就学习了长度、面积等相关概念,但事实上我们从未严格定义过长度、面积和体积.下面我们尝试定义这些概念.

我们可以把“长度”看作是1维实空间$\mathbb{R}$(即实数轴)的一个子集类$X$($\mathbb{R}$的每个子集不一定都有“长度”)到实数域的一个映射$m$.我们首先规定
$$m(\left[a,b\right])\coloneqq b-a.$$
其中$a\leqslant b$.这表明任何闭区间$\left[a,b\right]$的长度为$b-a$,并蕴含了实数轴上任意一点的长度为零.然后我们可以列出几条公理(姑且称它们为公理):设有实数轴上的一些点集构成的集类$\mathcal{M}$,对于每个$E\in\mathcal{M}$,都对应一个实数$m$,有以下性质:
\begin{enumerate}
	\item 非负性:$m(E)\geqslant 0$;
	\item 有限可加性:若$E_1,E_2,\cdots,E_n$两两不相交,则$m\left(\bigcup\limits_{i=1}^{n}E_i\right)=\sum\limits_{i=1}^{n}m(E_i)$;
	\item 正则性:$m(\left[a,b\right])=b-a$.
\end{enumerate}
若集合$A$可通过平面上的正交变换(平面的正交变换即为平移、旋转、反射以及它们的乘积)变成了$B$,则称$A$和$B${\heiti 全等}或{\heiti 合同}.我们规定,$m(A)=m(B)$当且仅当$A\cong B$.

由于任意一点的长度都是零,由可加性公理可知开区间$(a,b)$的长度也是$b-a$,半开半闭区间的长度亦然.为了让整个实数轴也有长度,我们规定$m$可以取到$+\infty$.

类似地,我们可以把面积看作是2维实空间$\mathbb{R}^2$(即实平面)的一个子集类$X$到实数域$\mathbb{R}$的一个映射$m$.我们首先规定一个邻边长分别为$a$和$b$的矩形$A$的面积为$a\cdot b,\ a,b\geqslant 0$.这蕴含了线段的面积为零.以上的三条定理可以“原封不动”地来刻画面积.依次下去,还可以进一步把长度、面积的概念推广到体积以及$n$维Euclid空间$\mathbb{R}^n$中.事实上物理中的{\heiti 功}(work),{\heiti 位移}(displacement),{\heiti 冲量}(impulse)都满足以上三条公理.此外,我们从长度公理中仅能求出有限个区间的并的长度,对于无限个点集的并,长度公理就无能为力了.因此,我们可以考虑用一个统一的概念来描述长度、面积等等,并设法扩充其测量的范围,这就引出了{\heiti 测度}(measure)的概念.

显然,一下子推广到不可数无穷多个区间的长度是不现实的,我们退而求其次,考虑可数个区间的"长度",就有Lebesgue提出的测度公理:

实数轴上的一些点集构成的集类$\mathcal{M}$,对于每个$E\in\mathcal{M}$,都对应一个实数$m$,满足:
\begin{enumerate}
	\item 非负性:$m(E)\geqslant 0$;
	\item 可数可加性:若$E_1,E_2,\cdots,E_n,\cdots$两两不相交,则$m\left(\bigcup\limits_{i=1}^{\infty}E_i\right)=\sum\limits_{i=1}^{\infty}m(E_i)$;
	\item 正则性:$m(\left[a,b\right])=b-a$.
\end{enumerate}
我们提出以下问题:
满足Lebesgue测度公理且在集类$\mathcal{M}$上定义的实函数$m(E)$是否存在?$\mathcal{M}$由哪些集合构成?是否每个集合都有测度?这就是本章要讨论的内容.
\section{Lebesgue外测度}\label{exterior measure}
\begin{definition}[Lebesgue外测度]
	设$E\subset\mathbb{R}^n$,定义$E$的Lebesgue外测度为
	$$m^{*}E=\inf\{\sum_{i=1}^{\infty}|I_n|,\ \sum_{i=1}^{\infty}I_n\supset E\}$$
	其中$I_n$是开域.
\end{definition}
外测度具有以下三条基本性质:
\begin{theorem}
	\begin{enumerate}
		\item 非负性:$m^*E\leqslant 0$,规定$m^*\varnothing=0$;
		\item 单调性:设$A\subset B$,则$m^*A\leqslant m^*B$;
		\item 次可数可加性:$m^*\left(\bigcup\limits_{i=1}^{\infty}E_i\right)\leqslant\sum\limits_{i=1}^{\infty}m^*(E_i)$.
	\end{enumerate}
\end{theorem}
\begin{proof}
	(1)显然成立.
	
	(2)的证明.设$A\subset B$,则任一列覆盖$B$的开域$\{I_n\}$一定也是覆盖$A$的,因而
	$$m^*A\leqslant\sum_{i=1}^{\infty}|I_n|,$$
	对所有能覆盖$B$的开域列取下确界即得
	$$m^*A\leqslant\inf\sum_{i=1}^{\infty}|I_i|=m^*B.$$
	
	(3)的证明.任给$\varepsilon>0$,由Lebesgue外测度定义,对每个$n$都应有一列开区间$I_{n,1},I_{n,2},\cdots,I_{n,m},\cdots$,使$E_n\subset\bigcup\limits_{m=1}^{\infty}I_{n,m}$且
	$$\sum_{m=1}^{\infty}|I_{n,m}|\leqslant m^*E_n+\frac{\varepsilon}{2^n.}$$
	从而
	$$\bigcup_{n=1}^{\infty}E_n\subset\bigcup_{n,m=1}^{\infty}I_{n,m},$$
	且
	$$\sum_{n,m=1}^{\infty}|I_{n,m}|=\sum_{n=1}^{\infty}\sum_{m=1}^{\infty}|I_{n,m}|\leqslant\sum_{n=1}^{\infty}\left(m^*E_n+\frac{\varepsilon}{2^n}\right)=\sum_{n=1}^{\infty}m^*E_n+\varepsilon.$$
	可见
	$$m^*\left(\bigcup_{n=1}^{\infty}E_n\right)\leqslant\sum_{n,m=1}^{\infty}|I_{n,m}|\leqslant\sum_{n=1}^{\infty}m^*E_n+\varepsilon.$$
	由于$\varepsilon$的任意性,得
	$$m^*\left(\bigcup\limits_{i=1}^{\infty}E_i\right)\leqslant\sum\limits_{i=1}^{\infty}m^*(E_i).$$
	
\end{proof}
\begin{theorem}
	设区间$I$,则$m^*I=|I|$.
\end{theorem}
\begin{proof}
	\begin{enumerate}
		\item 设$I$是闭区间.对于任给的$\varepsilon>0$,存在开区间$I'$,使得$I\subset I'$且
		$$|I'|<|I|+\varepsilon.$$
		由外测度定义,$m^*I<|I|+\varepsilon$,由$\varepsilon$的任意性,有
		$$m^*I\leqslant|I|.$$
		
		现在来证明$m^*I\geqslant |I|$.对于任给$\varepsilon>0$,存在一列开区间$\{I_i\}$,使$I\subset\bigcup\limits_{i=1}^{\infty}I_i$,且$\sum\limits_{i=1}^{\infty}|I_i|<m^*I+\varepsilon$.
		
		由Heine-Borel有限覆盖定理,在$\{I_i\}$中存在有限多个区间,不妨设为$I_1,I_2,\cdots,I_n$,使得$I\subset\bigcup\limits_{i=1}^{n}I_i$.
		
		因为$I=\bigcup\limits_{i=1}^{n}(I\cap I_i)$,于此$I\cap I_i$为区间,由初等几何易知
		$$|I|\leqslant\sum_{i=1}^{n}|I\cap I_i|,$$
		故
		$$|I|\leqslant\sum_{i=1}^{n}|I\cap I_i|\leqslant\sum_{i=1}^{n}|I_i|<m^*I+\varepsilon.$$
		由于$\varepsilon$的任意性,即得
		$$|I|\leqslant m^*I.$$
		于是$m^*I=|I|$.
		\item 设$I$为任意区间.作闭区间$I_1,I_2$使$I_1\subset I\subset I_2$且
		$$|I_2|-\varepsilon<|I|<|I_1|+\varepsilon$$
		($I_2$可取为$I$的闭包$\overline{I}$),则
		$$|I|-\varepsilon\leqslant|I_1|=m^*I_1\leqslant m^*I\leqslant m^*I_2=|I_2|<|I|+\varepsilon.$$
		由于$\varepsilon>0$的任意性,得
		$$m^*I=|I|.$$
	\end{enumerate}
	
\end{proof}
\section{Lebesgue可测集}
在\ref{exterior measure}节中,我们定义了Lebesgue外测度,它的一个优点是任何集合都有外测度,但是外测度只具有次可数可加性,不具有可数可加性.这意味着,如果把外测度当作测度看,使得任何集合都有测度,这是办不到的.这启发我们思考能否对外测度$m^*$的定义域进行限制,即设法在$\mathbb{R}^n$中找出一个集类$\mathcal{M}$,使得$\mathcal{M}$中的集合满足Lebesgue测度公理.

首先,$\mathcal{M}$对某些集合运算应该是封闭的.例如对$\mathcal{M}$中的集合作可数并(当然对有限并也成立,只需在后面添加可数个空集即可)、作交或作差运算后仍在$\mathcal{M}$中,而且对$\mathcal{M}$中一列互不相交的集合$\{E_i\}$,应当满足可数可加性:
$$m^*\left(\bigcup_{i=1}^{\infty}E_i\right)=\sum_{i=1}^{\infty}m^*E_i.$$

其次,由Lebesgue的测度公理(3),自然应该要求$\mathcal{M}$包含$\mathbb{R}^n$中的所有有限开域.又由于$\mathbb{R}^n$是一列有限开区间的可列并,所以$\mathcal{M}$也应该包括$\mathbb{R}^n$.

想要从$\mathbb{R}^n$中挑出集类$\mathcal{M}$,我们只需附加一个判断$\mathbb{R}$中的集合$E$属于$\mathcal{M}$的条件即可.我们试从可数可加性条件来思考.

设$E\subset \mathbb{R}^n$.如果$E\in\mathcal{M}$,由于$\mathbb{R}^n$中任何开区间$I$都属于$\mathcal{M}$,由$\mathcal{M}$的运算封闭性,则$I\cap E,\ I\cap E^c$都应该属于$\mathcal{M}$.但由$(I\cap E)\cap(I\cap E^c)=\varnothing$,$I=(I\cap E)\cup(I\cap E^c)$,所以由可数可加性,应该有
\begin{equation}\label{cara}
	m^*I=m^*(I\cap E)+m^*(I\cap E^c).
\end{equation}
反之,如果存在某个开区间$I$,使\ref{cara}式不成立,则$E$自然不应该属于$\mathcal{M}$.由上可见,对于$\mathbb{R}^n$中点集$E$是否属于$\mathcal{M}$,我们可以用\ref{cara}是否对$\mathbb{R}^n$中任何开区间成立来判断.事实上,我们有下列结论.
\begin{lemma}
	设$E\subset\mathbb{R}^n$,则\ref{cara}式对$\mathbb{R}^n$中任何开区间$I$都成立的充要条件是对$\mathbb{R}^n$中的任何点集$T$都有
	\begin{equation}\label{carath}
		m^*T=m^*(T\cap E)+m^*(T\cap E^c).
	\end{equation}
\end{lemma}
\begin{proof}
	充分性显然成立.下证必要性.设$T$为$\mathbb{R}^n$中的任意集合,则由外测度定义,对于任何$\varepsilon>0$,有一列开区间$\{I_n\}$使得
	$$T\subset\bigcup_{i=1}^{\infty}I_i,\quad\text{且}\sum_{i=1}^{\infty}|I_i|\leqslant m^*T+\varepsilon.$$
	但由于
	$$T\cap E\subset\bigcup_{i=1}^{\infty}(I_i\cap E),\quad T\cap E^c\subset \bigcup_{i=1}^{\infty}(I_i\cap E^c),$$
	故
	$$m^*(T\cap E)\leqslant\sum_{i=1}^{\infty}m^*(I_i\cap E),$$
	$$m^*(T\cap E^c)\leqslant\sum_{i=1}^{\infty}m^*(I_i\cap E^c).$$
	从而
	\begin{align*}
		m^*(T\cap E)+m^*(T\cap E^c)
		&\leqslant\sum_{i=1}^{\infty}m^*(I_i\cap E)+\sum_{i=1}^{\infty}m^*(I_i\cap E^c)\\
		&=\sum_{i=1}^{\infty}\left[m^*(I_i\cap E)+m^*(I_i\cap E^c)\right]\\
		&=\sum_{i=1}^{\infty}|I_i|\leqslant m^*T+\varepsilon.
	\end{align*}
	由于$\varepsilon$的任意性,即得
	$$m^*(T\cap E)+m^*(T\cap E^c)\leqslant m^*T.$$
	另一方面,由Lebesgue外测度的次可加性,有
	$$m^*(T\cap E)+m^*(T\cap E^c)\geqslant m^*T.$$
	故
	$$m^*(T\cap E)+m^*(T\cap E^c)=m^*T.$$
	
\end{proof}
\begin{remark}
	这个引理是由Carath\'eodory给出的,通常我们称\ref{carath}式为Carath\'eodory条件.
\end{remark}
现在,我们终于可以给出Lebesgue可测的定义.
\begin{definition}[Lebesgue可测]
	设$E$是$\mathbb{R}^n$中的点集,如果对任一点集$T$都有
	$$m^*T=m^*(T\cap E)+m^*(T\cap E^c),$$
	则称$E$是{\heiti Lebesgue可测的},也称为$L${\heiti 可测}.这时$E$的Lebesgue外测度$m^*E$即称为$E$的{\heiti Lebesgue测度},记为$mE$.\ Lebesgue可测集全体记为$\mathcal{M}$.
\end{definition}
由上述定义,我们可以得出Lebesgue测度的若干性质.
\begin{theorem}\label{submea}
	集合$E$可测的充要条件是对于任意$A\subset E,\ B\subset E^c$,总有
	$$m^*(A\cup B)=m^*A+m^*B.$$
\end{theorem}
\begin{proof}
	必要性\qquad 取$T=A\cup B$,则$T\cap E=A,\ T\cap E^c=B$,所以
	$$m^*(A\cup B)=m^*T=m^*(T\cap E)+m^*(T\cap E^c)=m^*A+m^*B.$$
	
	充分性\qquad 对于任意$T$,令$A=T\cap E$,$B=T\cap E^c$,则$A\subset E,\ B\subset E^c$且$A\cup B=T$,因此
	$$m^*T=m^*(A\cup B)=m^*A+m^*B=m^*(T\cap E)+m^*(T\cap E^c).$$
	
\end{proof}
\begin{theorem}[补集的可测性]
	$S$可测的充要条件是$S^c$可测.
\end{theorem}
\begin{proof}
	事实上,对于任意的$T$,
	$$m^*T=m^*(T\cap S)+m^*(T\cap S^c)=m^*(T\cap (S^c)^c)+m^*(T\cap S^c).$$
	
\end{proof}
\begin{theorem}[并集的可测性]
	设$S_1,\ S_2$都可测,则$S_1\cup S_2$也可测,并且当$S_1\cap S_2=\varnothing$时,对于任意集合$T$总有
	$$m^*\left[T\cap(S_1\cup S_2)\right]=m^*(T\cap S_1)+m^*(T\cap S_2).$$
\end{theorem}
\begin{proof}
	首先证明$S_1\cup S_2$的可测性,即证对于任意$T$总有
	$$m^*T=m^*(T\cap(S_1\cup S_2))+m^*(T\cap(S_1\cup S_2)^c).$$
	事实上,有
	\begin{align*}
		m^*T&=
		m^*(T\cap S_1)+m^*(T\cap S_1^c)\quad&\text{($S_1$可测)}\\
		&=m^*(T\cap S_1)+m^*\left[(T\cap S_1^c)\cap S_2\right]+m^*\left[(T\cap S_1^c)\cap S_2^c\right]\quad&\text{($S_2$可测)}\\			
	\end{align*}
	由De\ Morgan公式,
	$$m^*\left[(T\cap S_1^c)\cap S_2^c\right]=m^*\left[T\cap (S_1\cup S_2)^c\right]$$
	又因$S_1$可测,且$T\cap S_1\subset S_1,\ (T\cap S_1^c)\cap S_2\subset S_1^c$,故由定理\ref{submea},有
	$$m^*(T\cap S_1)+m^*\left[(T\cap S_1^c)\cap S_2\right]=m^*\left[T\cap(S_1\cup (S_1^c\cap S_2))\right]=m^*\left[T\cap(S_1\cup S_2)\right],$$
	整理,即得
	$$m^*T=m^*(T\cap(S_1\cup S_2))+m^*(T\cap(S_1\cup S_2)^c).$$
	
	其次当$S_1\cap S_2=\varnothing$时,因$S_1$可测,且$T\cap S_1\subset S_1,\ T\cap S_2\subset S_1^c$,故由定理\ref{submea},有
	$$m^*\left[T\cap(S_1\cup S_2)\right]=m^*(T\cap S_1)+m^*(T\cap S_2).$$
	
\end{proof}
\begin{corollary}[有限并的可测性]
	设$S_i\ (i=1,2,\cdots,n)$都可测,则$\bigcup\limits_{i=1}^nS_i$也可测,并且当$S_i\cap S_j=\varnothing\ (i\neq j)$时,对于任何集合$T$总有
	$$m^*\left(T\cap\left(\bigcup_{i=1}^{n}S_i\right)\right)=\sum_{i=1}^{n}m^*(T\cap S_i).$$
\end{corollary}
\begin{theorem}[交集的可测性]
	设$S_1,\ S_2$都可测,则$S_1\cap S_2$也可测.
\end{theorem}
\begin{proof}
	因为$S_1\cap S_2=\left[(S_1\cap S_2)^c\right]^c=\left[S_1^c\cup S_2^c\right]^c$,这就转化为了补集和并集的可测性结论.
\end{proof}
\begin{corollary}[有限交的可测性]
	设$S_i\ (i=1,2,\cdots,n)$都可测,则$\bigcap\limits_{i=1}^{n}S_i$也可测.
\end{corollary}
\begin{theorem}[差集的可测性]
	设$S_1,\ S_2$都可测,则$S_1\backslash S_2$也可测.
\end{theorem}
\begin{proof}
	因为$S_1\backslash S_2=S_1\cap S_2^c$,这就转化为了交集和补集的可测性结论.
\end{proof}
\begin{theorem}[可数可加性]\label{countable}
	设$\{S_i\}$是一列互不相交的可测集,则$\bigcup\limits_{i=1}^{\infty}S_i$也可测,且
	$$m\left(\bigcup_{i=1}^{\infty}S_i\right)=\sum_{i=1}^{\infty}mS_i.$$
\end{theorem}
\begin{proof}
	首先证明$\bigcup\limits_{i=1}^{\infty}S_i$的可测性.由有限并的可测性推论,对任意$n$,$\bigcup\limits_{i=1}^{n}S_i$可测,故对于任意$T$总有
	\begin{align*}
		m^*T
		&=m^*\left[T\cap\left(\bigcup_{i=1}^{n}S_i\right)\right]+m^*\left[T\cap\left(\bigcup_{i=1}^{n}S_i\right)^c\right]\\
		&\geqslant m^*\left[T\cap\left(\bigcup_{i=1}^{n}S_i\right)\right]+m^*\left[T\cap\left(\bigcup_{i=1}^{\infty}S_i\right)^c\right]\quad&\text{(外测度的单调性)}\\
		&=\sum_{i=1}^{n}m^*(T\cap S_i)+m^*\left[T\cap\left(\bigcup_{i=1}^{\infty}S_i\right)^c\right].\quad&\text{(有限并的可测性)}
	\end{align*}
	令$n\to\infty$得
	\begin{equation}\label{ntinfty}
		m^*T\geqslant\sum_{i=1}^{\infty}m^*(T\cap S_i)+m^*\left[T\cap\left(\bigcup_{i=1}^{\infty}S_i\right)^c\right].
	\end{equation}
	由外测度的次可数可加性,故有
	$$m^*T\geqslant m^*\left[T\cap\left(\bigcup_{i=1}^{\infty}S_i\right)\right]+m^*\left[T\cap\left(\bigcup_{i=1}^{\infty}S_i\right)^c\right].$$
	另一方面由于
	$$T=\left[T\cap\left(\bigcup_{i=1}^{\infty}S_i\right)\right]\cup\left[T\cap\left(\bigcup_{i=1}^{\infty}S_i\right)^c\right],$$
	又有
	$$m^*T\leqslant m^*\left[T\cap\left(\bigcup_{i=1}^{\infty}S_i\right)\right]+m^*\left[T\cap\left(\bigcup_{i=1}^{\infty}S_i\right)^c\right].$$
	因此
	$$m^*T= m^*\left[T\cap\left(\bigcup_{i=1}^{\infty}S_i\right)\right]+m^*\left[T\cap\left(\bigcup_{i=1}^{\infty}S_i\right)^c\right].$$
	这就证明了$\bigcup\limits_{i=1}^{\infty}S_i$的可测性.
	在\ref{ntinfty}式中,令$T=\bigcup\limits_{i=1}^{\infty}S_i$,这时由于$\left(\bigcup\limits_{i=1}^{\infty}S_i\right)\cap S_i=S_i$,便得
	$$m\left(\bigcup_{i=1}^{\infty}S_i\right)\geqslant\sum_{i=1}^{\infty}mS_i.$$
	另一方面,由外测度的次可数可加性,
	$$m\left(\bigcup_{i=1}^{\infty}S_i\right)\leqslant\sum_{i=1}^{\infty}mS_i.$$
	故
	$$m\left(\bigcup_{i=1}^{\infty}S_i\right)=\sum_{i=1}^{\infty}mS_i.$$
	
\end{proof}
\begin{corollary}[可数并的可测性]
	设$\{S_i\}$是一列可测集合,则$\bigcup\limits_{i=1}^{\infty}S_i$也可测.
\end{corollary}
\begin{proof}
	因$\bigcup\limits_{i=1}^{\infty}S_i$可表示为互不相交的集合的并:
	$$\bigcup\limits_{i=1}^{\infty}S_i=S_1\cup(S_2\backslash S_1)\cup\left[S_3\backslash(S_1\cup S_2)\right]\cup\cdots,$$
	由有限并、差、可数可加性结论即得.
\end{proof}
\begin{corollary}[可数交的可测性]
	设$\{S_i\}$是一列可测集合,则$\bigcap\limits_{i=1}^{\infty}S_i$也可测.
\end{corollary}
\begin{proof}
	因$\left(\bigcap\limits_{i=1}^{\infty}S_i\right)^c=\bigcup\limits_{i=1}^{\infty}S_i^c$,应用补与可数并的结论即得.
\end{proof}

\hspace*{\fill}

由上述性质的讨论,我们可知,Lebesgue可测集对可数并、可数交以及差集余集的运算都是封闭的.此外,定理\ref{countable}表明了Lebesgue测度具有可数可加性,它是满足Lebesgue测度公理的.下面,我们再介绍几个性质.

\begin{theorem}\label{zeng}
	设$\{S_i\}$是一列递增的可测集合
	$$S_1\subset S_2\subset\cdots\subset S_n\subset\cdots,$$
	令$S=\bigcup\limits_{i=1}^{\infty}S_i=\lim\limits_{n\to\infty}S_n$,则
	$$mS=\lim\limits_{n\to\infty}mS_n.$$
\end{theorem}
\begin{proof}
	因有
	$$S=S_1\cup(S_2\backslash S_1)\cup(S_3\backslash S_2)\cup\cdots\cup(S_n\backslash S_{n-1})\cup\cdots,$$
	其中各被并项都可测且互不相交,由Lebesgue测度的可数可加性,有(令$S_0=\varnothing$)
	\begin{align*}
		mS&=\sum_{i=1}^{\infty}m(S_i\backslash S_{i-1})=\lim\limits_{n\to\infty}\sum_{i=1}^{n}m(S_i\backslash S_{i-1})\\
		&=\lim\limits_{n\to\infty}m\left[\bigcup_{i=1}^{n}(S_i\backslash S_{i-1})\right]=\lim\limits_{n\to\infty}mS_n.
	\end{align*}
	
\end{proof}
\begin{theorem}
	设$\{S_i\}$是一列递降的可测集合
	$$S_1\supset S_2\supset\cdots\supset S_n\supset\cdots,$$
	令$S=\bigcap\limits_{i=1}^{\infty}S_i=\lim\limits_{n\to\infty}S_n$,则当$mS_1<\infty$时,
	$$mS=\lim\limits_{n\to\infty}mS_n.$$
\end{theorem}
\begin{proof}
	由于$S_n$可测,则可数交$S$也可测.又因$S_n$递降,从而$\{S_1\backslash S_n\}$递增,故由定理\ref{zeng}有
	$$\lim\limits_{n\to\infty}m\left[S_1\backslash S_n\right]=m\left[\bigcup_{i=1}^{\infty}(S_1\backslash S_n)\right]=m(S_1\backslash S).$$
	因$mS_1<\infty$及
	$$(S_1\backslash S_n)\cup S_n=S_1,$$
	$$m(S_1\backslash S_n)+mS_n=mS_1,$$
	有
	$$m(S_1\backslash S)=\lim\limits_{n\to\infty}m(S_1\backslash S_n)=mS_1-\lim\limits_{n\to\infty}mS_n.$$
	由于
	$$m(S_1\backslash S)=mS_1-mS,$$
	故
	$$mS=\lim\limits_{n\to\infty}mS_n.$$
\end{proof}
\begin{remark}
	条件$mS_1<\infty$是必要的.
\end{remark}
\begin{theorem}[平移不变性]
	对任意实数$\alpha$,定义映射$\tau_\alpha:x\to x+\alpha,\ x\in\mathbb{R}^n$.则对任何集$E\subset\mathbb{R}^n$,有$m^*E=m^*(\tau_\alpha E)$,且当$E$为Lebesgue可测时,$\tau_\alpha E$也Lebesgue可测(且测度不变).
\end{theorem}
\begin{proof}
	对任何一列开域$\{I_i\}$,$E\subset\bigcup\limits_{i=1}^{\infty}I_i$,同时就有$\tau_\alpha I_i$亦为开域,以及$\tau_\alpha E\subset\bigcup\limits_{i=1}^{\infty}(\tau_\alpha I_i)$,所以
	$$m^*E=\inf\left\{\sum_{i=1}^{\infty}|I_i|:E\subset\bigcup\limits_{i=1}^{\infty}I_i\right\}\geqslant m^*(\tau_\alpha E).$$
	但$\tau_\alpha E$再平移$\tau_{-\alpha}$后就是$E$,所以$m^*(\tau_\alpha E)\geqslant ,^*E$.这样就得到$m^*E=m^*(\tau_\alpha E)$.
	
	如果$E$为Lebesgue可测,那么对于任何$T\subset\mathbb{R}^n$,有
	$$m^*T=m^*(T\cap E)+m^*(T\cap E^c).$$
	由于$\tau_\alpha(T\cap E)=\tau_\alpha T\cap\tau_\alpha E,\ \tau_\alpha(T\cap E^c)=\tau_\alpha T\cap\tau_\alpha E^c$,因此从上式得到
	$$m^*(\tau_\alpha T)=m^*(\tau_\alpha T\cap\tau_\alpha E)+m^*(\tau_\alpha T\cap\tau_\alpha E^c),$$
	而上式中$\tau_\alpha T$为任意集,因此$\tau_\alpha E$为Lebesgue可测.
\end{proof}
定理说明,集$E\subset\mathbb{R}^n$经过平移后,它的外测度不变,对于Lebesgue可测集,平移后仍为Lebesgue可测.这个性质称为Lebesgue测度的平移不变性.

用类似的方法还可以证明Lebesgue测度的反射不变性.
\begin{theorem}[反射不变性]
	定义映射$\tau:x\to -x,\ x\in\mathbb{R}^n$.则对任何集$E\subset\mathbb{R}^n$,有$m^*E=m^*(\tau E)$,且当$E$为Lebesgue可测时,$\tau E$也Lebesgue可测(且测度不变).
\end{theorem}
证明不再赘述.

\section{可测集类}
这一节我们介绍常见的可测集.
\begin{theorem}
	\begin{enumerate}
		\item 凡外测度为零的集皆可测,称为{\heiti 零测度集}或{\heiti 零测集};
		\item 零测集的任何子集仍为零测集;
		\item 至多可数个零测集的并仍为零测集.
	\end{enumerate}
\end{theorem}
用Lebesgue测度的定义与简单性质即可证明,这里不再赘述.
\begin{theorem}[区间皆可测]
	区间$I$都是可测集,且$mI=|I|$.
\end{theorem}
\begin{proof}
	设$I_0$是异于区间$I$的任一开区间,则
	$$|I_0|=m^*(I_0\cap I)+m^*(I_0\cap I^c).$$
	事实上,在$\mathbb{R}$中显然,在$\mathbb{R}^2$中由于$I_0\cap I$为区间,而$I_0\cap I^c$可以分解成至多四个不相交的区间$I_i,\ i=1,2,3,4$,从而可证
	$$m^*(I_0\cap I^c)\leqslant\sum_{i=1}^{4}|I_i|,$$
	因此
	$$m^*(I_0\cap I)+m^*(I_0\cap I^c)\leqslant|I_0|,$$
	
	另一方面,反向不等式总成立,于是
	$$m^*(I_0\cap I)+m^*(I_0\cap I^c)=|I_0|,$$
	$\mathbb{R}^n$情形仿此.
	
	由Carath\'eodory引理及$m^*I_0=|I_0|$,对$\mathbb{R}^n$中任意点集$T$都有
	$$m^*T=m^*(T\cap I)+m^*(T\cap I^c).$$
	从而$I$可测.
\end{proof}
\begin{theorem}
	凡开集、闭集皆可测.
\end{theorem}
\begin{proof}
	任何非空开集可表示为至多可数个区间的并,而区间是可测的.开集既可测,闭集作为开集的余自然也可测.
\end{proof}
为了进一步拓广可测集类,我们给出下面的定义.
\begin{definition}[$\sigma$代数]
	设$\varOmega$是由$\mathbb{R}^n$的一些子集组成的集类,如果$\varOmega$满足条件
	\begin{enumerate}
		\item (包含空集)$\varnothing\in\varOmega$;
		\item (在补集下封闭)若$E\in\Omega$,则$E^c\in\varOmega$;
		\item (在可数并下封闭)若$E_n\in\varOmega,\ n=1,2,\cdots$,则$\bigcup\limits_{n=1}^{\infty}E_n\in\varOmega$.
	\end{enumerate}
	则称$\varOmega$是$\mathbb{R}^n$的一个$\sigma${\heiti 代数}.
\end{definition}
可以看出,$\mathbb{R}^n$中所有Lebesgue可测集全体组成的集类$\mathcal{M}$是一个$\sigma$代数(称之为Lebesgue代数).
\begin{definition}[测度]
	设$\varOmega$是$\mathbb{R}^n$上的一个$\sigma$代数.如果定义在$\varOmega$上的非负值集函数$\mu$满足条件
	\begin{enumerate}
		\item $\mu(\varnothing)=0$;
		\item 若$E_n\in\varOmega,\ n=1,2,\cdots$,且任意$n\neq m,\ E_n\cap E_m=\varnothing$,有
		$$\mu\left(\bigcup_{n=1}^{\infty}E_n\right)\sum_{n=1}^{\infty}\mu(E_n),$$
	\end{enumerate}
	则称$\mu$是$\varOmega$上的{\heiti (正)测度}.
\end{definition}
易见,Lebesgue测度$m$是定义在$\sigma$代数上的测度.

由$\sigma$代数的定义易知:如果$\{\varOmega_{\alpha}\}$是$\mathbb{R}^n$上的一族$\sigma$代数,则它们的交集$\bigcap\limits_{\alpha}\varOmega_{\alpha}$也是$\sigma$代数.
\begin{definition}[集类产生的$\sigma$代数]
	设$\varSigma$是$\mathbb{R}^n$的一个子集类,则称所有包含$\varSigma$的$\sigma$代数的交集为$\varSigma$产生的$\sigma$代数.
\end{definition}
由于$\mathbb{R}^n$全体子集组成的子集类是包含$\varSigma$的$\sigma$代数,因此包含$\varSigma$的$\sigma$代数不是空集,并且是包含$\varSigma$的最小的$\sigma$代数.
\begin{definition}[Borel代数]
	由$\mathbb{R}^n$中全体开集组成的子集类生成的$\sigma$代数,记为$\mathcal{B}$,称为{\heiti Borel代数},Borel代数里的元素称为{\heiti Borel集}.
\end{definition}
因为开集都是Lebesgue可测集,因此$\mathcal{B}\subset\mathcal{M}$,因而有以下定理.
\begin{theorem}
	凡Borel集都是Lebesgue可测集.
\end{theorem}
\begin{definition}[测度空间]
	若$\varOmega$是$\mathbb{R}^n$上的一个$\sigma$代数,$\mu$是$\varOmega$上的测度,则称$(\mathbb{R}^n,\varOmega,\mu)$为{\heiti 测度空间}.
\end{definition}
例如,上述$(\mathbb{R}^n,\mathcal{M},m)$和$(\mathbb{R}^n,\mathcal{B},m)$都是测度空间.
\begin{definition}
	设集合$G$可表示为一列开集$\{G_i\}$的交集:
	$$G=\bigcap\limits_{i=1}^{\infty}G_i,$$
	则称$G$为$G_{\delta}$型集.
	
	设集合$F$可表示为一列闭集$\{F_i\}$的并集:
	$$F=\bigcup\limits_{i=1}^{\infty}F_i,$$
	则称$F$为$F_{\delta}$型集.
\end{definition}
显然$G_{\delta}$型集及$F_{\delta}$型集都是Borel集.

我们已经知道,$\mathcal{B}\subset\mathcal{M}$,即Borel集都是Lebesgue可测集.但反之不成立.我们下面将讨论Lebesgue可测集合类中除了Borel集之外,还存在什么样的集合.

\begin{theorem}\label{g}
	设$E$是任一可测集,则一定存在$G_{\delta}$型集$G$,使$G\supset E$,且$m(G\backslash E)=0$.
\end{theorem}
\begin{proof}
		(1) 先证:对于任意$\varepsilon>0$,存在开集$G$,使$G\supset E$,且$m(G\backslash E)<\varepsilon$.
		
		先设$mE<\infty$,则由测度定义,有一列开区间$\{I_i\}(i=1,2,\cdots)$,使$\bigcup\limits_{i=1}^{\infty}I_i\supset E$,且
		$$\sum_{i=1}^{\infty}|I_i|<mE+\varepsilon.$$
		令$G=\bigcup\limits_{i=1}^{\infty}I_i$,则$G$为开集,$G\supset E$,且
		$$mE\leqslant mG\leqslant\sum_{i=1}^{\infty}mI_i=\sum_{i=1}^{\infty}|I_i|<mE+\varepsilon.$$
		因此,$mG-mE<\varepsilon$(这里用到$mE<\infty$),从而$m(G\backslash E)<\varepsilon$.
		
		\hspace*{\fill}
		
		其次,设$mE=\infty$,这时$E$必为无界集,但它总可表示成可数多个互不相交的有界可测集的并,即$E=\bigcup\limits_{n=1}^{\infty}E_n(mE_n<\infty)$,对每个$E_n$应用上面结果,可找到开集$G_n\supset E_n$使$m(G_n\backslash E_n)<\dfrac{\varepsilon}{2^n}$.
		
		令$G=\bigcup\limits_{n=1}^{\infty}G_n$,则$G$为开集,$G\supset E$,且
		$$G\backslash E=\bigcup\limits_{n=1}^{\infty}G_n\backslash\bigcup\limits_{n=1}^{\infty}E_n\subset\bigcup\limits_{n=1}^{\infty}(G_n\backslash E_n),$$
		$$m(G\backslash E)\leqslant\sum_{n=1}^{\infty}m(G_n\backslash E_n)<\varepsilon.$$
		(2) 依次取$\varepsilon_n=\dfrac{1}{n},\ n=1,2,\cdots$,由上述证明,存在开集$G_n\supset E$,使$m(G_n\backslash E)<\dfrac{1}{n}$.
		
		令$G=\bigcap\limits_{n=1}^{\infty}G_n$,则$G$为$G_{\delta}$型集,$G\supset E$,且
		$$m(G\backslash E)\leqslant m(G_n\backslash E)<\dfrac{1}{n},\ n=1,2,\cdots,$$
		故$m(G\backslash E)=0$.
\end{proof}
\begin{theorem}
	设$E$是任一可测集,则一定存在$F_{\delta}$型集$F$,使$F\subset E$,且$m(E\backslash F)=0$.
\end{theorem}
\begin{proof}
	因$E^c$也可测,由定理\ref{g}可知,存在$G_{\delta}$型集$G\supset E^c$,使$m(G\backslash E^c)=0$.
	
	令$F=G^c$,则$F$为$F_{\delta}$型集,$F\subset E$,且
	$$m(E\backslash F)=m(E\backslash G^c)=m(G\backslash E^c)=0.$$
	
\end{proof}
以上两个定理说明了只要有了全部$G_{\delta}$型集或$F_{\delta}$型集(它们只是Borel集的一部分)和全部Lebesgue零测集,就可以得到一切Lebesgue可测集.
\begin{theorem}[正则性]
	若$E$是一可测集,则
	\begin{enumerate}
		\item $mE=\inf\{mG|G\text{是开集},E\subset G\}$(外正则性);
		\item $mE=\sup\{mK|K\text{是紧集},K\subset E\}$(内正则性).
	\end{enumerate}
\end{theorem}
\begin{proof}
	(1)的证明:若$mE=\infty$,则对任意$G\supset E$,$mG=\infty$,因此(1)成立.
	
	若$mE<\infty$,则由定理\ref{g}的证明,对任意$\varepsilon>0$,存在开集$G\supset E,\ m(G\backslash E)<\varepsilon$,因此
	$$mG=m(G\backslash E)+mE<mE+\varepsilon.$$
	由确界定义,(1)成立.
	
	(2)的证明:若$E$有界,则存在有界闭区间$I$,使得$E\subset I$.对任意$\varepsilon>0$,存在开集$G\supset I\backslash E$,使得$m(G\backslash(I\backslash E))<\varepsilon$.令$K=I\backslash G$,则$K$是紧集,且
	$$E\backslash K=E\cap G\subset G\backslash(I\backslash E),$$
	故
	$$m(E\backslash K)<\varepsilon.$$
	于是当$E$有界时,(2)成立.
	
	若$E$无界,对任意$n$,令
	$$E_n=\{x|d(x,0)<n\}\cap E,$$
	则$\{E_n\}$单调可测,$\lim\limits_{n\to\infty}E_n=E$,且$\lim\limits_{n\to\infty}mE_n=mE$.由上述证明,存在紧集$K_n\subset E_n$,
	$$mE_n-\frac{1}{n}\leqslant mK_n\leqslant mE_n,\qquad n=1,2,\cdots,$$
	由此得到
	$$\lim\limits_{n\to\infty}mK_n=mE.$$
	因此无论$mE=\infty$或$mE<\infty$,(2)成立.
\end{proof}

\end{document}