\documentclass[12pt]{ctexart}
\usepackage{amsfonts,amssymb,amsmath,amsthm,geometry,graphicx,caption,color,xpatch}
\usepackage[colorlinks,linkcolor=blue,anchorcolor=blue,citecolor=green]{hyperref}
%define color 'orange'
\definecolor{orange}{RGB}{255,127,0}

%introduce theorem environment
\theoremstyle{definition}
\newtheorem{definition}{定义}
\newtheorem{theorem}{定理}
\newtheorem{lemma}{引理}
\newtheorem{property}{性质}
\newtheorem{example}{例}
\theoremstyle{plain}
\newtheorem*{solution}{\textcolor{green}{Soln}}
\newtheorem*{remark}{\textcolor{orange}{Rmk}}
\geometry{a4paper,scale=0.8}

%remove the dot after the theoremstyle
%\makeatletter
%\xpatchcmd{\@thm}{\thm@headpunct{.}}{\thm@headpunct{}}{}{}
%\makeatother

%article info
\title{\vspace{-2em}\textbf{Egorov定理}\vspace{-2em}}
\date{ }

\begin{document}
	\maketitle
	首先定义逐点收敛、一致收敛、几乎处处收敛和近一致收敛.
	\begin{definition}[逐点收敛]
		对于定义在$D\subset \mathbb{R}^n$上的一列函数$f,f_1,f_2,\cdots,f_n,\cdots$,对每一个$x\in D$,对$\forall \varepsilon>0$,$\exists N>0$,当$n>N$时,有
		$$|f_n(x)-f(x)|<\varepsilon,\qquad x\in D,$$
		则称函数列$\{f_n\}$在$D$上\textbf{逐点收敛}于$f$,记作
		$$\lim\limits_{n\to\infty}f_n(x)=f(x),\qquad x\in D.$$
	\end{definition}
	\begin{definition}[一致收敛]
		对于定义在$D\subset \mathbb{R}^n$上的一列函数$f,f_1,f_2,\cdots,f_n,\cdots$,对$\forall \varepsilon>0$,$\exists N>0$,当$n>N$时,有
		$$|f_n(x)-f(x)|<\varepsilon,\qquad x\in D,$$
		则称函数列$\{f_n\}$在$D$上\textbf{一致收敛}于$f$.
	\end{definition}
	上述定义在数学分析课程中已介绍,在此回顾一下,下面介绍测度意义下的收敛.
	\begin{definition}[几乎处处收敛]
		对于定义在$E\subset \mathbb{R}^n$上的一列广义实值函数$f,f_1,f_2,\cdots,f_n,\cdots$,若存在点集$Z\subset E$,满足$m(Z)=0$,使得
		$$\lim\limits_{n\to\infty}f_n(x)=f(x),\qquad x\in E\backslash Z,$$
		则称函数列$\{f_n(x)\}$在$E$上\textbf{几乎处处收敛}于$f(x)$,记作
		$$f_n(x)\to f(x),\qquad a.e.\ x\in E.$$
	\end{definition}
	\begin{definition}[近一致收敛]
		对于定义在$E\subset \mathbb{R}^n$上的一列可测函数$f,f_1,f_2,\cdots,f_n,\cdots$,若对$\forall\delta>0$,存在可测点集$E_{\delta}\subset E$,满足$m(E_{\delta})\leqslant\delta$,使得$f_n(x)$在$E\backslash E_{\delta}$上一致收敛,则称函数列$\{f_n(x)\}$在$E$上\textbf{近一致收敛}于$f(x)$.
	\end{definition}
	Egorov定理指出在测度意义下,函数列的逐点收敛与一致收敛“差不多”.下面先给出一个引理.
	\begin{lemma}
		对于定义在$\mathbb{R}^n$上的一列实值函数$f,f_1,f_2,\cdots,f_n,\cdots$,记$\{f_n(x)\}$不收敛于$f(x)$的点集$x$的全体组成的集合为$D$,则
		$$D=\bigcup_{k=1}^{\infty}\bigcap_{N=1}^{\infty}\bigcup_{n=N}^{\infty}\left\{x:\left|f_n(x)-f(x)\right|\geqslant\frac{1}{k}\right\}.$$
	\end{lemma}
	\begin{proof}
		对某不收敛点$x_0$,存在$\varepsilon_0>0$,对任意正整数$N>0$,必存在$n>N$,使得
		$$\left|f_n(x_0)-f(x_0)\right|\geqslant\varepsilon_0.$$
		记$E_{N,k}=\displaystyle\bigcup_{n=N}^{\infty}\left\{x:\left|f_n(x)-f(x)\right|\geqslant\dfrac{1}{k}\right\}$,则表示$n\geqslant N$时的全体不收敛点,$\displaystyle\bigcap_{N=1}^{\infty}E_{N,k}$表示取遍$N$的全体不收敛点,即对每个$N$,都存在不收敛点,$\displaystyle\bigcup_{k=1}^{\infty}$覆盖所有可能的$\varepsilon>0$,确保包含所有不收敛点.
	\end{proof}
	\begin{theorem}[Egorov定理]
		对于定义在$E\subset \mathbb{R}^n$上的一列可测函数$f,f_1,f_2,\cdots,f_n,\cdots$几乎处处有限,$m(E)<+\infty$,若$\{f_n(x)\}$在$E$上几乎处处收敛于$f(x)$,则$\{f_n(x)\}$在$E$上近一致收敛于$f(x)$.
	\end{theorem}
	\begin{proof}
		不妨假设这些函数在$E$上处处有限. 对任意$\delta>0$,先构造$E_{\delta}$.记$f_n(x)$不收敛于$f(x)$的点的全体组成的集合为$D$,则
		$$m\left(\bigcap_{N=1}^{\infty}\bigcup_{n=N}^{\infty}\left\{x:\left|f_n(x)-f(x)\right|\geqslant\frac{1}{k}\right\}\right)\leqslant m(D)=0.$$
		记$E_{N,k}=\displaystyle\bigcup_{n=N}^{\infty}\left\{x:\left|f_n(x)-f(x)\right|\geqslant\dfrac{1}{k}\right\}$,由递减集列测度与极限的可交换性,有
		$$\lim\limits_{N\to\infty}m(E_{N,k})=m\left(\lim\limits_{N\to\infty}E_{N,k}\right)=0,$$
		于是存在$N_k$,使得
		$$m(E_{N_k,k})<\frac{\delta}{2^k},$$
		令$E_{\delta}=\displaystyle\bigcup_{k=1}^{\infty}E_{N_k,k}$,则
		$$m(E_{\delta})\leqslant\sum_{i=1}^{\infty}m(E_{N_k,k})<\sum_{i=1}^{\infty}\frac{\delta}{2^k}=\delta.$$
		
		下面设$x\in E\backslash E_{\delta}$,对任意$\varepsilon>0$,存在正整数$k$使$\dfrac{1}{k}<\varepsilon$. 因为$x\notin E_{N_k,k}$,故当$n\geqslant N_k$时有$\left|f_n(x)-f(x)\right|<\dfrac{1}{k}<\varepsilon$. 所以函数列$\{f_n(x)\}$在$E\backslash E_{\delta}$上一致收敛于$f(x)$.
	\end{proof}
\end{document}