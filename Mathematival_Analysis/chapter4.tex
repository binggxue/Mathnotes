\chapter{函数的连续性}
连续函数是数学分析中着重讨论的一类函数.
\section{连续与间断}
\subsection{函数在一点的连续性}
\begin{definition}
	设函数$f$在某$U(x_0)$上有定义.若
	$$\lim\limits_{x\to x_0}f(x)=f(x_0),$$
	则称$f${\heiti 在点$x_0$连续}.
\end{definition}
记$\Delta x=x-x_0$,称为$x$在点$x_0$处的增量.设$y_0=f(x_0)$,相应的函数$y$在点$x_0$处的增量记为
$$\Delta y=f(x)-f(x_0)=f(x_0+\Delta x)-f(x_0)=y-y_0.$$
则函数在一点处连续的定义等价如下.
\begin{definition}
	设函数$f$在某$U(x_0)$上有定义.若
	$$\lim\limits_{\Delta x\to 0}\Delta y=0,$$
	则称$f${\heiti 在点$x_0$连续}.
\end{definition}
由于函数在一点处的连续性是通过极限来定义的,因而也可直接用$\varepsilon-\delta$语言来叙述.
\begin{definition}
	若对任给的$\varepsilon>0$,存在$\delta>0$,使得当$|x-x_0|<\delta$时,有
	$$|f(x)-f(x_0)|<\varepsilon,$$
	则称函数$f$在点$x_0$连续.
\end{definition}
\begin{definition}
	设函数$f$在某$U_+(x_0)$上有定义.若
	$$\lim\limits_{x\to x_0^+}f(x)=f(x_0),$$
	则称$f$在点$x_0${\heiti 右连续}.
	
	类似地,我们可以定义左连续.
\end{definition}
\begin{theorem}
	函数$f$在点$x_0$连续的充要条件是:$f$在$x_0$点既是右连续,又是左连续.
\end{theorem}

函数$f$在$x_0$处连续,意味着$f$在$x_0$处有极限,且极限值为$x_0$.而函数在$x_0$处有极限意味着它在这一点的上极限和下极限相等,且上极限和下极限都等于$f(x_0)$.从这个角度也可以刻画函数的逐点连续.为了叙述方便,我们给出函数振幅的概念.
\begin{definition}[函数在区间上的振幅]
	设区间$I$上的函数$f$.令
	$$\omega(I)\coloneqq\sup f(I)-\inf f(I),$$
	我们称$\omega(I)$为$f$在$I$上的{\heiti 振幅}(amplitude).
\end{definition}
振幅还有一种常用的等价定义.
\begin{proposition}
	设区间$I$上的函数$f$.令
	$$\omega=\sup\left[f(x_1)-f(x_2)\right],\quad(\forall x_1,x_2\in I),$$
	则$\omega=\omega(I)$.
\end{proposition}
\begin{proof}
	令$M=\sup f(I)$,$m=\inf f(I)$,只需证明$\omega=M-m$.
	
	(i)对于任意$x_1,x_2\in I$,都有
	$$m\leqslant f(x_1)\leqslant M,\qquad m\leqslant f(x_2)\leqslant M.$$
	因此
	$$|f(x_1)-f(x_2)|\leqslant M-m.$$
	于是可知$\omega\leqslant M-m.$
	
	(ii)对于任一$\varepsilon>0$都存在$x_1,x_2\in I$使得
	$$f(x_1)>M-\frac{\varepsilon}{2},\qquad f(x_2)<m+\frac{\varepsilon}{2}.$$
	因此
	$$|f(x_1)-f(x_2)|\geqslant f(x_1)-f(x_2)>M-m-\varepsilon.$$
	于是可知$\omega\geqslant M-m$.
	
	综上,有$\omega=M-m=\omega(I).$
	$\hfill\blacksquare$
\end{proof}
\begin{definition}[函数在一点的振幅]\label{def:amplitude}
	设区间$I$上的函数$f$,令
	$$\omega(x_0)\coloneqq\lim\limits_{\delta\to 0^+}\omega\left[U(x_0;\delta)\right]=\lim\limits_{\delta\to 0^+}\left[\sup\limits_{x\in U(x_0;\delta)}f(x)-\inf\limits_{x\in U(x_0;\delta)}f(x)\right].$$
	我们称$\omega(x_0)$为$f$在点$x_0$处的{\heiti 振幅}(amplitude).
\end{definition}
\begin{remark}
$\sup\limits_{x\in U(x_0;\delta)}f(x)$单调递减,而$\inf\limits_{x\in U(x_0;\delta)}f(x)$单调递增.因此$\sup\limits_{x\in U(x_0;\delta)}f(x)-\inf\limits_{x\in U(x_0;\delta)}f(x)$单调递增.又因为
$$\sup\limits_{x\in U(x_0;\delta)}f(x)-\inf\limits_{x\in U(x_0;\delta)}f(x)\geqslant 0,$$
因此定义式右侧有极限,这说明定义是合理的.
\end{remark}
用振幅的观点也可以刻画函数的逐点连续.
\begin{theorem}\label{amplitude}
	函数$f$在$x_0$处连续当且仅当$f$在$x_0$处的振幅$\omega(x_0)=0$.
\end{theorem}
\begin{proof}
	必要性\qquad 若$f$在$x_0$处连续,则对于任一$\varepsilon>0$存在$\delta>0$使得对于任意$x_1,x_2\in U(x_0;\delta)$都有
	$$|f(x_1)-f(x_0)|<\frac{\varepsilon}{2},\quad |f(x_2)-f(x_0)|<\frac{\varepsilon}{2}.$$
	因此
	$$|f(x_1)-f(x_2)|\leqslant|f(x_1)-f(x_0)|+|f(x_2)-f(x_0)|<\frac{\varepsilon}{2}+\frac{\varepsilon}{2}=\varepsilon.$$
	因此$\omega\left[U(x_0;\delta)\right]\leqslant\varepsilon$.令$\delta\to 0^+$得$0\leqslant\omega(x_0)\leqslant\varepsilon$.令$\varepsilon\to 0$即得$\omega(x_0)=0$.
	
	充分性\qquad 若$\omega(x_0)=0$,则
	$$\lim\limits_{\delta\to 0^+}\omega\left[U(x_0;\delta)\right]=0.$$
	因此对于任一$\varepsilon>0$,存在$\delta>0$使得$\omega\left[U(x_0;\delta)\right]<\varepsilon$.因此对于任一$x\in U(x_0;\delta)$都有
	$$|f(x)-f(x_0)|\leqslant\omega\left[U(x_0;\delta)\right]<\varepsilon.$$
	这表明$\lim\limits_{x\to x_0}f(x)=f(x_0)$.于是$f$在$x_0$处连续.$\hfill\blacksquare$
\end{proof}
不难看出函数在一点的振幅和函数在一点的上极限和下极限的定义只差了$x_0$这“一点”.
\begin{theorem}
	设函数$f$在$x_0$附近有定义.则
	$$\lim\limits_{\delta\to 0^+}\left[\sup\limits_{x\in\mathring{U}(x_0;\delta)}f(x)\right]=\lim\limits_{\delta\to 0^+}\left[\inf\limits_{x\in\mathring{U}(x_0;\delta)}f(x)\right]=f(x_0)\iff \lim\limits_{\delta\to 0^+}\left[\sup\limits_{x\in U(x_0;\delta)}f(x)\right]=\lim\limits_{\delta\to 0^+}\left[\inf\limits_{x\in U(x_0;\delta)}f(x)\right].$$
\end{theorem}
\begin{proof}
	由定理\ref{amplitude}可知
	\begin{align*}
		&\lim\limits_{\delta\to 0^+}\left[\sup\limits_{x\in\mathring{U}(x_0;\delta)}f(x)\right]=\lim\limits_{\delta\to 0^+}\left[\inf\limits_{x\in\mathring{U}(x_0;\delta)}f(x)\right]=f(x_0)\iff\limsup\limits_{x\to x_0}f(x)=\liminf\limits_{x\to x_0}f(x)=f(x_0)\\
		&\iff\lim\limits_{x\to x_0}f(x)=f(x_0)\iff f\text{在}x_0\text{处连续}\iff\omega(x_0)=0\\
		&\iff\lim\limits_{\delta\to 0^+}\left[\sup\limits_{x\in U(x_0;\delta)}f(x)-\inf\limits_{x\in U(x_0;\delta)}f(x)\right]=0\\
		&\iff\lim\limits_{\delta\to 0^+}\left[\sup\limits_{x\in U(x_0;\delta)}f(x)\right]=\lim\limits_{\delta\to 0^+}\left[\inf\limits_{x\in U(x_0;\delta)}f(x)\right].
	\end{align*}
	$\hfill\blacksquare$
\end{proof}
\subsection{区间上的连续函数}
\begin{definition}
	若函数$f$在区间$I$上的每一点都连续,则称$f$为$I$上的{\heiti 连续函数}(continuous function).对于闭区间或半开半闭区间的端点,函数在这些点上连续是指左连续或右连续.
\end{definition}
后面我们将证明任何初等函数在其定义区间上为连续函数.同时,也存在着在其定义区间上每一点都不连续的函数.
\subsection{间断点及其分类}
\begin{definition}[间断点]
	设函数在某$\mathring{U}(x_0)$上有定义.若$f$在点$x_0$无定义,或$f$在点$x_0$有定义而不连续,则称$x_0$为$f$的{\heiti 间断点}(point of discontinuity).
\end{definition}
我们对函数的间断点做如下分类.
\begin{definition}[可去间断点]
	若
	$$\lim\limits_{x\to x_0}f(x)=A,$$
	而$f$在$x_0$无定义或有定义但$f(x_0)\neq A$,则称$x_0$为$f$的可去间断点.
\end{definition}
\begin{definition}[跳跃间断点]
	若函数$f$在点$x_0$处的左右极限都存在,但
	$$\lim\limits_{x\to x_0^+}f(x)\neq \lim\limits_{x\to x_0^-}f(x),$$
	则称$x_0$为$f$的跳跃间断点.
\end{definition}
可去间断点和跳跃间断点统称为{\heiti 第一类间断点}.

函数的所有其他形式的间断点,即使得函数至少有一侧极限不存在的那些点,称为{\heiti 第二类间断点}.

前面介绍了用振幅的观点来刻画函数的连续点,类似地也可以用振幅的观点刻画不连续点.
\begin{theorem}
	$x_0$是函数$f$的一个间断点当且仅当$\omega(x_0)>0$.
\end{theorem}
设$I$上的函数$f$,把$f$的间断点的集合记作$D(f)$.用振幅的大小可以把不连续点归类.令
$$D_\delta\coloneqq\{x\in I|\omega(x)\geqslant\delta\}.$$
\begin{proposition}\label{prop:jianduan}
	设$I$上的函数$f$.则
	$$D(f)=\bigcup_{n=1}^\infty D_{1/n}.$$
\end{proposition}
\begin{proof}
	显然$D(f)\supseteq\displaystyle\bigcup_{n=1}^{\infty}D_{1/n}$,因此只需证明$D(f)\subseteq\displaystyle\bigcup_{n=1}^{\infty}D_{1/n}$.任取$x_0\in D(f)$,则$x_0$是$f$的一个间断点,即$\omega(x_0)>0$.因此存在$m$使得$\omega(x_0)\geqslant\frac{1}{m}$,这表明$x_0\in D_{1/m}$.因此$D(f)\subseteq\displaystyle\bigcup_{n=1}^{\infty}D_{1/n}$.于是可知
	$$D(f)=\bigcup_{n=1}^{\infty}D_{1/n}.$$
	$\hfill\blacksquare$
\end{proof}
\section{连续函数的性质}
\subsection{连续函数的局部性质}
\begin{theorem}[局部有界性]
	若函数$f$在点$x_0$连续,则$f$在某$U(x_0)$上有界.
\end{theorem}
\begin{theorem}[局部保号性]
	若函数$f$在点$x_0$连续,且$f(x_0)>0$,则对任何正数$r<f(x_0)$,存在某$U(x_0)$,使得对一切$x\in U(x_0)$,有
	$$f(x)>r.$$
\end{theorem}
\begin{theorem}[四则运算]
	若函数$f$和$g$在点$x_0$连续,则$f\pm g,f\cdot g,f/g$在有意义的情况下也都在点$x_0$连续.
\end{theorem}
以上三个定理的证明,都可从函数极限的有关定理直接推得.
\begin{theorem}[复合函数的连续性]
	若函数$f$在点$x_0$连续,$g$在点$u_0$连续,$u_0=f(x_0)$,则复合函数$g\circ f$在点$x_0$连续. 
\end{theorem}
\begin{proof}
	由于$g$在$u_0$连续,对任给的$\varepsilon>0$,存在$\delta_1>0$,使得当$|u-u_0|<\delta_1$时,有
	$$|g(u)-g(u_0)|<\varepsilon.$$
	又由$u_0=f(x_0)$及$u=f(x)$在点$x_0$连续,故对上述$\delta_1>0$,存在$\delta>0$,使得当$|x-x_0|<\delta$时,有$|u-u_0|=|f(x)-f(x_0)|<\delta_1$.由此,对任给的$\varepsilon>0$,存在$\delta>0$,当$|x-x_0|<\delta$时,有
	$$|g(f(x))-g(f(x_0))|<\varepsilon.$$
	这就证明了$g\circ f$在点$x_0$处连续.$\hfill\blacksquare$
\end{proof}
\begin{remark}
	根据连续性的定义,上述定理的结论可表示为
	$$\lim\limits_{x\to x_0}g(f(x))=g(\lim\limits_{x\to x_0}f(x))=g(f(x_0)).$$
\end{remark}
\subsection{反函数的连续性}
\begin{theorem}
	若函数$f$在$\left[a,b\right]$上{\heiti 严格单调并连续},则反函数$f^{-1}$在其定义域$\left[f(a),f(b)\right]$或$\left[f(b),f(a)\right]$上连续.
\end{theorem}
\begin{proof}
	不妨设$f$在$\left[a,b\right]$上严格递增.此时$f$的值域即$f^{-1}$的定义域为$\left[f(a),f(b)\right]$.任取$y_0\in(f(a),f(b))$,设$x_0=f^{-1}(y_0)$,则$x_0\in(a,b)$.于是对任给的$\varepsilon>0$,可在$(a,b)$上$x_0$的两侧各取异于$x_0$的点$x_1,x_2(x_1<x_0<x_2)$,使它们与$x_0$的距离小于$\varepsilon$.
	
	设与$x_1,x_2$对应的函数值分别为$y_1,y_2$,由$f$的严格递增性可知$y_1<y_0<y_2$.令
	$$\delta=\min\{y_2-y_0,y_0-y_1\},$$
	则当$y\in\mathring{U}(y_0;\delta)$时,对应的$x=f^{-1}(y)$的值都落在$x_1$与$x_2$之间,故有
	$$|f^{-1}(y)-f^{-1}(y_0)|=|x-x_0|<\varepsilon,$$
	这就证明了$f^{-1}$在$y_0$处连续,由$y_0$的任意性可知$f^{-1}$在$(f(a),f(b))$上连续.
	
	类似可证$f^{-1}$在其定义区间的端点$f(a)$与$f(b)$上分别为右连续和左连续,因此$f^{-1}$在$\left[f(a),f(b)\right]$上连续.$\hfill\blacksquare$
\end{proof}
\subsection{闭区间上连续函数的性质}
\begin{definition}[函数的最值]
	设$f$是定义在数集$D$上的函数.若存在$x_0\in D$,使得对一切$x\in D$,有
	$$f(x_0)\geqslant f(x)$$
	则称$f$在$D$上有最大值,并称$f(x_0)$为$f$在$D$上的最大值.
	
	若存在$x_0\in D$,使得对一切$x\in D$,有
	$$f(x_0)\leqslant f(x)$$
	则称$f$在$D$上有最小值,并称$f(x_0)$为$f$在$D$上的最小值.
\end{definition}
\begin{theorem}[有界性定理]
	若函数$f(x)$在闭区间$\left[a,b\right]$上连续,那么$f(x)$在闭区间$\left[a,b\right]$上有界.
\end{theorem}
\begin{proof}
	只需证明有上界的情况.用反证法.
	
	假设$f(x)$无上界,则存在$x_n\in\left[a,b\right]$,使得
	$$f(x_n)>n\qquad n=1,2,\cdots.$$
	由此得$\lim\limits_{n\to \infty}f(x_n)=+\infty$.
	
	另一方面,由于$\{x_n\}$是有界数列,由致密性原理,$\{x_n\}$有收敛的子列$\{x_{n_k}\}$,设$\lim\limits_{k\to\infty}x_{n_k}=x_0$.
	
	由于
	$$a\leqslant x_{n_k}\leqslant b,$$
	由极限的不等式性质推得
	$$a\leqslant x_0\leqslant b,$$
	故$f(x)$在$x_0$处连续.由归结原则,
	$$+\infty=\lim\limits_{n\to\infty}f(x_n)=\lim\limits_{k\to\infty}f(x_{n_k})=\lim\limits_{x\to x_0}f(x)=f(x_0).$$
	这与$f(x)$在$x_0$处连续矛盾.$\hfill\blacksquare$
\end{proof}
\begin{theorem}[最大、最小值定理]
	若函数$f(x)$在闭区间$\left[a,b\right]$上连续,则$f(x)$在$\left[a,b\right]$上有最大值与最小值.
\end{theorem}
\begin{proof}
	由有界性定理和确界原理,$f(x)$存在上确界
	$$\sup\limits_{x\in\left[a,b\right]}f(x)=M.$$
	下面证明:存在$\xi\in\left[a,b\right]$,使$f(\xi)=M$.用反证法,
	假设对一切$x\in\left[a,b\right]$,都有$f(x)<M$.令
	$$g(x)=\frac{1}{M-f(x)},\qquad x\in \left[a,b\right].$$
	显然$g(x)$在$\left[a,b\right]$上连续,且取正值,故$g$在$\left[a,b\right]$上有上界,记为$G$.则有
	$$0<g(x)=\frac{1}{M-f(x)}\leqslant G,\qquad x\in \left[a,b\right].$$
	从而推得
	$$f(x)\leqslant M-\frac{1}{G},\qquad x\in\left[a,b\right].$$
	这与$M$是$f(\left[a,b\right])$的上确界矛盾.故存在$\xi\in\left[a,b\right]$,使$f(\xi)=M$.
	
	同理可证$f$在$\left[a,b\right]$上有最小值.$\hfill\blacksquare$
\end{proof}
\begin{theorem}[介值定理]
	设函数$f(x)$在闭区间$\left[a,b\right]$上连续,且$f(a)\neq f(b)$.若$\mu$为介于$f(a)$和$f(b)$之间的任何实数,则至少存在一点$x_0\in(a,b)$,使得$f(x_0)=\mu.$
\end{theorem}
\begin{corollary}[根的存在定理]
	若函数在闭区间$\left[a,b\right]$上连续,且$f(a)f(b)<0$,则至少存在一点$x_0\in(a,b)$,使得$f(x_0)=0$.
\end{corollary}
要证介值定理,可以将其转化为证明根的存在定理.
\begin{proof}
	不妨设$f(a)<\mu<f(b)$.令$g(x)=f(x)-\mu$,则$g$也是$\left[a,b\right]$上的连续函数,且$g(a)<0,g(b)>0.$于是定理的结论转化为:存在$x_0\in\left[a,b\right]$,使得$f(x_0)=0$.
	
	设集合
	$$E=\{x|g(x)<0,x\in\left[a,b\right]\}.$$
	
	显然$E$为非空有界数集,故由确界原理,存在上确界$x_0\in\sup E.$另一方面,因为$g(a)<0,g(b)>0$,由连续函数的局部保号性,存在$\delta>0$,使得
	$$g(x)<0,\qquad x\in\left[a,a+\delta\right);$$
	$$g(x)>0,\qquad x\in\left(b-\delta,b\right].$$
	由此易见$x_0\neq a,x_0\neq b$,即$x_0\in(a,b)$.
	
	下证$g(x_0)=0$.用反证法,假设$g(x_0)\neq 0$,则$g(x_0)<0$.由局部保号性,存在$U(x_0;\eta)(\subset(a,b))$,使在其上$g(x)<0$,特别有$g(x_0+\frac{\eta}{2})\in E.$但这与$x_0=\sup E$相矛盾,故必有$g(x_0)=0$.$\hfill\blacksquare$
\end{proof}
\subsection{一致连续性}
函数$f$在区间上连续,是指$f$在该区间上的每一点都连续.下面讨论的一致连续性反映了函数在区间上更强的连续性.在连续性的定义中,对于给定的$\varepsilon>0$,在不同的点$x_0$处,相应的$\delta$不一定相同.我们提出这样的问题:对于任意的$\varepsilon>0$,是否存在适用于一切$x_0\in E$的$\delta$,使得只要$$x,x_0\in E,\qquad |x-x_0|<\delta,$$
就有$$|f(x)-f(x_0)|<\varepsilon?$$
由此我们引出一致连续的定义.
\begin{definition}[一致连续]
	设$E$是$\mathbb{R}$的一个子集,函数$f$在$E$上有定义.若对任意$\varepsilon>0$,存在$\delta>0$,使得只要
	$$x_1,x_2\in E,\qquad |x_1-x_2|<\delta,$$
	就有$$|f(x_1)-f(x_2)|<\varepsilon$$
	那么我们就说函数$f$在集合$E$上是{\heiti 一致连续}的.
\end{definition}
函数$f$在区间$I$上连续时,$\delta$的取值与$\varepsilon$和$x$都有关,因此我们写$\delta=\delta(\varepsilon,x)$表示$\delta$与$\varepsilon$和$x$的依赖关系.如果能做到$\delta$只与$\varepsilon$有关,而与$x$无关,或者说存在一个适合所有$x$的公共的$\delta=\delta(\varepsilon)$,那么函数不仅在$I$上连续,而且是一致连续.

一般地,函数在某区间上连续并不一定能推出一致连续,但对于闭区间却可以推出.于是我们有以下定理.
\begin{theorem}[Heine-Cantor一致连续性定理]
	若函数在闭区间$\left[a,b\right]$上连续,则$f$在$\left[a,b\right]$上一致连续.
\end{theorem}
\begin{proof}
	用反证法\qquad 假设存在$\varepsilon_0>0$,对任意$\delta>0$,存在点列$\{x_n\},\{y_n\}\in \left[a,b\right]$,使得$|x_n-y_n|<\delta$时,
	$$|f(x_1-f(x_2))|\geqslant\varepsilon_0.$$
	由于$\delta$的任意性,有
	$$\lim\limits_{n\to\infty}|x_n-y_n|=0.$$
	又$\{x_n\},\{y_n\}$有界,由致密性原理,存在收敛子列$\{x_{n_k}\},\{y_{n_k}\}$使得
	$$\lim\limits_{k\to\infty}x_{n_k}=\lim\limits_{k\to\infty}y_{n_k}=x_0.$$
	由极限的不等式性质可推知$a\leqslant x_0\leqslant b$.故$f(x)$在点$x_0$处连续.由归结原则,
	$$\lim\limits_{k\to\infty}|f(x_{n_k})-f(y_{n_k})|=0<\varepsilon_0.$$
	与假设矛盾.$\hfill\blacksquare$
\end{proof}
\begin{remark}
	闭区间确保了$x_0\in\left[a,b\right]$,如果是开区间,当$x_{n_k}$收敛到端点处时上述结论将不成立.
\end{remark}
如果把以上定理的条件放宽,可以得到以下结论,证明方法完全一样.
\begin{proposition}\label{prop:biji}
	设闭区间$I$上的函数$f$.若$D(f)$存在一个开覆盖$\{(\alpha_i,\beta_i)|i=1,2,\cdots\}$.令
	$$K=I\bigg\backslash\bigcup_{n=1}^{\infty}(\alpha_i,\beta_i).$$
	则对于任一$\varepsilon>0$都存在$\delta>0$使得当$x\in K,\ y\in I$且$|x-y|<\delta$时有$|f(x)-f(y)|<\varepsilon$.
\end{proposition}
\begin{proof}
	用反证法.假设结论不成立,则存在$\varepsilon_0>0$对于任一$n\in\mathbb{N}_+$都存在$x_n\in K,\ y_n\in I$使得
	$$|x_n-y_n|<\frac{1}{n},\quad |f(x_n)-f(y_n)|\geqslant\varepsilon_0.$$
	由于$\{x_n\}\subseteq K\subseteq I$,故$\{x_n\}$有界,由Weierstrass定理可知$\{x_n\}$有一个子列$x_{k_n}\to \xi\in K$.由于
	$$|y_{k_n}-\xi|\leqslant|y_{k_n}-x_{k_n}|+|x_{k_n}-\xi|<\frac{1}{k_n}+|x_{k_n}-\xi|\leqslant\frac{1}{n}+|x_{k_n}-\xi|.$$
	因此$y_{k_n}\to\xi$.由于$f$在$\xi$处连续,故
	$$\lim\limits_{n\to\infty}f(x_{k_n})=\lim\limits_{n\to\infty}f(y_{k_n})=f(\xi).$$
	则
	$$\lim\limits_{k\to\infty}|f(x_{n_k})-f(y_{n_k})|=0<\varepsilon_0.$$
	这与$|f(x_n)-f(y_n)|\geqslant\varepsilon_0$矛盾.于是可知命题成立.$\hfill\blacksquare$
\end{proof}
由一致连续性定理的证明,我们认识到可以用数列来描述一致连续性.
\begin{theorem}[一致连续性的数列式描述]
	设$E$是$\mathbb{R}$的一个子集,函数$f$在$E$上有定义.则$f$在$E$上一致连续的充要条件是:对任何满足条件$$\lim\limits_{n\to\infty}(x_n-y_n)=0$$
	的数列$\{x_n\}\in E$,都有$$\lim\limits_{n\to\infty}(f(x_n)-f(y_n))=0.$$
\end{theorem}
\section{初等函数的连续性}
由于初等函数由基本初等函数经过有限次四则运算和有限次复合运算得到,而前面已经证明了连续函数关于四则运算和复合运算的性质,因此我们只需讨论基本初等函数的连续性即可.
\begin{proposition}[常量函数]
	函数$y=c$\quad ($c$是常数)是连续的.
\end{proposition}
\begin{proof}
	$\forall\varepsilon>0$,取任意的$x_0$,在任意的$\mathring{U}(x_0)$中,都有$f(x)\in\mathring{U}(c)$,因此这个命题成立. $\hfill\blacksquare$
\end{proof}
\begin{proposition}[指数函数]
函数$f(x)=a^x$在$\mathbb{R}$上连续.
\end{proposition}
\begin{proof}
先证明$\lim\limits_{x\to 0}a^x=1.$
对任意$\varepsilon>0$,存在$\delta>0$使得
$$1-\varepsilon<a^{-\delta}<a^{\delta}<1+\varepsilon.$$
当$x\in\mathring{U}(0;\delta)$时,从而有
$$1-\varepsilon<a^{-\delta}<a^x<a^{\delta}<1+\varepsilon.$$
故$\lim\limits_{x\to 0}a^x=1.$
则对任意$x_0\in\mathbb{R}$,有
$$\lim\limits_{x\to x_0}a^{x-x_0}=1.$$
由$\varepsilon$的任意性,有
$$-\varepsilon a^{-x_0}<a^{x-x_0}-1<\varepsilon a^{-x_0}.$$
即$$\lim\limits_{x\to x_0}a^x=a^x_0.$$
$\hfill\blacksquare$
\end{proof}
\begin{proposition}[三角函数]
函数$f(x)=\sin x$和$f(x)=\cos x$在$\mathbb{R}$上连续.
\end{proposition}
\begin{proof}
对任意$\varepsilon>0$,取$\delta=\varepsilon$,则当$|x-x_0|<\delta$时,有
$$|\sin x-\sin x_0|=\big|2\cos \frac{x+x_0}{2}\sin\frac{x-x_0}{2}\big|\leqslant2\big|\sin\frac{x-x_0}{2}\big|\leqslant|x-x_0|<\varepsilon.$$
$$|\cos x-\cos x_0|=\big|-2\sin \frac{x+x_0}{2}\sin\frac{x-x_0}{2}\big|\leqslant2\big|\sin\frac{x-x_0}{2}\big|\leqslant|x-x_0|<\varepsilon.$$
$\hfill\blacksquare$
\end{proof}
由于反函数的连续性,可知对数函数和反三角函数也具有相应的连续性.对于幂函数$x^\alpha$,其可写成$x^\alpha=\text{e}^{\alpha\ln x}$从而看作函数$\text{e}^u$和$u=\alpha\ln x$的复合函数.从而推知幂函数的连续性.

以上我们完成了基本初等函数连续性的证明,于是有下述定理.
\begin{theorem}
任何初等函数都是在其定义区间的连续函数.
\end{theorem}