\chapter{微分中值定理及其应用}
在这一章里,我们要讨论怎样由导数$f'$的已知性质来推断函数$f$所应具有的性质.微分中值定理(包括Rolle定理、Lagrange定理、Cauchy定理、Taylor定理)正是进行这一讨论的有效工具.
\section{Lagrange定理}
本节首先介绍Lagrange定理以及它的预备定理——Rolle定理.

根据微分的定义,如果函数$f(x)$在点$x_0$可微,那么当$x\to x_0$时,有
$$f(x)-f(x_0)=f'(x_0)(x-x_0)+o(x-x_0).$$

以上公式称为{\heiti 无穷小增量公式},它反映了$x\to x_0$时函数值的变化情况.显然,这是一个局部的增量公式.我们由此思考:如果函数$f(x)$在闭区间$\left[a,b\right]$上连续,在开区间$(a,b)$上可导,是否也有一个刻画函数“整体上”的增量公式?也就是说是否存在一点$\xi$,对于任意的$x_1,x_2\in\left[a,b\right]$,有
$$f(x_1)-f(x_2)=f'(\xi)(x_1-x_2)?$$

从几何上看,我们要研究的问题就是是否存在$\xi$使得$f(x)$在$(\xi,f(\xi))$处的切线与经过$(x_1,f(x_1))$和$(x_2,f(x_2))$的直线平行.这个满足条件的$\xi$就称为{\heiti 中值}(mean value).我们关心中值的存在性,而并不关心它的具体位置.我们有以下Lagrange中值定理.
\begin{theorem}[Lagrange中值定理]
	若函数$f$满足:
	
	(i)$f$在闭区间$\left[a,b\right]$上连续;
	
	(ii)$f$在开区间$(a,b)$上可导,\\
	则在$(a,b)$上至少存在一点$\xi$使得
	\begin{equation}{\label{equ:lagrange}}
		f'(\xi)=\frac{f(b)-f(a)}{b-a}.
	\end{equation}
\end{theorem}
要证明此定理,我们可以先证明它的一个特殊情形——Rolle定理.
\begin{theorem}[Rolle定理]
	若函数$f$满足:
	
	(i)$f$在闭区间$\left[a,b\right]$上连续;
	
	(ii)$f$在开区间$(a,b)$上可导;
	
	(iii)$f(a)=f(b)$,\\
	则在$(a,b)$上至少存在一点$\xi$,使得
	$$f'(\xi)=0.$$
\end{theorem}
\begin{proof}
	因为$f$在闭区间$\left[a,b\right]$上连续,由最值定理,$f(x)$都最大值和最小值,分别用$M$和$m$来表示,下面分两种情况讨论:
	\begin{enumerate}
		\item $M=m$,则$f$在$\left[a,b\right]$上必为常数,从而结论显然成立.
		\item $M>m$,则因$f(a)=f(b)$,使得最大值$M$与最小值$m$至少有一个在$(a,b)$上的某点处取到,从而$\xi$是$f$的极值点.由Fermat定理,有
		$$f'(\xi)=0.$$
	\end{enumerate}
	$\hfill\blacksquare$
\end{proof}
下面是对Lagrange定理的证明.
\begin{proof}
	作辅助函数
	$$F(x)=f(x)-f(a)-\frac{f(b)-f(a)}{b-a}(x-a).$$
	显然$F(a)=F(b)$,且$F$满足Rolle定理的另两个条件,因此存在$\xi\in(a,b)$使
	$$F'(\xi)=f'(\xi)-\frac{f(b)-f(a)}{b-a}=0,$$
	即$$f'(\xi)=\frac{f(b)-f(a)}{b-a}.$$
	$\hfill\blacksquare$
\end{proof}

Lagrange定理的几何意义:在满足定理条件的曲线$y=f(x)$上至少存在一点$P(\xi,f(\xi))$,使得该曲线在该点处的切线平行于曲线两端点的连线$AB$.我们在证明中引入的辅助函数$F(x)$,正是曲线$y=f(x)$与直线$AB(y=f(a)+\frac{f(b)-f(a)}{b-a}(x-a))$之差.


Lagrange定理中的公式\ref{equ:lagrange}称为{\heiti Lagrange公式}.其还有几种等价形式如下:
\begin{align}
	f(b)-f(a)=f'(\xi)(b-a),\qquad a<\xi<b;\\
	\label{eq1}	f(b)-f(a)=f'(a+\theta(b-a))(b-a),\qquad 0<\theta<1;\\
	\label{eq2}	f(a+h)-f(a)=f'(a+\theta h)h,\qquad 0<\theta<1.
\end{align}

值得注意的是,Lagrange公式无论对于$a<b$还是$a>b$都成立,而$\xi$是介于$a$与$b$之间的一个确定的数.而\ref{eq1}和\ref{eq2}两式的特点在于把中值点$\xi$表示成了$a+\theta(b-a)$,使得不论$a,b$为何值,$\theta$总可为小于$1$的某一正数.
\begin{corollary}
	若函数$f$在区间$I$上可导,且$f'(x)\equiv 0,\ x\in I$,则$f$为$I$上的一个常量函数.
\end{corollary}
\begin{proof}
	任取两点$x_1,x_2\in I(\text{不妨设}x_1<x_2)$,在区间$\left[x_1,x_2\right]$上应用Lagrange定理,存在$\xi\in(x_1,x_2)\subset I$,使得
	$$f(x_2)-f(x_1)=f'(\xi)(x_2-x_1)=0.$$
	这就证得$f$在区间$I$上任何两点之值相等.
\end{proof}
进一步,我们有以下推论.
\begin{corollary}
	若函数$f$和$g$均在区间$I$上可导,且$f'(x)\equiv g'(x),\ x\in I$,则在区间$I$上$f(x)$与$g(x)$只相差某一常数,即
	$$f(x)=g(x)+c\ (c\text{为某一常数}).$$
\end{corollary}
\begin{corollary}[导数极限定理]
	设函数$f$在点$x_0$的某邻域$U(x_0)$上连续,在$\mathring{U}(x_0)$上可导,且极限$\lim\limits_{x\to x_0}f'(x)$存在,则$f$在点$x_0$可导,且
	$$f'(x_0)=\lim\limits_{x\to x_0}f'(x).$$
\end{corollary}
\begin{proof}
	分别按左、右导数证明结论成立.
	
	任取$x\in \mathring{U}_+(x_0)$,$f(x)$在$\left[x_0,x\right]$上满足Lagrange定理的条件,则存在$\xi\in(x_0,x)$,使得
	$$\frac{f(x)-f(x_0)}{x-x_0)}=f'(\xi).$$
	由于$x_0<\xi<x$,因此当$x\to x_0^+$时,有$\xi\to x_0^+$,对上式取极限,得
	$$\lim\limits_{x\to x_0^+}\frac{f(x)-f(x_0)}{x-x_0}=\lim\limits_{x\to x_0^+}f'(\xi)=f'(x_0+0).$$
	同理可得$f'_-(x_0)=f'(x_0-0).$
	
	因为$\lim\limits_{x\to x_0}f'(x)=k$存在,所以$f'(x_0+0)=f'(x_0-0)=k$,从而$f'_+(x_0)=f'_-(x_0)=k$,即$f'(x_0)=k$.
	$\hfill\blacksquare$
\end{proof}
\begin{remark}
	证明过程中要注意:$f'_+(x_0)=\lim\limits_{x\to x_0^+}\dfrac{f(x)-f(x_0)}{x-x_0}$,指的是$f(x)$在$x_0$处的左导数;而$f'(x_0+0)=\lim\limits_{x\to x_0}f'(x)$,指的是导函数$f'(x)$趋近于$x_0$时的左极限.该定理的本质说明了函数的左右导数等于导函数的左右极限,因此函数在某一点的导数等于其导函数在该点的极限.符合定理条件的$f(x)$的导函数在$x_0$点连续.导数极限定理适合于求分段函数在分段点处的导数.
\end{remark}
\section{Cauchy中值定理与L'Hospital法则}
\subsection{Cauchy中值定理}
Cauchy中值定理是形式更一般的微分中值定理,定理内容如下.
\begin{theorem}[Cauchy中值定理]
	设函数$f$和$g$满足:
	\begin{enumerate}
		\item 在闭区间$\left[a,b\right]$上都连续;
		\item 在开区间$(a,b)$上都可导;
		\item $f'(x)$和$g'(x)$不同时为零;
		\item $g(a)\neq g(b)$,
	\end{enumerate}
	则存在$\xi\in(a,b)$,使得
	$$\frac{f'(\xi)}{g'(\xi)}=\frac{f(b)-f(a)}{g(b)-g(a)}.$$
\end{theorem}
\begin{proof}
	作辅助函数
	$$F(x)=f(x)-f(a)-\frac{f(b)-f(a)}{g(b)-g(a)}(g(x)-g(a)).$$
	容易看出$F$在$\left[a,b\right]$上满足Rolle定理条件,故存在$\xi\in(a,b)$,使得
	$$F'(\xi)=f'(\xi)-\frac{f(b)-f(a)}{g(b)-g(a)}g'(\xi)=0.$$
	因为$g'(\xi)\neq 0$(否则由上式$f'(\xi)$也为零),所以可把上式改写成Cauchy定理的结论.
	$\hfill\blacksquare$
\end{proof}
\hspace*{\fill}

Cauchy中值定理与Lagrange定理的几何意义相类似.只是现在要把$f,g$写作以$x$为参量的参量方程
\begin{equation*}
	\left\{
	\begin{aligned}
		u=g(x),\\
		v=f(x),
	\end{aligned}
	\right.
\end{equation*}
它在$uOv$平面表示一段曲线.由于$\dfrac{f(b)-f(a)}{b-a}$表示连接该曲线两端的弦$AB$的斜率,而$$\frac{f'(\xi)}{g'(\xi)}=\frac{\d v}{\d u}\bigg|_{x=\xi}$$则表示该曲线上与$x=\xi$相对应的一点$(g(\xi),f(\xi))$处的切线的斜率,因此Cauchy中值定理的结论表明上述切线与弦$AB$互相平行.
\subsection{L'Hospital法则}
我们在极限理论中学习无穷小(大)量阶的比较时,已经遇到过两个无穷小量或两个无穷大量之比的极限.由于这种极限可能存在也可能不存在,因此我们将两个无穷小量或两个无穷大量之比的极限统称为{\heiti 不定式极限},分别记为$\dfrac{0}{0}$型或$\dfrac{\infty}{\infty}$型的不定式极限.下面我们将以导数研究不定式极限,这个方法通常称为{\heiti L'Hospital法则}.

{\heiti 1.\ $\dfrac{0}{0}$型不定式极限}
\begin{theorem}
	若函数$f$和$g$满足:
	\begin{enumerate}
		\item $\lim\limits_{x\to x_0}f(x)=\lim\limits_{x\to x_0}g(x)=0$;
		\item 在点$x_0$的某空心邻域$\mathring{U}(x_0)$上两者都可导,且$g'(x)\neq 0$;
		\item $\lim\limits_{x\to x_0}\frac{f'(x)}{g'(x)}=A$($A$为扩充后的任一实数),
	\end{enumerate}
	则
	$$\lim\limits_{x\to x_0}\frac{f(x)}{g(x)}=\lim\limits_{x\to x_0}\frac{f'(x)}{g'(x)}=A.$$
\end{theorem}
\begin{proof}
	补充定义$f(x_0)=g(x_0)=0$,使得$f$与$g$在点$x_0$连续.任取$x\in \mathring{U}(x_0)$,在区间$\left[x_0,x\right]$上应用Cauchy中值定理,有
	$$\frac{f(x)-f(x_0)}{g(x)-g(x_0)}=\frac{f'(\xi)}{g'(\xi)}\qquad x_0<\xi<x,$$即
	$$\frac{f(x)}{g(x)}=\frac{f'(\xi)}{g'(\xi)}.$$
	当令$x\to x_0$时,也有$\xi\to x_0$,故得
	$$\lim\limits_{x\to x_0}\frac{f(x)}{g(x)}=\lim\limits_{x\to x_0}\frac{f'(\xi)}{g'(\xi)}=\lim\limits_{x\to x_0}\frac{f'(x)}{g'(x)}=A.$$
	$\hfill\blacksquare$
\end{proof}
\begin{remark}
	对于其它极限过程,只要相应地修正条件2中的邻域即可得到同样的结论.
\end{remark}
{\heiti 2.\ $\dfrac{\bullet}{\infty}$型不定式极限}
\begin{theorem}
	若函数$f$和$g$满足:
	\begin{enumerate}
		\item 在$x_0$的某个右邻域$\mathring{U}(x_0)$上二者可导,且$g'(x)\neq 0$;
		\item $\lim\limits_{x\to x_0^+}g(x)=\infty$;
		\item $\lim\limits_{x\to x_0^+}\dfrac{f'(x)}{g'(x)}=A$($A$为扩充后的任一实数),
	\end{enumerate}
	则$$\lim\limits_{x\to x_0^+}\frac{f(x)}{g(x)}=A.$$
\end{theorem}
\begin{proof}
	先设$A\in \mathbb{R}$,则对任意$\varepsilon>0$,存在$\delta>0$,当$x\in(x_0,x_0+\delta)$时,有
	$$A-\varepsilon<\frac{f'(x)}{g'(x)}<A+\varepsilon.$$
	对于$(x,c)\subseteq (x_0,x_0+\delta)$,由Cauchy中值定理可知,存在$\xi\in(x,c)$使得
	$$\frac{f(x)-f(c)}{g(x)-g(c)}=\frac{f'(\xi)}{g'(\xi)}.$$
	因此
	$$A-\varepsilon<\frac{f(x)-f(c)}{g(x)-g(c)}<A+\varepsilon.$$
	即
	\begin{equation}{\label{proofhospital}}
		A-\varepsilon<\frac{\frac{f(x)}{g(x)}-\frac{f(c)}{g(x)}}{1-\frac{g(c)}{g(x)}}<A+\varepsilon.
	\end{equation}
	由于$\lim\limits_{x\to x_0^+}g(x)=\infty$,固定$c$,令不等式\ref{proofhospital}的右边的$x\to x_0^+$取上极限得
	$$\limsup\limits_{x\to x_0^+}\frac{f(x)}{g(x)}\leqslant A+\varepsilon.$$
	令$\varepsilon\to 0$,得
	$$\limsup\limits_{x\to x_0^+}\frac{f(x)}{g(x)}\leqslant A.$$
	同理,固定$c$,对不等式\ref{proofhospital}左边的$x\to x_0^+$取下极限得
	$$\liminf\limits_{x\to x_0^+}\frac{f(x)}{g(x)}\geqslant A-\varepsilon.$$
	令$\varepsilon\to 0$,得
	$$\liminf\limits_{x\to x_0^+}\frac{f(x)}{g(x)}\geqslant A.$$
	所以有
	$$\limsup\limits_{x\to x_0^+}\frac{f(x)}{g(x)}=\liminf\limits_{x\to x_0^+}\frac{f(x)}{g(x)}=\lim\limits_{x\to x_0^+}\frac{f(x)}{g(x)}=A.$$
	类似的可以证明$A=\pm\infty$或$\infty$的情形和其他极限过程,在此不再赘述.$\hfill\blacksquare$
\end{proof}
与Stolz定理一样,我们也有L'Hospital法则的推广形式.
\begin{theorem}
	设函数$f$和$g$在开区间$(a,b)$可导,$g(x)\neq 0(\forall x\in(a,b)\backslash\{x_0\})$,则对任意$x_0\in(a,b)$,有
	$$\liminf\limits_{x\to x_0}\frac{f'(x)}{g'(x)}\leqslant\liminf\limits_{x\to x_0}\frac{f(x)}{g(x)}\leqslant\limsup\limits_{x\to x_0}\frac{f(x)}{g(x)}\leqslant\limsup\limits_{x\to x_0}\frac{f'(x)}{g'(x)}.$$
\end{theorem}
\begin{remark}
	对其他极限过程,也有以上结论.
\end{remark}
从以上定理可看出,当$x\to x_0$时,若$\dfrac{f'(x)}{g'(x)}$存在,则可以断言$\dfrac{f(x)}{g(x)}$一定存在;反之,若$\dfrac{f'(x)}{g'(x)}$不存在,却不一定能说明$\dfrac{f(x)}{g(x)}$不存在.
\section{Taylor公式}
多项式函数是各类函数中最简单的一种,用多项式逼近函数是近似计算和理论分析的一个重要内容.
\subsection{带有Peano型余项的Taylor公式}
我们在学习导数和微分概念时已经知道,如果函数$f$在点$x_0$处可导,则有
$$f(x)=f(x_0)+f'(x_0)(x-x_0)+o(x-x_0).$$
即在点$x_0$附近,用一次多项式$f(x_0)+f'(x_0)(x-x_0)$逼近函数$f(x)$时,其误差为$(x-x_0)$的高阶无穷小量.然而在很多时候,用一次多项式逼近是不够的.下面我们研究用$n$次多项式逼近,则其误差为$o((x-x_0))$.

我们探索如下面形式的任一$n$次多项式:
$$p_n(x)=a_0+a_1(x-x_0)^1+a_2(x-x_0)^2+\cdots+a_n(x-x_0)^n=\sum_{k=0}^{n}a_k(x-x_0)^k.$$
逐次求其各阶导数,我们得到
$$p_n^{(k)}=k!a_k,\quad k=1,2,\cdots,n.$$
所以有
$$a_k=\frac{p_n^{(k)}}{k!}.$$
由此可见,多项式$p_n(x)$的各项系数由其在点$x_0$的各阶导数值所唯一确定.

对于一般函数$f$,设它在点$x_0$存在直到$n$阶的导数.由这些导数构造一个$n$次多项式
$$T_n(x)=f(x_0)+\frac{f'(x_0)}{1!}(x-x_0)+\frac{f''(x_0)}{2!}(x-x_0)^2+\cdots+\frac{f^{(n)}(x_0)}{n!}(x-x_0)^n=\sum_{k=0}^{n}\frac{f^{(k)}(x_0)}{k!}(x-x_0)^k.$$
我们将这个多项式$T_n$称为函数$f$在点$x_0$处的{\heiti Taylor多项式},$T_n(x)$的各项系数$\frac{f^{(k)}(x_0)}{k!}\ (k=1,2,\cdots,n)$称为{\heiti Taylor系数}.由上面对多项式系数的讨论,我们知道$f(x)$与其Taylor展开式$T_n(x)$在点$x_0$处有相同的函数值和相同的直至$n$阶的导数值,即
$$f^{(k)}(x_0)=T_n^{(k)}(x_0),\quad k=1,2,\cdots,n.$$
下面证明$f(x)-T_n(x)=o((x-x_0)^n)$.
\begin{proof}
	设$R_n(x)=f(x)-T_n(x),\ Q_n(x)=(x-x_0)^n$,即证
	$$\lim\limits_{x\to x_0}\frac{R_n(x)}{Q_n(x)}=0.$$
	易知
	$$R_n(x_0)=R_n'(x_0)=\cdots=R_n^{(n)}(x_0)=0,$$
	$$Q_n(x_0)=Q_n'(x_0)=\cdots=Q_n^{(n-1)}(x_0)=0,\ Q_n^{(n)}(x_0)=n!.$$
	它们满足L'Hospital法则中的“0/0”型,即
	\begin{enumerate}
		\item $\lim\limits_{x\to x_0}R_n^{(n-1)}(x)=\lim\limits_{x\to x_0}Q_n^{(n-1)}(x)=0$;
		
		\item 在点$x_0$的空心邻域$\mathring{U}(x_0)$上两者都可导,且$Q_n^{(n)}(x)=n!\neq 0$;
		
		\item $\lim\limits_{x\to x_0}\dfrac{R_n^{(n)}}{Q_n^{(n)}}=0$.
		
	\end{enumerate}
	故有
	$$\lim\limits_{x\to x_0}\frac{R_n^{(n-1)}}{Q_n^{(n-1)}}=\lim\limits_{x\to x_0}\frac{R_n^{(n)}}{Q_n^{(n)}}=0.$$
	以此类推,我们有
	$$\lim\limits_{x\to x_0}\frac{R_n}{Q_n}=0.$$
	$\hfill\blacksquare$
\end{proof}
现在我们有以下定理.
\begin{theorem}
	若函数$f$在点$x_0$存在直至$n$阶导数,则有$f(x)=T_n(x)+o((x-x_0)^n)$,即
	\begin{equation}{\label{peanotaylor}}
		f(x)=f(x_0)+f'(x_0)(x-x_0)+\frac{f''(x_0)}{2!}+\cdots+\frac{f^{(n)}}{n!}(x-x_0)^n+o((x-x_0)^n)=\sum_{k=0}^{n}\frac{f^{(k)}(x_0)}{k!}(x-x_0)^k.
	\end{equation}
\end{theorem}
我们将上述定理中的\ref{peanotaylor}式称为{\heiti Taylor公式},将$R_n(x)=f(x)-T_n(x)$称为Taylor公式的{\heiti 余项},形如$o((x-x_0)^n)$的余项称为{\heiti Peano型余项},因此\ref{peanotaylor}式又称为{\heiti 带有Peano型余项的Taylor公式}.\\
\begin{remark}
	若$f(x)$在$x_0$附近满足
	$$f(x)=p_n(x)+o((x-x_0)^n),$$
	其中$p_n(x)$为$n$阶多项式,这并不能说明$p_n(x)$必定是$f$的Taylor多项式.
\end{remark}
\begin{remark}
	满足$f(x)=p_n(x)+o((x-x_0)^n)$的多项式$p_n(x)$是唯一的.若函数$f$满足上述定理的条件,满足$f(x)=p_n(x)+o((x-x_0)^n)$的多项式$p_n(x)$只可能是$f$的Taylor多项式$T_n$.
\end{remark}
常用的Taylor公式是在$x_0=0$时的特殊形式:
$$f(x)=f(0)+f'(0)x+\frac{f''(0)}{2!}x^2+\cdots+\frac{f^{(n)}(0)}{n!}+o(x^n).$$
它称为带有Peano余项的{\heiti Maclaurin公式}.

下面给出一些常用的带有Peano余项的Maclaurin公式,读者可自行验证.
\begin{align*}
	&(1)\ \text{e}^x=1+x+\frac{x^2}{2}+\cdots+\frac{x_n}{n!}+o(x^n);\\
	&(2)\ \sin x=x-\frac{x^3}{3!}+\frac{x^5}{5!}+\cdots+(-1)^{m-1}\frac{x^{2m-1}}{(2m-1)!}+o(x^{2m});\\
	&(3)\ \cos x=1-\frac{x^2}{2!}+\frac{x^4}{4!}+\cdots+(-1)^m\frac{x^{2m}}{(2m)!}+o(x^{2m+1});\\
	&(4)\ \ln(1+x)=x-\frac{x^2}{2}+\frac{x^3}{3}+\cdots+(-1)^{n-1}\frac{x^n}{n}+o(x^n);\\
	&(5)\ (1+x)^\alpha=1+\alpha x+\frac{\alpha(\alpha-1)}{2!}+\cdots+\frac{\alpha(\alpha-1)\cdots(\alpha-n+1)}{n!}x^n+o(x^n);\\
	&(6)\ \frac{1}{1-x}=1+x+x^2+\cdots+x^n+o(x^n).&
\end{align*}

利用上述Maclaurin公式可间接求得其他一些函数的Maclaurin公式或Taylor公式,还可用来求某种类型的极限.
\subsection{带有Lagrange型余项的Taylor公式}
上面我们从微分近似出发,推广得到$n$次多项式逼近函数的Taylor公式\ref{peanotaylor}.Peano型余项是{\heiti 定性}的,下面我们将Taylor公式构造一个{\heiti 定量}形式的余项,即Lagrange余项,便于对误差进行具体的计算和估计.
\begin{theorem}[Taylor定理]
	若函数$f$在$\left[a,b\right]$上存在直至$n$阶的导函数,在$(a,b)$上存在$(n+1)$阶导函数,则对任意给定的$x,x_0\in\left[a,b\right]$,至少存在一点$\xi\in(a,b)$,使得
	\begin{equation}{\label{lagtaylor}}
		f(x)=f(x_0)+f'(x_0)(x-x_0)+\frac{f''(x_0)}{2}(x-x_0)^2+\cdots+\frac{f^{(n)}(x_0)}{n!}(x-x_0)^n+\frac{f^{(n+1)}(\xi)}{(n+1)!}(x-x_0)^{n+1}.
	\end{equation}
\end{theorem}
\begin{proof}
	作辅助函数
	$$F(t)=f(x)-\left[f(t)+f'(t)(x-t)+\cdots+\frac{f^{(n)}(t)}{n!}(x-t)^n\right],$$
	$$G(t)=(x-t)^{n+1}.$$
	所要证明的式\ref{lagtaylor}即为
	$$F(x_0)=\frac{f^{(n+1)}(\xi)}{(n+1)!}G(x_0).$$
	即证$$\frac{F(x_0)}{G(x_0)}=\frac{f^{(n+1)}(\xi)}{(n+1)!}.$$
	不妨设$x_0<x$,则$F(t)$与$G(t)$在$\left[x_0,x\right]$上连续,在$(x_0,x)$上可导,且
	$$F'(t)=-\frac{f^{(n+1)}(\xi)}{n!}(x-t)^n,$$
	$$G'(t)=-(n+1)(x-t)^n\neq 0.$$
	又因$F(x)=G(x)=0$,由Cauchy中值定理得
	$$\frac{F(x_0)}{G(x_0)}=\frac{F(x_0)-F(x)}{G(x_0)-G(x)}=\frac{F'(\xi)}{G'(\xi)}=\frac{f^{(n+1)}(\xi)}{(n+1)!},\quad \xi\in(x_0,x)\subset(a,b).$$
	$\hfill\blacksquare$
\end{proof}

式\ref{lagtaylor}同样称为{\heiti Taylor公式},它的余项为
$$R_n(x)=f(x)-T_n(x)=\frac{f^{(n+1)}(\xi)}{(n+1)!}(x-x_0)^{n+1},$$
$$\xi=x_0+\theta(x-x_0),\quad (0<\theta<1).$$
我们将该余项称为{\heiti Lagrange型余项}.因此\ref{lagtaylor}式又称为{\heiti 带有Lagrange型余项的Taylor公式}.

注意到$n=0$时,式\ref{lagtaylor}即为Lagrange中值公式
$$f(x)-f(x_0)=f'(\xi)(x-x_0).$$
所以Taylor定理可以看作Lagrange中值定理的推广.

当$x_0=0$时,得到Taylor公式
$$f(x)=f(0)+f'(0)x+\frac{f''(0)}{2!}x^2+\cdots+\frac{f^{(n)}(0)}{n!}x^n+\frac{f^{(n+1)}(\theta x)}{(n+1)!}x^{n+1}\quad (0<\theta<1).$$
上式称为带有Lagrange余项的Maclaurin公式.

下面将带有Peano余项的Maclaurin公式改写成了带有Lagrange余项的Maclaurin公式.($0<\theta<1$)
\begin{align*}
	(1)&\ \text{e}^x=1+x+\frac{x^2}{2}+\cdots+\frac{x_n}{n!}+\frac{e^{\theta x}}{(n+1)!}x^{n+1},\quad x\in(-\infty,+\infty);\\
	(2)&\ \sin x=x-\frac{x^3}{3!}+\frac{x^5}{5!}+\cdots+(-1)^{m-1}\frac{x^{2m-1}}{(2m-1)!}+(-1)^m\frac{\cos \theta x}{(2m+1)!}x^{2m+1},\quad x\in(-\infty,+\infty);\\
	(3)&\ \cos x=1-\frac{x^2}{2!}+\frac{x^4}{4!}+\cdots+(-1)^m\frac{x^{2m}}{(2m)!}+(-1)^{m+1}\frac{\cos \theta x}{(2m+2)!}x^{2m+2},\quad x\in(-\infty,+\infty);\\
	(4)&\ \ln(1+x)=x-\frac{x^2}{2}+\frac{x^3}{3}+\cdots+(-1)^{n-1}\frac{x^n}{n}+(-1)^n\frac{x^{n+1}}{(n+1)(1+\theta x)^{n+1}},\quad x>-1;\\
	(5)&\ (1+x)^\alpha=1+\alpha x+\frac{\alpha(\alpha-1)}{2!}+\cdots+\frac{\alpha(\alpha-1)\cdots(\alpha-n+1)}{n!}x^n+\\
	&\frac{\alpha(\alpha-1)\cdots(\alpha-n)}{(n+1)!}(1+\theta x)^{\alpha-n-1}x^{n+1},\quad x>-1;\\
	(6)&\ \frac{1}{1-x}=1+x+x^2+\cdots+x^n+\frac{x^{n+1}}{(1-\theta x)^{n+2}},\quad |x|<1.&
\end{align*}
\subsection{Taylor公式的应用}
\section{函数的单调性}
\begin{theorem}
	设函数$f(x)$在区间$I$上可导,则$f(x)$在$I$上递增的充要条件是$f'(x)\geqslant 0$;$f(x)$在$I$上递减的充要条件是$f'(x)\leqslant 0$.
\end{theorem}
\begin{proof}
	只需证递增的情况即可.设$f$为递增函数,则对每一$x_0\in I$,当$x\neq x_0$时,有
	$$\frac{f(x)-f(x_0)}{x-x_0)}\geqslant 0.$$
	令$x\to x_0$,即得$f'(x_0)\geqslant 0$.
	
	反之,若$f(x)$在区间$I$上恒有$f'(x)\geqslant 0$,则对任意$x_1,x_2\in I\ (\text{设}x_1<x_2)$,由Lagrange中值定理,存在$\xi\in(x_1,x_2)$,使得
	$$f(x_2)-f(x_1)=f'(\xi)(x_2-x_1)\geqslant 0$$.
	所以$f$在$I$上为递增函数.$\hfill\blacksquare$
\end{proof}
\begin{theorem}
	若函数$f$在$(a,b)$上可导,则$f$在$(a,b)$上严格递增(递减)的充要条件为
	
	(1)对$\forall x\in(a,b),\ f'(x)\geqslant 0\ (f'(x)\leqslant 0)$;
	
	(2)在$(a,b)$的任意子区间上$f'(x)\not\equiv 0$.
\end{theorem}
\begin{proof}
	用反证法.
	
	必要性\qquad (i)是显然的.对(ii),假设存在$(c,d)\subset(a,b)$,对于$\forall x\in(c,d)$,$f'(x)\equiv 0$,则对任意$x_1,x_2\in (c,d)(\text{设}x_1<x_2)$,对任意$\xi\in(x_1,x_2)$,有$f(x_2)-f(x_1)=f'(\xi)(x_2-x_1)=0$.与严格递增矛盾.
	
	充分性\qquad (i)是显然的.对(ii),假设不严格递增,则存在$(c,d)\subset(a,b)$,对于任意$x_1,x_2\in (c,d)(\text{设}x_1<x_2)$,对任意$\xi\in(c,d)$,有$f'(\xi)(x_2-x_1)=f(x_2)-f(x_1)=0$,由于$x_1\neq x_2$,故$f'(\xi)=0$,即对任意$x\in(c,d)$,$f'(x)\equiv 0$,这与(ii)矛盾.
	$\hfill\blacksquare$
\end{proof}
\begin{corollary}
	设函数在区间$I$上可微,若$f'(x)>0(<0)$,则$f$在$I$上严格递增(递减).
\end{corollary}
\begin{theorem}[Darboux定理]
	若函数$f$在$\left[a,b\right]$上可导,且$f'_+(a)\neq f'_-(b)$,$k$为介于$f'_+(a),f'_-(b)$之间任一实数,则至少存在一点$\xi\in(a,b)$,使得
	$$f'(\xi)=k.$$
\end{theorem}
\begin{proof}
	设$F=f(x)-kx$,则$F(x)$在$\left[a,b\right]$上可导,且
	$$F'_+(a)\cdot F'_-(b)=(f'_+(a)-k)(f'_-(b)-k)<0.$$
	不妨设$F'_+(a)>0,F'_-(b)<0$.由命题\ref{prooffermat},分别存在$x_1\in\mathring{U}_+(a),x_2\in\mathring{U}_-(b)$,且$x_1<x_2$,使得
	$$F(x_1)>F(a),F(x_2)>F(b).$$
	因为$F$在$\left[a,b\right]$可导,所以连续.根据最值定理,存在一点$\xi\in\left[a,b\right]$,使$F$在点$\xi$取得最大值.$\xi\neq a,b$,这就说明$\xi$是$F$的极大值点.由Fermat定理得$F'(\xi)=0$,即
	$$f'(\xi)=k.$$ $\hfill\blacksquare$
\end{proof}
\begin{remark}
	有时候称上述定理为{\heiti 导函数介值定理}.
\end{remark}
\begin{corollary}
	设函数$f(x)$在区间$I$上满足$f'(x)\neq 0$,那么$f(x)$在区间$I$上严格单调.
\end{corollary}
\section{函数的极值与最值}
函数的极值不仅在实际问题中占有重要的地位,而且也是函数性态的一个重要特征.

Fermat定理已经告诉我们,可导函数在点$x_0$取极值的必要条件是$f'(x_0)=0$.下面讨论充分条件.
\begin{theorem}[极值的第一充分条件]
	设$f$在点$x_0$连续,在某邻域$\mathring{U}(x_0;\delta)$上可导.
	
	(i)若当$x\in(x_0-\delta,x_0)$时$f'(x)\leqslant 0$,当$x\in (x_0,x_0+\delta)$时$f'(x)\geqslant 0$,则$f$在点$x_0$取得极小值;
	
	(ii)若当$x\in(x_0-\delta,x_0)$时$f'(x)\geqslant 0$,当$x\in (x_0,x_0+\delta)$时$f'(x)\leqslant 0$,则$f$在点$x_0$取得极大值.
\end{theorem}
\begin{proof}
	只需证(i)即可.由定理的条件,$f$在$(x_0-\delta,x_0)$上递减,在$(x_0,x_0+\delta)$上递增,又由$f$在点$x_0$上连续,故对任意$x\in U(x_0;\delta)$,恒有
	$$f(x)\geqslant f(x_0).$$
	即$f$在点$x_0$取得极小值.可类似证明(ii)的情况.$\hfill\blacksquare$
\end{proof}
\begin{theorem}[极值的第二充分条件]
	设$f$在$x_0$的某邻域$U(x_0;\delta)$上一阶可导,在$x_0$处二阶可导,且$f'(x_0)=0,f''(x_0)\neq 0$.
	
	(i)若$f''(x_0)<0$,则$f$在$x_0$取得极大值.
	
	(ii)若$f''(x_0)>0$,则$f$在$x_0$取得极小值.
\end{theorem}
\begin{proof}
	由条件,可得$f$在$x_0$处的二阶Taylor公式
	$$f(x)=f(x_0)+f'(x_0)(x-x_0)+\frac{1}{2!}f''(x_0)(x-x_0)^2+o((x-x_0)^2).$$
	由于$f'(x_0)=0$,因此
	$$f(x)-f(x_0)=\left[\frac{f''(x_0)}{2}+o(1)\right](x-x_0)^2.$$
	又因$f''(x_0)\neq 0$,故存在正数$\delta'\leqslant\delta$,当$x\in U(x_0;\delta')$时,$\dfrac{1}{2}f''(x_0)$和$\dfrac{1}{2}f''(x_0)+o(1)$同号.所以,当$f''(x_0)<0$时,$\dfrac{1}{2}f''(x_0)+o(1)<0$,从而对任意$x\in \mathring{U}(x_0;\delta')$,有
	$$f(x)-f(x_0)<0.$$
	即$f$在$x_0$取极大值.同理,对$f''(x_0)>0$,可得$f$在$x_0$取极小值.
	$\hfill\blacksquare$
\end{proof}
\begin{theorem}[极值的第三充分条件]
	设$f$在$x_0$的某邻域上存在直到$n-1$阶导函数,在$x_0$处$n$阶可导,且$f^{(k)}(x_0)\ (k=1,2,\cdots,n-1)$,$f^{(n)}(x_0)\neq 0$,则
	
	(i)当$n$为偶数时,$f$在$x_0$取得极值,且当$f^{(n)}(x_0)<0$时取极大值,$f^{(n)}(x_0)>0$时取极小值.
	
	(ii)当$n$为奇数时,$f$在$x_0$处不取极值.
\end{theorem}
\begin{proof}
	与极值的第二充分条件证明类似.$f$在$x_0$的$n$阶Taylor公式
	$$f(x)=f(x_0)+f'(x_0)(x-x_0)+\cdots+\frac{f^{(n)}(x_0)}{n!}(x-x_0)^{n}+o((x-x_0)^n).$$
	其中$f^{(k)}(x_0)\ (k=1,2,\cdots,n-1)$,则只剩
	$$f(x)-f(x_0)=\left[\frac{f^{(n)}(x_0)}{n!}+o(1)\right](x-x_0)^{(n)}.$$
	
	当$n$为偶数时,$(x-x_0)^n>0$,$x_0$两侧某邻域内$f(x)$同号.$f^{(n)}(x_0)<0$时,$f(x)-f(x_0)<0$.则$f(x)$在$x_0$处取极大值,同理,对$f^{(n)}(x_0)>0$,可得$f$在$x_0$取极小值.
	
	当$n$为奇数时,$(x-x_0)^n$在$x_0$的两侧异号,因此$f(x)-f(x_0)$也在$x_0$的两侧异号,故不取极值.$\hfill\blacksquare$
\end{proof}

根据闭区间上连续函数的基本性质,若函数$f$在闭区间$\left[a,b\right]$上连续,则$f$在$\left[a,b\right]$上一定有最大、最小值.这是我们求连续函数的最大、最小值的理论保证.若函数$f$的最大(小)值点$x_0$在开区间$(a,b)$上,则$x_0$必定是$f$的极大(小)值点.又若$f$在$x_0$可导,则$x_0$还是一个稳定点.所以我们只要比较$f$在区间内部的所有稳定点、不可导点和区间端点上的函数值,就能从中找到$f$在$\left[a,b\right]$上的最大值和最小值.
\section{函数的凸性与拐点}
\begin{definition}[凸函数和凹函数]
	设$f$为定义在区间$I$上的函数,若对$I$上的任意两点$x_1,x_2$和任意实数$\lambda\in(0,1)$,都有
	$$f(\lambda x_1+(1-\lambda)x_2)\leqslant\lambda f(x_1)+(1-\lambda)f(x_2),$$
	则称$f$为$I$上的{\heiti 凸函数}(convex function).反之,如果总有
	$$f(\lambda x_1+(1-\lambda)x_2)\geqslant\lambda f(x_1)+(1-\lambda)f(x_2),$$
	则称$f$为$I$上的{\heiti 凹函数}(concave function).
\end{definition}
如果将上述的不等式改为严格不等式,则相应的函数称为{\heiti 严格凸函数}和{\heiti 严格凹函数}.

容易证明:若$-f$为区间$I$上的凸函数,则$f$为区间$I$上的凹函数.因此,只需讨论凸函数的性质即可.
\begin{theorem}
	$f$为$I$上的凸函数的充要条件是:对于$I$上的任意三点$x_1<x_2<x_3$,总有
	$$\frac{f(x_2)-f(x_1)}{x_2-x_1}\leqslant\frac{f(x_3)-f(x_2)}{x_3-x_2}.$$
\end{theorem}
\begin{proof}
	必要性\qquad 记$\lambda=\dfrac{x_3-x_2}{x_2-x_1}$,则$x_2=\lambda x_1+(1-\lambda)x_3$.由$f$的凸性知道
	\begin{align*}
		f(x_2)&=f(\lambda x_1+(1-\lambda)x_3)\leqslant\lambda f(x_1)+(1-\lambda)f(x_3)\\
		&=\frac{x_3-x_2}{x_3-x_1}f(x_1)+\frac{x_2-x_1}{x_3-x_1}f(x_3),
	\end{align*}
	从而有$$(x_3-x_1)f(x_2)\leqslant(x_3-x_2)f(x_1)+(x_2-x_1)f(x_3),$$
	$$(x_3-x_2)f(x_2)+(x_2-x_1)f(x_2)\leqslant (x_3-x_2)f(x_1)+(x_2-x_1)f(x_3).$$
	整理后即得结论.
	
	充分性\qquad 在$I$上任取两点$x_1,x_3(x_1<x_3)$,在$\left[x_1,x_3\right]$上任取一点$x_2=\lambda x_1+(1-\lambda)x_3,\lambda\in(0,1)$,即$\lambda=\dfrac{x_3-x_2}{x_3-x_1}$.由必要性推导的逆过程,可得
	$$f(\lambda x_1+(1-\lambda)x_3)\leqslant\lambda f(x_1)+(1-\lambda)f(x_3),$$
	故$f$为$I$上的凸函数.$\hfill\blacksquare$
\end{proof}
\begin{remark}
	如果$f$是$I$上的严格凸函数,则定理中的“$\leqslant$”可改为“$<$”.
\end{remark}
\begin{remark}
	同理可证加强的定理
	$$\frac{f(x_2)-f(x_1)}{x_2-x_1}\leqslant\frac{f(x_3)-f(x_1)}{x_3-x_1}\leqslant\frac{f(x_3)-f(x_2)}{x_3-x_2}.$$
	如果$f$是$I$上的严格凸函数,则定理中的“$\leqslant$”亦可改为“$<$”.
\end{remark}
\begin{theorem}
	设$f$是区间$I$上的可导函数,则下述论断互相等价:
	\begin{enumerate}
		\item $f$为$I$上的凸函数;
		\item $f'$为$I$上的增函数;
		\item 对$I$上任意两点$x_1,x_2$,有
		$$f(x_2)\geqslant f(x_1)+f'(x_1)(x_2-x_1).$$
	\end{enumerate}
\end{theorem}
\begin{proof}
	$1\to 2$\qquad 任取$I$上两点$x_1,x_2(x_1<x_2)$及充分小的正数$h$.由于$x_1-h<x_1<x_2<x_2+h$,根据$f$的凸性有
	$$\frac{f(x_1)-f(x_1-h)}{h}\leqslant\frac{f(x_2)-f(x_1)}{x_2-x_1}\leqslant\frac{f(x_2+h)-f(x_2)}{h}.$$
	由$f$是可导函数,令$h\to 0$,得
	$$f'(x_1)\leq\frac{f(x_2-f(x_1))}{x_2-x_1}\leqslant f'(x_2),$$
	所以$f'$是$I$上的递增函数.
	
	$2\to 3$\qquad 在以$x_1,x_2(\text{不妨设}x_1<x_2)$为端点的区间上,由Lagrange定理和$f'$递增条件,有
	$$f(x_2)-f(x_1)=f'(\xi)(x_2-x_1)\geqslant f'(x_1)(x_2-x_1).$$
	移项后即得
	$$f(x_2)\geqslant f(x_1)+f'(x_1)(x_2-x_1).$$
	
	$3\to 1$\qquad 设$x_1,x_2$为$I$上任意两点,$x_3=\lambda x_1+(1-\lambda)x_2,\ \lambda\in(0,1)$.由论断3,并利用$x_1-x_3=(1-\lambda)(x_1-x_2)$与$x_2-x_3=\lambda(x_2-x_1)$,有
	$$f(x_1)\geqslant f(x_3)+f'(x_3)(x_1-x_3)=f(x_3)+(1-\lambda)f'(x_3)(x_1-x_2),$$
	$$f(x_2)\geqslant f(x_3)+f'(x_3)(x_2-x_3)=f(x_3)+\lambda f'(x_3)(x_2-x_1).$$
	分别用$\lambda$和$1-\lambda$乘上列两式并相加,得
	$$f(x_1)+(1-\lambda)f(x_2)\geqslant f(x_3)=f(\lambda x_1+(1-\lambda)x_2)$$
	即$f$为$I$上的凸函数.$\hfill\blacksquare$
\end{proof}
\begin{remark}
	论断3的几何意义是:曲线$y=f(x)$总是在它的任一切线的上方.这是可导凸函数的几何特征.对于凹函数也有类似结论.
\end{remark}
\begin{corollary}
	设$f$为区间$I$上的二阶可导函数,则在$I$上$f$为凸函数的充要条件是
	$$f''(x)\geqslant 0,\ x\in I.$$
	$f$为凹函数的充要条件是
	$$f''(x)\leqslant 0,\ x\in I.$$
\end{corollary}
\begin{definition}[拐点]
	设曲线$y=f(x)$在点$(x_0,f(x_0))$处有穿过曲线的切线.且在切点近旁,曲线在切线的两侧分别是严格凸和严格凹的,这时称点$(x_0,f(x_0))$为曲线$y=f(x)$的{\heiti 拐点}(inflection point).
\end{definition}
由定义可见,拐点正是凸和凹曲线的分界点.

易证下面有关拐点的定理.
\begin{theorem}
	若$f$在$x_0$二阶可导,则$(x_0,f(x_0))$是曲线拐点的必要条件是$f''(x_0)=0$.
\end{theorem}
\begin{theorem}
	设$f$在$x_0$可导,在某邻域$\mathring{U}(x_0)$上二阶可导.若在$\mathring{U}_+(x_0)$和$\mathring{U}_-(x_0)$上$f''(x_0)$的符号相反,则$(x_0,f(x_0))$为曲线的一个拐点.
\end{theorem}
然而,若$(x_0,f(x_0))$是曲线$y=f(x)$的一个拐点,$y=f(x)$在$x_0$处的导数不一定存在.例如$y=\sqrt[3]{x}$在$x=0$时的情况.
\section{方程的近似解}