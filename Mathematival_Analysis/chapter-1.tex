\chapter*{序}
数学分析是数学学习的基础课程之一。这份讲义是基于华东师范大学数学系编写的《数学分析》(第五版上下册)编写而成。我们依次介绍了实数理论、极限与连续、一元函数微分学、一元函数积分学、无穷级数、多元函数微分学、含参量积分、曲线曲面积分、重积分等内容。对书上的定理作了总结与补充,以整体的观点、层层递进的逻辑关系进行叙述。读者可能会在初次阅读时感到晦涩、难懂,这是因为此讲义的编写顺序并不符合我们的认知顺序。事实上,在数学史中,我们是先创造了微积分,后来因数学大厦的地基不够充实,从而发展了连续性理论、极限理论以及实数理论,这与我们的叙述方向刚好是相反的。由此可见,在历史的发展中,人们对数学的认知是由模糊与感觉转变成越来越精细的定义与逻辑的推理。

在数学的学习中,我们要找出各个分支的相同点与不同点。例如,对于上极限和下极限的知识,我们有集合的上极限和下极限,数列的上极限和下极限,函数的上极限和下极限,它们都是普遍存在的,并通过夹逼的方式来定义具体的极限。我们还可以从这里推广、延伸,例如利用定义Darboux上和与Darboux下和的方式来定义Darboux积分,在测度论中Lebesgue定义了外测度和内测度来定义Lebesgue测度。它们的思想都是相通的,值得我们细细体会。又例如,在反常积分部分,我们分别研究了无穷积分和瑕积分的敛散性,实际上,我们都是从定义、非负函数的判别法则和一般函数的判别法则出发的,给出了Cauchy准则、比较原则、Dirichlet判别法和Abel判别法。事实上,在数项级数、函数项级数敛散性的研究中也有类似的判别法,我们可以整体地看待这些法则。此外,还有很多很多相通的实例不胜枚举。