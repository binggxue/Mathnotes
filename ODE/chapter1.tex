\chapter{基本概念}
\section{微分方程及其解的定义}
\begin{definition}[常微分方程]
	已知$F(z_0,z_1,\cdots,z_{n+1})$为关于$z_0,z_1,\cdots,z_{n+1}$的已知函数,则关于$y=y(x)$的方程
	$$F\left(x,y,\frac{\d y}{\d x},\frac{\d^2y}{\d x^2},\cdots,\frac{\d^n y}{\d x^n}\right)=0$$
	或
	$$F(x,y,y',\cdots,y^{(n)})=0$$
	称为{\heiti 常微分方程}(Ordinary Differential Equation,简称ODE).称$x$为{\heiti 自变量},$y$为{\heiti 未知函数}.
\end{definition}
\begin{remark}
	导数实际出现的最高阶数$n$称为常微分方程的{\heiti 阶}.
\end{remark}
\begin{remark}
	若$F$关于$z_1,\cdots,z_{n+1}$为线性函数,则ODE为{\heiti 线性的},若$F$关于$z_{n+1}$为线性函数,则ODE为{\heiti 拟线性的}.
\end{remark}
\begin{remark}
	若$\dfrac{\partial F}{\partial z_0}\equiv 0$,则ODE是{\heiti 自治的}.
\end{remark}
\begin{example}
	例如,下面的方程都是常微分方程:
	\begin{align*}
		&(1)y''=-\frac{1}{y^2},\\
		&(2)y''=ky,\\
		&(3)y'=\alpha y,\\
		&(4)y'=x^2+y^2,\\
		&(5)\frac{\d^2 y}{\d x^2}=1+y^2,\\
		&(6)y''+yy'=x,\\
		&(7)\frac{\d^2\theta}{\d t^2}+a^2\theta=0.
	\end{align*}
\end{example}
\begin{example}
	例如,下面的方程都不是常微分方程:
	\begin{align*}
		&(1)y'(x)=y(y(x)),\\
		&(2)y'(x)=y(x-1),\\
		&(3)\int_{0}^{x}y(t)\d t+y(x)=x.
	\end{align*}
\end{example}
在ODE的例子中,(1)(2)(3)是线性的,除(5)外都是拟线性的.可见,所有线性ODE都是拟线性的,但不是拟线性的ODE都是线性的,也就是说,线性是拟线性的一种特殊情形.

从自治性来看,(1)(2)(3)(5)(7)都是自治的,其余是非自治的.我们可以看出,自治的ODE是不含关于$x$的项的.

对未知函数的个数进行推广,我们就得到了常微分方程组:
$$\mathbf{F}(x,\mathbf{y},\mathbf{y}',\cdots,\mathbf{y}^{(n)})=\mathbf{0}.$$
其中$\mathbf{F}=(F_1,F_2,\cdots,F_m)$,$\mathbf{y}=(y_1,y_2,\cdots,y_m)$.这里$m\geqslant 2$,因为当$m=1$时就是ODE的定义.例如$m=2$的情形,有
\begin{equation*}
	\left\{
	\begin{aligned}
		&F_1(x,y_1,y_2,y_1',y_2',\cdots,y_1^{(n)},y_2^{(n)})=0\\
		&F_2(x,y_1,y_2,y_1',y_2',\cdots,y_1^{(n)},y_2^{(n)})=0
	\end{aligned}
	\right.
\end{equation*}

对自变量的个数进行推广,我们就得到了{\heiti 偏微分方程}(Partial Differential Equation).例如
$$x\frac{\partial f}{\partial x}+y\frac{\partial f}{\partial y}+z\frac{\partial f}{\partial z}+f=0$$
是一阶线性偏微分方程. 其中$x,y$和$z$为自变量,$f=f(x,y,z)$为未知函数.本课程主要讲述常微分方程,有时简称其为微分方程或方程.

\begin{definition}
	设函数$y=\varphi(x)$在区间$J$上连续,且有直到$n$阶的导数.如果把$y=\varphi(x)$及其相应的各阶导数代入$F(x,y,y',\cdots,y^{(n)})=0$,得到关于$x$的恒等式,即
	$$F(x,\varphi(x),\varphi(x)',\cdots,\varphi(x)^{(n)})=0$$
	对一切$x$都成立,则称$y=\varphi(x)$为微分方程在$J$上的一个{\heiti 解}.
\end{definition}
\begin{example}
	$$\frac{\d^2\theta}{\d t^2}+a^2\theta=0.$$
\end{example}
\begin{solution}
	对任意的常数$C_1,C_2$,
	$$\theta=C_1\sin at+C_2\cos at$$
	是$(-\infty,+\infty)$上的一个解.
\end{solution}
微分方程的解可以包含一个或几个任意常数(与方程的阶数有关),而有的解不包含任意常数. 为了确切表达任意常数的个数,我们定义通解和特解的概念.
\begin{definition}[通解与特解]
	设$n$阶微分方程$F(x,y,y',\cdots,y^{(n)})=0$的解
	$$y=\varphi(x,C_1,C_2,\cdots,C_n)$$
	包含$n$个{\heiti 独立的}任意常数$C_1,C_2,\cdots,C_n$,则称它为{\heiti 通解}.如果方程的解不包含任意常数,则称它为{\heiti 特解}.
\end{definition}
\begin{remark}
	当任意常数一旦确定之后,通解也就变成了特解.
\end{remark}
\begin{remark}
	这里所说的$n$个任意常数$C_1,C_2,\cdots,C_n$是独立的,其含义是Jacobi行列式不为$0$.
\end{remark}
定解问题:我们简单介绍Cauchy初值条件以及边值条件.
\begin{definition}[Cauchy初值条件]
	对于函数$y$,如果它满足:
	\begin{enumerate}[(1)]
		\item 它是微分方程$F(x,y,y',\cdots,y^{(n)})=0$的解;
		\item 它在同一点$x_0$处满足初始条件,取给定的值,即
		$$y(x_0)=y_0,\ y'(x_0)=y_1,\cdots,y^{(n-1)}(x_0)=y_{n-1}.$$
	\end{enumerate}
	则称$y$满足Cauchy初值条件.
\end{definition}
\begin{definition}[边值条件]
	边值条件是在求解微分方程时使用的一类条件,它们指定了方程解在定义域边界上的行为.
\end{definition}
在常微分方程情况下,边值条件通常涉及以下两种类型:
\begin{enumerate}[(1)]
	\item Dirichlet条件:直接指定了未知函数在边界上的值.例如对于区间上的二阶ODE,Dirichlet条件形如
	$$y(a)=\alpha,\quad y(b)=\beta,$$
	这里$\alpha,\beta$是给定的常数.
	\item Neumann条件:指定了未知函数在边界上的导数值.例如对于区间上的二阶ODE,Neumann条件形如
	$$y'(a)=\gamma,\quad y'(b)=\delta,$$
	这里$\gamma,\delta$也是给定的常数.
\end{enumerate}
除此以外,还有其他类型的边值条件,如混合边值条件(同时包含函数值和导数值的条件)和周期性边值条件(对于周期函数).

\section{几何解释}
考虑一阶微分方程
$$\frac{\d y}{\d x}=f(x,y),$$
其中$f(x,y)$是平面区域$G$内的连续函数. 假设
$$y=\varphi(x)\quad(x\in I)$$
是方程的解,(其中$I$是解的存在区间),则$y=\varphi(x)$在$Oxy$平面上是一条光滑的曲线$\varGamma$,称它为微分方程的{\heiti 积分曲线}或{\heiti 解曲线}.

由于$y=\varphi(x)$,我们可以将微分方程改写为
$$\varphi'(x)=f(x,y),$$
亦即$\varGamma$在其上任一点$P_0(x_0,y_0)$的切线斜率为$f(x_0,y_0)$,则切线方程为
$$y=y_0+f(x_0,y_0)(x-x_0),$$
即使我们并不知道积分曲线$\varGamma:y=\varphi(x)$是什么.

这样,在区域$G$内的每一点$P(x,y)$,我们可以做一个以$f(P)$为斜率的短小的直线段$l(P)$,来标明积分曲线(如果存在)在该点的切线方向. 称$l(P)$为微分方程在$P$点的{\heiti 线素},称区域$G$联同上述全体线素为微分方程的{\heiti 线素场}或{\heiti 方向场}.

由此可见,微分方程的任何积分曲线$\varGamma$与它的线素场是吻合的,即积分曲线所到之处与线素均相切. 反之,如果一条连续可微的曲线$\Lambda$与微分方程的线素场吻合,则$\Lambda$是微分方程的一条积分曲线.

\hspace*{\fill}

在构造方程$\frac{\d y}{\d x}=f(x,y)$的线素场时,通常利用由关系式$f(x,y)=k$确定的曲线$L_k$,称它为线素场的{\heiti 等斜线}. 显然,等斜线上各点线素的斜率都等于$k$,因此,等斜线简化了线素场逐点构造的方法.

\hspace*{\fill}

这里需指出,一阶微分方程$\dfrac{\d y}{\d x}=f(x,y)$在许多情况下取如下形式:
$$\frac{\d y}{\d x}=-\frac{P(x,y)}{Q(x,y)},$$
其中,$P(x,y)$和$Q(x,y)$是区域$G$内的连续函数.

当$Q(x_0,y_0)\neq 0$时,方程的右端函数$\dfrac{P(x,y)}{Q(x,y)}$在$(x_0,y_0)$点的近旁是连续的. 因此,方程的线素场在$(x_0,y_0)$点附近是完全确定的. 然而,如果$Q(x_0,y_0)=0$,那么线素场在$(x_0,y_0)$点就失去意义.

但是,只要$P(x_0,y_0)\neq 0$,我们就可以把方程改写为
$$\frac{\d y}{\d x}=-\frac{Q(x,y)}{P(x,y)},$$
这里需要把$x=x(y)$看作未知函数. 此时,微分方程的右端函数$\dfrac{Q(x,y)}{P(x,y)}$在$(x_0,y_0)$点近旁是连续的. 因此它在那里的线素场也是确定的.

这样,当$P(x_0,y_0)$和$Q(x_0,y_0)$不同时为零时,我们可以在$(x_0,y_0)$近旁考虑上述两个微分方程,虽然它们的未知函数略有不同. 此时,我们可以把它们统一写成下面的对称形式:
$$P(x,y)\d x+Q(x,y)\d y=0.$$

只是当$P(x_0,y_0)=Q(x_0,y_0)=0$时,上述的三个微分方程在$(x_0,y_0)$点都是不定式,因此线素场在$(x_0,y_0)$点没有意义. 我们称这样的点为相应微分方程的{\heiti 奇异点}.

虽然在奇异点微分方程是不定式,但是在积分曲线族的分布中奇异点是关键性的点. 之后我们引入动力系统的概念,这里的奇异点将称为相应动力系统的奇点.