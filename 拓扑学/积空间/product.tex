\documentclass[12pt]{ctexart}
\usepackage{amsfonts,amssymb,amsmath,amsthm,geometry,enumerate}
\usepackage[colorlinks,linkcolor=blue,anchorcolor=blue,citecolor=green]{hyperref}
%introduce theorem environment
\theoremstyle{definition}
\newtheorem{definition}{定义}
\newtheorem{theorem}{定理}
\newtheorem{lemma}{引理}
\newtheorem{corollary}{推论}
\newtheorem{property}{性质}
\newtheorem{example}{例}
\theoremstyle{plain}
\newtheorem*{solution}{解}
\newtheorem*{remark}{注}
\geometry{a4paper,scale=0.8}

%article info
\title{\vspace{-2em}\textbf{积空间}\vspace{-2em}}
\date{ }
\begin{document}
	\maketitle
	\begin{definition}[积拓扑]
		设$X$,$Y$为拓扑空间,任意开集$U\subset X$,$V\subset Y$,则集族$\beta=\{U\times V\}$是一个拓扑基,称$\beta$生成的$X\times Y$上的拓扑为\textbf{积拓扑}.
	\end{definition}
	需要说明$\beta$是一个拓扑基,因为$\bigcup \beta=X\times Y$,$\beta$中任意两个集合的交仍在$\beta$中.
	\begin{definition}[投影映射]
		对任意$x\in X,\ y\in Y$,映射$p_1:X\times Y\to X,\ (x,y)\to x$和$p_2:X\times Y\to Y,\ (x,y)\to y$称为$X\times Y$上的\textbf{投影映射}.
	\end{definition}
	\begin{theorem}
		设$X\times Y$带有积拓扑,则它的投影映射为连续映射,且把开集映射到开集.
	\end{theorem}
	\begin{proof}
		设开集$U\subset X$,映射$p_1:X\times Y\to X,\ (x,y)\to x$,则$p_1^{-1}(U)=U\cap Y$为开集,故$p_1$为连续映射. 只需考虑基础开集$U\times V\in\beta$,$p_1(U\times V)=U$为$X$中的开集,故$p_1$把开集映射到开集. 对于$p_2$同理.
	\end{proof}
	\begin{theorem}
		$X\times Y$上的积拓扑是满足投影映射连续的最小拓扑.
	\end{theorem}
	\begin{proof}
		设我们在$X\times Y$上有一些拓扑满足投影映射连续,设开集$U\subset X$,$V\subset Y$,则$p_1^{-1}(U)\cap p_2^{-1}(V)=U\times V$是开集. 这个拓扑包含了积拓扑中所有的基础开集,于是这个拓扑至少要和积拓扑一样大.
	\end{proof}
	下文中,若无特殊说明,$X\times Y$均指积拓扑空间.
	\begin{theorem}
		设$X$,$Y$,$Z$为拓扑空间,映射$f:Z\to X\times Y$连续当且仅当$p_1\circ f:Z\to X$和$p_2\circ f:Z\to Y$都连续.
	\end{theorem}
	\begin{proof}
		若$f$连续,连续映射的复合仍连续,故$p_1\circ f:Z\to X$和$p_2\circ f:Z\to Y$都连续.
		
		若$p_1\circ f$和$p_2\circ f$都连续,对任意开集$U\subset X$,$V\subset Y$,有$U\times V$是开集,且
		$$f^{-1}(U\times V)=(p_1\circ f)^{-1}(U)\cap(p_2\circ f)^{-1}(V),$$
		其中$(p_1\circ f)^{-1}(U)$和$(p_2\circ f)^{-1}(V)$都是开集,于是$f^{-1}(U\times V)$是开集,$f$连续.
	\end{proof}
	\begin{theorem}
		积空间$X\times Y$是Hausdorff空间当且仅当$X$和$Y$都是Hausdorff空间.
	\end{theorem}
	\begin{proof}
		若$X\times Y$是Hausdorff空间,对不同的$x_1,\ x_2\in X$,$y\in Y$,存在基础开集$U_1\times V_1,\ U_2\times V_2\subset X\times Y$满足$(x_1,y)\in U_1\times V_1$,$(x_2,y)\in U_2\times V_2$,$y\in V_1\cap V_2$,那么$U_1\cap U_2=\varnothing$. 对$y$同理.
		
		若$X$和$Y$均是Hausdorff空间,仅考虑$x_1\neq x_2$时即可. 则对不同的$x_1,\ x_2\in X$,有$U_1\ni x_1,\ U_2\ni x_2$,使得$U_1\cap U_2=\varnothing$,那么$\left(U_1\times Y\right)\cap\left(U_2\times Y\right)=\varnothing$.
	\end{proof}
	\begin{lemma}
		设$X$是拓扑空间,$\beta$是$X$的一个拓扑基,则$X$为紧的当且仅当$\beta$中的一些基础开集可以组成$X$的一个开覆盖,且存在有限的子覆盖.
	\end{lemma}
	\begin{proof}
		设$\beta$中的一些基础开集可以组成$X$的一个开覆盖,且存在有限的子覆盖,对$X$的任一开覆盖$\mathcal{F}$,对任意$\mathcal{F}$中的集合,都可以由$\beta$中的若干基础开集生成,取所有用到的基础开集,构成$\beta'$,则$\bigcup\beta'=\bigcup\mathcal{F}=X$,所以$\beta'$是$X$的一个开覆盖,且存在有限子覆盖. 那么对于$\beta'$的有限子覆盖中的基础开集,都能从$\mathcal{F}$中选出一个集合覆盖,这样构造了$\mathcal{F}$的有限子覆盖,故$X$是紧的. 反之显然.
	\end{proof}
	\begin{theorem}\label{procom}
		积拓扑空间$X\times Y$是紧的当且仅当$X$和$Y$都是紧的.
	\end{theorem}
	\begin{proof}
		若$X\times Y$是紧的,由于$p_1:X\times Y\to X$,$p_2:X\times Y\to Y$都是连续映射,于是$X$和$Y$都是紧的.
		
		若$X$和$Y$都是紧的,对于任意$x\in X$,考虑$p_2\big|_{\{x\}}:\{x\}\times Y\to Y$,则容易验证这是同胚映射. 由于$Y$是紧的,那么$\{x\}\times Y$是紧的,考虑基础开集组成的任一开覆盖存在有限子覆盖
		$$U_1^{x}\times V_1^{x},\ U_2^{x}\times V_2^{x},\cdots,\ U_{n_x}^x\times V_{n_x}^x$$
		覆盖$\{x\}\times Y$,而且这些$U_i^{x}$覆盖了$U^{x}=\bigcap_{i=1}^{n_x}U_i^{x}$.
		
		由于$X$是紧的,所以$X$存在有限的开覆盖
		$$U^{x_1},U^{x_2},\cdots,U^{x_s},$$
		于是
		$$X\times Y=\bigcup_{i=1}^sU^{x_i}\times Y\subset\bigcup_{i=1}^{s}\left(\bigcap_{j=1}^{n_{x_i}}U_j^{x_i}\times V_{j}^{x_i}\right)=\bigcup_{i=1}^{s}\bigcap_{j=1}^{n_{x_i}}\left(U_j^{x_i}\times V_{j}^{x_i}\right),$$
		即由基础开集组成的$X\times Y$的开覆盖存在有限子覆盖,故$X\times Y$是紧的.
	\end{proof}
	\begin{theorem}[Tychonoff]
		任意个紧空间的积空间仍是紧的.
	\end{theorem}
	\begin{remark}
		这里的“任意”包括有限和无限的情形,是定理\ref{procom}的延伸,Tychonoff定理最终被证明与选择公理等价.
	\end{remark}
\end{document}