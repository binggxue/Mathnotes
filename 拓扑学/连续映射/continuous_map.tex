\documentclass[12pt]{ctexart}
\usepackage{amsfonts,amssymb,amsmath,amsthm,geometry,color,enumerate}
\usepackage[colorlinks,linkcolor=blue,anchorcolor=blue,citecolor=green]{hyperref}
%introduce theorem environment
\theoremstyle{definition}
\newtheorem{definition}{定义}
\newtheorem{theorem}{定理}
\newtheorem{lemma}{引理}
\newtheorem{corollary}{推论}
\newtheorem{property}{性质}
\newtheorem{example}{例}
\theoremstyle{plain}
\newtheorem*{solution}{解}
\newtheorem*{remark}{注}
\geometry{a4paper,scale=0.8}

%remove the dot after the theoremstyle
%\makeatletter
%\xpatchcmd{\@thm}{\thm@headpunct{.}}{\thm@headpunct{}}{}{}
%\makeatother

%article info
\title{\vspace{-2em}\textbf{连续映射}\vspace{-2em}}
\date{ }
\begin{document}
	\maketitle
	\begin{definition}[连续映射]
		设拓扑空间$X$和$Y$,映射$f:X\to Y$,若对$Y$中的任一开集$U$,$f^{-1}(U)$为$X$中的开集,则称$f$为从$X$到$Y$的\textbf{连续映射}.
	\end{definition}
	\begin{definition}[同胚映射]
		设映射$f:X\to Y$满足以下条件:
		\begin{enumerate}
			\item $f$是双射;
			\item $f$是连续映射;
			\item $f^{-1}$是连续映射.
		\end{enumerate}
		则称$f$为从$X$到$Y$的\textbf{同胚映射},称拓扑空间$X$和$Y$\textbf{同胚}.
	\end{definition}
	\begin{theorem}
		两个连续映射的复合仍为连续映射.
	\end{theorem}
	\begin{proof}
		设$f:X\to Y$,$g:Y\to Z$为两个连续映射,则对$Z$中任意开集$U$,$g^{-1}(U)$为$Y$中的开集,$f^{-1}\left(g^{-1}(U)\right)$为$X$中的开集. 于是$g\circ f$为连续映射.
	\end{proof}
	\begin{theorem}
		设连续映射$f:X\to Y$,$A\subset X$,并有$A$上的子空间拓扑,则$f$在$A$上的限制映射$f\big|_A:A\to Y$也是连续映射.
	\end{theorem}
	\begin{proof}
		设开集$O\subset Y$,注意到$(f\big|_A)^{-1}(O)=f^{-1}(O)\cap A$为开集.
	\end{proof}
	\begin{definition}[恒等映射]
		设映射$\mathrm{id}:X\to X$,$\mathrm{id}(x)=x,\ x\in X$,称$\mathrm{id}$为$X$上的\textbf{恒等映射}.
	\end{definition}
	\begin{definition}[嵌入映射]
		设$A\subset X$,定义映射$i:A\to X$,$i(x)=x,\ x\in A$,称为$X$的\textbf{嵌入映射}.
	\end{definition}
	\begin{theorem}
		下面五个命题等价.
		\begin{enumerate}
			\item $f:X\to Y$是一个连续映射;
			\item 若$\beta$是$Y$的一个拓扑基,则$\beta$的任意元素的原像都是开的;
			\item 对任意$A\subset X$,$f(\overline{A})\subset\overline{f(A)}$;
			\item 对任意$B\subset Y$,$\overline{f^{-1}(B)}\subset f^{-1}(\overline{B})$;
			\item $Y$中任意闭集的原像都是闭的.
		\end{enumerate}
	\end{theorem}
	\begin{proof}
		1$\to$ 2 : $f$为连续映射,则开集的原像都是开集,而$\beta$中的元素为开集,则原像仍为开集.
		
		2$\to$ 3 :设$A\subset X$,显然$f(A)\subset\overline{f(A)}$,考虑$x\in \overline{A}\backslash A$,$f(x)\notin f(A)$,即证$f(x)$为$f(A)$的极限点. 考虑$Y$的拓扑基$\beta$,对$B\in \beta$,存在邻域$N$,使得$f(x)\in B\subset N$,下证$N\backslash\{f(x)\}\cap f(A)\neq\varnothing$.
		
		由于$B$的原像仍为开的,故$x\in f^{-1}(B)$为一开集,又$x$为$A$的极限点,故$f^{-1}(B)\backslash\{x\}\cap A\neq \varnothing$,于是$B\backslash\{f(x)\}\cap f(A)\neq\varnothing$,而$B\subset N$,故有$N\backslash\{x\}\cap f(A)\neq\varnothing$.
		
		3$\to$ 4 : $f^{-1}(B)\in X$,则$f\left(\overline{f^{-1}(B)}\right)\subset\overline{f\left(f^{-1}(B)\right)}\subset\overline{B}$,则
		$$\overline{f^{-1}(B)}= f^{-1}\circ f\left(\overline{f^{-1}(B)}\right)\subset f^{-1}(\overline{B}).$$
		
		4$\to$ 5 : 对任意闭集$B\subset Y$,有$B=\overline{B}$,则由命题4,有
		$$f^{-1}(B)=f^{-1}(\overline{B})\supset\overline{f^{-1}(B)},$$
		而显然$f^{-1}(B)\subset\overline{f^{-1}(B)}$,于是$f^{-1}(B)=\overline{f^{-1}(B)}$,故$f^{-1}(B)$为闭集.
		
		5$\to$ 1 : 对任意闭集$B\subset Y$,有$f^{-1}(B)$为闭集,则$Y\backslash B$和$X\backslash f^{-1}(B)$均为开集,而$f^{-1}(Y\backslash B)=X\backslash f^{-1}(B)$为开集,故$f$为连续映射.
	\end{proof}
\end{document}