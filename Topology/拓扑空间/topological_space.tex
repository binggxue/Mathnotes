\documentclass[12pt]{ctexart}
\usepackage{amsfonts,amssymb,amsmath,amsthm,geometry,color,enumerate}
\usepackage[colorlinks,linkcolor=blue,anchorcolor=blue,citecolor=green]{hyperref}
%introduce theorem environment
\theoremstyle{definition}
\newtheorem{definition}{定义}
\newtheorem{theorem}{定理}
\newtheorem{lemma}{引理}
\newtheorem{corollary}{推论}
\newtheorem{property}{性质}
\newtheorem{example}{例}
\theoremstyle{plain}
\newtheorem*{solution}{解}
\newtheorem*{remark}{注}
\geometry{a4paper,scale=0.8}

%remove the dot after the theoremstyle
%\makeatletter
%\xpatchcmd{\@thm}{\thm@headpunct{.}}{\thm@headpunct{}}{}{}
%\makeatother

%article info
\title{\vspace{-2em}\textbf{拓扑空间}\vspace{-2em}}
\date{ }

\begin{document}
	\maketitle
	拓扑无需保距,研究形状的连续变形性质. 首先定义拓扑空间.
	\begin{definition}[拓扑空间]
		设$X$是一非空集合,集类$\mathcal{T}\subset 2^X$,满足以下条件:
		\begin{enumerate}
			\item $\varnothing,\ X\in\mathcal{T}$;
			\item 对任意并封闭:$\forall U_a\in\mathcal{T},a\in I$,则$\bigcup_{a\in I}U_a\in\mathcal{T}$;
			\item 对有限交封闭:$\bigcap_{a=1}^{n}U_a\in\mathcal{T}$.
		\end{enumerate}
		则称$(X,\mathcal{T})$为一个\textbf{拓扑空间},$\mathcal{T}$为集合$X$上的\textbf{拓扑}. $\mathcal{T}$中的元素$U_a$称为\textbf{开集}.
	\end{definition}
	\begin{definition}
		$2^X$是$X$上最大的拓扑,称为\textbf{离散拓扑}.
	\end{definition}
	\begin{definition}
		$\{\varnothing\ ,X\}$是$X$上最小的拓扑,称为\textbf{平凡拓扑}.
	\end{definition}
	有了开集的定义,自然地定义它的补集为闭集.
	\begin{definition}[闭集]
		对$X$的子集$A$,若$X\backslash A$为开集,则称$A$为\textbf{闭集}.
	\end{definition}
	\begin{definition}[邻域]
		设$x\in X$,如果集合$N\subset X$满足:存在开集$U$,使得$x\in U\subset N$,则称$N$为$x$的\textbf{邻域},称$N\backslash\{x\}$为$x$的\textbf{去心邻域}.
	\end{definition}
	\begin{definition}[内核]
		$A$中包含的所有开集的并称为$A$的内核,记作$\mathring{A}$.
	\end{definition}
	\begin{theorem}
		$A$是开集当且仅当$A=\mathring{A}$.
	\end{theorem}
	\begin{proof}
		 设$A$是开集,$\forall x\in A$,$\exists U_a\in A$,使得$x\in U_a\subset\mathring{A}$,于是$A\subset\mathring{A}$; $\mathring{A}=\bigcup_{a\in I}U_a,\ U_a\in A$,因此$\mathring{A}\subset A$. 故$A=\mathring{A}$.
		
		反之,设$A=\mathring{A}=\bigcup_{a\in I}U_a$,由拓扑空间定义,开集的任意并仍为开集,即$A$为开集.
	\end{proof}
	\begin{definition}[极限点]
		设$x\in X$,$A\subset X$. 若$x$的任一去心邻域中均有至少一点属于$A$,即包含$x$的任一开集$U$中,有
		$$\left(U\backslash\{x\}\right)\cap A\neq\varnothing,$$
		则称$x$为$A$的\textbf{极限点},又称\textbf{聚点}.
	\end{definition}
	\begin{theorem}\label{closeeq}
		$A$是闭集当且仅当$A$包含了它的全部极限点.
	\end{theorem}
	\begin{proof}
		设$A$是闭集,则$X\backslash A$是开集,由于开集是它的任意元素的邻域,于是$X\backslash A$的任一点都不是$A$的极限点,于是$A$包含了它的全部极限点.
		
		反之,设$A$包含了它的全部极限点,则对任意$x\in X\backslash A$,总存在$x$的邻域$N$使得$N\cap A=\varnothing$,于是$N\subset X\backslash A$,故$X\backslash A$是它的任意元素的邻域,故$X\backslash A$为开集,$A$为闭集.
	\end{proof}
	\begin{definition}[导集]
		$A$的极限点的集合称为$A$的\textbf{导集},记作$A'$.
	\end{definition}
	\begin{definition}[闭包]
		$A$中的元素和它的极限点组成的集合,即$A$本身和$A$的导集的并集,称为$A$的\textbf{闭包},记作$\overline{A}$.
	\end{definition}
	由定义立即有$\overline{A}=A\cup A'$.
	\begin{theorem}
		$A$的闭包是包含$A$的最小闭集.
	\end{theorem}
	\begin{proof}
		对$\forall x\in X\backslash \overline{A}$,存在$x$的开邻域$U$不包含$A$的极限点,于是$x\in U\subset X\backslash \overline{A}$,于是$X\backslash\overline{A}$是开集,故$\overline{A}$是闭集. 
		
		对任意闭集$B\supset A$,由定理\ref{closeeq},$B$包含$A$的全部极限点以及$B$的极限点,于是$\overline{A}\subset B$. 故$A$的闭包是包含$A$的最小闭集.
	\end{proof}
	\begin{remark}
		该定理也就是说,$A$的闭包是包含$A$的全体闭集的交.
	\end{remark}
	\begin{corollary}
		$A$是闭集当且仅当$A=\overline{A}$.
	\end{corollary}
	\begin{definition}[稠密性]
		设$X$为拓扑空间,$A\subset X$,若$\overline{A}=X$,即对任意$x\in U\subset X$,$U\backslash\{x\}\cap A\neq\varnothing$,则称$A$是\textbf{稠密的}.
	\end{definition}
	\begin{definition}[边界]
		$A$的闭包与$X\backslash A$的交称为$A$的边界,记作$\partial A$.
	\end{definition}
	用数学语言描述,即$\partial A=\overline{A}\cap\overline{X\backslash A}$.
	\begin{definition}[拓扑子空间]
		设$(X,\mathcal{T})$为拓扑空间,$Y\subset X$,定义$Y$上的子空间拓扑:
		$$\mathcal{T}_Y=\{O\cap Y|O\in\mathcal{T}\}.$$
		称$(Y,\mathcal{T}_Y)$为$X$的\textbf{拓扑子空间}.
	\end{definition}
	\begin{definition}[有限补拓扑]
		设集合$X=\mathbb{R}$,赋予\textbf{有限补拓扑},对任意开集$U\subset\mathbb{R}$,$X\backslash U$是有限的或者为整个空间$X$.
	\end{definition}
	对任一无限集$A\subset X$,$X$中的任意元素$x$都是$A$的极限点. 反之,在这一拓扑下,有限集没有极限点.
	\begin{definition}[拓扑基]
		设拓扑空间$(X,\mathcal{T})$,若一族开集$\beta$通过任意的一些并可以得到$X$中的任一开集,则称$\beta$为\textbf{拓扑基},$\beta$中的元素称为\textbf{基础开集},也称$\mathcal{T}$为$\beta$\textbf{生成的拓扑}.
	\end{definition}
	等价地,我们可以说对任意$x\in X$及其邻域为$N$,都存在开集$O\subset \beta$,使得$x\in O\subset N$.
	\begin{theorem}
		设$\beta$是$X$的一个非空子集族,若$\beta$中元素的有限交仍在$\beta$中,$\bigcup\beta\in\beta$,则$\beta$给出$X$上的拓扑,且为其拓扑基.
	\end{theorem}
	\begin{proof}
		设$U,V\in\beta$,令
		$$\mathcal{T}=\left\{U=\bigcup_{V\in\beta'}V\bigg|\beta'\subset\beta\right\},$$
		下面证明$\mathcal{T}$是$X$上的拓扑.
		\begin{enumerate}
			\item $\varnothing,\ X\in\mathcal{T}$. 令$\beta'=\varnothing$,有$\varnothing=\bigcup_{V\in\beta'}V\in\mathcal{T}$.\ 令$\beta'=X$,有$X=\bigcup_{V\in\beta'}V\in\mathcal{T}$.
			\item 设$U_a\in\mathcal{T},\ a\in I$. 设$U_a=\bigcup_{V\in\beta_a}V$,其中$\beta_a\subset\beta$.则
			$$\bigcup_{a\in I}U_a=\bigcup_{a\in I}\left(\bigcup_{V\in\beta_a}V\right)=\bigcup_{V\in\beta_a,a\in I}V=\bigcup_{V\in\bigcup_{a\in I}\beta_a}V\in\mathcal{T}.$$
			\item 设$U_1=\bigcup_{V\in\beta_1}V$,$U_2=\bigcup_{V\in\beta_2}V$,这里$\beta_1, \beta_2\in\beta$. 有
			$$U_1\cap U_2=\bigcup_{V_1\in\beta_1,V_2\in\beta_2}V_1\cap V_2.$$
			对任意$x\in V_1\cap V_2$,存在$W_x\in \beta$,使得$X\in W_x\subset V_1\cap V_2$. 于是有
			$$V_1\cap V_2\subset\bigcup_{x\in V_1\cap V_2}W_x\subset V_1\cap V_2,$$
			于是
			$$V_1\cap V_2=\bigcup_{x\in V_1\cap V_2}W_x\in\mathcal{T}.$$
		\end{enumerate}
		所以$\mathcal{T}$是$X$上的拓扑,并且由$\mathcal{T}$的构造,$\beta$是$\mathcal{T}$的拓扑基.
	\end{proof}
\end{document}