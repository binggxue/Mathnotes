\documentclass[12pt]{ctexart}
\usepackage{amsfonts,amssymb,amsmath,amsthm,geometry,enumerate}
\usepackage[colorlinks,linkcolor=blue,anchorcolor=blue,citecolor=green]{hyperref}
%introduce theorem environment
\theoremstyle{definition}
\newtheorem{definition}{定义}
\newtheorem{theorem}{定理}
\newtheorem{lemma}{引理}
\newtheorem{corollary}{推论}
\newtheorem{property}{性质}
\newtheorem{example}{例}
\theoremstyle{plain}
\newtheorem*{solution}{解}
\newtheorem*{remark}{注}
\geometry{a4paper,scale=0.8}

%article info
\title{\vspace{-2em}\textbf{Tietze扩张定理}\vspace{-2em}}
\date{ }
\begin{document}
	\maketitle
	\begin{definition}[度量空间]
		设定义在集合$X\times X$上的一个实值函数$d$,对任意$x,y,z\in X$,满足
		\begin{enumerate}
			\item $d(x,y)\geqslant 0\iff x=y$取等;
			\item $d(x,y)=d(y,x)$;
			\item $d(x,y)+d(y,z)\geqslant d(x,z)$,
		\end{enumerate}
		则称$d$为$X$上的一个\textbf{度量},$(X,d)$为一个\textbf{度量空间}.
	\end{definition}
	度量空间是一种特殊的拓扑空间,我们赋予了开集的意义. 
	\begin{definition}[开球]
		对$x\in X$,定义$x$的\textbf{开球}$B(x,\varepsilon)=\left\{y\in X:d(x,y)<\varepsilon\right\}$,其中$\varepsilon>0$,称为开球$B$的\textbf{半径}.
	\end{definition}
	\begin{definition}
		设$O\subset X$,对任意$x\in O$,总存在开球$B(x,\varepsilon)\subset O$,则称$O$为\textbf{开集}.
	\end{definition}
	容易验证,这样定义的开集满足拓扑公理.
	
	度量空间中离散的两个点,可以被两个不交的开集包含,称这种性质为Hausdorff性. 一般地,可以定义Hausdorff空间.
	\begin{definition}
		对任意$x,y\in X$,$x\neq y$,若总存在开集$O_x$,$O_y$满足$O_x\cap O_y=\varnothing$,
		则称$X$为\textbf{Hausdorff空间}.
	\end{definition}
	并非所有拓扑空间都具有Hausdorff性,例如有限补拓扑空间,包含任意两个离散的点的开集总相交.
	\begin{definition}[点到点集的距离]
		设$x\in X$,$A\subset X$,定义$x$到$A$的距离
		$$d(x,A)=\inf\left\{d(x,y):y\in A\right\}.$$
	\end{definition}
	\begin{lemma}\label{conti}
		定义的$d(x,A)$对$x$是连续的.
	\end{lemma}
	\begin{proof}
		对任意$\varepsilon>0$,$x_0\in X$,使得$d(x,x_0)<{\varepsilon}/{2}$,存在$y\in A$满足$d(x_0,y)<d(x_0,A)+\varepsilon/2$,则有
		$$d(x,A)\leqslant d(x,y)\leqslant d(x,x_0)+d(x_0,y)<d(x_0,A)+\varepsilon,$$
		将$x$与$x_0$互换,有$d(x_0,A)<d(x,A)+\varepsilon$,故$|d(x,A)-d(x_0,A)|<\varepsilon$.
	\end{proof}
	\begin{lemma}
		$d(x,A)=0\iff x\in\overline{A}$.
	\end{lemma}
	\begin{proof}
		$\Rightarrow$: $d(x,A)=\inf d(x,y)=0 , y\in A$,于是对任意$\varepsilon>0$,存在$y_0$,有
		$$d(x,y_0)<\inf d(x,y)+\varepsilon=\varepsilon,$$
		 故$B(x,\varepsilon)\backslash\{x\}\cap A\neq\varnothing$.
		 
		$\Leftarrow$: $x\in\overline{A}$,故存在开集$U\ni x$满足$U\backslash\{x\}\cap A\neq\varnothing$,对任意$\varepsilon>0$,总存在$y_0\in A$,使得$d(x,y_0)<\varepsilon$,故$d(x,A)=\inf d(x,y_0)=0$.
	\end{proof}
	\begin{lemma}[Urysohn]\label{urysohn}
		设$A$,$B$为$X$上的两个不交闭集,则存在$X$上的连续函数$f$使得$|f(x)|\leqslant 1$,$f\big|_A=1$,$f\big|_B=-1$.
	\end{lemma}
	\begin{proof}
		因为$A$,$B$为不交闭集,则$d(x,A)+d(x,B)\neq 0$,定义函数
		$$f(x)=\frac{d(x,B)-d(x,A)}{d(x,A)+d(x,B)},$$
		则由引理\ref{conti},得$f(x)$为符合条件的连续函数.
	\end{proof}
	\begin{theorem}[Tietze]
		定义在度量空间中一闭集上的连续实值函数可以延拓到整个空间.
	\end{theorem}
	\begin{proof}
		设$F$为$X$上的一闭集,$f$是定义在$F$上的连续函数. 先考虑$f$有界的情形,即存在$M>0$,使得$|f(x)|\leqslant M$.
		
		设$A=\left\{x\in F:M/3 \leqslant f(x)\leqslant M\right\}$,$B=\left\{x\in F:-M\leqslant f(x)\leqslant -M/3\right\}$,由于连续映射下,闭集的原像仍为闭集,于是$A$,$B$为不交闭集. 由引理\ref{urysohn},定义函数
		$$g_1(x)=\frac{M}{3}\cdot\frac{d(x,B)-d(x,A)}{d(x,A)+d(x,B)},\qquad x\in X,$$
		则$|g_1(x)|\leqslant \dfrac{M}{3}$,$|f(x)-g_1(x)|\leqslant\dfrac{2M}{3}$,下面以$f(x)-g_1(x)$为新的连续函数来研究,则有$$|g_2(x)|\leqslant \dfrac{2}{3}\cdot\dfrac{M}{3},$$
		$$|f(x)-g_1(x)-g_2(x)|\leqslant\dfrac{2}{3}\cdot\dfrac{2M}{3}.$$
		同理,对任意$k\in\mathbb{Z}^{+}$,有
		$$|g_k(x)|\leqslant\left(\frac{2}{3}\right)^{k-1}\cdot\frac{M}{3},$$
		$$\left|f(x)-\sum_{i=1}^{k}g_i(x)\right|\leqslant\frac{2^kM}{3^k},$$
		由Weierstrass判别法可知$\sum g_k(x)$是一致收敛的,记其和函数为$g(x)$,则
		$$g(x)=\sum_{k=1}^{\infty}g_k(x)=f(x).$$
		且
		$$|g(x)|\leqslant\sum_{k=1}^{\infty}|g_k(x)|\leqslant M\sum_{k=1}^{\infty}\frac{2^{k-1}}{3^k}=M.$$
		当$f$无界时,考虑$\arctan f(x)$即可.
	\end{proof}
\end{document}