\documentclass[12pt]{ctexart}
\usepackage{amsfonts,amssymb,amsmath,amsthm,geometry,enumerate}
\usepackage[colorlinks,linkcolor=blue,anchorcolor=blue,citecolor=green]{hyperref}
\usepackage[all]{xy}
%introduce theorem environment
\theoremstyle{definition}
\newtheorem{definition}{定义}
\newtheorem{theorem}{定理}
\newtheorem{lemma}{引理}
\newtheorem{corollary}{推论}
\newtheorem{property}{性质}
\newtheorem{example}{例}
\theoremstyle{plain}
\newtheorem*{solution}{解}
\newtheorem*{remark}{注}
\geometry{a4paper,scale=0.8}

%article info
\title{\vspace{-2em}\textbf{商空间}\vspace{-2em}}
\date{ }
\begin{document}
	\maketitle
	\begin{definition}[分划]
		设$X$为拓扑空间,$\mathcal{P}=\left\{M_i:M_i\cap M_j=\varnothing, i\neq j,\bigcup_{i\in I}M_i=X\right\}$则称$\mathcal{P}$为$X$的一个\textbf{分划}.
	\end{definition}
	\begin{definition}[商拓扑]
		设$\mathcal{P}$是$X$的一个分划,$Y$中的元素是$\mathcal{P}$中的若干集合,定义满射$\pi:X\to Y$,$\pi(x)=M_{i_x}\ni x$,$Y$带有的拓扑是满足$\pi$连续的最大的拓扑. 即对任意$O\subset Y$,$O$是开集当且仅当$\pi^{-1}(O)\subset X$是开集,则称$Y$带有的拓扑为\textbf{商拓扑}.
	\end{definition}
	商拓扑空间$Y$就是把$X$分划后的每一类的点黏合为一个点得到的空间.
	\begin{theorem}\label{natrual}
		设$Y$为商空间,$Z$为任意拓扑空间,则$f:Y\to Z$连续当且仅当$f\circ\pi:X\to Z$连续.
	\end{theorem}
	\begin{proof}
		设开集$U\subset Z$,$f^{-1}(U)$是开集当且仅当$\pi^{-1}\circ f^{-1}(U)=\left(f\circ\pi\right)^{-1}(U)$是开集.
	\end{proof}
	\begin{definition}[商映射]
		设$f:X\to Y$是连续满射,$Y$上的拓扑是满足$f$连续的最大拓扑,即$O\subset Y$是开集当且仅当$f^{-1}(O)\subset X$是开集,则称$f$是\textbf{商映射}.
	\end{definition}
	下面推论可由\ref{natrual}推广得到.
	\begin{corollary}
		设$f:X\to Y$是商映射,$Z$是任意拓扑空间,则$g:Y\to Z$连续当且仅当$g\circ f:X\to Z$连续.
	\end{corollary}
	自然映射$\pi$是特殊的商映射,将$X$映为它的一个分划生成的商空间$Y_{*}$.
	\begin{theorem}
		设$f:X\to Y$是商映射,则$Y_{*}$和$Y$同胚.
	\end{theorem}
	\begin{proof}
		$\{f^{-1}(y):y\in Y\}$构成$X$的一个划分. 设$h:Y_{*}\to Y$,$h(f^{-1}(y))=y$. 则$h$是双射,$h=\pi\circ f^{-1}$连续,$h^{-1}=f\circ\pi^{-1}$连续,于是$h$为同胚映射.
	\end{proof}
	\begin{remark}
		为便于理解,画出交换图如下. 
	\end{remark}
	\begin{displaymath}
		\xymatrix{ 
			X \ar[r]^f \ar[dr]_{\pi} & Y \ar[d]^h \\
					 		& Y_{*}  
		}
	\end{displaymath}
	\begin{theorem}
		设$f:X\to Y$是连续满射,若$f$将开集映为开集,或将闭集映为闭集,则$f$是商映射.
	\end{theorem}
	\begin{proof}
		不妨设$f$将开集映为开集,设$U\subset Y$,$f^{-1}(U)\subset X$为开集当且仅当$U=f\circ f^{-1}(U)\subset Y$为开集,于是$f$是商映射,闭集时也可类似证明.
	\end{proof}
	\begin{corollary}
		设连续满射$f:X\to Y$,$X$为紧空间,$Y$为Hausdorff空间,则$f$是商映射.
	\end{corollary}
	\begin{proof}
		只需证明$f$将闭集映为闭集. 设闭集$C\subset X$,则$C$为紧的,$f(C)$为紧的,而$Y$是Hausdorff空间,于是$f(C)$是闭的.
	\end{proof}
	\begin{definition}
		若映射$f$将开集映为开集,则$f$称为\textbf{开映射},若映射$g$将闭集映为闭集,则称$g$为\textbf{闭映射}.
	\end{definition}
	由上可知,开或闭连续满射都是商映射,但商映射不一定是开的或闭的. 例如,存在商映射,既不是开的也不是闭的.
\end{document}