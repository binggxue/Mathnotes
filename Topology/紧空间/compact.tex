\documentclass[12pt]{ctexart}
\usepackage{amsfonts,amssymb,amsmath,amsthm,geometry,enumerate}
\usepackage[colorlinks,linkcolor=blue,anchorcolor=blue,citecolor=green]{hyperref}
%introduce theorem environment
\theoremstyle{definition}
\newtheorem{definition}{定义}
\newtheorem{theorem}{定理}
\newtheorem{lemma}{引理}
\newtheorem{corollary}{推论}
\newtheorem{property}{性质}
\newtheorem{example}{例}
\theoremstyle{plain}
\newtheorem*{solution}{解}
\newtheorem*{remark}{注}
\geometry{a4paper,scale=0.8}

%article info
\title{\vspace{-2em}\textbf{紧空间}\vspace{-2em}}
\date{ }

\begin{document}
	\maketitle
	\begin{definition}[开覆盖]
		设$X$为拓扑空间,若存在一族开集组成的集族$\mathcal{F}$,使得$\bigcup_{i\in I}O_i=X,\ O_i\in\mathcal{F}$,则称$\mathcal{F}$为$X$的一个\textbf{开覆盖}.
	\end{definition}
	\begin{definition}[子覆盖]
		设$\mathcal{F}$是$X$的一个开覆盖,若存在$\mathcal{F}'\subset\mathcal{F}$,且$\mathcal{F}'$也是$X$的一个开覆盖,则称$\mathcal{F}'$是$\mathcal{F}$的\textbf{子覆盖}.
	\end{definition}
	\begin{definition}[紧性]
		若拓扑空间$X$的任一开覆盖都包含有限子覆盖,则称$X$为\textbf{紧的}.
	\end{definition}
	下面给出紧空间的若干性质.
	\begin{theorem}\label{compact}
		紧空间在连续映射下的像仍是紧的.
	\end{theorem}
	\begin{proof}
		不妨设连续映射$f:X\to Y$是满射. 设$\mathcal{F}$是$Y$的一族开覆盖,由连续映射的定义,$\left\{f^{-1}(O):O\in\mathcal{F}\right\}$是开集,$f^{-1}(O_i)_{i\in I}$给出了$X$的一个开覆盖,又$X$是紧的,故开覆盖存在有限子覆盖,记为$\left\{f^{-1}(O_i),\ 1\leqslant i\leqslant k\right\}$,有$\bigcup_{i=1}^{k}f^{-1}(O_i)=X$,$f(X)=\bigcup_{i=1}^{k}O_i=Y$,于是存在$\mathcal{F}$的有限子覆盖,故$Y$是紧的.
	\end{proof}
	\begin{theorem}\label{close}
		紧空间的任一闭子集仍是紧的.
	\end{theorem}
	\begin{proof}
		设$X$为紧空间,$C\subset X$为闭集,则$X\backslash C$为开集,存在$C$的一族开覆盖$\mathcal{F}=\left\{O_i:i\in I\right\}$,$C\subset\bigcup_{i\in I}O_i$,于是$\bigcup_{i\in I}O_i\cup(X\backslash C)$是$X$的一个开覆盖,而$X$是紧的,故存在有限子覆盖$\{O_i\ (1\leqslant i\leqslant k),\ X\backslash C\}$,于是$O_i\ (1\leqslant i\leqslant k)$是$C$的一族开覆盖,为$\mathcal{F}$的有限子覆盖,故$C$是紧的.
	\end{proof}
	紧空间的任一紧子集却不一定是闭的,有以下结论.
	\begin{theorem}\label{Hcom}
		Hausdorff空间的紧子集是闭的.
	\end{theorem}
	\begin{proof}
		设紧集$A$是Hausdorff空间$X$的子集,即证$X\backslash A$为开集.
		设$x\notin A$,$z_i\in A$,则对任意开集$V_i\ni z_i$,有$U_i\ni x_i$使得$U_i\cap V_i=\varnothing$,由于$A$为紧集,故存在正整数$k$,使得$A\subset\bigcup_{i=1}^{k}V_i$,令$V=\bigcup_{i=1}^{k}V_i$,$U=\bigcap_{i=1}^{k}U_i$,则$U\cap V=\varnothing$,故$U\cap A=\varnothing$,$U\subset X\backslash A$,于是$X\backslash A$为开集,$A$为闭集.
	\end{proof}
	同胚映射需要保证双射、连续以及逆映射连续,连续双射未必是同胚映射,在从紧空间到Hausdorff空间的映射却是成立的.
	\begin{theorem}
		从紧空间到Hausdorff空间的连续双射是同胚映射.
	\end{theorem}
	\begin{proof}
		设$f:X\to Y$,闭集$C\subset X$,则由定理\ref{close},$C$为紧集. 由定理\ref{compact},$f(C)$为紧的,由定理\ref{Hcom},$f(C)$是闭的,由连续映射等价命题,$f^{-1}$连续,故$f$为同胚映射.
	\end{proof}
	\begin{theorem}[Bolzano-Weierstrass]
		紧空间的任一无限子集必有至少一个极限点.
	\end{theorem}
	\begin{proof}
		设$X$为紧空间,$S\subset X$没有极限点,下证$S$为有限集.
		
		对任意$x\in X$,存在开集$O(x)$,满足
		\begin{equation*}
			O(x)\cap S=\left\{
			\begin{aligned}
				&\varnothing,\ x\notin S,\\
				&\{x\},\ x\in S,
			\end{aligned}
			\right.
		\end{equation*}
		否则$S$有极限点. 取遍$x$,可以得到一族$O(x)$构成$X$的开覆盖. 由于$X$是紧的,故开覆盖存在有限子覆盖,有$x_1,\cdots,x_k$,使得$X\subset\bigcup_{i=1}^{k}O(x_i)$,而$O(x_i)$至多包含$S$中的一个点,于是$S$为有限集.
	\end{proof}
	\begin{theorem}[Heine-Borel]
		实数轴上任一闭区间是紧集.
	\end{theorem}
	\begin{proof}
		对任意$\left[a,b\right]\in\mathbb{R}$,定义它的一族开覆盖$\mathcal{F}$. 设$A\subset \left[a,b\right]$,以如下方式定义.
		$$X=\left\{x\in\left[a,b\right]:\text{存在}\mathcal{F}\text{的有限子覆盖包含}\left[a,x\right]\right\}.$$
		于是$X$非空($a\in X$)且有界$x\leqslant b$,由确界原理,$X$有上确界. 设$s=\sup X$,下证$s\in X$且$s=b$.
		
		设$s\in O\in\mathcal{F}$,由于$O$是开集,故存在$\varepsilon>0$,$\left(s-\varepsilon,s\right]\subset O$,若$s<b$,则有$(s-\varepsilon,s+\varepsilon)\subset O$. 对任意$\varepsilon>0$,有$s-\varepsilon\in X$,于是$\left[a,s-\varepsilon\right]\subset X$,故$\left[a,s-\varepsilon\right]$可被$\mathcal{F}$的有限子覆盖$\bigcup_{i=1}^{k}U_i$包含,又$\left(s-\varepsilon,s\right]\subset O$,于是$\left[a,s\right]$可被$O\cup\bigcup_{i=1}^{k}U_i$包含,故$s\in X$.
		
		若$s<b$,则$s+\varepsilon/2\in(s-\varepsilon,s+\varepsilon)\subset O$,有$s+\varepsilon/2\in X$,这与$s$是$X$的上确界矛盾!于是$s=b$.
	\end{proof}
	\begin{corollary}
		Euclidean空间上的有界闭集是紧集.
	\end{corollary}
	\begin{proof}
		由两个紧空间的积空间仍为紧空间即得.
	\end{proof}
	\begin{theorem}\label{comclose}
		Euclidean空间上的紧集是有界闭集.
	\end{theorem}
	\begin{proof}
		设$A\subset\mathbb{R}^n$,Euclidean空间显然是Hausdorff空间,由定理\ref{Hcom},$A$是闭集. 构造一列开球$\left\{B(0,r_i),\ r_k=k,\ k\in\mathbb{Z}^+\right\}$,是$\mathbb{R}^n$的一个开覆盖,自然是$A$的一个开覆盖. 又$A$是紧集,故该开覆盖存在有限子覆盖,于是$r_i$有限,$A$有界.
	\end{proof}
	于是,在Euclidean空间中,有界闭集和紧集是等价的.
	\begin{theorem}
		定义在紧空间上的连续实值函数有界且能达到边界.
	\end{theorem}
	\begin{proof}
		设$f:X\to\mathbb{R}$,由定理\ref{compact},$f(X)\subset\mathbb{R}$是紧的,由定理\ref{comclose},$f(X)$是有界闭集. 又闭集的性质,存在$x_1,x_2\in X$,$f(x_1)=\inf f(X)$,$f(x_2)=\sup f(X)$,于是函数有界且能达到边界.
	\end{proof}
	\begin{theorem}[Lebesgue数引理]
		设$X$为紧度量空间,$\mathcal{F}$为$X$的一个开覆盖,则存在$\delta>0$(称为Lebesgue数),使得对任意$x\in X$,存在$U\in\mathcal{F}$,$B(x,\delta)\subset U$.
	\end{theorem}
	\begin{proof}
		反证法,设序列$\{A_n\}$是$X$的一列子集,每一项都不包含于$\mathcal{F}$的任一开集中,且序列的直径趋于$0$. 对每个$n\in\mathbb{Z}^+$,存在$x_n\in A_n$. 对于序列$\{x_n\}$,要么包含有限个离散的点,存在某点无限重复出现,要么包含无限个点,由于$X$是紧的,故存在极限点. 
		
		记无限重复出现的点或极限点为$p$,设$U\in\mathcal{F}$是包含$p$的开集,则存在$\varepsilon>0$,$B(p,\varepsilon)\subset U$,存在足够大的正整数$N$,使得$A_N$的直径小于$\varepsilon/2$且$x_N\in B(p,\varepsilon/2)$,于是对$x\in A_N$,有$d(x_N,p)<\varepsilon/2$,$d(x,x_N)<\varepsilon/2$,于是
		$$d(x,p)<d(x,x_N)+d(x_N,p)<\varepsilon.$$
		即$A_N\subset U$,这与$A_n$的构造假设矛盾!
	\end{proof}
	\begin{definition}[局部紧]
		若拓扑空间$X$中每一点都有紧的邻域,则称$X$是\textbf{局部紧的}.
	\end{definition}
	\begin{definition}[单点紧化]
		
	\end{definition}
\end{document}